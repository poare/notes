\documentclass[11pt, oneside]{article}   	% use "amsart" instead of "article" for AMSLaTeX format
\usepackage[margin = 1in]{geometry}                		% See geometry.pdf to learn the layout options. There are lots.
\geometry{letterpaper}                   		% ... or a4paper or a5paper or ... 
%\geometry{landscape}                		% Activate for rotated page geometry
%\usepackage[parfill]{parskip}    		% Activate to begin paragraphs with an empty line rather than an indent
\usepackage{graphicx}				% Use pdf, png, jpg, or eps§ with pdflatex; use eps in DVI mode
								% TeX will automatically convert eps --> pdf in pdflatex		
\usepackage{amssymb}
\usepackage{amsmath}
\usepackage[shortlabels]{enumitem}
\usepackage{float}
\usepackage{tikz-cd}
\usepackage{subcaption}
\usepackage{simpler-wick}
\usepackage[compat=1.0.0]{tikz-feynman}   %note you need to compile this in LuaLaTeX for diagrams to render correctly

\usepackage{verbatim}
\usepackage{amsthm}
\usepackage{hyperref}

%%%%%%%%%%%%%%%%%%%%%%%%%%%%%%%%%%%%%%%%%%%%%%%%
%%%%%%%%%%%%%%% CUSTOM MATH ENVIRONMENTS %%%%%%%%%%%%%%%
%%%%%%%%%%%%%%%%%%%%%%%%%%%%%%%%%%%%%%%%%%%%%%%%

\usepackage{mdframed}
\usepackage{xparse}
\usepackage{framed}		% Colored boxes. \begin{shaded} to use the package
\usepackage{minted}

\definecolor{lightgray}{rgb}{0.93, 0.93, 0.93}
\definecolor{lightpurple}{rgb}{0.9, 0.7, 1.0}
\definecolor{lightblue}{rgb}{0.2, 0.7, 0.7}
%\definecolor{lightred}{rgb}{0.8, 0.2, 0.2}
\definecolor{lightred}{rgb}{0.99, 0.0, 0.0}
\definecolor{lightgreen}{rgb}{0.2, 0.6, 0.2}
\definecolor{magenta}{rgb}{0.9, 0.2, 0.9}

\colorlet{shadecolor}{lightgray}		% 40% purple, 40% white
\colorlet{defcolor}{lightpurple!40}
\colorlet{thmcolor}{lightblue!20}
\colorlet{excolor}{lightred!30}
\colorlet{rescolor}{lightgreen!40}
\colorlet{intercolor}{magenta!40}

% Definition
\newcounter{dfnctr}
\newenvironment{definition}[1][]{
\stepcounter{dfnctr}
%\protected@edef\@currentlabelname{dfnctr}
\ifstrempty{#1}
{\mdfsetup{
frametitle={
\tikz[baseline=(current bounding box.east),outer sep=0pt]
\node[anchor=east,rectangle,fill=defcolor]
{\strut Definition~\arabic{dfnctr}};}}
}
{\mdfsetup{
frametitle={
\tikz[baseline=(current bounding box.east),outer sep=0pt]
\node[anchor=east,rectangle,fill=defcolor]
{\strut Definition~\arabic{dfnctr}:~#1};}}
}
\mdfsetup{innertopmargin=3pt,linecolor=lightpurple,
linewidth=2pt,topline=true,
frametitleaboveskip=\dimexpr-\ht\strutbox\relax,}
%\begin{mdframed}[skipabove=2cm, splittopskip=\baselineskip]\relax%
\begin{mdframed}[]\relax%
}{\end{mdframed}}

% Theorem
\newcounter{thmctr}
\newenvironment{theorem}[1][]{
\stepcounter{thmctr}
\ifstrempty{#1}
{\mdfsetup{
frametitle={
\tikz[baseline=(current bounding box.east),outer sep=0pt]
\node[anchor=east,rectangle,fill=thmcolor]
{\strut Theorem~\arabic{thmctr}};}}
}
{\mdfsetup{
frametitle={
\tikz[baseline=(current bounding box.east),outer sep=0pt]
\node[anchor=east,rectangle,fill=thmcolor]
{\strut Theorem~\arabic{thmctr}:~#1};}}
}
\mdfsetup{innertopmargin=3pt,linecolor=lightblue!60,
linewidth=2pt,topline=true,
frametitleaboveskip=\dimexpr-\ht\strutbox\relax,}
\begin{mdframed}[]\relax%
}{\end{mdframed}}

% Corollary
\newcounter{corctr}
\newenvironment{corollary}[1][]{
\stepcounter{corctr}
\ifstrempty{#1}
{\mdfsetup{
frametitle={
\tikz[baseline=(current bounding box.east),outer sep=0pt]
\node[anchor=east,rectangle,fill=thmcolor]
{\strut Corollary~\arabic{corctr}};}}
}
{\mdfsetup{
frametitle={
\tikz[baseline=(current bounding box.east),outer sep=0pt]
\node[anchor=east,rectangle,fill=thmcolor]
{\strut Corollary~\arabic{corctr}:~#1};}}
}
\mdfsetup{innertopmargin=3pt,linecolor=lightblue!60,
linewidth=2pt,topline=true,
frametitleaboveskip=\dimexpr-\ht\strutbox\relax,}
\begin{mdframed}[]\relax%
}{\end{mdframed}}

% Proposition
\newcounter{propctr}
\newenvironment{prop}[1][]{
\stepcounter{propctr}
\ifstrempty{#1}
{\mdfsetup{
frametitle={
\tikz[baseline=(current bounding box.east),outer sep=0pt]
\node[anchor=east,rectangle,fill=thmcolor]
{\strut Proposition~\arabic{propctr}};}}
}
{\mdfsetup{
frametitle={
\tikz[baseline=(current bounding box.east),outer sep=0pt]
\node[anchor=east,rectangle,fill=thmcolor]
{\strut Proposition~\arabic{propctr}:~#1};}}
}
\mdfsetup{innertopmargin=3pt,linecolor=lightblue!60,
linewidth=2pt,topline=true,
frametitleaboveskip=\dimexpr-\ht\strutbox\relax,}
\begin{mdframed}[]\relax%
}{\end{mdframed}}

% Lemma
\newcounter{lemctr}
\newenvironment{lemma}[1][]{
\stepcounter{lemctr}
\ifstrempty{#1}
{\mdfsetup{
frametitle={
\tikz[baseline=(current bounding box.east),outer sep=0pt]
\node[anchor=east,rectangle,fill=thmcolor]
{\strut Lemma~\arabic{lemctr}};}}
}
{\mdfsetup{
frametitle={
\tikz[baseline=(current bounding box.east),outer sep=0pt]
\node[anchor=east,rectangle,fill=thmcolor]
{\strut Lemma~\arabic{lemctr}:~#1};}}
}
\mdfsetup{innertopmargin=3pt,linecolor=lightblue!60,
linewidth=2pt,topline=true,
frametitleaboveskip=\dimexpr-\ht\strutbox\relax,}
\begin{mdframed}[]\relax%
}{\end{mdframed}}

% Example
\newcounter{exctr}
\newenvironment{example}[1][]{
\stepcounter{exctr}
\ifstrempty{#1}
{\mdfsetup{
frametitle={
\tikz[baseline=(current bounding box.east),outer sep=0pt]
\node[anchor=east,rectangle,fill=excolor]
{\strut Example~\arabic{exctr}};}}
}
{\mdfsetup{
frametitle={
\tikz[baseline=(current bounding box.east),outer sep=0pt]
\node[anchor=east,rectangle,fill=excolor]
{\strut Example~\arabic{exctr}:~#1};}}
}
\mdfsetup{innertopmargin=3pt,linecolor=excolor,
linewidth=2pt,topline=true,
frametitleaboveskip=\dimexpr-\ht\strutbox\relax,}
\begin{mdframed}[]\relax%
}{\end{mdframed}}

% Resources
\newcounter{resctr}
\newenvironment{resources}[1][]{
\stepcounter{resctr}
\ifstrempty{#1}
{\mdfsetup{
frametitle={
\tikz[baseline=(current bounding box.east),outer sep=0pt]
\node[anchor=east,rectangle,fill=rescolor]
{\strut Resources};}}
}
{\mdfsetup{
frametitle={
\tikz[baseline=(current bounding box.east),outer sep=0pt]
\node[anchor=east,rectangle,fill=rescolor]
{\strut Resources};}}
}
\mdfsetup{innertopmargin=3pt,linecolor=rescolor,
linewidth=2pt,topline=true,
frametitleaboveskip=\dimexpr-\ht\strutbox\relax,}
\begin{mdframed}[]\relax%
}{\end{mdframed}}

% Interlude
\newcounter{interctr}
\newenvironment{interlude}[1][]{
\stepcounter{interctr}
\ifstrempty{#1}
{\mdfsetup{
frametitle={
\tikz[baseline=(current bounding box.east),outer sep=0pt]
\node[anchor=east,rectangle,fill=intercolor]
{\strut Example~\arabic{interctr}};}}
}
{\mdfsetup{
frametitle={
\tikz[baseline=(current bounding box.east),outer sep=0pt]
\node[anchor=east,rectangle,fill=intercolor]
{\strut Interlude~\arabic{interctr}:~#1};}}
}
\mdfsetup{innertopmargin=3pt,linecolor=intercolor,
linewidth=2pt,topline=true,
frametitleaboveskip=\dimexpr-\ht\strutbox\relax,}
\begin{mdframed}[]\relax%
}{\end{mdframed}}

%%%%%%%%%%%%%%%%%%%%%%%%%%%%%%%%%%%%%%%%%%%%%%%%
%%%%%%%%%%%%%%%%%% MATH COMMANDS %%%%%%%%%%%%%%%%%%%
%%%%%%%%%%%%%%%%%%%%%%%%%%%%%%%%%%%%%%%%%%%%%%%%

\usepackage{slashed}
\usepackage{bm}
\usepackage{cancel}

% Equation
\def\eq{\begin{equation}\begin{aligned}}
\def\qe{\end{aligned}\end{equation}}

% Common mathbb's
\newcommand{\N}{\mathbb{N}}
\newcommand{\R}{\mathbb{R}}
\newcommand{\Z}{\mathbb{Z}}
\newcommand{\Q}{\mathbb{Q}}

% make arrow superscripts
\DeclareFontFamily{OMS}{oasy}{\skewchar\font48 }
\DeclareFontShape{OMS}{oasy}{m}{n}{%
         <-5.5> oasy5     <5.5-6.5> oasy6
      <6.5-7.5> oasy7     <7.5-8.5> oasy8
      <8.5-9.5> oasy9     <9.5->  oasy10
      }{}
\DeclareFontShape{OMS}{oasy}{b}{n}{%
       <-6> oabsy5
      <6-8> oabsy7
      <8->  oabsy10
      }{}
\DeclareSymbolFont{oasy}{OMS}{oasy}{m}{n}
\SetSymbolFont{oasy}{bold}{OMS}{oasy}{b}{n}
\DeclareMathSymbol{\smallleftarrow}     {\mathrel}{oasy}{"20}
\DeclareMathSymbol{\smallrightarrow}    {\mathrel}{oasy}{"21}
\DeclareMathSymbol{\smallleftrightarrow}{\mathrel}{oasy}{"24}
\newcommand{\vecc}[1]{\overset{\scriptscriptstyle\smallrightarrow}{#1}}
\newcommand{\cev}[1]{\overset{\scriptscriptstyle\smallleftarrow}{#1}}
\newcommand{\cevvec}[1]{\overset{\scriptscriptstyle\smallleftrightarrow}{#1}}

% Other commands
\newcommand{\im}{\mathrm{im}}
\newcommand{\supp}{\mathrm{supp}}
\newcommand{\Tr}{\mathrm{Tr}}
\newcommand{\dbar}{d\hspace*{-0.08em}\bar{}\hspace*{0.1em}}
\newcommand{\Hom}{\mathrm{Hom}}
\newcommand{\Span}{\mathrm{span}}

% to use a black and white box environment, use \begin{answer} and \end{answer}
\usepackage{tcolorbox}
\tcbuselibrary{theorems}
\newtcolorbox{answerbox}{sharp corners=all, colframe=black, colback=black!5!white, boxrule=1.5pt, halign=flush center, width = 1\textwidth, valign=center}
\newenvironment{answer}{\begin{center}\begin{answerbox}}{\end{answerbox}\end{center}}