\documentclass[11pt, oneside]{amsart}   	% use "amsart" instead of "article" for AMSLaTeX format
\usepackage[margin = 1in]{geometry}                		% See geometry.pdf to learn the layout options. There are lots.
\geometry{letterpaper}                   		% ... or a4paper or a5paper or ... 
%\geometry{landscape}                		% Activate for rotated page geometry
%\usepackage[parfill]{parskip}    		% Activate to begin paragraphs with an empty line rather than an indent
\usepackage{graphicx}				% Use pdf, png, jpg, or eps§ with pdflatex; use eps in DVI mode
								% TeX will automatically convert eps --> pdf in pdflatex		
\usepackage{amssymb}
\usepackage{amsmath}
\usepackage[shortlabels]{enumitem}
\usepackage{float}
\usepackage{tikz-cd}

\usepackage{amsthm}
\theoremstyle{definition}
\newtheorem{definition}{Definition}[section]
\newtheorem{theorem}{Theorem}[section]
\newtheorem{corollary}{Corollary}[theorem]
\newtheorem{lemma}[theorem]{Lemma}

\newcommand{\N}{\mathbb{N}}
\newcommand{\R}{\mathbb{R}}
\newcommand{\Z}{\mathbb{Z}}
\newcommand{\Q}{\mathbb{Q}}

%SetFonts

%SetFonts


\title{Homology}
\author{Patrick Oare}
\date{}							% Activate to display a given date or no date

\begin{document}
\maketitle

Like the fundamental group, homology is an important algebraic invariant of topological spaces which can 
be used to describe the properties of such spaces. Although it is more complicated to define, it provides 
a powerful tool to use to classify spaces. The essence of homology is to define a sequence of functors 
$H_n : Top\rightarrow Ab$ which are invariant up to homotopy. A particular advantage of $H_n(\cdot)$ 
over homotopy groups $\pi_n(\cdot)$ is that the homology groups are abelian, and thus their structure 
is much easier to understand than their non-abelian counterparts. 

We will begin by studying simplicial homology, which is a particular example of a homology theory. Intuitively, 
the simplicial homology groups of a space will classify how many ``n-dimensional holes" the space has. 
Eventually, we will generalize the notion of homology to an arbitrary chain complex and examine 
chain complex structures that may arise as we consider different examples of spaces. 

\section{Simplicial Homology}

We begin with simplicial homology, which is a particular way to construct homology groups on a space. We 
will aim to study a space $X$ based on maps from simpler spaces (the $n$-simplices) into $X$.

\begin{definition}[Standard $n$-simplex]
	For $n > 0$, the \textbf{standard $n$-simplex} $\Delta^n$ is the convex hull of the standard basis 
	$\{e_0, ..., e_n\}$ of $\mathbb R^{n + 1}$, i.e. we have:
	\begin{equation}
		\Delta^n := \left\{\sum_{k = 0}^n t_k e_k : \sum_{k = 0}^n t_k = 1, t_k\geq 0\right\}
	\end{equation}
	The numbers $(t_0, ..., t_k)$ are called \textbf{barycentric coordinates}. 
\end{definition}

The standard $n$-simplex is an extraordinary simple space, and just looks like a hyper-triangle embedded 
into $\mathbb R^{n + 1}$. We can describe the $n$-simplex as an ordered pair $[v_0, ..., v_n]$, where 
each $v_i\in \{0, ..., n\}$ and none are repeated. This assigns each simplex an implicit orientation, and 
we can use this to manipulate such simplices. For each $n$, we have inclusion maps:
\begin{equation}
	d^i : \Delta^{n - 1}\rightarrow\Delta^n, \;\; 0\leq i\leq n
\end{equation}
where we map $[v_0, ..., v_{n - 1}]$ to $[v_0, ..., \hat v_i, ..., v_n]$, where the $\hat v_i$ denotes that we 
omit the vertex $v_i$. 

\begin{definition}[Singular $n$-simplex]
	Let $X$ be a topological space. Then a \textbf{singular $n$-simplex} is a map:
	\begin{equation}
		\sigma : \Delta^n\rightarrow X
	\end{equation}
	We define $Sin_n(X)$ to be the set of all singular $n$-simplices in $X$. 
\end{definition}

Note that the face maps induce a canonical map between the singular $n$-simplices on $X$ through the 
following composition of maps, which will call $d_i : Sin_n(X)\rightarrow Sin_{n - 1}(X)$, $\sigma\mapsto \sigma\circ d^i$:
\[
	\begin{tikzcd}
		\Delta^{n - 1}\arrow[r, "d^i"]\arrow[dr, swap, "d_i\sigma"] & \Delta^n\arrow[d, "\sigma"] \\
		 & X \\
	\end{tikzcd}
\]
%$$
%	\Delta^{n - 1}\xrightarrow{d_i}\Delta^{n}\xrightarrow{\sigma} X
%$$

For $i\leq j$, the maps $d_i$ satisfy (and the face maps $d^i$ satisfy the same identity): 
\begin{equation}
	d_i d_j = d_{j + 1} d_i
\end{equation}

We wish to consider a variant of $Sin_n(X)$ which allows us to add and subtract simplices in a natural 
way. This will allow us to consider boundaries of simplices, which will be intimately connected to homology.  
For example, consider a 1-simplex which is a closed loop, i.e. $\gamma : [0, 1] = \Delta^1\rightarrow X$. 
We wish to be able to differentiate this loop from an open loop. We can almost do this with the definitions 
we have already made: notice that $d_0(\gamma) = \gamma(1) = \gamma(0) = d_1(\gamma)$. However, 
we need a way to take differences to make this precise. As such, we will consider the free abelian 
group generated by $Sin_n(X)$.
\begin{definition}[Singular $n$-chain]
	Let $S_n(X)$ be the free abelian group generated by $Sin_n(X)$:
	\begin{equation}
		S_n(X) := \mathbb Z Sin_n(X)
	\end{equation}
	We call an element of $S_n(X)$ a \textbf{singular $n$-chain}, and we may write each $n$-chain as:
	\begin{equation}
		\sum_{k = 1}^n a_k\sigma_k
	\end{equation}
	for $\sigma_k\in Sin_n(X)$. 
\end{definition}

\begin{definition}[Boundary operator]
	We define the \textbf{boundary operator} (also called a differential) $d : Sin_n(X)\rightarrow 
	S_{n - 1}(X)$ by:
	\begin{equation}
		\sigma\mapsto\sum_{i = 0}^n (-1)^i d_i\sigma
	\end{equation}
	This extends uniquely to a homomorphism:
	\begin{equation}
		d : S_n(X)\rightarrow S_{n - 1}(X)
	\end{equation}
\end{definition}

We will use this boundary map extensively, as it allows us to go between simplices of different dimensions. 
Intuitively, the boundary map just gives you the oriented face of the $n$-simplex; for example, the 
boundary of $\Delta^2$ is just an oriented (not filled in) triangle. Simplices which are killed by the 
differential are ``boundaries" of closed regions, in a sense which we will make precise.

\begin{definition}[$n$-cycle]
	An $n$-chain $c$ in $X$ is a \textbf{$n$-cycle} if $dc = 0$. We denote the set of all $n$-cycles by:
	\begin{equation}
		Z_n(X) := ker(d : S_n(X)\rightarrow S_{n - 1}(X)
	\end{equation}
\end{definition}

\begin{definition}[$n$-boundary]
	An $n$-chain $b$ in $X$ is a \textbf{$n$-boundary} if $b\in im(d)$. We denote the set of all 
	$n$-boundaries by:
	\begin{equation}
		B_n(X) := im(d : S_{n + 1}(X)\rightarrow S_n(X))
	\end{equation}
\end{definition}

\begin{theorem}
	The boundary operator satisfies:
	\begin{equation}
		d^2 = 0
	\end{equation}
	This implies that $B_n(X)\subseteq Z_n(X)$, so every boundary is a cycle. 
\end{theorem}

\begin{definition}[Graded abelian group]
	A \textbf{graded abelian group} is a sequence of abelian groups indexed by $\mathbb Z$. A 
	\textbf{chain complex} is a graded abelian group $\{A_n\}_{n\in\mathbb Z}$ together with 
	homomorphisms $d : A_n\rightarrow A_{n - 1}$ such that $d^2\equiv 0$. We will draw a chain 
	complex like so:
	\[
		\vspace{-.7cm}
		\begin{tikzcd}
			...\arrow[r, "d"] & A_{n + 1}\arrow[r, "d"] & A_n\arrow[r, "d"] & A_{n - 1}\arrow[r, "d"] & ... \\
		\end{tikzcd}
	\]
\end{definition}

An $n$-cycle in $Z_n(X)$ is closed in the way that a closed loop is closed-- it has no edges and all its 
faces connect with itself. The distinction between cycles and boundaries will give us homology groups. 
A nice way to visualize the difference is to consider an annulus in $\mathbb R^2$, $A = \{(x, y)\in\mathbb 
R^2 : x^2 + y^2\in [1, 3]\}$. A circle wrapping the annulus $x^2 + y^2 = 2$ is a cycle because it is a 
closed path; the boundary operator sends it to 0 since it links back to itself. However, it is \textbf{not} 
a boundary. It could almost be the boundary of $U := \{(x, y) : x^2 + y^2 \leq 2\}\cap A$, but this is not a 
simplex. $U$ is 2-dimensional and resembles the simplex $\Delta^2$, but unfortunately has a hole in it. 

In a manner such as this, homology will tell us about the holes that a space has by considering the 
quotient of the $n$-cycles by the $n$-boundaries. 

\begin{definition}
	For a topological space $X$, the $n$th \textbf{singular homology group} of $X$ is:
	\begin{equation}
		H_n(X) := Z_n(X) / B_n(X)
	\end{equation}
\end{definition}

Homology can also be defined in the same way for an arbitrary chain complex, and often once we have a 
chain complex we will forget about the underlying space it comes from. Note that homology groups 
$H_n(X)$ are always abelian, since $Z_n(X)$ is an abelian group. 

\begin{definition}[Semi-simplicial set]
	A collection of sets $K_n$ for $n\geq 0$ together with maps $d_i : K_n\rightarrow K_{n - 1}$ satisfying $d_i d_j = 
	d_{j + 1}d_i$ for $i\leq j$ is called a \textbf{semi-simplicial set}. 
\end{definition}

This notion generalizes the structure of the set $Sin_n(X)$. What we have done in the previous few pages is summarized 
as follows: to each topological space, we associate a semi-simplicial set $Sin_n(X)$. To each of these sets, we create the 
free abelian group $S_n(X)$ generated on these simplices. This forms a chain complex $(S_*(X), d)$, and we take the 
homology of said chain complex to form the homology of the space. 

For a basic computation, consider the homology of the one point space $X = \{*\}$. There is a single $n$ simplex for each 
$n$ because $X$ only has one point, which we will denote by $C_*^n$, so $S_*(X) = \mathbb Z\{C_*^n\}$. The composition 
of a face map with $C_*^n$ is $C_*^{n - 1}$ because we are simply sending the simplex of one dimension smaller to $*$, 
so $d_i C_*^n = C_*^{n - 1}$. But, this implies that:
\begin{equation}
	d C_*^n = \begin{cases}
		C_*^{n - 1} & n\textnormal{ even} \\
		0 & n\textnormal{ odd} \\
	\end{cases}
\end{equation}
because for odd $n$ we have an even number of face maps, so the alternating sum $d = \sum_{i = 0}^n (-1)^n d_i$ cancels 
itself and vanishes. For even $n$ we have one surviving $d_i$, so $d C_*^n = C_*^{n - 1}$. So, the chain complex of 
$n$-chains on $X$ is the following sequence (here I have labeled each copy of $\mathbb Z$ with which copy of $S_n(X)$ 
it is):
\[
	\begin{tikzcd}
		...\arrow[r, "id"] & \mathbb Z\arrow[r, "0"] & \mathbb Z\arrow[r, "id"] & \mathbb Z\arrow[r, "0"]
		& \mathbb Z\arrow[r, "0"] & 0\arrow[r] & ... \\
		 & S_3(X) & S_2(X) & S_1(X) & S_0(X) & \\
	\end{tikzcd}
\]
For the even and nonzero $n$, the kernel of $id : S_n(X)\cong\mathbb Z\rightarrow \mathbb Z$ is 0, so $Z_n(X) = 0$ and 
hence $H_n(X) = 0$. Similarly for odd $n$, the kernel of $0 : \mathbb Z\rightarrow\mathbb Z$ is $\mathbb Z$, but the image 
of the previous map is $im(id) = \mathbb Z$, hence $Z_n(X) = B_n(X) = \mathbb Z$, so the homology $H_n(X)$ vanishes as 
well. However, for $n = 0$, we note that $Z_0(X) = \mathbb Z$ and $B_0(X) = \{0\}$, hence $H_0(X) = \mathbb Z$. 
Hence to summarize:
\begin{equation}
	H_n(*) = \begin{cases}
	0 & n > 0 \\
	\mathbb Z & n = 0
	\end{cases}
\end{equation}

We will now briefly consider the functorial nature of homology before 
moving on to lay out some definitions in category theory. 

\begin{definition}[Chain map]
	Let $C_*$ and $D_*$ be two chain complexes. A \textbf{chain map} $f_* : C_*\rightarrow D_*$ is a collection of 
	homomorphisms $f_n : C_n\rightarrow D_n$ such that the following diagram commutes:
	\begin{equation}\begin{tikzcd}
	...\arrow[r, "d"] & C_{n + 1}\arrow[d, "f_{n + 1}"]\arrow[r, "d"] & C_n\arrow[d, "f_n"]\arrow[r, "d"] & C_{n - 1}\arrow[d, 
	"f_{n - 1}"]\arrow[r, "d"] & ... \\
	...\arrow[r, "d"] & D_{n + 1}\arrow[r, "d"] & D_n\arrow[r, "d"] & D_{n - 1}\arrow[r, "d"] & ... \\
	\end{tikzcd}\end{equation}
\end{definition}

A chain map is just a sequence of maps which respect the differential. Now, we show that a continuous map from $X$ to $Y$ 
induces a chain map on the $n$-chains of $X$, which we will show in turn induces a morphism on homology. 

\begin{theorem}
	Let $f : X\rightarrow Y$ be a continuous map. Then $f$ induces a chain map:
	\begin{equation}
		f_* : S_*(X)\rightarrow S_*(Y)
	\end{equation}
\end{theorem}

\begin{proof}
	The induced map is very much a natural one, and we shall map the generators $\sigma$ of $S_n(X)$ to $\sigma\mapsto 
	f\circ\sigma : \Delta^n\rightarrow Y$, which extends to a homomorphism $f_* : S_*(X)\rightarrow S_*(Y)$ by linearity. We 
	must show that $f_*$ is in fact a chain map. We have:
	\begin{equation}
		d_i (f_*\sigma) = d_i (f\circ\sigma) = (f\circ\sigma)\circ d^i = f(d_i\sigma)
	\end{equation}
	hence summing this up, we see that $df_* = f_*d$. 
\end{proof}

\begin{theorem}
	Any chain map $f_* : C_*\rightarrow D_*$ induces a homomorphism on homology $f_* : H_*(C)\rightarrow H_*(D)$. 
\end{theorem}

\begin{proof}
	We will restrict each $f_n : C_n\rightarrow D_n$ to the $n$-cycles $Z_n(C)$ and show that the map $f_n : Z_n(C)
	\rightarrow Z_n(D)$ and is well defined. Suppose that $x\in Z_n(C)$, so $dx = 0$. Then $df_n(x) = f_{n - 1}(dx) = 0$ 
	because $f_*$ is a chain map, hence $f_n(x)\in ker (d : D_n\rightarrow D_{n - 1})$, and the map is well defined on the 
	n-cycles. Projecting $Z_n(D)$ onto the homology with the map $\pi_D : Z_n(D)\rightarrow H_n(D)$, we have the 
	diagram:
	\[\begin{tikzcd}
		Z_n(C)\arrow[d, "\pi_C"]\arrow[r, "f_n"] & Z_n(D)\arrow[d, "\pi_D"] \\
		H_n(C)\arrow[r, dashed, "\tilde f_n"] & H_n(D) \\
	\end{tikzcd}\]
	and we will show that $\tilde f_n : H_n(C)\rightarrow H_n(D), [c]\mapsto \pi_D f_n(c)$, i.e. that we have a well defined 
	lift from the quotient. Suppose that $[c] = [c']\in H_n(C)$, so $c = c' + db$ for $b\in C_{n + 1}$. Then we must show that 
	$\tilde f([c]) = \tilde f([c']) + \tilde f([db]) = \tilde f([c'])$, so we have:
	\begin{equation}
		\tilde f([db]) = \pi_D f_n(db) = \pi_D (d f_{n + 1}b) = 0
	\end{equation}
	because $df_{n + 1} b\in B_n(D) = im(d : D_{n + 1}\rightarrow D_n)$. Hence the map is well defined, so we have a 
	valid morphism $H_n(C)\rightarrow H_n(D)$ for all $n$. 
\end{proof}

\newpage
\section{Categorical Considerations}

So, we have seen that any continuous map $f : X\rightarrow Y$ induces a well defined map $f_* : H_*(X)\rightarrow H_*(Y)$ 
on the homology. In other words, \textbf{$H_n(\cdot)$ is a covariant functor from $Top$ to $Ab$}. We will be using many 
of its functorial properties, and now may be a good time to brush up on your category theory. We will recall a few definitions 
which will be important to us. We denote the objects of a category $\mathcal C$ by $obj(\mathcal C)$, and the morphisms 
from $X$ to $Y$ by $\mathcal C(X, Y)$. 

\begin{definition}[Natural transformation]
	Let $F, G : \mathcal C\rightarrow\mathcal D$ be two functors. A \textbf{natural transformation} $\Theta : F\rightarrow G$ 
	consists of, for each $X\in obj(\mathcal C)$, a map $\Theta_X : F(X)\rightarrow G(X)$ such that for each 
	$f : X\rightarrow Y$ in $\mathcal C$, the following diagram commutes (in the category $\mathcal D$). 
	\begin{equation}\begin{tikzcd}
		F(X)\arrow[r, "\Theta_X"]\arrow[d, "F(f)"] & G(X)\arrow[d, "G(f)"] \\
		F(Y)\arrow[r, "\Theta_Y"] & G(Y) \\
	\end{tikzcd}\end{equation}
\end{definition}

A natural transformation is essentially a way of mapping one functor into another while respecting the structure of the functor; 
it is essentially a ``morphism" of functors. A functor takes you between categories $\mathcal C\rightarrow\mathcal D$, but 
a natural transformation takes you between functors. The canonical example of this is that for a small category $\mathcal C$ 
(recall $\mathcal C$ is small if $obj(\mathcal C)$ is a set), we can define the category of functors $Fun(\mathcal C, 
\mathcal D)$. The objects of this category are functors $F : \mathcal C\rightarrow\mathcal D$, and the morphisms of this 
category are natural transformations $\Theta : F\rightarrow G$. We also recall the definition of the opposite category:

\begin{definition}[Opposite category]
	For a category $\mathcal C$, define its \textbf{opposite category} $\mathcal C^{op}$ to be the category with:
	\begin{itemize}
		\item Objects $obj(\mathcal C^{op}) = obj(\mathcal C)$. 
		\item Morphisms $\mathcal C^{op}(X, Y) = \mathcal C(Y, X)$. 
	\end{itemize}
\end{definition}

We can formalize some of the notions that we've discussed about simplices by defining the \textbf{simplex category} $\Delta$. 
This category has the following data:
\begin{itemize}
	\item \textbf{Objects} are the sets $[n] := \{0, 1, ..., n\}$. 
	\item \textbf{Morphisms} are weakly order preserving maps $\phi : [n]\rightarrow [m]$. By weakly order preserving, we 
	mean that $i < j\implies \phi(i)\leq\phi(j)$. 
\end{itemize}

With this notation, the face maps are maps $d^i : [n - 1]\rightarrow [n]$
\begin{equation}
	d^i(k) = \begin{cases}
		k & k < i \\
		k + 1 & k\geq i
	\end{cases}
\end{equation}
The counterpart of the face maps are \textbf{degeneracy maps} $s^i : [n + 1]\rightarrow [n]$ which repeat the value $i$ are:
\begin{equation}
	s^i(k) = \begin{cases}
		k & k \leq i \\
		k - 1 & k > i
	\end{cases}
\end{equation}
Now, we can define simplicial objects.
\begin{definition}[Simplicial object]
	Let $\mathcal C$ be a category. A \textbf{simplicial object} in $\mathcal C$ is a functor $K : \Delta^{op}\rightarrow 
	\mathcal C$. A \textbf{semi-simplicial object} is a functor $K : \Delta_{inj}^{op}\rightarrow\mathcal C$, where 
	$\Delta_{inj}$ denotes the subcategory of $\Delta$ whose morphisms are injective. We use the phrase \textbf{simplicial 
	set} to mean a simplicial object in $Set$. 
\end{definition}

Suppose that we fix a space $X$. Then the set of simplicial complexes in $X$ is a functor:
\begin{equation}
	Sin_* : \Delta^{op}\rightarrow Set
\end{equation}
On objects of these categories, this assigns to each simplex $[n] = \Delta^n$ the set $Sin_n(X) = Top(\Delta^n, X)$. The 
reason this is from $\Delta^{op}$ to $Set$ is because of how morphisms are mapped by this functor. For the face map 
$d^i : \Delta^{n - 1} = [n - 1]\rightarrow \Delta^n = [n]$, we induce a morphism:
\begin{equation}
	d_i = Sin_*(d^i) : Sin_n(X) = Sin_*([n])\rightarrow Sin_{n - 1}(X) = Sin_*([n - 1])
\end{equation}
i.e. $Sin_*$ reverses the arrows between objects, as seen in the following diagram:
\[\begin{tikzcd}
\left[ n - 1\right]\arrow[r, "d^i"]\arrow[d, "Sin_*"] & \left[ n\right]\arrow[d, "Sin_*"] \\
Sin_{n - 1}(X) & Sin_n(X) \arrow[l, "d_i"] \\
\end{tikzcd}\]

We denote the category of simplicial sets to be \textbf{sSet}. This category has the following data:
\begin{itemize}
	\item Objects are simplicial sets, i.e. functors $\Delta^{op}\rightarrow Set$. 
	\item Morphisms are the natural transformations between these functors.
\end{itemize}

Now, we consider certain types of maps. We will use this shortly to define reduced homology.

\begin{definition}[Split epimorphism]
	A morphism $f : X\rightarrow Y$ in $\mathcal C$ is called a \textbf{split epimorphism} if there is $g : Y\rightarrow X$ 
	(such a $g$ is called a \textbf{section} of $f$) such that:
	\[\begin{tikzcd}
		Y\arrow[r, "g"]\arrow[dr, swap, "1_Y"] & X\arrow[d, "f"] \\ 
		& Y\\
	\end{tikzcd}\]
	A morphism $g : Y\rightarrow X$ in $\mathcal C$ is called a \textbf{split monomorphism} if there is $f : Y\rightarrow X$ 
	which makes the above diagram commute.
\end{definition}

\begin{lemma}
	A morphism is an isomorphism iff it is a split epimorphism and a split monomorphism.
\end{lemma}

\begin{lemma}
	Any functor sends split epimorphisms to split epimorphisms and split monomorphisms to split monomorphisms.
\end{lemma}
Epimorphisms are the categorical generalization of surjective morphisms, and monomorphisms generalize injectivity. For 
example, if $\mathcal C = Set$, then an epimorphism is exactly a surjective map. Let us examine the category $\mathcal 
C = Ab$ of abelian groups. Suppose that $f : A\rightarrow B$ is a split epimorphism. Then we have a section $g : B\rightarrow 
A$, and we can consider the inclusion $\iota : ker(f)\hookleftarrow A$. Then we have an isomorphism:
\begin{equation}
	\iota\oplus g : ker(f)\oplus B\xrightarrow{\sim} A
\end{equation}
This should be reasonably intuitive because a split epimorphism is a surjection, so $B\cong A / ker(f)$, and  
the presence of a section means we can actually solve the extension problem in this case. 

We now put these ideas into practice. First recall the definition of terminal and initial objects in a category.
\begin{definition}
	Let $\mathcal C$ be a category. A \textbf{terminal object} is an object $X\in obj(\mathcal C)$ such that for each $A\in 
	obj(\mathcal C)$, there is a morphism (which is unique) $A\rightarrow X$. An \textbf{initial object} is an object $X$ 
	such that for each $A\in obj(\mathcal C)$, there is a morphism $X\rightarrow A$. 
\end{definition}

In the category \textbf{Top}, an initial object is $\emptyset$ and a terminal object is the one-point space $\{*\}$ (which we 
often denote by $*$). So, let $X$ be a topological space. Then we get a map:
\begin{equation}
	\xi_n : H_n(X)\longrightarrow H_n(*) = \begin{cases} \mathbb Z & n = 0 \\ 0 & \textnormal{else} \end{cases}
\end{equation}
because $H_n$ is a covariant functor. On the $0$-chains in $X$, note that $S_0(X)$ is generated by the points in $X$ and 
each element $c\in S_0(X)$ is of the form $c = \sum_{i = 1}^m a_i x_i$ with $a_i\in\mathbb Z$ and $x_i\in X$. So, we get 
a map $S_0(X)\rightarrow S_0(*), c\mapsto\sum_{i = 1}^m a_i$, which induces the map $\xi_0 : H_0(X)\rightarrow H_0(*)$. 

If $X\neq\emptyset$, then $X\rightarrow *$ is a split epimorphism, so $\xi_n$ is a split epimorphism as well. Thus we must 
be able to factor its kernel out as a summand:
\begin{equation}
	H_n(X)\cong H_n(*)\oplus ker(\xi_n)
\end{equation}
We will call this kernel the reduced homology group of $X$.
\begin{definition}[Reduced homology group]
	Let $X\neq\emptyset$ be a topological space. Then the \textbf{reduced $n$th homology group} of $X$ is defined as 
	$\tilde H_n(X) = ker(\xi_n)$, and the homology splits as a sum:
	\begin{equation}
		H_*(X) = H_*(*)\oplus \tilde H_*(X)
	\end{equation}
\end{definition}
In particular, note that the homology is the reduced homology for each $H_n(X)$ as long as $n > 0$. For the $n = 0$ case, 
the reduced homology factors out a copy of $\mathbb Z$:
\begin{equation}
	H_0(X) = \mathbb Z\oplus\tilde H_0(X)
\end{equation}
The reduced homology essentially compensates for the fact that $*$ has a nontrivial 0th homology group, so we can 
extract only the interesting mathematics from more complicated spaces and not be required to carry a copy of $\mathbb Z$ 
with us everywhere we go. 

Note that the interpretation of the $0$th homology group is that it characterizes the path-components of a space. In this 
case because $*$ has a single path component, it has $H_0(*) = \mathbb Z$. On the other hand, a space like $S^1\coprod 
S^1$ has $H_0(S^1\coprod S^1) = \mathbb Z\oplus\mathbb Z$ because it has two path components. 

We now consider the effect of homotopy on the homology groups of the space. Like the fundamental group, we will show 
that the homotopy groups $H_n(X)$ are invariant under homotopy, which offers another way to categorize spaces, just like 
$\pi_1(X)$. 

\newpage
\section{Homotopy}

Recall the definition of a homotopy. If you are unfamiliar with this concept, it may be useful to check out my notes on 
the subject.

\begin{definition}[Homotopy]
	Let $f, g : X\rightarrow Y$ be continuous maps. A \textbf{homotopy} from $f$ to $g$ is a continuous map:
	\begin{equation}
		h : X\times I\rightarrow Y
	\end{equation}
	where $I = [0, 1]$ is the unit interval, such that $h(x, 0) = f(x)$ and $h(x, 1) = g(x)$ for each $x\in X$. We say that 
	$f$ and $g$ are \textbf{homotopic}, and $h$ is a \textbf{homotopy} between them. We denote this by $f\simeq_h g$. 
\end{definition}

Note that $\simeq$ is an equivalence relation on functions. We let $[X, Y]$ denote the homotopy classes of maps 
from $X$ to $Y$:
\begin{equation}
	[X, Y] := Top(X, Y) / \simeq
\end{equation}

As in studying the fundamental group, homotopy will be an essential tool for our study of homology. Namely (although 
we will not prove this until later), homology is invariant under homotopy, and so if $H_*(X)\neq H_*(Y)$, then the spaces 
$X$ and $Y$ cannot be homotopy equivalent, much less so homeomorphic.
\begin{theorem}[Homotopy invariance of homology]
	View $H_*(\cdot)$ as a functor from \textbf{Top} to \textbf{Ab}. If $f_0\simeq f_1 : X\rightarrow Y$, then:
	\begin{equation}
		H_*(f_0) = H_*(f_1) : H_*(X)\rightarrow H_*(Y)
	\end{equation}
	and homology cannot distinguish between homotopic maps. 
\end{theorem}

\begin{definition}
	The \textbf{homotopy category} of topological spaces \textbf{Ho(Top)} is the category with:
	\begin{itemize}
		\item Objects: Same as \textbf{Top}.
		\item Morphisms: $\textbf{Ho(Top)}(X, Y) := [X, Y]$
	\end{itemize}
\end{definition}

\begin{definition}[Homotopy equivalence]
	A map $f : X\rightarrow Y$ is a \textbf{homotopy equivalence} if $[f]\in [X, Y]$ is an isomorphism. Equivalently, $f$ 
	is a homotopy equivalence if there is $g : Y\rightarrow X$ such that $[f\circ g] = [1_Y]$ and $[g\circ f] = [1_X]$. 
	In this case, we call $g$ a \textbf{homotopy inverse} of $f$, and we say that $X$ and $Y$ are \textbf{homotopy 
	equivalent}. 
\end{definition}

Note that because of the homotopy invariance of homology, $H_* : \textbf{Top}\rightarrow\textbf{Ab}$ factors through 
\textbf{Ho(Top)} as $\textbf{Top}\rightarrow\textbf{Ho(Top)}\rightarrow\textbf{Ab}$. An easy corollary of the homotopy 
invariance of homology is that homotopy equivalent spaces have the same homology groups, as seen below.

\begin{corollary}
	Homotopy equivalence induces an isomorphism in $H_*$, i.e. if $X$ is homotopy equivalent to $Y$, then they have the 
	same homology groups. 
\end{corollary}
\begin{proof}
	If $f : X\rightarrow Y$ and $g : Y\rightarrow X$ are homotopy inverses, then $H_*(f)\circ H_*(g) = H_*(id_X) = 
	1_{H_*(X)}$, and similarly for the other way. So, $H_*(f)$ and $H_*(g)$ are inverses of one another and hence 
	isomorphisms.
\end{proof}

A nice diagram to remember when dealing with the homotopy invariance of homology is the following commutative square 
of compositions (where $\pi$ is the quotient map taking $\textbf{Top}(X, Y)\rightarrow [X, Y]$):
\[\begin{tikzcd}
	\textbf{Top}(X, Y)\times\textbf{Top}(Y, Z)\arrow[d, "\pi"]\arrow[r, "\circ"] & \textbf{Top}(X, Z)\arrow[d, "\pi"] \\
	\left[X, Y\right]\times \left[Y, Z\right]\arrow[r, "\circ"] & \left[X, Z\right] \\
\end{tikzcd}\]

Homotopy equivalence is a nicer type of map in many ways than homormorphism, simply because it is weaker. However, 
note that it does not preserve many nice topological properties of spaces, including compactness, metrizability, being 
Hausdorff, second countability, and dimension (as a topological manifold).
The essential reasoning behind this is that homotopy equivalence allows one to crush a space into a point, as long as there 
are no holes in the region of interest. It only preserves the structure of the space in the broadest sense, and if there are 
parts of the space that may deformation retract onto other parts (like a dangling fiber), then no information about this 
need be preserved under homotopy equivalence.

To directly illustrate an example of one of these properties being lost, consider the sphere $S^{n - 1}\rightarrow \mathbb R^n
\setminus\{0\}$. Then $\mathbb R^n\setminus\{0\}$ deformation retracts onto $S^{n - 1}$ because the projection map 
$v\rightarrow \frac{v}{||v||}$ onto the sphere is a homotopy equivalence. However, the space $\mathbb R^n\setminus\{0\}$ is 
certainly not a compact space, but $S^{n - 1}$ is a compact space. Hence when working with homotopy equivalence, it is 
essential to remember that it is not nearly as strong a map as a homeomorphism, and that some properties may be lost in 
translation from one space to another.

\begin{definition}
	A space $X$ is \textbf{contractible} if the unique map $X\rightarrow\{*\}$ is a homotopy equivalence. 
\end{definition}

\begin{corollary}
	Let $X$ be contractible. Then $H_0(X) \cong\mathbb Z$ and $H_n(X)\cong 0$ for $n > 0$, i.e. we have:
	\begin{equation}
		\tilde H_*(X) = 0
	\end{equation}
\end{corollary}

If a space is contractible, it has ``no holes in its surface", and can be crushed down into the one point space by homotopy. 
This implies that it has both a trival fundamental group and trivial reduced homology groups. However, note that the converse 
is not true: if a space has trivial reduced homology groups, it need not be contractible. This is closely related to the Poincar\'e 
conjecture, and because the proof of this is extremely sophisticated we will not have the tools to prove this in these notes. 
Another way to formulate contractibility is if $X$ can deformation retract onto a point, which we will define now.

\begin{definition}[Deformation retract]
	Let $X$ be a space. An inclusion $\iota : A\hookrightarrow X$ is a \textbf{deformation retract} if there the homotopy 
	relative to $A$ from the identity into a map $X\rightarrow A$, i.e. we have a map $h : X\times I\rightarrow A$ such that 
	the following hold:
	\begin{enumerate}
		\item $h(x, 0) = x$ for each $x\in X$.
		\item $h(a, t) = a$ for $a\in A$ and $t\in I$.
		\item $h(x, 1)\in A$ for each $x\in X$. 
	\end{enumerate}
\end{definition}

We will now work towards a proof of the homotopy invariance of homology. To do this, we take a brief detour into the concept 
of a chain homotopy.

\begin{definition}[Chain homotopy]
	Let $C_*, D_*$ be chain complexes and $f_*, g_* : C_*\rightarrow D_*$ be chain maps. Then a \textbf{chain homotopy} 
	$h_* : f_*\simeq g_*$ is a chain map $h_n : C_n\rightarrow D_{n + 1}$:
	\[\begin{tikzcd}
	...\arrow[r, "d"] & C_{n + 1}\arrow[r, "d"]\arrow[d, "f_{n + 1}"]\arrow[d, swap, "g_{n + 1}"] & C_n\arrow[r, "d"]\arrow[d, "f_n"]
	\arrow[d, swap, "g_n"]\arrow[ld, "h_n"] & C_{n - 1}\arrow[r, "d"]\arrow[d, "f_{n - 1}"]\arrow[d, swap, "g_{n - 1}"]\arrow[ld, 
	"h_{n - 1}"] & ... \\
	...\arrow[r, "d"] & D_{n + 1}\arrow[r, "d"] & D_n\arrow[r, "d"] & D_{n - 1}\arrow[r, "d"] & ... \\
	\end{tikzcd}\]
	such that:
	\begin{equation}
		hd + dh = g - f
	\end{equation}
\end{definition}

Chain homotopy is an equivalence relation on the space of maps between topological spaces, which is easy to verify 
straight from the definition. The reason we are considering chain homotopy is because chain maps which are chain 
homotopic will induce the same maps on homology. 

\begin{theorem}
	If $f_0\simeq f_1 : C_*\rightarrow D_*$ are chain homotopic, then they induce the same maps on homology:
	\begin{equation}
		(f_0)_* = (f_1)_* : H_*(C_*)\rightarrow H_*(D_*)
	\end{equation}
\end{theorem}

\begin{proof}
	Recall that the induced map is the natural one, $f_*([x]) = [f(x)]$ for a map $f : C_*\rightarrow D_*$. So, we must 
	show that for $x\in Z_n(C_*)$, $f_0(x)$ and $f_1(x)$ are in the same coset when we mod $Z_n(D_*)$ out by $B_n(D_*)$ 
	to form the homology, i.e. $f_0(x) - f_1(x) = da$ with $a\in D_{n + 1}$. Let $x\in Z_n(C_*)$. Then:
	\begin{equation}
		f_0(x) - f_1(x) = (dh + hd)(x) = d(hx) + h(dx) = d(hx)\in B_n(D_*)
	\end{equation}
	because $dx = 0$ for a cycle $x$. Thus these induce the same maps on homology. 
\end{proof}

\subsection{Proof of the Homotopy Invariance of Homology}
Using the above theorem, we can reduce proving the homotopy invariance of homology to showing that homotopic maps $f, g : 
X\rightarrow Y$ induce chain homotopic maps $f_*, g_* : S_*(X)\rightarrow S_*(Y)$ on the $n$-chains, since these will by 
default induce the same maps on homology. We will need the following lemma:
\begin{lemma}
	Let $k : f\simeq g$ be a chain homotopy between $f, g : C_*\rightarrow D_*$ and let $j : D_*\rightarrow E_*$ 
	be a chain map. Then $j\circ k : j\circ f\simeq j\circ g$ is a chain homotopy between $j\circ f$ and $j\circ g$. 
\end{lemma}
\begin{proof}
	This is immediate once you draw out the diagrams. Because $j$ is a chain map, $dj = jd$. So, $(jh)d + d(jh) = 
	jhd + jdh = j(hd + dh) = j(f - g) = jf - jg$ because $hd + dh = f - g$ as $h$ is a chain homotopy. 
\end{proof}

Now, consider an arbitrary homotopy $h : X\times I\rightarrow Y$ from $f : X\rightarrow Y$ to $g : X\rightarrow Y$. Note 
that we have canonical injections $\iota_0 : X\times\{0\}\hookrightarrow X\times I$ and $\iota_1 : X\times \{1\}\hookrightarrow 
X\times I$ such that the diagram commutes:
\[\begin{tikzcd}
	X\times\{0\} \arrow[dr, "\iota_0"] \arrow[drr, "f", bend left=20]&  &  \\
	& X\times I\arrow[r, "h"] & Y \\
	X\times \{1\} \arrow[ur, "\iota_1"] \arrow[urr, "g", bend right=20] & & \\
\end{tikzcd}\]

We are trying to prove that we can find a chain homotopy $h_* : f_*\rightarrow g_* : H_*(X)\rightarrow H_*(Y)$ to prove the 
homotopy invariance. Since $H_*$ is a covariant functor, note that $f_* = h_*\circ\iota_0$ and $g_* = h_*\circ\iota_1$. 
Using the lemma, if we show that we can find a chain homotopy $k_* : (\iota_0)_*\simeq (\iota_1)_*$, then we can 
compose the chain homotopy with $h_*$ to get a chain homotopy $h_*\circ k_* : f_*\simeq g_*$. Thus, if we can 
find a map:
\begin{equation}
	k_n : S_n(X)\rightarrow S_{n + 1}(X\times I)\;\;\;\;\;\;\;\;\;\;\;\;\;\;\;\;\;\;\;\;\;\;\;\;\;\;\;\;\;\;\;\;\; dk + kd = (\iota_0)_* - (\iota_1)_*
\end{equation}
then we have proved the theorem. To construct such a map, view $X\times I\cong X\times\Delta^1$. We can view $k$ as  
a map $k : \textbf{Top}(\Delta^n, X)\rightarrow \textbf{Top}(\Delta^{n + 1}, X\times\Delta^1)$, and if we can find a way to 
construct a map:
\begin{equation}
	\gamma : \Delta^{n + 1}\rightarrow\Delta^n\times\Delta^1
\end{equation}
then given $\sigma\in\textbf{Top}(\Delta^n, X)$, we can construct a map $\tau\in\textbf{Top}(\Delta^{n + 1}, X)$, such that:
\[\begin{tikzcd}
	\Delta^{n + 1}\arrow[r, "\gamma"] \arrow[dr, "\tau", bend right] & \Delta^n\times\Delta^1 
	\arrow[d, shift right=2ex, "\sigma"] \arrow[d, shift 
	left = 2ex, hook] \\
	& X\times I \\
\end{tikzcd}\]

To do this, we will take for granted a theorem that will be proven later in these notes in Section~\ref{subsec:products}: the 
existence of a \textbf{cross product} on homology. 

\begin{theorem}
	There is a natural bilinear map $\times : S_p(X)\times S_q(Y)\rightarrow S_{p + q}(X \times Y)$, called the \textbf{cross 
	product}, such that the following conditions hold.
	\begin{enumerate}
		\item $\times$ is natural, so given maps $f : X\rightarrow X'$ and $g : Y\rightarrow Y'$, the following diagram 
		commutes:
		\[\begin{tikzcd}
			S_p(X)\times S_q(Y) \arrow[r, "\times"]\arrow[d, "{(f_*, g_*)}"] & S_{p + q}(X \times Y)\arrow[d, "(f \times g)_*"] \\
			S_p(X')\times S_q(Y')\arrow[r, "\times"] & S_{p + q}(X'\times Y') \\
		\end{tikzcd}\]
		where $f_*$ and $g_*$ are the maps induced on the $n$-chains and $(f\times g)_*$ is the map induced by 
		$f\times g : X\times Y\rightarrow X'\times Y', (x, y)\mapsto (f(x), g(y))$. 
		\item $\times$ is bilinear, i.e. $(a + a')\times b = a\times b + a'\times b$ and $a\times (b + b') = a\times b + a\times 
		b'$. 
		\item $d$ is an \textit{anti-derivation} with respect to $\times$, so
		\begin{equation}
			d(a\times b) = da\times b + (-1)^p a\times db
		\end{equation}
		\item $\times$ is normalized. For $x\in X$, $y\in Y$, and each $a\in S_p(X), b\in S_q(Y)$, given the inclusions 
		$j_x : Y\hookrightarrow X\times Y, y\mapsto (x, y)$ and $\iota_y : X\rightarrow X\times Y$, the induced maps 
		$(j_x)_* : S_*(Y)\rightarrow S_*(X\times Y)$ and $(\iota_y)_* : S_*(X)\rightarrow S_*(X\times Y)$ satisfy the 
		following equations:
		\begin{equation}
			(j_x)_*(b) = C_x^0\times b
		\end{equation}
		\begin{equation}
			(\iota_y)_* (a) = a\times C_y^0
		\end{equation}
	\end{enumerate}
\label{thm:cross_product}
\end{theorem}

Using the cross product, we can explicitly construct a chain homotopy $k_* : (\iota_0)_*\simeq (\iota_1)_*$. Pick 
a map $j : \Delta^1\rightarrow I$ such that $d_0 j = C_1^0$ and $d_1 j = C_0^0$, which is a very obtuse way of saying $j(0) = 
1$ and $j(1) = 0$, i.e. $j$ is an orientation reversing path\footnote{Recall that $C_p^n$ is the 
constant map $\Delta^n\rightarrow X$ with image $p\in X$.}. Then define our map $k_n : S_n(X)\rightarrow S_{n + 1}(X\times I)$ as follows:
\begin{equation}
	k_n\sigma := (-1)^n\sigma\times j
\end{equation}
Now we take the boundary of this map. Using the properties of $\times$, we have:
\[
	d(k_n\sigma) = (-1)^n d(\sigma\times j) = (-1)^n (d\sigma\times j + \sigma\times dj) = -(-1)^{n - 1} (d\sigma\times j) 
	+ \sigma\times (d_0j - d_1j)
\]
\[
	= -k_{n - 1}(d\sigma) + \sigma\times C_1^0 - \sigma\times C_0^0
\]
where we use that $k_{n - 1}(d\sigma) = (-1)^{n - 1} d\sigma\times j$. Now, using the normalization of the cross product 
for the induced maps $(\iota_0)_*, (\iota_1)_* : S_n(X)\rightarrow S_{n}(X\times I)$, we can write\footnote{These are exactly 
the maps $\iota_y : X\rightarrow X\times I$ in the theorem at $y = 0$ and $y = 1$.} $\sigma\times C_1^0 = (\iota_1)_*(\sigma)$ 
and $\sigma\times C_0^1 = (\iota_0)_*(\sigma)$. Thus we see that:
\begin{equation}
	dk\sigma = -kd\sigma + ((\iota_1)_* - (\iota_0)_*)\sigma
\end{equation}
and we see that $k_*$ is indeed a chain homotopy from $(\iota_0)_*$ to $(\iota_1)_*$, which completes the proof.

Note that there is still a black box in this proof: namely, proof of the existence of the cross product. We will prove this in a 
later chapter when we discuss it further in detail. The proof ends up requiring the fact that the product $\Delta^p\times\Delta^q$ 
has a trivial $p + q - 1$ homology group; this can either be shown explicitly, or one can prove a weaker version of the 
homotopy invariance of homology for star shaped regions, since $\Delta^p\times\Delta^q$ is convex and hence star shaped. 
However, this is not difficult to prove and can be shown by considering the straight line homotopy. 

\newpage
\section{Sequences and Relative Homology}
The tone of these notes is now going to pivot quite a bit and begin preparation for homological algebra, which was developed 
originally to aid in the study of homology and cohomology. We will be considering abstract sequences of objects and 
drawing lots of arrows, and our proofs will take a more diagrammatic tone than previously. We begin with the definition of 
a sequence.

\begin{definition}[Sequence]
	A \textbf{sequence} of abelian groups is a diagram of the form:
	\[\begin{tikzcd}
		...\arrow[r] & C_{n + 1}\arrow[r, "f_{n + 1}"] & C_n\arrow[r, "f_{n}"] & C_{n - 1}\arrow[r] & ... \\
	\end{tikzcd}\]
	such that $f_n\circ f_{n - 1} = 0$ for each $n$. A sequence is \textbf{long exact} if $im(f_n) = ker(f_{n - 1})$ for each $n$, 
	i.e. if its homology vanishes identically. 
\end{definition}
In other words, a sequence is a chain complex, and for any such chain complex $im(f_n)\subseteq ker(f_{n - 1})$, the 
central property that enables the definition and study of homology. If the sequence only has a finite number of nonzero 
elements, it is understood that we have just left out the 0 elements after the last nonzero one. 

\begin{definition}
	A \textbf{short exact sequence} (SES) is a long exact sequence of the form:
	\[\begin{tikzcd}
		0\arrow[r] & A\arrow[r, "\iota", hook] & B\arrow[r, "p", twoheadrightarrow] & C\arrow[r] & 0 \\
	\end{tikzcd}\]
\end{definition}
Note here we have drawn the arrows in a certain way to suggest specific properties. Namely, for the first and last arrows to 
be exact, then $A$ must embed into $B$ and $B$ must project onto $C$. Additionally, if this sequence is exact then 
$A\cong ker(p)$, because $\iota$ embeds $A$ as $A\cong im(\iota)\subseteq B$ and $im(\iota) = ker(p)$. Similarly, 
we must have $C\cong coker(\iota)$, because $coker(\iota) = B / im(\iota)\cong B / ker(p)\cong C$ as $p$ is surjective. 
Recall the cokernel of a map is defined as:
\begin{equation}
	coker(f : A\rightarrow B) := B / im(f)
\end{equation}

Intuition for the cokernel of a map is a bit more subtle than intuition for the kernel. This is because 
the cokernel is not strictly a group (or module, etc.) as we do not necessarily have $im(f)\trianglelefteq B$, and so we cannot 
consider its algebraic properties. However, an easy way to view the cokernel is by considering how much $f$ deviates from 
being a surjection. Just as how if the kernel of $f$ grows larger, it is less close to being injective, we have a similar 
property for the cokernel: as the cokernel grows, $f$ is less and less close to being a surjection. In particular, $f$ is a surjection 
iff $coker(f) = 0$, so this gives a nice analog to how a map is injective iff $ker(f) = 0$. 

\begin{definition}
	We define $\bf{Top_2}$ to be the category of pairs $(X, A)$ where $X$ is a topological space and $A$ is a subspace. The 
	morphisms $(X, A)\rightarrow (Y, B)$ are continuous maps $f : X\rightarrow Y$ such that $f(A)\subseteq B$. 
\end{definition}

Note that $\bf{Top}$ is a subcategory in $\bf{Top_2}$, which can be seen by either sending $X\mapsto (X, \emptyset)$ or 
$X\mapsto (X, X)$. 


\begin{lemma}
	Let $B_*$ be a chain complex with subchain complex $A_*\subseteq B_*$. Then there is a unique structure of a chain 
	complex on the graded abelian groups $C_* := B_* / A_*$ such that the canonical projection $B_*\rightarrow C_*$ 
	is a chain map. 
\end{lemma}

\begin{proof}
	The following diagram outlines the situation we are in, where the diagram is exact vertically and horizonatally:
	\[ \begin{tikzcd}
		\cdots \arrow[r] & A_{n + 1} \arrow[r, "d"] \arrow[d, hook] & A_n\arrow[r, "d"] \arrow[d, hook] & A_{n - 1}\arrow[r, "d"] 
		\arrow[d, hook] & \cdots \\
		\cdots \arrow[r] & B_{n + 1} \arrow[r, "d"] \arrow[d, "\pi", twoheadrightarrow] & B_n\arrow[r, "d"]\arrow[d, "\pi", 
		twoheadrightarrow] & B_{n - 1} \arrow[r, "d"] \arrow[d, "\pi", twoheadrightarrow] & \cdots \\
		\cdots \arrow[r] & C_{n + 1}\arrow[r, "d"] & C_n\arrow[r, "d"] & C_{n - 1}\arrow[r, "d"] & \cdots \\
	\end{tikzcd} \]
	We need to define the maps $d : C_n\rightarrow C_{n - 1}$ to make $(C_*, d)$ into a chain complex, where each 
	$C_n = B_n / A_n$. Define:
	\begin{equation}
		d([b]) := [db]
	\end{equation}
	To show this is well defined, suppose that $b = b' + a$ with $b, b'\in B_n$ and $a\in A_n$ (i.e. $[b] = [b']$). Then:
	\begin{equation}
		db = db' + da\implies [db] = [db']
	\end{equation}
	because $[da] = 0$ as $d$ is a chain map on the complex $A_*$, so $da\in A_{n - 1}$ which is killed by the quotient. 
	This is indeed a chain complex, because:
	\begin{equation}
		d^2[b] = d[db] = [d^2 b] = 0
	\end{equation}
	Finally, this definition makes $\pi : B_*\rightarrow C_*$ into a chain map by definition, because $\pi(b) = [b]$. Furthermore, 
	this makes it clear the definition is unique, so we are done.
\end{proof}

\begin{definition}[Relative singular chain]
	A \textbf{relative singular chain} is a functor $S_* : \bf{Top_2}\rightarrow\bf{Chain}$, where $\bf{Chain}$ is the 
	category of chain complexes, which maps:
	\begin{equation}
		S_*(X, A) := S_*(X) / S_*(A)
	\end{equation}
	which is well defined as $S_*(A)$ is a subcomplex of $S_*(X)$ for a subspace $A$ of $X$. 
\end{definition}

\begin{definition}[Relative Homology]
	The \textbf{relative homology} of $(X, A)\in\bf{Top_2}$ is:
	\begin{equation}
		H_*(X, A) := H_*(S_*(X, A))
	\end{equation}
\end{definition}

Note that the relative homology \textbf{is not} the quotient of the homology groups. However, it is related, and we will 
show that in certain situations the relative homology behaves just like a quotient. 

As an example, consider $X = \Delta^n$ and $A = \partial X = \partial\Delta^n$. Then the identity $\iota : 
\Delta^n\rightarrow\Delta^n$ is an element of $Sin_n(\Delta^n$, and the boundary satisfies $d\iota\in S_*(\partial\Delta^n) 
= S_*(A)$. So, $[\iota]$ is a relative cycle ($[\iota]\in Z_n(X, A)$) because $d[\iota] = [d\iota] = 0$ in $S_*(X, A) = S_*(X) / 
S_*(A)$, hence $[\iota]$ defines an element of the relative homology $H_n(\Delta^n, \partial\Delta^n)$, which is 
interesting because $\iota$ is not a representative of an element in the homology of $\Delta^n$. Eventually, we will see that 
$H_n(\Delta^n, \partial\Delta^n)\cong\mathbb Z$ with $[\iota]$ as a generator. 

Suppose we have a short exact sequence of chain complexes
\begin{equation}
	0\longrightarrow A_*\longrightarrow B_*\longrightarrow C_*\longrightarrow 0
\end{equation}
In particular, this is the case when $A_* = S_*(A)$, $B_* = S_*(X)$, and $C_* = S_*(X, A)$ for $(X, A)\in\bf{Top_2}$, so 
anything proved with a sequence of this form can be used for relative simplicial homology as well. When we apply 
the functor $H_*(\cdot)$, we unfortunately do not get a short exact sequence, although some parts of this SES stay 
exact when dropped to the level of homology. We make this precise in the following theorem, which will be very important 
as we continue our study of homology. 

\begin{theorem}
	Let $0\rightarrow A_*\xrightarrow{f_*} B_*\xrightarrow{g_*} C_*\rightarrow 0$ be a short exact sequence of chain 
	complexes. Then there is a natural homomorphism:
	\begin{equation}
		\partial : H_n(C_*)\rightarrow H_{n - 1}(A_*)
	\end{equation}
	such that the following sequence is long exact:
	\[\begin{tikzcd}
		\cdots\arrow[r] & H_{n + 1}(A_*)\arrow[r] & H_{n + 1}(B_*)\arrow[r] \arrow[d, phantom, ""{coordinate, name=Z}] & 
		H_{n + 1}(C_*) \arrow[dll, "\partial", rounded corners, 
		to path={ -- ([xshift=2ex]\tikztostart.east)
                |- (Z) [near end]\tikztonodes
                -| ([xshift=-2ex]\tikztotarget.west)
                -- (\tikztotarget)}] & \\
		& H_n(A_*)\arrow[r] & H_n(B_*)\arrow[r] \arrow[d, phantom, ""{coordinate, name=W}] & H_n(C_*) 
		\arrow[dll, "\partial", rounded corners, 
		to path={ -- ([xshift=2ex]\tikztostart.east)
                |- (W) [near end]\tikztonodes
                -| ([xshift=-2ex]\tikztotarget.west)
                -- (\tikztotarget)}] & \\
		 & H_{n - 1}(A_*)\arrow[r] & H_{n - 1}(B_*)\arrow[r] & H_{n - 1}(C_*)\arrow[r] & \cdots \\
	\end{tikzcd}\]
\end{theorem}

\begin{proof}
	We begin by showing exactness at the $\rightarrow H_n(B_*)\rightarrow$ joint. Let $\bar f_*, \bar g_*$ be the 
	maps induced on the homology by $f_*, g_*$, i.e. $\bar f([a]) := [f(a)]$. Then: 
	\begin{equation}
		\bar g_n\circ\bar f_n ([a]) = [g\circ f(a)] = [0] = 0
	\end{equation}
	so $im(\bar f_*)\subseteq ker(\bar g_*)$. The reverse inclusion is a bit more involved, and we will need to chase the 
	following diagram, which is the statement that the above SES is exact (in the future I will not draw this diagram 
	explicitly). 
	\[\begin{tikzcd}
		& 0\arrow[d] & 0\arrow[d] & 0\arrow[d] & \\
		\cdots\arrow[r, "d"] & A_{n + 1}\arrow[r, "d"]\arrow[d, "f_{n + 1}"] & A_n\arrow[r, "d"]\arrow[d, "f_{n}"] & A_{n - 1}
		\arrow[r, "d"]\arrow[d, "f_{n - 1}"] & \cdots \\
		\cdots\arrow[r, "d"] & B_{n + 1}\arrow[r, "d"]\arrow[d, "g_{n + 1}"] & B_n\arrow[r, "d"]\arrow[d, "g_n"] & B_{n - 1}
		\arrow[r, "d"]\arrow[d, "g_{n - 1}"] & \cdots \\
		\cdots\arrow[r, "d"] & C_{n + 1}\arrow[r, "d"]\arrow[d] & C_n\arrow[r, "d"]\arrow[d] & C_{n - 1}\arrow[r, "d"]\arrow[d] & 
		\cdots \\
		& 0 & 0 & 0 & \\ 
	\end{tikzcd} \]
	Pick $[b]\in ker(\bar g_n)$. Then $[g_nb] = 0$ in $H_n(C)$, so $g_nb = dc$ for $c\in C_{n + 1}$. Lift $c$ to $b'\in B_{n + 
	1}$, so $c = g_{n + 1}b'$. Then $g_nb = dc = dg_{n + 1} b' = g_n db'$, so $g_n(b - db') = 0$, which implies $b - db' = 
	f_n a$ for $a\in A_n$. Then it is clear that $[b] = [db'] + [f_n a] = \bar f_n([a])$, so $im(\bar f_*) = ker(\bar g_*)$. 
	
	We now turn to the connecting homomorphism $\partial$, which we define via a diagram chase. Let $[c]\in H_n(C_*)$, 
	so $c\in C_n$ and $dc = 0$. Lift $c$ to $b\in B_n$, so $g_{n - 1} db = d g_n b = dc = 0\implies db\in ker(g_{n - 1}) = 
	im(f_{n - 1})$, so we can lift $db$ to $a\in A_{n - 1}$ so $f_{n - 1}a = db$. We use this element $a$ to define 
	$\partial[c]$ as $\partial [c] := [a]$. 
	
	Now, we show this is well defined. Suppose $[c] = [c']$, so $c = c' + dc''$ with $c''\in C_{n + 1}$. Let $b'\in B_n$ be 
	a lift of $c'$ and $b''\in B_{n + 1}$ be a lift of $c''$, so $c' = g_n b'$, $c'' = g_{n + 1}b''$, and $dc'' = g_n(b' - b)$. 
	Applying $dg_{n + 1}$ to $b''$, we see $dg_{n + 1}b'' = dc'' = g_n(b' - b)$, but because $g_*$ is a chain map, we also 
	have $g_n db'' = dg_{n + 1}b''$, hence $g_n(b' - b - db'') = 0$ and we lift $b' - b - db''$ to $a''\in A_n$. 
	Note that $db = f_{n - 1}(a)$ and $db' = f_{n - 1}(a')$, where $a'$ is defined for $b'$ as above. But then $df_{n}(a'') = 
	d(b' - b - db'') = db' - db$, and also $df_n(a'') = f_{n - 1} da''$, and by the definition of $a$ and $a'$ we have 
	$db' - db = f_{n - 1}(a - a') = f_{n - 1}(da'')\implies f_{n - 1}(a - a' - da'') = 0$. But $f_{n - 1}$ is injective, so this implies 
	$a = a' + da''$, so $[a] = [a']$ and $\partial$ is thus well defined. 
\end{proof}




\newpage
\label{subsec:products}
\section{Products on Homology}

\subsection{The cross product}

The cross product is a map $\times : S_p(X)\times S_q(Y)\rightarrow S_{p + q}(X\times Y)$, which we will show extends to 
homology and has many nice properties. The basic intuition is to separate our simplices into factors and then map them into a 
larger simplex in the product space in a natural way. We will prove the existence by induction. Recall 
Theorem~\ref{thm:cross_product}, which we restate here for convenience:

\begin{theorem}
	There is a natural bilinear map $\times : S_p(X)\times S_q(Y)\rightarrow S_{p + q}(X \times Y)$, called the \textbf{cross 
	product}, such that the following conditions hold.
	\begin{enumerate}
		\item $\times$ is natural, so given maps $f : X\rightarrow X'$ and $g : Y\rightarrow Y'$, the following diagram 
		commutes:
		\[\begin{tikzcd}
			S_p(X)\times S_q(Y) \arrow[r, "\times"]\arrow[d, "{(f_*, g_*)}"] & S_{p + q}(X \times Y)\arrow[d, "(f \times g)_*"] \\
			S_p(X')\times S_q(Y')\arrow[r, "\times"] & S_{p + q}(X'\times Y') \\
		\end{tikzcd}\]
		where $f_*$ and $g_*$ are the maps induced on the $n$-chains and $(f\times g)_*$ is the map induced by 
		$f\times g : X\times Y\rightarrow X'\times Y', (x, y)\mapsto (f(x), g(y))$. 
		\item $\times$ is bilinear, i.e. $(a + a')\times b = a\times b + a'\times b$ and $a\times (b + b') = a\times b + a\times 
		b'$. 
		\item $d$ is an \textit{anti-derivation} with respect to $\times$, so
		\[
			d(a\times b) = da\times b + (-1)^p a\times db
		\]
		\item $\times$ is normalized. For $x\in X$, $y\in Y$, and each $a\in S_p(X), b\in S_q(Y)$, given the inclusions 
		$j_x : Y\hookrightarrow X\times Y, y\mapsto (x, y)$ and $\iota_y : X\rightarrow X\times Y$, the induced maps 
		$(j_x)_* : S_*(Y)\rightarrow S_*(X\times Y)$ and $(\iota_x)_* : S_*(X)\rightarrow S_*(X\times Y)$ satisfy:
		\[
			(j_x)_*(b) = C_x^0\times b
		\]
	\end{enumerate}
\end{theorem}

\begin{theorem}
	There is a natural bilinear normalized map:
	\begin{equation}
		\times : H_q(X)\times H_q(X)\rightarrow H_{p + q}(X\times Y)
	\end{equation}
\end{theorem}

\subsection{Kunneth Formulas}

\end{document}