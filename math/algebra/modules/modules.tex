\documentclass[11pt, oneside]{amsart}   	% use "amsart" instead of "article" for AMSLaTeX format
\usepackage[margin = 1in]{geometry}                		% See geometry.pdf to learn the layout options. There are lots.
\geometry{letterpaper}                   		% ... or a4paper or a5paper or ... 
%\geometry{landscape}                		% Activate for rotated page geometry
%\usepackage[parfill]{parskip}    		% Activate to begin paragraphs with an empty line rather than an indent
\usepackage{graphicx}				% Use pdf, png, jpg, or eps§ with pdflatex; use eps in DVI mode
								% TeX will automatically convert eps --> pdf in pdflatex		
\usepackage{amssymb}
\usepackage{amsthm}
\usepackage{mathtools}
\usepackage{float}
\usepackage{tikz-cd}

\theoremstyle{definition}
\newtheorem{definition}{Definition}[section]
\newtheorem{theorem}{Theorem}[section]
\newtheorem{corollary}{Corollary}[theorem]
\newtheorem{lemma}[theorem]{Lemma}

\title{Modules}
\author{Patrick Oare}
%\date{}							% Activate to display a given date or no date

\begin{document}
\maketitle

\section{Introduction}

At their most basic level, modules are essentially dumb vector spaces. They have the same structure as vector spaces, but 
instead of being equipped with a multiplicative action from a vector space, they are equipped with an action from a ring. 
Although this seems like a small thing to change, module theory is a \textit{much} more difficult subject to work with than vector 
space theory. We will see that free modules have a similar structure to vector spaces because they look like products of the 
base ring $R$. However, most modules are not free. 

These notes will cover the basic definitions of module theory and then study the basics of homological algebra. We will define 
the tensor product $\otimes$ of two modules, then study certain derived functors of interest. Many of the tools that we will 
consider in the latter half of these notes are fundamental for the study of algebraic topology, especially when working with 
homology groups and cohomology algebras. The study of simplicial homology is typically the first place that one encounters 
using many of these ideas outside of a purely abstract sense, and these notes should help set a foundation for that subject. 

\section{Basic Definitions and $Hom$}

For these notes, we will work over a ring $R$. It may or may not be commutative or have identity, and when we need these 
properties we will specify them. We begin with the definition of a module. 

\begin{definition}[$R$-module]
	A \textbf{left $R$-module} is a triple $(M, +, \cdot)$ consisting of a set $M$, a binary operation $(M, +)$, and a map 
	$\cdot : R\times M\rightarrow M$ such that:
	\begin{enumerate}
			\item $(M, +)$ is an Abelian group.
			\item $(r_1r_2)m = r_1(r_2 m)$.
			\item $r(m_1 + m_2) = rm_1 + rm_2$.
			\item$ (r_1 + r_2)m = r_1m + r_2m$.
			\item $1m = m$ if $R$ has unity.
	\end{enumerate}
	In other words, we have a left action of $(R, \times)$ on $M$ compatible with its group structure. A 
	\textbf{right $R$-module} is defined in an analogous manner with a right $R$-action. 
\end{definition}

We will typically drop the modifiers ``left" and ``right" unless they are important to the theorem we are trying to prove. Similarly, 
unless it is important we will drop the modifier ``$R$" and simply refer to these objects as modules. For a fixed ring $R$, 
$R$-modules form a category \textbf{RMod} with the following morphisms:
\begin{definition}
	Given two modules $M, N$, a \textbf{$R$-module homomorphism} is a map $f : M\rightarrow N$ such that $f(m + n) = 
	f(m) + f(n)$ and $f(rm) = rf(m)$ for $r\in R$ and $m, n\in M$. 
\end{definition}
For left modules, a more suggestive notation is to write $mf := f(m)$. With this manner, then the linearity of scalar multiplication 
simply means that writing $rmf$ is ambiguous, because $r(mf) = (rm)f$. 

Given two modules $M, N$, we denote by $Hom_R(M, N)$ the set of all $R$-module homomorphisms between them (and will 
often suppress the $R$). For a general ring, $Hom_R(M, N)$ is an abelian group. For $\phi, \psi\in Hom(M, N)$, define:
\begin{equation}
	(\phi + \psi)(m) := \phi(m) + \psi(m)
\end{equation}
If $R$ is commutative, then $Hom_R(M, N)$ additionally gets the structure of a $R$-module by defining the multiplication:
\begin{equation}
	(rf)(m) := r(fm)
\end{equation}

$Hom$ also respects maps between modules. Fix a module $L$ and consider a map $\phi : M\rightarrow N$. If we consider the 
action of $Hom(\cdot, L)$ on these modules, we get an induced map:
\begin{align}
	\phi^* : Hom(N, L) &\rightarrow Hom(M, L) \\
	f & \mapsto f\circ\phi
\end{align}
and considering the action of $Hom(L, \cdot)$:
\begin{align}
	\phi_* : Hom(L, M) &\rightarrow Hom(L, N) \\
	g &\mapsto \phi\circ g
\end{align}
Notice that $phi^*$ switches the arrow directions and $\phi_*$ preserves the arrow directions. This implies that $Hom(\cdot, L)$ is a 
\textbf{contravariant functor}, and $Hom(L, \cdot)$ is a covariant functor. In other words:
\begin{theorem}
	$Hom_R(\cdot, \cdot)$ is a \textbf{bifunctor} from \textbf{RMod} to \textbf{Ab}. If $R$ is a commutative ring, then it is in fact a 
	bifunctor from \textbf{RMod} to \textbf{RMod}. 
\end{theorem}

This is a nice functor, but unfortunately it is not precisely exact, i.e. it does not preserve exact sequences. However, it is close, as we will 
see in the following theorem.
\begin{theorem}
	$Hom_R(\cdot, \cdot)$ is a \textbf{left-exact} functor, i.e. given an exact sequence of $R$-modules:
	\begin{equation}\begin{tikzcd}
		0\arrow[r] & A\arrow[r, "\phi"] & B\arrow[r, "\psi"] & C\arrow[r] & 0
	\end{tikzcd}\end{equation}
	the following sequences are exact:
	\begin{equation}\begin{tikzcd}
		0\arrow[r] & Hom_R(M, A)\arrow[r, "\phi_*"] & Hom_R(M, B)\arrow[r, "\psi_*"] & Hom_R(M, C)
	\end{tikzcd}\end{equation}
	\begin{equation}\begin{tikzcd}
		Hom_R(A, M) & Hom_R(B, M)\arrow[l, "\psi^*"] & Hom_R(C, M)\arrow[l, "\phi^*"] & 0\arrow[l]
	\end{tikzcd}\end{equation}	
\end{theorem}

\begin{proof}

\end{proof}

\section{The Tensor Product}

\subsection{Graded algebras}

In this section we consider naturally graded algebras that arise in the studies of modules; the tensor algebra and the 
exterior algebra. The tensor algebra is the ``mother of all algebras", and is very important in all areas of math. The exterior 
algebra $\Lambda^*M$ over $R$ is derived from the tensor algebra, and shows up most commonly in the 
definition of a differential form. The algebra operation over $\Lambda^*M$ is the wedge product, which we will define abstractly 
and show how to use it on computations. 

\begin{definition}[Graded module]
	An algebra $(M, +, \cdot, \times)$ ($\times : M\times M\rightarrow M$ is the ring multiplication) is called \textbf{graded} if 
	$M$ can be decomposed as a direct sum of submodules:
	\begin{equation}
		M = \bigoplus_{k = 0}^\infty M_k
	\end{equation}
	and the algebra product acts on submodules as $\times : M_k\times M_\ell\rightarrow M_{k + \ell}$. 
\end{definition}
The first example of a graded module we will consider is the tensor algebra. Another example of a graded algebraic object 
that you may have studied before is the homology group and the cohomology algebra, which are naturally graded as 
they are sums of 

\begin{definition}[Tensor algebra]
	Let $M$ be a $R$-module space. We define $\bigotimes_0 V := R$, and for $k\geq 1$:
	\begin{equation}
		\bigotimes_k M := M\otimes\left(\bigotimes_{k - 1} M\right)
	\end{equation}
	i.e. $\bigotimes_k M$ equals $M\otimes M\otimes ...\otimes M$ with $k$ copies of $M$. The \textbf{tensor algebra} of 
	$M$ is the module:
	\begin{equation}
		\bigotimes M := \bigoplus_{k = 0}^\infty \left(\bigotimes_k M\right)
	\end{equation}
	equipped with the ring multiplication $\otimes: (\bigotimes_k M)\times (\bigotimes_\ell M)\rightarrow 
	\bigotimes_{k + \ell} M$ (which means that $\otimes$ is a valid multiplication on $\bigotimes M$ and makes it into an 
	algebra). 
\end{definition}
The tensor algebra is naturally graded as it is a direct sum of $\bigotimes_k M$. We will use it to define another graded 
algebra, called the \textit{exterior algebra} of $M$. This will be the space of interest for differential forms. 

\begin{definition}[Exterior algebra]
	The \textbf{$k$th exterior power} of a module $M$ is the quotient algebra:
	\begin{equation}
		\Lambda^k M := \bigotimes_k M / \langle\{x_1\otimes ... \otimes x_k : x_i = x_j, i\neq j\}\rangle
	\end{equation}
	where $\langle ...\rangle$ denotes the hull of such a subset. The \textbf{exterior algebra} of $M$ is:
	\begin{equation}
		\Lambda^* M :=\bigoplus_{k = 0}^\infty\Lambda^kM
	\end{equation}
	equipped with the natural ring multiplication $\wedge : \Lambda^k M\times\Lambda^\ell M\rightarrow\Lambda^{k + \ell}
	M$ defined by:
	\begin{equation}
		[x]\wedge [y] := [x\otimes y]
	\end{equation}
	i.e. $\wedge$ is the image of $\otimes$ in $\Lambda^* M$. 
\end{definition}
The exterior algebra is a naturally graded algebra with exterior multiplication. It is the ``best" algebra over $M$ which is 
antisymmetric, in the sense that $x\wedge y = - y \wedge x$, and more generally for $\sigma\in S_n$:
\begin{equation}
	x_1\wedge ...\wedge x_n = sgn(\sigma) x_{\sigma 1}(\wedge ...\wedge x_{\sigma n})
\end{equation}
By ``best", we mean that it satisfies a universal property outlined in the following theorem. 
\begin{theorem}
	Suppose that $\phi : M^k\rightarrow N$ is an alternating multilinear map. Denote by $\iota : M^k\rightarrow 
	\Lambda^k M$ the canonical embedding $(m_1, ..., m_k)\mapsto m_1\wedge ...\wedge m_k$. Then $\phi$ 
	factors through a linear map of $R$-modules $\tilde\phi : \Lambda^k M\rightarrow N$, i.e. the diagram commutes:
	\begin{equation}\begin{tikzcd}
		M^k\arrow[r, "\iota"]\arrow[dr, "\phi"] & \Lambda^k M, \arrow[d, "\tilde\phi"] \\
		& N
	\end{tikzcd}\end{equation}
\end{theorem}

\end{document}