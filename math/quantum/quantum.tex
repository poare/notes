\def \root {../../}			% path to root (/notes)
\input{\root/template/preamble.tex}

\title{Mathematical Quantum Mechanics}
\author{Patrick Oare}
\date{}							% Activate to display a given date or no date

\begin{document}
\maketitle

 \begin{resources}
The main resource for these notes is Brian Hall's great book \textit{Quantum Theory for Mathematicians}. In addition, there are other good resources online that are helpful.
\begin{itemize}
	\item \href{https://terrytao.wordpress.com/2012/10/07/some-notes-on-weyl-quantisation/}{Terence Tao's notes on Weyl Quantization}.
\end{itemize}
\end{resources}

\section{Preliminaries}

Studying quantum mechanics (QM) rigorously requires a lot of prerequisites. In physics, we tend to think of the math behind QM as linear algebra. Once you learn linear algebra, you're typically equipped with most of the mathematical knowledge you'll need to understand a QM course. However, when we study QM rigorously, there's more that we need than just linear algebra: specifically, we need to use \textbf{functional analysis}. Functional analysis is the study of function spaces: these function spaces are vector spaces (here's where the linear algebra enters) that are equipped with specific topologies and metrics. The essential reason that function spaces make linear algebra complicated is because \textit{they are uncountably-infinite dimensional vector spaces}. We tend to sweep this all under the rug in a QM course, and pretend that uncountably-infinite vector spaces are no different than finite vector spaces: unfortunately, there are a lot of differences that make function spaces much more difficult to deal with (albeit more rewarding!) than simple finite-dimensional vector spaces. That's what these notes will explore. 

\section{Classical Mechanics}

\section{Quantization}

\section{Weyl quantization and phase-space QM}

\end{document}