\def \root {../../}			% path to root (/notes)
\input{\root/template/preamble.tex}

\title{Point-Set Topology}
\author{Patrick Oare}
\date{}							% Activate to display a given date or no date

\begin{document}
\maketitle

\section{Axioms and Definitions}

	In topology, we take as given our open sets which satisfy certain axioms; namely, we want to be able to combine these open sets as we did in metric 
	spaces. However, we will make our definition more general to include a broader range of spaces which will give us more flexibility and more useful 
	tools that we can use.
	
	\begin{definition}[Topological Space] A \textbf{topological space} is a pair $(X, \tau)$ consisting of a set $X$ and a collection $\tau$ of subsets of $X$, where 
	$\tau$ satisfies the following axioms.
		\begin{enumerate}
			\item $X$ and $\emptyset$ are in $\tau$.
			\item $\tau$ is closed under arbitrary unions, i.e. if $\{U_\alpha\}\subset\tau$, then:
			\eq
				\bigcup_\alpha U_\alpha\in\tau.
			\qe
			\item $\tau$ is closed under finite intersections, i.e. if $\{U_i\}_{i = 1}^n\subset\tau$, then:
			\eq
				\bigcap_{i = 1}^n U_i\in\tau.
			\qe
		\end{enumerate}
	We call the set $\tau$ a \textbf{topology} on the set $X$ and we call the members of $\tau$ \textbf{open sets}. 
	\end{definition}
	
	For the rest of these notes, unless otherwise specified, $(X, \tau)$ will be a topological space. Also notice that in this definition, we basically just took the 
	things that we liked about how to combine open sets from metric spaces and turned them into axioms. Because of this, every metric space admits a 
	canonical topology; called the \textbf{metric topology}, the open sets are just the ones that are open via the metric. 
	
	Given any set $S$, there are two topologies that are always admitted on $S$. The first is called the \textbf{discrete topology}, in which every subset of $S$ 
	is open. The second is called the \textbf{trivial topology}, in which the open sets are $\{\emptyset, S\}$. Neither of these are very interesting, though they 
	sometimes turn up as malicious counterexamples.
	
	Note that there is a standard way to turn any subset of a topological space into a topological space in its own right.
	
	\begin{definition}[Subspace Topology]
		Let $A\subseteq X$. Then the \textbf{subspace topology} on $A$ is given by $U\subseteq A$ open iff $U = A\cap V$ for some open set $V\subseteq X
		$. We refer to $A$ as a \textbf{subspace} of $A$ if it is endowed with the subspace topology.
	\end{definition}
	
	\subsection{Combining Topologies}
	
	We also have the notion of how ``large" a topology is. Intuitively, one topology is finer than another if it has more open sets; essentially, we can break the 
	space up into open sets, and a larger topology will break the space up into smaller pieces, while a coarser topology will break the space up into larger 
	pieces.
	
	\begin{definition}
		Let $\tau_1$ and $\tau_2$ be two topologies on $X$. We say that $\tau_1$ is \textbf{larger (finer, stronger)} than $\tau_2$ if $\tau_2\subset\tau_1$. We say 
		that $\tau_1$ is \textbf{smaller (coarser, weaker)} than $\tau_2$ if $\tau_1\subset\tau_2$.
	\end{definition}
	
	Generally, we will work with the elements of a topological space in topology. However, it is useful to know how we can combine open sets from different 
	topologies. Much as in how a space can have different metrics, a space can also have different topologies, and the properties it has in one topology need 
	not be the same properties it has in another topology.
	
	\begin{prop}
		Let $\mathcal C$ be a collection of topologies on $X$. Then:
		\eq
			\bigcap_{\tau\in\mathcal C}\tau.
		\qe
		is a topology on $X$.
	\end{prop}
	
	\begin{prop}
		Let $S$ be any nonempty collection of subsets of $X$. Then there is a smallest topology on $X$ that contains $S$, namely:
		\eq
			\tau := \bigcap \left\{\tau : \tau\textnormal{ a topology with } S\subseteq\tau \right\}.
		\qe
	\end{prop}
	
	\subsection{Generalization of Metric Space Properties}
	
	As we dealt with closed sets in metric spaces, we can equally well define closed sets in topology. We will now generalize many of the notions we saw in metric 
	spaces to arbitrary topological spaces.
	
	\begin{definition}[Closed Set]
		A subset $E\subset X$ is a \textbf{closed set} if $E^c$ is open.
	\end{definition}
	
	We now define the important notion of a neighborhood of a point. A neighborhood will allow us to formally ask the question, ``what points in $X$ are close to $x$?" using the ideas we've developed of open sets. 
	\newline
	\begin{definition}[Neighborhood]
		Let $x\in X$. A \textbf{neighborhood} of $x$ is an open set $U$ that contains $x$.
	\end{definition}
	
	Neighborhoods give us a practical way to show that a set is open-- suppose that we are trying to determine if $V\subset X$ is open. Then it suffices to show 
	that given $x\in V$, we can find a neighborhood $U_x$ of $x$ contained in $V$, for if we have found such a collection of neighborhoods $\{U_x\}_{x\in V}$, 
	then:
	\eq
		\bigcup_{x\in V}U_x = V,
	\qe
	for the forward containment follows because each $U_x\subseteq V$, and the reverse because $V\subseteq\{x : x\in V\}$. Because open sets are closed 
	under arbitrary unions, if we can find such a neighborhood for every $x\in V$ it guarantees that $V$ is open, because we will have written it as a union 
	of open sets.

	We can define the notion of interior and of a limit point equally well in a topological space as in a metric space. A limit point of a set is a point that is ``close" to 
	the set in a topological sense, i.e. that every neighborhood of a limit point intersects the set in some point. The interior is just as in metric spaces, where we 
	can find an open neighborhood around any point in the interior contained that is contained in the set.
	
	\begin{definition}[Interior]
		Let $A\subset X$. Then we define the \textbf{interior} of $A$, denoted $A^\circ$, to be the union of all the open sets contained in $A$.
	\end{definition}
	
	\begin{definition}[Limit Point]
		Let $A\subseteq X$, $p\in X$. Then $p$ is a \textbf{limit point} of $A$ if for each neighborhood $p\in U$, we have:
		\eq
			U\cap(A\setminus\{p\})\neq\emptyset.
		\qe
	\end{definition}
	
	Subtracting $p$ out of $A$ guarantees that the points of $A$ have to be sufficiently close to $p$, and that it is not enough to just have $p\in A$ (consider 
	the natural numbers $\mathbb N$. These do not have a limit point in the standard topology of $\mathbb R$.). To show $p$ is not a limit point, we can 
	simply show it has a disjoint neighborhood from $A\setminus\{p\}$. Just as limit points generalize, convergent sequences generalize as well.
	
	\begin{definition}[Convergence]
		Let $X$ be a topological space, and let $(x_n)$ be a sequence in $X$. Let $x\in X$. Then we say that $(x_n)$ \textbf{converges} to $x$, denoted $x_n
		\rightarrow x$, if for each open neighborhood $U\ni x, \exists N\in\mathbb N$ such that $x_n\in U, \forall n > N$.
	\end{definition}
	
	For the most part, convergence plays a much smaller role in topology than it does in the theory of metric spaces. To relate this definition to its metric space 
	counterpart, suppose $x_n\rightarrow x$. Then for any open set $U\ni x$ (i.e. $\epsilon > 0$ in metric spaces), the sequence lies in $U$ \textbf{eventually}. 
	Open sets may look very strange in this space, but the basic notion is the same: for any tolerance (open set around $x$), the sequence eventually gets within 
	this tolerance and stays within this tolerance. Now, we may relate closed sets and limit points.
	
	\begin{definition}[Closure, Dense]
		Let $A\subseteq X$. We define the \textbf{closure} of $A$ to be the set:
		\eq
			\overline A := A\cup\{p\in X : p \textnormal{ is a limit point of $A$}\}.
		\qe
		If $\overline A = X$, we say that $A$ is \textbf{dense} in $X$.
	\end{definition}
	
	\begin{theorem}
		$A\subseteq X$ is closed iff $A$ contains all its limit points, i.e. $A = \overline A$.
	\end{theorem}
	
	\begin{proof}
		Suppose $A$ is closed and $p\notin A$ is a limit point of $A$. Then $A^c$ is an open neighborhood of $p$ and is disjoint from $A\setminus\{p\}$, a contradiction to $p$ being a limit point of $A$. Conversely, suppose $A$ contains all its limit points. Then each $p\in A^c$ is not a limit point, and so we may find a disjoint open neighborhood to $A$ of $p$, $U_p$. But then $A^c = \bigcup_{p\in A^c} U_p$, and hence is open.
	\end{proof}
	
	In a sense, the closure of $A$ is the ``best" closed set that contains $A$, in that if any other closed set $B$ contains $A$, then it must contain $\overline A$ as well.
	
	\begin{theorem}
		The closure $\overline A$ is the smallest closed set containing $A$.
	\end{theorem}
	
	\begin{proof}
		We must show that $\overline A$ is closed, and that given any $A\subseteq B\subseteq X$ with $B$ closed, $\overline A\subseteq B$. We prove the latter first; if this is the case, then limit points of $A$ are also limit points of $B$, and $B$ must contain all its limit points, so $\overline A\subseteq B$. We show $\overline A$ is closed, i.e. that $\overline{\overline A} = \overline A$, or that $\overline A^c$ is open. Let $x\in \overline A^c$, so $x$ is not a limit point of $A$ and $\exists U_x$ an open neighborhood of $x$ such that $U_x\cap (A\setminus \{x\}) = U_x\cap A = \emptyset$. Now, suppose $y\in\overline A$ is in $U_x$. This is a contradiction because $U_x\cap(A\setminus\{y\}) = \emptyset$, so $y\notin U_x\implies U_x\cap \overline A = \emptyset$, so $\overline A^c$ is open. 
	\end{proof}
	
	\begin{corollary}
		$A$ is closed iff $A = \overline A$. 
	\end{corollary}
	
	\subsection{Bases}
	
	It is often quite difficult to specify an explicit list of open sets in a topological space. Instead, we specify a subset of open sets that can ``create" every other open 
	set, called a base.
	
	\begin{definition}[Base]
		A \textbf{base} for a topological space $(X, \tau)$ is a collection of open sets $B$ such that any set $A\in\tau$ is a union of open sets in $B$.
	\end{definition}
	
	For example, a base for the standard topology on $\mathbb R^n$ is the collection of open balls with radius $\epsilon > 0$ centered at an arbitrary $x\in\mathbb R^n$. Note that \textbf{bases are not unique} whatsoever; we can always add an extra open set to a base to make a new base. Now, let's see how we can generate a base from a family of subsets of $X$. 
	
	\begin{prop}
		Let $X$ be a set, $B$ a collection of subsets of $X$. If $B$ has the property that for $U, V\in B$, $U\cap V$ is a union of elements in $B$, then the collection of unions of elements of $B$ is a topology with a base given by $B$. 
	\end{prop}
	
	\begin{prop}
		Let $X$ be a set, $S$ any collection of subsets of $X$ such that $\bigcap\{V\in S\} = X$. Then the set of finite intersections of elements of $S$ is a base for a topology (the topology in the previous proposition). In this case, we say that $S$ is a \textbf{subbase} for the topology.
	\end{prop}
	
	\begin{definition}[Subbase]
		Let $\tau$ be a topology on $X$, and $S\subseteq\tau$ such that $X = \bigcup S$. If the topology for which $S$ is a subbase is $\tau$, we say that 
		$S$ is a \textbf{subbase} for $\tau$. 
	\end{definition}
	
	All a subbase means is that we generate a base by intersecting subbase elements. Condition that $\bigcup S = X$ is just so that $X$ is in our topology. 
	So, a subbase allows us to generate a base by taking finite intersections of subbase elements, and a base in turn allows us to generate a topology by taking 
	arbitrary unions of base elements. We will see later that in some cases, it is more natural to specify a subbase. Note that \textbf{a base is note necessarily 
	a subbase}. Indeed, if we take a subbase on a metric space to be the collection of open balls, then the base it generates is not the collection of open balls 
	on the metric space, because the intersection of two balls is not guaranteed to be a ball.

\section{Continuity}

	As we did with the axioms for a topology, we motivate the topological definition of continuity with the definition from continuity we use from metric spaces. 
	The general notion of continuity from a metric space that involves the metric $d$ cannot be easily generalized to topological spaces, but we can use the 
	theorem relating pullbacks of open sets in metric spaces. That is:
	
	\begin{definition}[Continuity]
		A map $f : X\rightarrow Y$ between topological spaces is \textbf{continuous} if for every open set $V$ in $Y$, 
		\eq
			f^{-1}(V)\subseteq X
		\qe
		is open. 
	\end{definition}
	
	We also define special types of maps, called open maps and closed maps, that will help us a bit when we get to homeomorphisms. Note that these definitions are different than the ones we have for continuity in that they deal with \textit{pushing forward} sets, while continuity describes how a map \textit{pulls back} sets. 
	
	\begin{definition}[Open and Closed]
		Let $f : X\rightarrow Y$ be a map. We say $f$ is \textbf{open} if $f$ maps open sets into open sets; we say $f$ is \textbf{closed} if $f$ maps closed sets 
		to closed sets.
	\end{definition}
	
	Even though a map is open / closed, it does not imply that the map is closed / open. Note that any function out of a discrete topological space is continuous, 
	simply because every set is open in the discrete topology. Continuity respects the subspace topology nicely. Here is a slew of easy theorems to prove about 
	continuity:
	
	\begin{theorem}
		If $A\subseteq X$ is a subspace, then the inclusion map $\iota : A\rightarrow X$ is continuous. Furthermore, if $f : X\rightarrow Y$ is any continuous 
		map, then $f|_A : A\rightarrow Y$ is continuous, where $f|_A = f\circ\iota$.
	\end{theorem}
	
	The first part follows essentially from the definition of a subspace, and the second is an immediate corollary of the next theorem, which shows that we can 
	compose continuous maps:
	
	\begin{theorem}
		If $f : X\rightarrow Y$ and $g : Y\rightarrow Z$ are continuous functions, then:
		\eq
			g\circ f : X\rightarrow Z
		\qe
		is also continuous.
	\end{theorem}
	
	\begin{theorem}
		Let $f : X\rightarrow Y$ be a map. The following are equivalent:
		\begin{enumerate}
			\item $f$ is continuous. 
			\item $f^{-1}(A)$ is closed whenever $A\subseteq Y$ is closed. 
			\item If $\{U_\alpha\}$ is a base (or subbase) for the topology on $Y$, then $f^{-1}(U_\alpha)$ is open for all $\alpha$.
		\end{enumerate}
	\end{theorem}
	
	\begin{proof}
		These follow immediately due to a couple of set theory identities:
		\begin{align}
			f^{-1}(A\cup B) = f^{-1}(A)\cup f^{-1}(B) &&
			f^{-1}(A^c) = (f^{-1}(A))^c.
		\end{align}
	\end{proof}
	
	The last identity is especially useful; since we typically don't know what every open set in a topology looks like, it is much easier to prove that $f$ pulls back 
	base elements into open sets than knowing that $f$ pulls back all open sets into open sets. Now, we turn to the idea of topological equivalence. In 
	algebra, we say that two structures are the same up to isomorphism if they share the same algebraic structure. There is a similar notion in topology; 
	we use the homeomorphism to describe a nice, topology-preserving map between spaces.
	
	\begin{definition}[Homeomorphism]
		A function $f : X\rightarrow Y$ is a \textbf{homeomorphism} if $f$ is a continuous bijection with a continuous inverse. We call $X$ and $Y$ \textbf{
		homeomorphic spaces} if there is a homeomorphism between them, and write $X\cong Y$.
	\end{definition}
	
	Homeomorphisms are isomorphisms in the category of topological spaces. Any homeomorphism $f$ induces a bijection between open sets of $X$ and open 
	sets of $Y$, because since $f^{-1}$ is continuous, $f$ pushes open sets of $X$ into open sets of $Y$. So, if $X\cong Y$, their topologies are in bijection, and it 
	makes sense to identify them as topologically the same.
	
	As an example, consider the stereographic projection $f : S^n\setminus\{(0, 0, ..., 1)\}\rightarrow \mathbb R^n$. This is a homeomorphism from $S^n$ 
	minus one point to $\mathbb R^n$, and we will use this quite a bit in this course.
	
	
\section{Constructions}

	\subsection{Abstract Constructions}
	
	By an abstract construction, I mean a construction which specifies another space as a universal object. Given a map to or from a topological space, we will 
	consider natural topologies that we can put on the domain or the codomain. 
	
	\begin{definition}[Initial Topology]
		Let Let $\{(X_\alpha, \tau_\alpha)\}_\alpha$ be a collection of topological spaces, $Y$ a set, and $f_\alpha : 
		Y\rightarrow X_\alpha$ a map for each $\alpha$.Then the weakest topology on $Y$ making each $f_\alpha$ 
		continuous is called the \textbf{initial topology}.
	\end{definition}
	
	\begin{definition}[Final Topology]
		Let $\{(X_\alpha, \tau_\alpha)\}_\alpha$ be a collection of topological spaces, $Y$ a set, and $f_\alpha : 
		X_\alpha\rightarrow Y$ a map for each $\alpha$. Then the strongest topology making each $f_\alpha$ continuous is 
		called the \textbf{final topology}
	\end{definition}
	
	Note that these two constructions are dual to one another. To remember them, remember that ``initial" and ``final" specify 
	the side of the map which we are trying to put a topology on. If we are given a map from a set \textbf{into} a space, then 
	we put the initial topology on it, and vice versa. Remembering if we should put the strongest or weakest topology on the 
	set is a bit harder, but still reasonably easy. For example, suppose we have the map $f : X\rightarrow Y$ from a space $X$ 
	into a set $Y$, and we wish to put a topology $\tau_Y$ on $Y$ making $f$ continuous. To make $f$ continuous, we don't 
	need to add any sets to $\tau_Y$ at all; we can simply let $\tau_Y = \{\emptyset, Y\}$, and then these will trivially pull back 
	open sets of $X$. But, this is not interesting, hence we don't want the final topology on $Y$ to be the weakest topology 
	making $f$ continuous; rather, we want an interesting topology! This means instead we should make the final topology on 
	$Y$ the \textbf{strongest} topology making $f$ continuous, as then we will get a nontrivial topology. The same line of 
	reasoning will show that we should consider the initial topology to be the weakest one making such a function continuous, 
	because the strongest topology we can put on the domain of a map to make it continuous is the power set itself.
	
	The first topology we will construct with these notions is the quotient topology.
	
	\begin{definition}[Quotient Topology]
		Let $\{(X_\alpha, \tau_\alpha)\}$ be a collection of topological spaces, $Y$ a set, and $f_\alpha : X_\alpha\rightarrow 
		Y$ a map for each $\alpha$. Then, the final topology on $Y$ is called the \textbf{quotient topology}. 
	\end{definition}
	
	Essentially, the quotient topology maps the natural structure of the open sets in $X$ onto $Y$. For another look at this, 
	see the section below about identification spaces. Given a collection $\{((X_\alpha, \tau_\alpha), f_\alpha)\}$ mapping into 
	$Y$ as above, we can explicitly construct the quotient topology on $Y$. Let $X = \bigoplus_\alpha X_\alpha$ be the 
	disjoint union of all the $X_\alpha$'s, and define $f : X\rightarrow Y$ by $f|_{X_\alpha} = f_\alpha$. Put a topology on $X$ 
	which has base $\tau_X := \bigoplus_\alpha \tau_{X_\alpha}$. Then we can define the quotient topology on $Y$ via this 
	map $f$, where a subset of $Y$ is open iff its preimage in $X$ is open. 
	
	Now, all that was a construction of a final topology. With an initial topology, the biggest difference is that our base is only 
	\textbf{finite intersections} of preimages of base elements. For example, suppose we are putting the initial topology on 
	$Y$, and we have maps $f_\alpha : Y\rightarrow X_\alpha$. Then a base for the initial topology is:
	\eq
		\beta = \left\{\bigcap_{j = 1}^n f_{\alpha_j}^{-1}(U_j) : U_{\alpha_j}\subseteq X_{\alpha_j}\textnormal{ is open.} \right\}
	\qe
	Feel free to justify this to yourself; this will be important later when we consider products. Finally, with initial topologies we 
	have a nice characterization of continuous maps. There is an equivalent theorem for final topologies.
	
	\begin{prop}
		Let $Y$ be equipped with the initial topology from maps $f_\alpha : Y\rightarrow X_\alpha$. Let $Z$ be another 
		topological space with a map $g : Z\rightarrow Y$. Then $g$ is continuous iff $f_\alpha\circ g$ is continuous for each 
		$\alpha$.
		\[
		\begin{tikzcd}
			{} & Z \arrow[d, "g"] & {} \\
			{} & Y \arrow[ld, bend right, "f_\alpha"] \arrow[d, "f_\beta"] \arrow[rd, bend left] & {} \\
			X_\alpha & X_\beta & \cdots \\
		\end{tikzcd}
		\]
	\end{prop}	
	
	\subsection{Concrete Constructions}

	Here, we will provide concrete examples of constructions. Some of the information will be repeated from the first section to this section, but everything should 
	be compatible, i.e. the original definition we gave of a quotient space will be the same as the definition we will give in this section, just looked at in a different 
	way. 
	
	\subsection{Disjoint Union}
	
	Let $X$ and $Y$ be topological spaces. Then we can put a topology on the disjoint union:
	\eq
		X\sqcup Y
	\qe
	by letting a base for the topology be of the form $\{U_\alpha\}\cup\{V_\beta\}$, where $\{U_\alpha\}$ are the open sets of $X$ and $\{V_\beta\}$ are the open 
	sets of $Y$. \textit{Note the union is disjoint}; even if $X$ and $Y$ come from the same space originally and have nonempty intersection, we still must 
	identify the elements of $X$ separately from the elements of $Y$. For example, if $X = \{1, 2, 3\}$ and $Y = \{3, 4, 5\}$, then $X\sqcup Y = \{1, 2, 3_X, 3_Y, 
	4, 5\}$.
	
	\subsection{Product Space}
	
	If $X$ and $Y$ are topological spaces, we can turn their Cartesian product 
	\eq
		X\times Y = \{(x, y) : x\in X, y\in Y\}
	\qe
	into a topological space by specifying a base for the topology as:
	\eq
		\left\{U\times V : U\subseteq X\textnormal{ open, } V\subseteq Y\textnormal{ open}\right\}.
	\qe
	Essentially, we take products of all the open sets as our base, and then union them to get all the open sets in the topology. In the case of $\mathbb R^2 = 
	\mathbb R\times\mathbb R$, the base we get is from the box topology; i.e. the base for $\mathbb R^2$ from the product topology is:
	\eq
		\{(a, b)\times (c, d) : a < b, c < d\}.
	\qe
	In this topology, the unit disk $B^2$ is \textbf{not} a base element; it is an open set, however, because it is a union of open boxes. Now, we turn our attention 
	to projection maps. Every product space comes equipped with two projection maps, 
	\eq
	\begin{cases}
		p_1 : X\times Y\rightarrow X & (x, y)\mapsto x \\
		p_2 : X\times Y\rightarrow Y & (x, y)\mapsto y
	\end{cases}
	\qe
	which are naturally surjective.
	
	\begin{theorem}
		If $X\times Y$ is a product space, then $p_1$ and $p_2$ are continuous and take open sets into open sets. Furthermore, \textbf{the product topology 
		is the coarsest topology for which $p_1$ and $p_2$ are continuous}.
	\end{theorem}
	
	\begin{proof}
		$p_1$ is continuous because it pulls back open $U\subseteq X$ to $U\times Y$, which is a base element. Suppose $A\subseteq X\times Y$ is open. 
		Then
		\eq
			A = \bigcup_{\alpha, \beta}U_\alpha\times V_\beta
		\qe
		and so
		\eq
			p_1(A) = p_1\left(\bigcup_{\alpha, \beta}U_\alpha\times V_\beta \right) = \bigcup_\alpha U_\alpha
		\qe
		is open in $X$. Suppose $\tau$ is another topology making $p_1$ and $p_2$ continuous; we will show that the product topology is contained in 
		$\tau$. Because $p_1$ is continuous, every open set $U\subseteq X$ pulls back (via $p_1$) to $U\times Y\in \tau$, and every open set $V\subseteq 
		Y$ pulls back (via $p_2$) to $X\times V\in\tau$. This implies that $U\times V = (U\times Y)\cap (X\times V)\in\tau$, and so the base for the product 
		topology is contained in $\tau$. Hence $\tau$ contains the product topology, so $\tau$ is finer than the product topology.
	\end{proof}
	
	\begin{theorem}
		A function $f : Z\rightarrow X\times Y$ is continuous iff $p_1\circ f : Z\rightarrow X$ and $p_2\circ f : Z\rightarrow Y$ are continuous.
	\end{theorem}
	
	\begin{proof}
		The forward direction is trivial. Suppose the converse, that $p_1\circ f$ and $p_2\circ f$ are continuous. Let $U\times V$ be open in $X\times Y$. 
		Then $(p_1\circ f)^{-1}(U) = f^{-1}(U\times Y)$ and $(p_2\circ f)^{-1}(V) = f^{-1}(X\times V)$ are open in $Z$, and because
		\eq
			f^{-1}(U\times V) = f^{-1}((U\times Y)\cap (X\times V)) = f^{-1}(U\times Y)\cap f^{-1}(X\times V)
		\qe
		is the intersection of two open sets, $f^{-1}(U\times V)$ is open as well, hence $f$ is continuous.
	\end{proof}
	
	Now, we can consider general product. Up until now, we have considered products of two (and by induction, finite) spaces. However, many of the things that 
	we have said hold more generally if we take arbitrary products.
	
	\begin{definition}[Product Space]
		Let $\{(X_\alpha, \tau_\alpha)\}_{\alpha\in A}$ be a collection of topological spaces. Let:
		\eq
			Y = \prod_{\alpha\in A} X_\alpha := \left\{h : A\rightarrow\bigoplus_\alpha X_\alpha | h(\alpha)\in X_\alpha, \forall\alpha\in A \right\}
		\qe
		equipped with projection maps $\pi_\alpha : Y\rightarrow X_\alpha, (x_\beta)_{\beta\in A}\mapsto x_\alpha$. Then the \textbf{product topology} on $Y$ 
		is the initial topology of $Y$ under the maps $\{\pi_\alpha\}_{\alpha\in A}$.
	\end{definition}
	
	First, note that we can view $Y$ as functions $h : A\rightarrow\bigoplus_\alpha X_\alpha$, since with each sequence $(x_\alpha)_{\alpha\in A}$, we can 
	define $h(\alpha) := x_\alpha$ (this is a more general way to define a product than to by viewing it as a sequence of tuples). As a base for the product 
	topology, we get \textbf{finite} intersections of preimages of open sets in each $X_\alpha$. This means that a subbase for the product topology is:
	\eq
		S = \{\pi_\alpha^{-1}(U) : U\subseteq X_\alpha\textnormal{ is open.}\}
	\qe
	Explicitly, the open sets for the product topology are of the form:
	\eq
		\bigcap_{j = 1}^n\pi_{\alpha_j}^{-1}(U_{\alpha_j})
	\qe
	where each $U_{\alpha_j}$ is open in $X_{\alpha_j}$. For an example, let $I$ be the interval $[0, 1]$ in $\mathbb R$, and let $B(I)$ denote that 
	set of all functions $I\rightarrow I$. Then we can equivalently define $B$ as a product; for $i\in I$, define $X_i := I$. Let $X := \prod_{i\in I} X_i$. Then, there is a 
	natural homeomorphism:
	\eq
		B(I)\xrightarrow{\sim}X
	\qe
	by mapping each $f\mapsto (f(i))_{i\in I}$. This product topology is interesting to consider; if we view elements of $X$ as functions $f : I\rightarrow I$, then 
	the open sets are finite intersections of preimages of open sets in $X_i$. This means the following: fix a finite number of points $\{i_j\}_{j = 1}^n$ in $I$. 
	For each $j$, take an open subset $U_j$ of $[0, 1]$. Then, let $\mathcal O$ be the set $\{f\in X : f(x_j) \in U_j\forall j = 1, ..., n\}$. Then $\mathcal O$ is an 
	open set, and every open set in the product topology is of this form. Essentially, the open sets in $X$ contain functions whose values are specified to be in 
	a given open set at a finite number of points in $I$. The values the functions take on not in this finite range of values can be arbitrary.
	
	\subsection{Identification Spaces}
	
	This is the most complicated construction we will encounter; the idea behind identification spaces it to ``identify" points on an existing space $X$ as equal, 
	and endow this new space with a topology inherited from $X$. We may formalize this notion like so:
	
	\begin{definition}[Identification Space]
		Let $P$ be a partition of $X$, i.e. $P$ is a collection of disjoint subsets of $X$ such that $X = \bigcup_{A\in P}A$. We define $\pi : X\rightarrow P$ to be 
		the map $x\mapsto A_x$, where $x\in A_x\in P$, and define the \textbf{identification space} $Y$ by setting the points of $Y$ to be the elements of $P$ 
		(i.e. as sets, $Y = P$), and we say a subset $A\subseteq Y$ is open iff $\pi^{-1}(A)\subseteq X$ is open.
	\end{definition}
	
	\begin{theorem}
		The projection map $\pi : X\rightarrow Y$ is continuous, and the identification topology is the largest (finest) topology on $Y$ making $\pi$ continuous.
	\end{theorem}
	
	This is essentially a restatement of the definition. We have a universal property for the quotient similar to the one for the product, whose proof is quite 
	easy:
	
	\begin{theorem}
		If $Y$ is an identification space and $Z$ is any space, then $f : Y\rightarrow Z$ is continuous iff $f\circ\pi : X\rightarrow Z$ is continuous. 
	\end{theorem}
	
	Now, we turn to the subject of attaching maps and quotient spaces. 
	
	\begin{definition}[Attaching Map]
		Let $X, Y$ be topological spaces, $A\subseteq X$, and $f : A\rightarrow Y$ a continuous function. Construct a partition $P$ on $X\sqcup Y$ by 
		letting $P$ contain $\{x\}$ for $x\in X\setminus A$, $\{y\}$ for $y\in Y\setminus im(f)$, and $\{y\}\cup f^{-1}(\{y\})$ for $y\in im(f)$. We denote the 
		identification space produced by $P$ by:
		\eq
			X\cup_f Y
		\qe
		and call $f$ the \textbf{attaching map}.
	\end{definition}
	
	What we have done is essentially used the map $f$ to glue together $X$ and $Y$ along $A$. We have identified any points in the image of $f$ with its 
	preimage as equal, so we can view this as joining the spaces along the set $A$. We may use the notion of an attaching map to construct a quotient space, 
	one in which we collapse an entire subset of a space to a point.
	
	\begin{definition}[Quotient Space]
		Let $A\subseteq X$, and let $Y = \{*\}$ be the one-point space. Let $f : A\rightarrow Y$ be the (only) map $a\mapsto *$. Then we define the 
		\textbf{quotient space} $X / A$ to be the identification space $X\cup_f Y$
	\end{definition}
	
	The intuitive view of a quotient space is that we have collapsed $A$ into a single point. For example, if we quotient the endpoints of a line segment out, 
	the resulting space is homeomorphic to a circle. 
	
	Now \textbf{suppose we have a continuous surjective $f : X\rightarrow Y$, and that the topology on $Y$ is the largest for which $f$ is continuous}. We can 
	define a partition $P$ on $X$ by taking sets of the form $f^{-1}(\{y\})$, for $y\in Y$, and let $Y^*$ be this identification space with projection map $\pi : X
	\rightarrow Y^*$. 
	
	\begin{lemma}
		A function $g : Y\rightarrow Z$ is continuous iff the map $g\circ f : X\rightarrow Z$ is continuous.
	\end{lemma}
	
	\begin{proof}
		The forward direction is obvious, so suppose $g\circ f$ is continuous. Let $V\subseteq Z$ be open, so $(g\circ f)^{-1}(V) = f^{-1}(g^{-1}(V))$ is open 
		in $X$. Let $U := g^{-1}(V)$. If $U$ is not open, then we may add $U$ to the topology on $Y$ and preserve the continuity of $f$ as $f$ pulls $U$ 
		back into an open set in $X$. But since the topology on $Y$ is the largest making $f$ continuous, this implies that $U$ is open in $Y$, and hence 
		$g^{-1}(V)$ is open, so $g$ is continuous.
	\end{proof}
	
	\begin{theorem}
		If the above conditions are satisfied ($f$ a continuous surjection and the topology on $Y$ the largest making $f$ continuous), then $Y$ and $Y^*$ 
		are homeomorphic by the map:
		\begin{align}
			f_* : Y^*\rightarrow Y && f_*(\{f^{-1}(y)\}) := y.
		\end{align}
	\end{theorem}
	
	\begin{proof}
		We define a map $f_* : Y^*\rightarrow Y$ by $f_*(\{f^{-1}(y)\}) = y$. Then $f_*$ is injective because if $f^{-1}(\{y_1\})\neq f^{-1}(\{y_2\})$, then $y_1\neq 
		y_2$, and $f_*$ is surjective because $f^{-1}(\{y\})$ is nonempty for each $y\in Y$ due to the surjectivity of $f$. Now, note that $f = f_*\circ\pi$ and 
		$\pi = f_*^{-1}\circ f$, and this implies that $f_*$ and its inverse are continuous by the previous theorems and lemma.
	\end{proof}
	
	Let's notice that the only place we used the condition about the topology on $Y$ being the finest making $f$ continuous was the lemma, so what can we 
	assume given an arbitrary continuous surjection $f : X\rightarrow Y$? We can still construct the identification space $Y^*$, and the induced map $f_* : Y^*
	\rightarrow Y$ is a continuous bijection. However, it is \textbf{not} necessarily a homeomorphism, as we cannot use the lemma that we proved. Note 
	that (I think) this is the topological equivalent of the first isomorphism theorem-- although if we don't make any assumptions about the topology on $Y$, 
	we can only assume that $f_*$ is a continuous bijection, not a homeomorphism.
	
	\begin{theorem}
		Let $f : X\rightarrow Y$ be a continuous surjection that maps open sets to open sets. Then $Y$ is an identification space of $X$, and in fact $Y\cong 
		Y^*$. 
	\end{theorem}
	
	\begin{proof}
		If $f$ maps open sets to open sets, then the topology on $Y$ is the largest making $f$ continuous, because any set in $Y$ that is not open does 
		not pull back to an open set in $X$, and hence if we add any open sets to $Y$, they do not pull back to open sets in $X$ under $f$ and hence 
		$f$ would cease to be continuous. So, this is equivalent to the previous theorem. 
	\end{proof}
	
	As one final example, we have the relation:
	\eq
		S^n\cong B^n / S^{n - 1}
	\qe
	where $B^n$ is the unit ball, and we use the relations $S^n\setminus\{p\}\cong \mathbb R^n$ for any point $p$.

	Now, we will move on to study topological properties that are invariant under homeomorphism; the two main properties 
	we will consider are \textbf{compactness} and \textbf{connectedness}.

\newpage

\section{Compactness}
	
	Compactness in an arbitrary topological space is a generalization of ``finiteness" in $\mathbb R^n$. Essentially, compact 
	sets are not homeomorphic to ``infinite sized" set. We will define compactness topologically in terms of covers:
	
	\begin{definition}[Open Cover, Subcover]
		An \textbf{open cover} of a topological space $X$ is a collection of open sets $\{U_\alpha\}$ such that:
		\eq
			X = \bigcup_\alpha U_\alpha.
		\qe
		If $\{U_\alpha\}$ and $\{V_\beta\}$ are open covers of $X$ and $\{V_\beta\}\subseteq\{U_\alpha\}$, then we say 
		$\{V_\beta\}$ is a \textbf{subcover} of $\{U_\alpha\}$.
	\end{definition}
	
	\begin{definition}[Compactness]
		A space $X$ is \textbf{compact} if every open cover of $X$ contains a finite subcover. A subset $A\subseteq X$ is 
		\textbf{compact} if it is compact in the subspace topology. 
	\end{definition}
	
	Showing a space is not compact is much easier than showing it is compact; just exhibit an open cover that has no finite 
	subcover. For example, the open cover $\{B_k(\vec{0})\}_{k\in\mathbb N}$ of $\mathbb R^n$ has no finite subcover, and 
	hence $\mathbb R^n$ is not compact. Compactness is preserved under continuous mappings.
	
	\begin{prop}
		The continuous image of a compact set is compact. In other words, if $f : X\rightarrow Y$ is continuous and $X$ is 
		compact, then $\im(f)\subseteq Y$ is compact as well.
	\end{prop}
	
	\begin{proof}
		Assume WLOG that $f$ is surjective, and let $\{V_\alpha\}$ be an open cover of $Y$, so $Y = \bigcup_\alpha V_
		\alpha$. Then we also have an open cover of $X$ given by $\{f^{-1}(V_\alpha)\}$, as $f$ is continuous and 
		surjective. But $X$ is compact, so this has a finite subcover $\{f^{-1}(V_{\alpha_i})\}_{i = 1}^n$, and we have that:
		\eq
			Y = f(X) = f\left( \bigcup_{i = 1}^n f^{-1}(V_{\alpha_i}) \right) = \bigcup_{i = 1}^n f \left( f^{-1}(V_{\alpha_i}) \right) = \bigcup_{i = 1}^n 
			V_{\alpha_i}
		\qe
		and so we have found a finite subcover of $\{V_\alpha\}$, and hence $Y$ is compact.
	\end{proof}
	
	\begin{prop}
		A closed subset of a compact space is compact.
	\end{prop}
	\begin{proof}
		Let $C$ be a compact space, and let $A\subseteq C$ be a closed subset of $C$. This proof is almost immediate, since we can union an open cover of $A$ together with $A^c$ to get an open cover of $C$. Let $\{U_\alpha\}$ be an open cover for $A$. Then $\{U_\alpha\}\cup \{A^c\}$ is an open cover of $C$, hence has a finite subcover $\{U_i\}\cup \{A^c\}$ (this subcover need not contain $A^c$, but we write it here for convenience), and $\{U_i\}$ is an open cover of $A$, which completes the proof. 
	\end{proof}
	
	However, note that the converse of this is not true, namely that there are compact subsets of a compact space which 
	are not closed. For example, take a finite number of points and give them the indiscrete topology. Then these will 
	not be closed, but they will be compact, because any finite set is compact. There are also some equivalent formations of 
	compactness that are sometimes easier to use in a proof. One of these will rely on a property called the finite intersection 
	property, and we will use it reasonably frequently.
	
	\begin{definition}[Finite Intersection Property]
		Let $X$ be a set, and $\xi$ a collection of subsets of $X$. We say that $\xi$ has the \textbf{Finite Intersection 
		Property (FIP)} if the intersection of any finite number of elements of $\xi$ is nonempty.
	\end{definition}
	
	\begin{theorem}
		A space $(X, \tau)$ is compact if given any family of closed subsets $\xi$ with the FIP, then $\bigcap\xi\neq
		\emptyset$.
	\end{theorem}
	
	An example of a collection of subsets with the FIP is the tower $\{[0, \frac{1}{n}]\}_{n\in\mathbb N}$ of $\mathbb R$. This is because if we take a finite intersection, it will always be equal to the smallest interval, and if we intersect the entire family together, we will have $0$ in the intersection. 
		
	Next, we recall some theorems that you may remember from analysis about compact spaces.
	
	\begin{theorem}[Bolzano-Weierstrauss]
		An infinite subset of a compact space has a limit point.
	\end{theorem}
	
	\begin{theorem}
		If $X$ is compact and $f : X\rightarrow\mathbb R$ is continuous, then $f$ is bounded and attains its bounds.
	\end{theorem}
	
	\begin{theorem}[Heine-Borel]
		A subset of $\mathbb R$ is compact iff it is closed and bounded.
	\end{theorem}
	
	We will now study how compact spaces behave under products. It turns out that they act rather nicely, and compactness will be preserved under products. We will state Tychonoff's theorem, which is a very important theorem in all areas of analysis.
	
	\begin{lemma}
		If $\{U_\alpha\}$ is a base for the topology on $X$, then $X$ is compact iff every open cover $C$ of $X$ such that 
		$C\subseteq\{U_\alpha\}$ has a 
		finite subcover.
	\end{lemma}
	
	\begin{prop}
		$X\times Y$ is compact iff $X$ and $Y$ are compact.
	\end{prop}
	
	This gives the immediate corollary:
	
	\begin{corollary}
		$A\subseteq\mathbb R^n$ is compact iff $A$ is closed and bounded.
	\end{corollary}
	
	\subsection{Tychonoff's Theorem}
	
	This is actually not the strongest statement that we can make about products of compact spaces. It turns out that in 
	addition to finite products, we can extend this to arbitrary, possibly \textbf{uncountable}, products of spaces.
	
	\begin{theorem}[Tychonoff]
		Let $\{(X_\alpha, \tau_\alpha)\}_\alpha$ be a collection of compact spaces. Then:
		\eq
			\prod_\alpha X_\alpha
		\qe
		is compact.
	\end{theorem}
	
	Before we are able to prove this, we need to study some properties of sets, namely we will need to introduce the 
	Axiom of Choice and Zorn's Lemma. Tychonoff's theorem will end up being equivalent to the AoC, just as Zorn's 
	lemma is, i.e. if you assume Tychonoff's theorem, then your set theory will implicitly have the AoC built in. 
	
	\begin{definition}[Axiom of Choice]
		Given any family of non-empty sets, there is a set that contains one element from each of these sets
	\end{definition}
	
	The axiom of choice seems very intuitive, but it can lead to some very strange constructions. For example, the axiom 
	of choice can be used to prove that every vector space has a basis. However, what is a basis for $\mathbb R$ as a 
	vector space over $\mathbb Q$? It often seems impossible to exhibit explicit constructions that are a consequence 
	of the axiom of choice, hence why it is so controversial. Another statement that is equivalent to the AoC is 
	Zorn's Lemma.
	
	\begin{definition}[Partial order]
		A \textbf{partial order} on a set $P$ is a relation $\leq$ such that:
		\begin{enumerate}
			\item $a\leq b$ and $b\leq c$ implies $a\leq c$.
			\item $a\leq b$ and $b\leq a$ implies $a = b$.
		\end{enumerate}
		A set $P$ is called a \textbf{partially ordered set (poset)} if it has a partial order.
	\end{definition}
	
	\begin{definition}[Total order, chain]
		A \textbf{total order} on a set $P$ is a partial order with the additional property that for any $a, b\in P$, either 
		$a\leq b$ or $b\leq a$. A \textbf{chain} in a poset $P$ is a totally ordered subset of $P$. 
	\end{definition}
	
	\begin{definition}[Maximal element]
		Let $P$ be a poset. A \textbf{maximal element} of $P$ is an element $a$ such that $\not\exists b\in P$ such that 
		$a < b$. 
	\end{definition}
	
	\begin{definition}[Inductively ordered]
		We say $P$ is \textbf{inductively ordered} if for each chain $\mathcal C\subseteq P$, there is some $b\in P$ with 
		$a < b$ for all $a\in\mathcal C$ (b may or may not be in $\mathcal C$). 
	\end{definition}
	
	\begin{theorem}[Zorn's Lemma]
		Let $P$ be inductively ordered. Then for each chain $\mathcal C$, there is a maximal element $b$ with 
		$a\leq b, \forall a\in\mathcal C$.
	\end{theorem}
	
	Note that by a maximal element, we do not mean an element $a$ with $a\geq x$ for each $x\in P$; instead, we mean 
	that no element can be larger than $a$. These are only equivalent if the set we are interested in has a total order. 
	For example, if we use the partial order $\subseteq$ on the collection of subsets of a set, then we may have two 
	subsets which are incomparable. We can still have a maximal element $A$ such that $B\not\subseteq A$, because 
	our definition only requires that we do not have $A\subset B$. 
	
	With these prerequisites, we can prove Tychonoff's theorem. Let $X = \prod_\alpha X_\alpha$. We will show that if 
	$\mathcal C$ is any collection of closed subsets of $X$ with the FIP, then $\mathcal C\neq\emptyset$. We define:
	\eq
		\Theta := \left\{D : \mathcal C\subseteq D\textnormal{ is a family of subsets of } X \textnormal{ s.t. } D \textnormal{ has 
		the FIP.} \right\}.
	\qe
	Note that the elements $D$ of $\Theta$ are not necessarily closed. Now, $\Theta$ form a poset by inclusion. First, 
	show that $\Theta$ is inductively ordered. Then, we have a maximal element of $D^*$ of $\Theta$ by Zorn's Lemma.
	Then $D^*$ satisfies:
	\begin{enumerate}
		\item $Z_1, Z_2\in D^*\implies Z_1\cap Z_2\in D^*$.
		\item Let $Y\subseteq X$. If $Y\cap Z\neq\emptyset$ for each $Z\in D^*$, then $Y\in D^*$.
	\end{enumerate}
	Now, for $D\in\Theta$ and for any $\alpha$, we claim that $\mathcal F_\alpha :=\{\pi_\alpha(Z) : Z\in D\}$ has the FIP. 
	TODO.
	
	As we stated before, Tychonoff's theorem, like Zorn's Lemma, is equivalent to the AoC.
	
	\begin{theorem}
		Tychonoff's theorem implies the AoC.
	\end{theorem}
	
	\begin{proof}
		Let $\{X_\alpha\}_{\alpha\in A}$ be a collection of non-empty sets. Form $\prod_\alpha X_\alpha$, and we wish to 
		show that this set is not empty. Let $X = \bigoplus_\alpha X_\alpha$, and let $w$ be a symbol (can take it to 
		be a set not in $\bigoplus_\alpha X_\alpha$). For each $\alpha$, set:
		\eq
			Y_\alpha := X_\alpha\cup\{w\}.
		\qe
		For each $Y_\alpha$, define $\tau_\alpha$ to be $\{X_\alpha, \{w\}, Y_\alpha, \emptyset\}$. Each $(Y_\alpha, 
		\tau_\alpha)$ is a compact space (as the topology is finite), so $\prod_\alpha Y_\alpha$ is compact. For each 
		$\alpha$, let $C_\alpha = \pi_\alpha^{-1}(X_\alpha)$. This is closed because $X_\alpha\subseteq Y_\alpha$ is 
		closed, and we claim that $\{C_\alpha\}_{\alpha\in A}$ has the FIP. Given $\{C_{\alpha_j}\}_{j = 1}^n$, we can 
		choose $x_{\alpha_j}\in X_{\alpha_j}$, and let $y\in Y$ be defined as:
		\eq
			y_\alpha := \begin{cases}
				x_{\alpha_j} & \alpha = \alpha_j \\
				w & else
			\end{cases}
		\qe
		so $y\in \bigcap_{j = 1}^n C_{\alpha_j}$, and thus $\{C_\alpha\}_{\alpha\in A}$ has the FIP. But, because 
		$\prod_\alpha Y_\alpha$ is compact, this implies that $\bigcap_\alpha C_\alpha\neq\emptyset$. But $z\in 
		\bigcap_\alpha C_\alpha$ has $z_\alpha\in X_\alpha$ for each $\alpha$, and hence $z\in\prod_\alpha X_\alpha$.
	\end{proof}
	
\subsection{Locally Compact Spaces}

	Many topological spaces are not compact, but they are ``nice" enough that we can find compact neighborhoods of 
	points. 
	
	\begin{definition}[Locally Compact]
		A topological space $(X, \tau)$ is \textbf{locally compact} if for each $x\in X$, there is an open set $\mathcal O
		\in\tau$ such that $x\in\mathcal O$ and $\overline{\mathcal O}$ is compact.
	\end{definition}
	
	\begin{prop}
		Let $(X, \tau)$ be locally compact. Let $C$ be a compact subset of $X$. Then $\exists\mathcal O\in\tau$ such 
		that $C\subseteq\mathcal O$ and $\overline{\mathcal O}$ is compact.
	\end{prop}
	
	\begin{proof}
		For each $x\in C$, we can find an open neighborhood $\mathcal O_x$ such that $x\in\mathcal O_x$ and 
		$\overline{\mathcal O_x}$ is compact. So, the set $\{\mathcal O_x\}_{x\in C}$ is an open cover for $C$, and 
		hence has a finite subcover because $C$ is compact. Call this subcover $\{\mathcal O_{x_j}\}_{j = 1}^n$, and 
		let:
		\eq
			\mathcal O = \bigcup_{j = 1}^n\mathcal O_{x_j}.
		\qe
		But, $ D := \bigcup_{j = 1}^n\overline{\mathcal O_{x_j}}$ is compact because it is a finite union of compact sets, 
		and $\overline{\mathcal O}\subseteq D$ because $\mathcal O\subseteq D$ and $\overline{\mathcal O}$ is the 
		smallest closed set containing $\mathcal O$. This implies that $\overline{\mathcal O}$ is compact because it 
		is a closed subspace of a compact space $D$, and hence we are done (in fact, $\overline{\mathcal O} = D$).
	\end{proof}
	
	The canonical example of a locally compact space is $\mathbb R$, because for each real number, we can find a 
	closed and bounded interval surrounding this number.

\newpage
	
\section{Connectedness}
	
	The next major topological property that we consider is connectedness. 
	
	\begin{definition}[Connectedness]
		A space $X$ is \textbf{connected} if whenever $X = A\cup B$ with $A, B$ open and $A\cap B = \emptyset$, then 
		either $A = \emptyset$ or $B = \emptyset$. 
	\end{definition}
	
	There are a number of equivalent ways to define a connected space. $X$ is connected iff:
	\begin{enumerate}
		\item if $X = A\cup B$ with $A, B$ open and nonempty, then $\overline A\cap B = \emptyset$ or $A\cap\overline B = 
			\emptyset$. 
		\item if $A\subseteq X$ is open and closed, then $A = X$ or $A = \emptyset$. 
		\item if $A\subseteq X$ has empty boundary, then $A = X$ or $A = \emptyset$. 
		\item if $f : X\rightarrow \{1, 2\}$ is continuous and $\{1, 2\}$ has the discrete topology, then $f$ is constant.
	\end{enumerate}
	
	\begin{theorem}
		$\mathbb R$ is connected.
	\end{theorem}
	
	\begin{proof}
		Assume $\mathbb R = A\cup B$ with $A$ and $B$ open and disjoint; note this implies they are both closed as well. Pick $x\in A, y\in B$ and assume 
		WLOG that $x < y$. Consider the set
		\eq
			X := \{b\in [x, y]: [b, y]\subseteq B]\}.
		\qe
		Note that $y\in X$, so $X\neq\emptyset$ and $x$ gives a lower bound for $X$, so $x\leq\inf X$, hence $I := \inf X$ 
		exists. This is a limit point of $X$ by definition, and because $X\subseteq B$, $I\in\overline B = B$, and hence $I
		\notin A$. But $B$ is open, so $\exists\epsilon > 0$ such that $(I - \epsilon, I + \epsilon)\subseteq B\implies I-
		\frac{\epsilon}{2}\in X$, but this is a contradiction to $I$ being the infimum of $X$.
	\end{proof}
	
	\begin{prop}
		A nonempty $X\subseteq\mathbb R$ is connected iff $X$ is an interval.
	\end{prop}
	
	For many of these connectedness proofs, think of why intuitively the set should not be connected. For example, if you 
	know a set is connected, it should have no ``holes", and you should be able to use this to prove that it is an interval. For 
	the theorem above, supposing that the set is not an interval will give you a ``hole", $p$, that is surrounded by points in 
	the set, and you can use this $p$ to find a contradiction to connectedness. Now, like compactness, connectedness 
	behaves well under continuous maps.
	
	\begin{prop}
		If $f : X\rightarrow Y$ is a continuous map and $X$ is connected, then $f(X)$ is connected.
	\end{prop}
	
	\begin{proof}
		Assume WLOG that $Y = f(X)$. Suppose $Y = A\cap B$ with $A$ and $B$ open, and $A\cap B = \emptyset$. Then $X = f^{-1}(A)\cap f^{-1}(B)$, which 
		are open, and $f^{-1}(A)\cap f^{-1}(B) = \emptyset$, hence $f^{-1}(A)$ or $f^{-1}(B)$ is the empty set because $X$ is connected. This implies that 
		$A = \emptyset$ or $B = \emptyset$. 
	\end{proof}
	
	\begin{lemma}
		Suppose $\{A_i\}$ is a collection of connected subspaces of $X$ such that $\bigcap_i A_i\neq\emptyset$. Then 
		\eq
			\bigcup_i A_i
		\qe
		is connected.
	\end{lemma}
	
	Note this is an arbitrary union of connected subsets, and doesn't have to be finite or even countable. 
	
	\begin{proof}
		Suppose that $\bigcup_i A_i = B\cup C$ with $B$ and $C$ open and disjoint, and let $p\in\bigcap_i A_i$. Let $p\in B$ WLOG. For each $A_i$, 
		$A_i\cap B$ and $A_i\cap C$ are disjoint and open in the subspace topology, and $A_i = (A_i\cap B)\cup (A_i\cap C)$, which implies that either 
		$A_i\cap B$ or $A_i\cap C$ is the empty set. But $p\in B\cap(\bigcap_i A_i)$, hence $A_i\cap B\neq\emptyset\implies A_i\cap C = \emptyset$. 
		This holds for each $A_i$, so
		\eq
			C = \bigcup_i (A_i\cap C) = \emptyset
		\qe
		and thus $\bigcup_i A_i$ is connected.
	\end{proof}
	
	\begin{theorem}
		$X\times Y$ is connected iff $X$ and $Y$ are connected.
	\end{theorem}
	
	\begin{proof}
		The forward direction is immediate because of the continuity of the projection maps $p_1$ and $p_2$. Suppose $X$ and $Y$ are connected. We 
		have $\{x\}\times Y\cong Y$ and $X\times\{y\}\cong X$ for any $x\in X, y\in Y$, so these are connected. Define 
		\eq
			A_{x, y} := (X\times \{y\})\cup (\{x\}\times Y)
		\qe
		which is connected because these two sets intersect at $(x, y)$. Fix $y_0\in Y$. Then:
		\eq
			X\times Y = \bigcup_{x\in X} A_{x, y_0}
		\qe
		and these $A_{x, y_0}$ intersect at $X\times \{y_0\}$, hence $X\times Y$ is connected.
	\end{proof}
	
	This gives immediate corollary:
	\begin{corollary}
		The spaces $\mathbb R^n$ and $S^n$ are connected.
	\end{corollary}
	
	\begin{proof}
		The first follows immediately; for $S^n$, note that $S^n\setminus\{p\}\cong\mathbb R^n$ is connected, and so $S^n = (S^n\setminus\{N\})\cup(S^n
		\setminus\{S\})$, where $N$ and $S$ are the north and south poles.
	\end{proof}
	
	We can break spaces up into connected pieces, which have some nice properties. In the case of the real line, there is only one maximal connected piece, 
	as the space is connected. However, if we look at a union of intervals, there are multiple pieces, each of which are connected individually. We give this 
	notion a name.
	
	\begin{definition}[Component]
		A \textbf{component} of a space $X$ is a maximal connected subspace. 
	\end{definition}
	
	The final property of connectedness we will study is path connectedness; this is a property in topological spaces that you can connect any two points together 
	via a curve.
	
	\begin{definition}[Path, Path-Connected]
		A \textbf{path} in a space $X$ is a continuous function $\gamma : [0, 1]\rightarrow X$. We say the path is from $\gamma(0)$ to $\gamma(1)$. A 
		space is called \textbf{path-connected} if for each $x\neq y$ in $X$, there is a path from $x$ to $y$.
	\end{definition}
	
	\begin{prop}
		A path-connected space is connected.
	\end{prop}
	
	\begin{proof}
		Suppose $X$ is path-connected and $X = A\cup B$ with $A, B$ disjoint and open; suppose $A$ and $B$ are nonempty. Pick $x\in A$, $y\in B$, and we 
		have a path $\gamma$ connecting them from $x$ to $y$. Then $[0, 1] = \gamma^{-1}(A)\cup\gamma^{-1}(B)$ is a disjoint union of open sets, so 
		because this is an interval, this implies one of these is empty; suppose WLOG that $\gamma^{-1}(A) = \emptyset$. But $\gamma(0) = x\in A$, so 
		this is a contradiction, hence $A$ or $B$ is empty.
	\end{proof}

\newpage

\section{Separability}

	In this section, we will consider separability of topological spaces. We are going to encounter some non-intuitive 
	properties that spaces may have. In my head, when I think of a topological space, I think of $\mathbb R^n$; however, 
	many spaces are not like this at all. When we consider more general spaces, we may not have a metric or a well defined 
	notion of distance, and the open sets do not act as we would expect them. A major property that some spaces have is 
	whether points or sets can be separated by open sets; in other words, how can you tell topologically that two points or 
	sets are distinct? The first examples we will consider is when points or sets can be separated.
	
	\begin{definition}[Hausdorff]
		A space $X$ is \textbf{Hausdorff} if given $x\neq y$ in $X$, we can find disjoint open neighborhoods $x\in U_x$, 
		$y\in U_y$ such that 
		\eq
			U_x\cap U_y = \emptyset.
		\qe
	\end{definition}
	
	\begin{definition}[Regular]
		A space $X$ is \textbf{regular} if for every closed $A\subseteq X$ and every $x\notin A$, there are disjoint open 
		sets $\mathcal O, U$ such that $A\subseteq\mathcal O$ and $x\in U$.
	\end{definition}
	
	\begin{definition}[Normal Space]
		Let $X$ be a topological space. We say it is \textbf{normal} if for any two disjoint closed subsets $\mathcal C_0$ 
		and $\mathcal C_1$ of $X$, there are disjoint open sets $\mathcal O_0$ and $\mathcal O_1$ such that 
		$\mathcal C_0\subseteq\mathcal O_0$, $\mathcal C_1\subseteq\mathcal O_1$, and:
		\eq
			\mathcal O_0\cap\mathcal O_1 = \emptyset.
		\qe
	\end{definition}
	
	For the most part, we will be interested in Hausdorff and normal spaces. Regular spaces are a nice middle ground 
	between the two, but they are not as interesting as the others. In a Hausdorff space, we can separate points. In a 
	regular space, we can separate points from sets. Finally, in a normal space we can separate sets. Note that normal 
	spaces are not necessarily Hausdorff (if singleton sets are closed, then they are). Many books in literature assume that 
	normal spaces are Hausdorff as well, but we will not require this. For example, $X$ with the indiscrete topology is normal 
	in our case, but \textbf{not} Hausdorff. Likewise, there are examples of Hausdorff spaces which are not normal. First, we 
	will show some properties of normal spaces.
	
	\begin{definition}[Metrizable]
		A topology $\tau$ on a set $X$ is \textbf{metrizable} if there is a metric on $X$ which induces $\tau$
	\end{definition}
	
	\begin{theorem}
		Every metric space is normal. To be more precise, every metrizable topology is normal.
	\end{theorem}
	
	\begin{proof}
		Let $(X, d)$ be a metric space. Let $C_0, C_1$ be closed and disjoint subsets of $X$. For $x\in C_0$, choose 
		$\epsilon_x > 0$ such that $B(x, \epsilon_x)\subseteq C_1^C$, and likewise for $y\in C_1$ choose $\epsilon_y > 0$ 
		with $B(y, \epsilon_y)\subseteq C_0^C$ (as $C_0, C_1$ are closed, we can do this). Now, let:
		\eq
			\mathcal O_0 &:= \bigcup_{x\in C_0} B\left(x, \frac{\epsilon_x}{3} \right) \\
			\mathcal O_1 &:= \bigcup_{y\in C_1} B\left(y, \frac{\epsilon_y}{3} \right).
		\qe
		Note that $\mathcal O_0$ and $\mathcal O_1$ are open and contain $C_0, C_1$. We claim that these are disjoint; 
		let $z\in\mathcal O_0\cap\mathcal O_1$. Then, $z\in B(x, \frac{\epsilon_x}{3})$ and in $B(y, \frac{\epsilon_y}{3})$. 
		Assume WLOG that $\epsilon_x \leq\epsilon_y$. Then $d(x, y)\leq d(x, z) + d(z, y) < \frac{\epsilon_x}{3} + 
		\frac{\epsilon_y}{3} \leq \frac{2\epsilon_x}{3} < \epsilon_x\implies y\in B(x, \epsilon_x)\subseteq C_1^C$, a 
		contradiction $y\in C_1$. Hence these sets are disjoint, which completes the proof.
	\end{proof}
	
	Now, for one of the fundamental theorems of point-set topology. We will soon use this to prove the Tietze Extension 
	theorem, which is another theorem of great importance. Before we state Urysohn's Lemma, we will prove a separate 
	lemma. This lemma will show us that given an open subset of a topological space containing a closed set, we can 
	``squeeze" another open set between the closed set and original open set. 
	
	\begin{lemma}
		Let $X$ be normal and $C$ a closed subset of $X$. Let $\mathcal O$ be an open subset of $X$ containing $C$. 
		Then, there is an open set $U$ such that $C\subseteq U\subseteq\overline U\subseteq\mathcal O$.
	\end{lemma}
	
	\begin{proof}
		Note that $\mathcal O^C$ is a closed subset disjoint from $C$. Because $X$ is normal, we can find disjoint open 
		subsets $U, V$ such that $C\subseteq U$ and $\mathcal O^C\subseteq V$. So, $U\subseteq\mathcal O$. But, $U
		\subseteq V^C\subseteq\mathcal O$, and because $V^C$ is a closed set containing $U$, we have $\overline U
		\subseteq V^C$. Hence $\overline U\subseteq\mathcal O$, and so we are done.
	\end{proof}
	
	\begin{theorem}[Urysohn's Lemma]
		Let $X$ be a normal space and $C_0, C_1\subseteq X$ disjoint and closed subsets of $X$. Then, there is a 
		continuous function $f : X\rightarrow [0, 1]$such that $f|{C_0}\equiv 0$ and $f|_{C_1}\equiv 1$. 
	\end{theorem}
	
	\begin{proof}
		We will sketch the proof for metric spaces, and I may fill in the general proof later. For any $A\subseteq X$, define 
		$\rho_A : X\rightarrow\mathbb R$ by:
		\eq
			\rho_A(x) := \inf \left\{d(x, y) : y\in A \right\}
		\qe
		i.e. $\rho_A$ is the distance from a point to $A$. One can show that $\rho_A(x) = 0$ if and only if $x\in \overline A$. 
		Now, let $A, B$ be closed and disjoint subsets of $X$. Define:
		\eq
			f(x) := \frac{\rho_A(x)}{\rho_A(x) + \rho_B(x)}.
		\qe
		Note $\rho_A(x) + \rho_B(x)$ is never zero, because this would only be the case if they both vanished. However, 
		because they are both closed, this could vanish only if there was an element in $\overline A = A$ and $\overline B = 
		B$, but we assumed they are disjoint, hence this sum is never 0. This implies that $f$ is continuous. We also have 
		$f|_A\equiv 0$ and $f|_B\equiv 1$, hence we are done.
	\end{proof}
	
	Now, we will use Urysohn's Lemma to prove the Tietze Extension theorem. We first prove a lemma which we will 
	use.
	
	\begin{lemma}
		For any space $(X, \tau)$ and closed $A\subseteq X$, if $C\subseteq A$ is closed in $A$, then $C$ is closed in 
		$X$.
	\end{lemma}
	
	\begin{proof}
		In this case, $A\setminus C\subseteq A$ is open in $A$, so $A\setminus C = U\cap A$ for some open $U\in 
		\tau$. But, this implies that $C = A\cap U^C$ is open in $X$. 
	\end{proof}
	
	\begin{theorem}[Tietze Extension Theorem]
		Let $(X, \tau)$ be a normal space with $A\subseteq X$ closed. Let $f : A\rightarrow\mathbb R$ be continuous. 
		Then there is an extension $\tilde f : X\rightarrow\mathbb R$ of $f$ which is continuous. Furthermore, if $f : 
		A\rightarrow [a, b]$, then we can choose $\tilde f : X\rightarrow [a, b]$. 
	\end{theorem}
	
	\begin{proof}
		We will sketch the proof. First, consider the case when $f : A\rightarrow [0, 1]$, which will prove this for all 
		compact sets because we can find a homeomorphism onto $[a, b]$. Essentially, we will construct a sequence 
		of maps using Urysohn's Lemma to extend $f$. The easiest way to visualize this proof is with a ``mountain 
		range" of values of $f$ on $A$. Here is the idea. $f$ originally maps $A$ onto the interval $[0, 1]$. We will 
		``flatten" the range of this out by defining inductively functions $f_n$ which map $A$ onto the interval $[0, 
		(\frac{2}{3})^n]$. We will take the low and high points of each part of the mountain range to apply Urysohn's 
		lemma to, and using this we will be able to construct a family of functions $g_n$ which sum to the original 
		function on $A$.
		
		We also need to prove this for the unbounded case, i.e. $f : A\rightarrow\mathbb R$ is unbounded. Let 
		$h : \mathbb R\rightarrow (-1, 1)$ be a homeomorphism. Then $g = h\circ f : A\rightarrow (-1, 1)\subseteq [-1, 1]$, 
		so we can extend $g$ to $\tilde g : X\rightarrow [-1, 1]$. Now, we need to determine if $\tilde g$ maps anything 
		into $0$ or $1$, since we want to define an extension of $g$ which maps $X$ into $(-1, 1)$. Let $D = 
		\tilde g^{-1}(\{-1\}\cup\{1\})$, which is closed. Then note that $A$ and $D$ are disjoint, because everything in 
		$A$ maps into $g(A)\subset (-1, 1)$. Then we can apply Urysohn's lemma to $A$ and $D$ to get a map 
		$k : X\rightarrow [0, 1]$ such that $k|_D = 0$ and $k|_A = 1$. Now, we will use $k$ to kill off any of the values 
		of $\tilde g$ which are at the endpoints $-1$ or $1$, which is $D$. Thus $k\cdot\tilde g : X\rightarrow (0, 1)$ is a 
		continuous extension of $g$ to $(0, 1)$, and we can take $\tilde f := h^{-1}\circ (k\cdot\tilde g)$, and so we are done.
	\end{proof}
	
	\subsection{Compactness and Hausdorff Spaces}
	
	There are a number of important theorems relating compact spaces and Hausdorff spaces, enough to warrant a full 
	subsection on the topic. 
	
	\begin{prop}
		If $f : X\rightarrow Y$ is a continuous bijection from a compact space to a Hausdorff space, then $f$ is a 
		homeomorphism. 
	\end{prop}
	
	\begin{proof}
		Suppose $B\subseteq X$ is closed. Then $B$ is compact, and because $f$ is continuous, $f(B)$ is compact in $X$. 
		But a compact subset of a Hausdorff space is closed (proof next), so $f(B)$ is closed, and $f$ takes closed sets to 
		closed sets. Because $f$ is a bijection, this implies that $f^{-1}$ pulls back open sets to open sets as well, so $f$ is 
		a homeomorphism. 
	\end{proof}
	
	\begin{prop}
		Compact subsets of Hausdorff spaces are closed, i.e. if $X$ is Hausdorff and $A\subseteq X$ is compact, then $A$ 
		is closed. 
	\end{prop}
	
	\begin{proof}
		Let $x\in A^c$, and pick $z\in A$. Then we have disjoint open neighborhoods $x\in U_x$, $z\in V_z$, and let us vary 
		$z$ throughout $A$ to get an open cover $\{V_z\}_{z\in A}$ of $A$. Then this has a finite subcover 
		$\{V_{z_i}\}_{i = 1}^n$, and consider:
		\eq
			U := \bigcap_{i = 1}^n U_{x_i}.
		\qe
		$U$ is an open neighborhood of $x$, and we have $A\subseteq \bigcup_{i = 1}^n V_{z_i}$, and that $U\cap V_{z_i} 
		= \emptyset$ by construction. Then:
		\eq
			U\cap A \subseteq U\cap \left(\bigcup_{i = 1}^n V_{z_i} \right) = \bigcup_{i = 1}^n U\cap V_{z_i} = \bigcup_{i = 1}^n
			\emptyset = \emptyset.
		\qe
		and hence we have found a neighborhood of $x$ contained in $A^c$, so $A^c$ is open.
	\end{proof}
	
	Compact Hausdorff spaces themselves are very nice. We can show that they are both regular and normal.
	
	\begin{prop}
		Any compact Hausdorff space is regular.
	\end{prop}
	
	\begin{proof}
		Let $X$ be a compact Hausdorff space. Let $A\subseteq X$ be closed, and $x\notin A$. Note that because $A$ 
		is a closed subset of a compact space, it is compact. For $y\in A$, because $X$ is Hausdorff we can separate $x$ 
		and $y$ by open sets, i.e. we have $x\in U_y$ and $y\in V_y$ such that $U_y\cap V_y = \emptyset$. But, 
		$\{V_y\}_{y\in A}$ is an open cover for $A$, and hence we have a finite subcover $\{V_{y_j}\}_{j = 1}^n$ for $A$. 
		So, let:
		\begin{align}
			U := \bigcap_{j = 1}^n U_{y_j} && \mathcal O := \bigcup_{j = 1}^n V_{y_j}.
		\end{align}
		Then $x\in U$ and $A\subseteq \mathcal O$, and both of these are open since $U$ is a finite intersection of 
		open sets. Furthermore, these are disjoint because if $z\in U\cap\mathcal O$, then $z\in V_{y_k}$ for some 
		$k$, but $z\in U\implies z\in U_{y_k}$ which is disjoint from $V_{y_k}$, a contradiction. Thus $U\cap
		\mathcal O = \emptyset$, so we are done.
	\end{proof}
	
	\begin{prop}
		Any compact Hausdorff space is normal.
	\end{prop}
	
	\begin{proof}
		Let $X$ be a compact Hausdorff space, and let $C, D\subseteq X$ be closed and disjoint. Both of these spaces 
		are compact. By the previous proposition, $X$ is regular, hence for each $x\in D$ we can find disjoint open 
		subsets $\mathcal O_x$ and $U_x$ such that $x\in U_x$ and $C\subseteq\mathcal O_x$. Now, we have an 
		open cover of $D$ given by $\{U_x\}_{x\in D}$, so we get a finite subcover $\{U_{x_j}\}_{j = 1}^n$. So, let:
		\begin{align}
			\mathcal O :=\bigcap_{j = 1}\mathcal O_{x_j} && U := \bigcup_{j = 1}^n U_{x_j}.
		\end{align}
		Then we have $C\subseteq\mathcal O$ and $D\subseteq\mathcal U$, $U\cap\mathcal O = \emptyset$, and both 
		of these are open because they are finite intersections and unions, hence $X$ is normal.
	\end{proof}
	
	Note the proof of each of these is essentially the same. Pick an open cover of the set you are interested in by ranging 
	over all points in the set, use compactness to make this finite, then intersect or union the relevant sets until you 
	can cover the entire set with one open set.
	
	\begin{prop}
		Let $X$ be a set with two topologies, $\tau_1$ and $\tau_2$ such that $\tau_2\subseteq\tau_1$, so $\tau_1$ is 
		finer than $\tau_2$. If $(X, \tau_1)$ is compact, then so is $(X, \tau_2)$. If $(X, \tau_2)$ is Hausdorff, then so 
		is $(X, \tau_1)$. 
	\end{prop}
	
	\begin{prop}
		If $(X, \tau_1)$ and $(X, \tau_2)$ are both compact and Hausdorff and $\tau_1\subseteq \tau_2$, then $\tau_1 = 
		\tau_2$. 
	\end{prop}
	
	Now, we will turn our attention to Hausdorff spaces which are locally compact. When we write LCH, this means ``locally 
	compact and Hausdorff". 
	
	\begin{prop}
		Let $(X, \tau)$ be LCH, and let $C\subseteq X$ be compact. Then for $\mathcal O\in\tau$ with $C\subseteq
		\mathcal O$, there is an open set $U\in\tau$ such that $C\subseteq U\subseteq \overline U\subseteq O$, and 
		$\overline U$ is compact. 
	\end{prop}
	
	\begin{proof}
		$X$ is locally compact, so we can find an open set $V$ such that $C\subseteq V$ with $\overline{V}$ compact. 
		Let $W := V\cap \mathcal O$, which is an open set containing $C$ such that $W\subseteq\mathcal O$.  
		We also have that $\overline W$ is compact and Hausdorff, so $\overline W$ is normal. This implies that 
		$C$ is normal. Let $B := \overline W\setminus W$. This is closed and disjoint from $C$ because $C\subseteq W$. 
		This is closed and disjoint from $C$ and contained in $\overline W$ which is normal, so we have open sets 
		$U, Z$ disjoint such that $C\subseteq U$ and $B\subseteq Z$. Then $U\subseteq Z^C\cap\overline W$ which is 
		closed, so $\overline U\subseteq Z^C\cap\overline W$. Thus $\overline U\subseteq B^C = (\overline W\setminus 
		W)^C\cap\overline W = W\subseteq \mathcal O$, and $\overline U$ is compact, so we are done.
	\end{proof}
	
	\begin{definition}[Support]
		Let $f$ be a continuous function on $X$ with values in a normed vector space $V$. Then the \textbf{support} of $f$ 
		is:
		\eq
			\supp(f) := \overline{\{x\in X : f(x)\neq 0\}}.
		\qe
		If $\supp(f)$ is compact, then we say that $f$ has \textbf{compact support}. Let $C_C(X, V)$ be the vector space of 
		continuous functions of compact support.
	\end{definition}
	
	\begin{prop}
		Let $(X, \tau)$ be LCH. Let $C\subseteq X$ be compact, and let $\mathcal O\in\tau$ with $C\subseteq\mathcal O$. 
		Then there is a continuous function $f : X\rightarrow [0, 1]$ such that $f\equiv 1$ on $C$, $f\equiv 0$ on $\mathcal
		 O^C$, and $f$ has compact support. 
	\end{prop}
	
	\begin{proof}
		By the previous proposition, we can find a set $U$ such that $\overline U$ is compact and $C\subseteq U\subseteq
		\overline U\subseteq \mathcal O$. So, $\overline U$ is normal. We can apply the previous proposition again 
		to $U$ and find $V$ such that $\overline V$ is compact and $C\subseteq V\subseteq\overline V\subseteq U$. 
		Let $B := \overline U\setminus V$, which is closed in $\overline U$. Then because $B$ and $C$ are disjoint, 
		by Urysohn's lemma we can find a continuous function $f : \overline U\rightarrow [0, 1]$ such that $f|_C\equiv 1$ 
		and $f|_B\equiv 0$. For $x\notin\overline U$, set $f(x) := 0$. Then $f|_C\equiv 1$ and $f|_{\mathcal O^C}\equiv 
		0$. If $x\in U$, then $f$ is continuous at $x$ because $f$ was defined to be continuous on $\overline U$. If 
		$x\in \overline{V^C}$, then $f(x) = 0$, so it is continuous on $\overline{V^C}$ as well. Thus $f$ is continuous, 
		and we are done.
	\end{proof}

\newpage

\section{Banach Spaces and Metric Spaces}

\subsection{Banach Spaces}

	Let $X$ be any set, and $V$ a normed vector space (we will often use $V = \mathbb R$). Then $V$ inherits a metric from 
	its norm, and likewise a topology from this metric. 
	
	\begin{definition}
		$S\subseteq V$ is \textbf{bounded} if $\exists r\in\mathbb R$ such that $S\subseteq B(0, r)$. We say a function 
		$f : X\rightarrow V$ is bounded if $f(X)$ is bounded. Denote by $B(X, V)$ the set of all bounded functions 
		$X\rightarrow V$. We define a norm on $B(X, V)$ by:
		\eq
			||f||_\infty := \sup_{x\in V} ||f(x)||.
		\qe
	\end{definition}
	
	\begin{definition}[Banach Space]
		Let $V$ be a normed vector space with norm $||\cdot||$. If $V$ is complete with respect to $||\cdot||$, then we say 
		$V$ is a \textbf{Banach space}.
	\end{definition}
	
	Banach spaces are a very interesting object of study in their own right. We will barely be able to scratch the surface, 
	but they will be an object that we see a lot of for the rest of this course. Note that $B(X, V)$ is a vector space as 
	well under the operations of pointwise addition and pointwise scaling. Now, we have the following proposition.
	
	\begin{prop}
		Let $V$ be a Banach space. Then $B(X, V)$ is a Banach space. 
	\end{prop}
	
	\begin{proof}
		Let $(f_n)$ be a Cauchy sequence in $B(X, V)$. Note that the sequence $(f_n(x))$ is Cauchy in $V$. For $x\in X$, 
		we can define:
		\eq
			f(x) := \lim f_n(x).
		\qe
		This is a valid definition because $V$ is a Banach space, and hence Cauchy sequences converge. We show that 
		$f_n\rightarrow f$ with respect to $||\cdot||_\infty$. Let $\epsilon > 0$. We need to show that we can find $N\in
		\mathbb N$ such that $||f - f_n||_\infty < \epsilon$, for each $n \geq N$. Pick $N\in\mathbb N$ such that 
		$||f_n - f_m||_\infty < \frac{\epsilon}{2}$ for $n, m\geq N$, which exists because $(f_n)$ is Cauchy. Then for 
		each $x\in X$, we can choose $m \geq N$ such that $||f(x) - f_m(x)|| < \frac{\epsilon}{2}$ because $f_n(x)
		\rightarrow f(x)$. But then:
		\eq
			||f_n(x) - f(x)||\leq ||f_n(x) - f_m(x)|| + ||f_m(x) + f(x)|| < \frac{\epsilon}{2} + \frac{\epsilon}{2} = \epsilon
		\qe
		and hence we are done.
	\end{proof}
	
	\begin{prop}
		Let $(X, \tau)$ be a topological space, and let $C_b(X, V)$ be the set of all closed and bounded continuous 
		functions from $X$ to a Banach space $V$. Then $C_b(X, V)$ is a closed subspace of $B(X, V)$. 
	\end{prop}
	
	\begin{proof}
		It is easy to show $C_b(X, V)$ is a subspace. So, will show it is closed. Let $(f_n)$ be a sequence in 
		$C_b(X, V)$ that converges to $f\in B(X, V)$ w.r.t $||\cdot||_\infty$. We show that $f$ is continuous, so let 
		$\epsilon > 0$ and $v\in V$. We need to show $f^{-1}(B(v, \epsilon))$ is open in $X$. TODO.
	\end{proof}
	
\subsection{Compact Metric Spaces}

	Compact metric spaces give us an environment to extend the notion about compactness in $\mathbb R^n$. Namely, 
	we will seek to characterize compact subsets of a metric space, just as the Heine-Borel theorem characterized 
	compact subsets of Euclidean space. We first need to generalize the notion of boundedness in $\mathbb R^n$.
	
	\begin{definition}[Totally Bounded]
		A metric space $(X, d)$ is \textbf{totally bounded} if for every $\epsilon > 0$, there are a finite number of 
		$\epsilon$-balls that cover $X$.
	\end{definition}
	
	\begin{prop}
		Let $X$ be a metric space, and $A\subseteq X$. If $A$ is totally bounded, then so is $\overline A$. 
	\end{prop}
	
	\begin{proof}
		Let $\epsilon > 0$. Then $A$ is totally bounded, so we can contain $A$ in a finite number of $\frac{\epsilon}{2}$ 
		balls centered at $\{z_j\}_{j = 1}^n$. But, $A$ is dense in $\overline A$, so for any $x\in\overline A$, we can find an 
		element $y\in A$ such that $x\in B(y, \frac{\epsilon}{2})$. But $y$ is contained in some $\frac{\epsilon}{2}$ ball about 
		some point $z_k$ for some $k$, and so $x$ is contained in the $\epsilon$-ball about $z_k$, and hence $\overline 
		A$ is covered by a finite number of $\epsilon$-balls.
	\end{proof}
	
	We will also deal with completeness in metric spaces. Complete metric spaces will have a special place in the study of 
	compact metric spaces, as seen in the following proposition.
	
	\begin{prop}
		Let $(X, d)$ be a metric space. If $X$ is not complete, then it is not compact.
	\end{prop}
	
	\begin{proof}
		Because $X$ is not complete, we can find a Cauchy sequence $(x_n)$ in $X$ such that $x_n$ does not converge. 
		Pick $x\in X$. Then $x_n\not\rightarrow x$, so there is $\epsilon_x > 0$ such that for each $M\in\mathbb N$, 
		there is $m > M$ with $d(x_m, x) > \epsilon_x$. Because $(x_n)$ is Cauchy, there is $N\in\mathbb N$ such 
		that $n, m\geq N\implies d(x_n, x_m) < \frac{\epsilon}{2}$. Now, there is $m > N$ such that $d(x_m, x) > 
		\epsilon_x$ by what we already argued. TODO.
	\end{proof}
	
	\begin{prop}
		Let $(X, d)$ be a complete metric space. Then if $X$ is also totally bounded, then $X$ is compact.
	\end{prop}
	
	\begin{proof}
		TODO.
	\end{proof}
	
	\begin{theorem}
		Let $(X, d)$ be a complete metric space with $A\subseteq X$. If $A$ is totally bounded, then $\overline A$ is 
		compact.
	\end{theorem}
	
	This follows immediately from the last few theorems, because $A$ is totally bounded implies $\overline A$ is as well, 
	and any closed subset of a complete metric space is complete. Hence we can apply the previous proposition to show 
	that $\overline A$ is compact. Now, we will examine conditions on when a subset of $X$ is totally bounded.
	
	\begin{definition}[Bounded]
		Let $(X, d)$ be a metric space. A subset $A\subseteq X$ is \textbf{bounded} if $\exists K\geq 0$ such 
		that $d(x, y) < K$ for each $x, y\in A$. Let $Z$ be a set. A function $f : Z\rightarrow X$ is bounded if 
		$f(Z)$ is bounded.
	\end{definition}
	
	\begin{theorem}
		Let $B(X, Y)$ be the collection of bounded functions $f : X\rightarrow Y$, for a set $X$ and a 
		metric space $Y$. Then for $f, g\in B(X, Y)$, set:
		\eq
			d_\infty (f, g) := \sup_{x\in X}\{d(f(x), g(x))\}.
		\qe
		Then $(B(X, Y), d_\infty)$ is a metric space. If $Y$ is complete, then $B(X, Y)$ is complete.
	\end{theorem}
	
	Note that $d_\infty$ is defined on $B(X, Y)$ because each of these functions are bounded. For the rest of this section, 
	we will be working in $B(X, Y)$ with the metric $d_\infty$. Suppose that $X$ is a topological space. We again examine 
	$B(X, Y)$ with the metric $d_\infty$. If $\tau$ is the topology on $X$, then let:
	\eq
		C_b(X, Y) = \{f \in B(X, Y) : f \textnormal{ is closed and bounded.}\}.
	\qe
	Suppose that we are given $\mathcal F\subseteq C_b(X, Y)$, where $\mathcal F$ is totally bounded. Then for $\epsilon 
	> 0$, we can find a finite number of $g_1, ..., g_n\in \mathcal F\subseteq C_b(X, Y)$ such that the balls of radius 
	$\epsilon$ with $d_\infty$ around each $g_j$ cover $\mathcal F$. Let $x\in X$ be given. Then for each $j$, we 
	can find $\mathcal O_j\in\tau$ such that $x, y\in\mathcal O_j\implies d(g_j(x), g_j(y)) < \epsilon$ because $g_j$ is 
	continuous. Now, let:
	\eq
		\mathcal O_x = \bigcap_{j = 1}^n\mathcal O_j\in\tau.
	\qe
	Then $x\in \mathcal O_x$ and if $y\in\mathcal O_x$, we have $d(g_j(x), g_j(y)) < \epsilon$ for each $j$. Let $f\in
	\mathcal F$. Since the $g_j$ cover $\mathcal F$, we can find $g_j$ such that $d_\infty(g_j, f) < \epsilon$. So, we 
	have:
	\eq
		d(f(x), f(y)) \leq d(f(x), g_j(x)) + d(g_j(x), g_j(y)) + d(g_j(y), g(y)) < 3\epsilon.
	\qe
	Rename the $\epsilon$ to $\frac{\epsilon}{3}$. So, we have shown that if $\mathcal F$ is a totally bounded family 
	of functions in $C_b(X, Y)$ and if $x\in X$ and $\epsilon > 0$, then there is an open set $\mathcal O_x$ such 
	that $y\in \mathcal O_x$ implies $d(f(x), f(y)) < \epsilon$ for each $f\in\mathcal F$. This $\mathcal O_x$ works for 
	every $f\in\mathcal F$. Essentially, the family of functions $\mathcal F$ are all smooth enough to ensure ``continuity" 
	for each function in $\mathcal F$. This motivates the following definition:
	
	\begin{definition}[Equicontinuity]
		Let $(X, \tau)$ be a topological space and $(Y, d)$ a metric space. Let $\mathcal F\subseteq C(X, Y)$. We say that 
		$\mathcal F$ is \textbf{equicontinuous} at $x\in X$ if for each $\epsilon > 0$, there exists $\mathcal O_x\in\tau$ 
		such that $y\in\mathcal O_x\implies d(f(x), f(y)) < \epsilon$ for each $f\in\mathcal F$. If $\mathcal F$ is 
		equicontinuous at each point in $X$, we say that $\mathcal F$ is equicontinuous.
	\end{definition}
	
	Equicontinuity is essentially a generalization of continuity applied to a family of functions. It will have an important role to play with regards to total boundedness. The other definition we must make first before we can discuss that is that of pointwise boundedness.
	
	\begin{definition}[Pointwise totally bounded]
		Let $\mathcal F\subseteq C_b(X, Y)$ be a family of functions. We say that $\mathcal F$ is \textbf{pointwise 
		totally bounded} for each $x\in X$, the set $\{f(x) : f\in\mathcal F\}$ is bounded, i.e. if there are a finite 
		number of $\epsilon$-balls in $Y$ which cover $\{f(x) : f\in\mathcal F\}$.
	\end{definition}
	
	Now, for the main part of the Arzela-Ascoli theorem. This theorem makes it immensely easier to prove that a 
	subset of $C_b(X, Y)$ is totally bounded, for in general it is quite difficult.
	
	\begin{theorem}
		Let $(X, \tau)$ be a compact topological space and $(M, d)$ a metric space. If $\mathcal F\subseteq C_b(X, M)$ 
		is equicontinuous and pointwise totally bounded, then $\mathcal F$ is totally bounded for $d_\infty$.
	\end{theorem}
	
	\begin{theorem}[Arzela-Ascoli]
		Let $(X, \tau)$ be a compact space, and let $(M, d)$ be a complete metric space. Then $\mathcal F$ is compact in
		$(C(X, M), d_\infty)$ iff the following hold:
		\begin{enumerate}
			\item $\mathcal F$ is equicontinuous.
			\item $\mathcal F$ is pointwise totally bounded.
			\item $\mathcal F$ is closed in $C_b(X, Y)$.
		\end{enumerate}
	\end{theorem}
	
	Arzela-Ascoli immediately follows from previous theorem, because a subset of a complete metric space is compact iff it is 
	closed and totally bounded.

\newpage

\section{Random Set Theory Identities}

	\begin{enumerate}
		\item Maps respect unions, complements, intersections, and set difference:
		\eq
		f^{-1}(A\cup B) &= f^{-1}(A)\cup f^{-1}(B) \\
		f^{-1}(A\cap B) &= f^{-1}(A)\cap f^{-1}(B) \\
		f^{-1}(A^c) &= f^{-1}(A)^c \\
		f^{-1}(A\setminus B) &= f^{-1}(A)\setminus f^{-1}(B).
		\qe
		\item For any map $f$:
		\eq
			A\subseteq f^{-1}(f(A)).
		\qe
		and if $f$ is surjective, then we have equality:
		\eq
			A = f^{-1}(f(A)).
		\qe
	\end{enumerate}

\end{document}