\documentclass[11pt, oneside]{amsart}   	% use "amsart" instead of "article" for AMSLaTeX format
\usepackage{geometry}                		% See geometry.pdf to learn the layout options. There are lots.
\geometry{letterpaper}                   		% ... or a4paper or a5paper or ... 
%\geometry{landscape}                		% Activate for rotated page geometry
%\usepackage[parfill]{parskip}    		% Activate to begin paragraphs with an empty line rather than an indent
\usepackage{graphicx}				% Use pdf, png, jpg, or eps§ with pdflatex; use eps in DVI mode
								% TeX will automatically convert eps --> pdf in pdflatex		
\usepackage{amssymb}
\usepackage{amsthm}
\usepackage{mathtools}
\usepackage{float}

\theoremstyle{definition}
\newtheorem{definition}{Definition}[section]
\newtheorem{theorem}{Theorem}[section]
\newtheorem{corollary}{Corollary}[theorem]
\newtheorem{lemma}[theorem]{Lemma}

%SetFonts

%SetFonts

\title{Math 250A Lecture Recaps (Groups)}
\author{Patrick Oare}
%\date{}							% Activate to display a given date or no date

\begin{document}
\maketitle
\section{8/24 (Groups, Direct Product)}

\begin{itemize}

	\item \textbf{Groups}: A group can be equivalently defined as:
		
		\begin{itemize}
			
			\item The set of symmetries of an object
			
			\item A pair $(G, \cdot)$ where $G$ is a set and $\cdot : G\times G\rightarrow G$ is a map satisfying:
			
				\begin{enumerate}
					
					\item Associativity. We have $(ab)c = a(bc)$ for every $a, b, c\in G$.
					
					\item Identity. There is an $e\in G$ with $ea = ae = a$ for each $a\in G$.
					
					\item Inverse. For each $a\in G$, we have $a^{-1}\in G$ such that $aa^{-1} = a^{-1}a = e$.
					
				\end{enumerate}
				
				$G$ is \textbf{Abelian} if the operation commutes.
			
		\end{itemize}
		
	\item \textbf{Actions of a group on itself.}
	
	A \textbf{symmetry} preserves an action. An \textbf{action} is a map $\cdot: G\times S\rightarrow S$ with $g_1\cdot(g_2\cdot s) = (g_1g_2)\cdot 
	s$ and $e\cdot s = s$ for each $g_1, g_2\in G, s\in S$. Every group has 4 left actions on itself and 4 right actions on itself. The 4 left actions 
	are:
	
		\begin{enumerate}
		
			\item $g\cdot s := s$
			
			\item $g\cdot s := gs$
			
			\item $g\cdot s := sg^{-1}$
			
			\item $g\cdot s := gsg^{-1}$
		
		\end{enumerate}
	
	The right actions of a group are similar. 
	
	\item \textbf{Lagrange's Theorem}: The order of any subgroup $H\leq G$ divides $|G|$.
	
	Suppose $G$ acts on a set $S$ \textbf{transitively}, so there is one orbit. Fix $s\in S$, and let $H = G_s$ be the \textbf{isotropy} group 
	(stabilizer) of $s$. Then, we have a bijection between $G / H$ and the points of $S$. The forward direction of the map is mapping $t\in S$ to
	$t\mapsto\{g\in G : gs = t\}$. This is a coset of $H$ because if $gs = t$, then $(gh)s = g(hs) = gs$, so $gH$ is contained in the set. The reverse 
	direction of the map is given by $gH\mapsto gs$. This is well defined because if $g_1H = g_2H$, then $g_1 = g_2h$, so $g_1s = g_2hs$ = 
	$g_2s$.
	
	This also gives us that if we fix $s\in S$:
	$$
		|G| = |H|\#(cosets) = |G_s|\times|S|
	$$
	which allows us to do some nice counting arguments. For example, the number of rotations of the icosahedron (triangular face, 20-sided) is 
	60, because the group of its rotations ($A_4$) acts transitively on the icosahedron. If we fix a face, there are $3$ rotations that stabilize it, and 
	the number of elements in $S$ is 20, hence we have $60$ rotations.
	
	\item \textbf{Direct Product}: If $H, K\leq G$ such that:
		
		\begin{enumerate}
		
			\item $H\cap K = \{e\}$
			
			\item $G = HK$
			
			\item Either $H, K\trianglelefteq G$ or $\forall h\in H, k\in K$, $hk = kh$
		
		\end{enumerate}
	
	Then $G \cong H\times K$.
		

\end{itemize}

\section{8/29 (Normal Subgroups, Semidirect Product)}

\begin{itemize}

	\item \textbf{Normal Subgroups}: Equivalently, we have $H\trianglelefteq G$ if:
		
		\begin{enumerate}
		
			\item $\forall g\in G, gHg^{-1} = H$
			
			\item The set of left cosets of $H$ equal the set of right cosets of $H$.
			
			\item $H$ is a union of conjugacy classes.
		
		\end{enumerate}

	\item \textbf{Cauchy's Theorem}: If $p$ is a prime and divides $|G|$, then $G$ has an element of order $p$.
	
	\textbf{Clarification of proof for $G$ abelian}: Begin with $G$ abelian and induct on $|G|$. Pick an element of prime power $q$ in $G$-- one 
	way to do this is to pick $x\in G\setminus \{e\}$ with $|x| = aq$ for $a\in\mathbb{N}$ and $q$ a prime power, then take $x^a$, which has 
	order $q$. If $p | q$ then we are done, as $q = p^\alpha$ and so $x^{p^{\alpha - 1}}$ has order $p$. Else, $G/\langle x \rangle$ has $p | 
	(G : \langle x \rangle)$, so $G / \langle x \rangle$ has an element $b$ of order $p$ by induction. We \textbf{lift} $b$ to $a\in G$ (so we find 
	an element $a\in G$ such that $b = aH$), and let $|a| = m$. As $a^m = e$, $(aH)^m = H$, and thus $|aH| = p$ divides $m$. So, $a^{m/p}$ 
	has order $p$ in $G$, and we are done.
	
	\item \textbf{Theorem}: Let $H\leq G$ be a subgroup with $(G : H) = 2$. Then, $H\trianglelefteq G$.
	
	This is because if $(G : H) = 2$, then $H$ has two left cosets and two right cosets. As $H$ is itself a left coset and a right coset and the 
	left/right cosets partition $G$, the elements not in $H$ must be both the left and the right coset, and hence $G/H = H\backslash G$.
	
	\item \textbf{Semidirect Product}: We can define the \textbf{semidirect product} of subgroups $A$ and $B$ of $G$, $G = A\rtimes B$, if:
	
	\begin{enumerate}
	
		\item $A\trianglelefteq G$ (not necessarily $B$).
		
		\item $A\cap B = \{e\}$.
		
		\item $G = AB$ (note that as $A\trianglelefteq G$, $AB = BA$).
	
	\end{enumerate}
	
	If $A$ and $B$ are arbitrary groups and $\psi : B\rightarrow Aut(A)$ is a homomorphism, then the group law on $A\times B$ given by:
	$$
		(a_1, b_1)\cdot (a_2, b_2) := (a_1\psi(b_1)(a_2), b_1b_2)
	$$
	gives a semidirect product of $A$ and $B$. 
	
	When $A\trianglelefteq G$ and $B\leq G$, then the inner semidirect product is isomorphic to the outer semidirect product given with the 
	map $\psi: B\rightarrow Aut(A)$, $b\mapsto \gamma_b$, where $\gamma_b$ is conjugation by $b$ by the correspondence $(a, b)\mapsto 
	ab$.
	
	Note that the \textbf{number of semidirect products of A with B} is the \textbf{number of homomorphisms $\psi : B\rightarrow Aut(A)$}
	
	\item \textbf{Short Exact Sequences}: An exact sequence is a sequence of morphisms $(f_i)$ where $f_i : A_i\rightarrow A_i + 1$ such that 
	$im(f_i) = ker(f_{i + 1})$. A short exact sequence is an exact sequence of the form:
	$$
		0\longrightarrow G' \overset{f}{\longrightarrow} G\overset{g}{\longrightarrow} G''\longrightarrow 0
	$$
	In this case, $f$ is injective, $g$ is surjective, and $G/G'\cong G''$ by the first isomorphism theorem.
	
	Given a sequence of the above form, the \textbf{extension problem} is to find all groups $G$ that make the sequence exact. Note that 
	\textbf{in general it is not the case that either $G\cong G'\times G''$ or $G\cong G'\rtimes G''$}.
	
	\item \textbf{Automorphisms of $\mathbb Z/n\mathbb Z$ and $\mathbb Z$}: This group has automorphism group:
	$$
		Aut(\mathbb Z/n\mathbb Z) \cong (\mathbb Z/n\mathbb Z)^*
	$$
	where the star indicates these are the elements coprime to $n$. This group has order $\phi(n)$, where $\phi$ is the totient function. This is 
	because any automorphism is completely determined by its action on a generator of $\mathbb Z/n\mathbb Z$, and so we have $\phi(n)$ 
	automorphisms. In the case where $n = p$ is prime, we have:
	$$
		Aut(\mathbb Z/p\mathbb Z) \cong \mathbb Z / (p - 1)\mathbb Z
	$$
	For the group $(\mathbb Z, +)$, we have:
	$$
		Aut(\mathbb Z) \cong \{\pm 1\}
	$$
	as there are only two automorphisms, the identity, and the one sending $1\mapsto -1$ ($1$ and $-1$ are the generators of $\mathbb Z$).
	
	\item \textbf{Groups of order 6}: There are two groups of order 6, $\mathbb{Z}/6\mathbb{Z}$ and $S_3$.
	
	If $G$ has an element of order 6, $G \cong \mathbb Z/6\mathbb Z$, so suppose not. $G$ has an element $y$ of order 2 and $x$ of order 3 
	by Cauchy. $H := \langle x\rangle$ has index 2 in $G$ and is hence normal. Let $K := \langle y \rangle$. We will show $G\cong H\rtimes 
	K$. Every nonidentity element of $H$ has order $3$, so $y\notin H$, and $G / H = \{H, yH\}$. For every $a\in G$, 
	either $a\in H$, or $aH = yH$, in which case $a = yh$, $h\in H$, which imples $a\in KH = HK$. Thus $G = HK$, and as $H \cap K = \{e\}$ 
	(as they are generated by elements of coprime order), $G\cong H\rtimes K \cong \mathbb Z/3\mathbb Z\rtimes\mathbb Z / 2\mathbb Z$. This is 
	characterized by the homomorphisms $\mathbb Z / 2\mathbb Z\rightarrow Aut(\mathbb Z/3\mathbb Z)\cong \mathbb Z/2\mathbb Z$. The trivial 
	homomorphism gives the direct product which is isomorphic to $\mathbb Z/6\mathbb Z$ as $gcd(2, 3) = 1$, and the nontrivial one sends $1$ 
	(where $0$ is the identity) to the automorphism $x\mapsto x^{-1}$), and this yields a group isomorphic to $S_3$.
	
	\item \textbf{Groups of order 8}: The groups of order 8 are $\mathbb Z/8\mathbb Z$, $(\mathbb Z/2\mathbb Z)^3$, $Q$ (quaternion group), 
	and $D_8$.
	
	Suppose $|G| = 8 = 2^3$. $G$ has an element of order 2 by Cauchy. If every element of $G\setminus \{e\}$ is order 2, we can view $G$ as 
	a vector space over $F_2$ and hence $G\cong (\mathbb Z/2\mathbb Z)^3$. Another way to view this is to pick $x\in G\setminus\{1\}$. Then, 
	$|x| = 2$, and pick $y\in G\setminus\langle x\rangle$. $\langle x\rangle\cap\langle y\rangle$ is trivial, so $\langle x\rangle\langle y\rangle
	\cong \langle x\rangle\times\langle y\rangle$. Pick $z\in G\setminus\langle x\rangle\langle y\rangle$, and $\langle z\rangle\cap\langle x
	\rangle\langle y\rangle = \{e\}$, so $G\cong \langle x\rangle\times \langle y\rangle\times\langle z\rangle\cong (\mathbb Z/2\mathbb Z)^3$.
	
	Now suppose every non-identity element in $G$ does not have order 2. If there is an element of order $8$, then $G\cong\mathbb Z /8 
	\mathbb Z$, so suppose not, and $G$ has nonidentity elements of order $2$ or $4$. Let $g\in G$ have order $4$. Then, $H := \langle g
	\rangle\trianglelefteq G$ as it has index 2.n Pick $x\in G$ such that $xH\neq H$. We have $xHx^{-1} = H$, so $xgx^-1 = g'$ with $g'\in H$. 
	Conjugation by $x$ is an automorphism, so it preserves order, thus $xgx^{-1} = g$ or $xgx^{-1} = g^3$. Also, $(xH)^2 = H$, so $x^2\in H$.
	\textbf{If $x^2 = g^3$, then change $g$ to $g^-1$} and carry out the same argument. We then have a table of presentations and can deduce 
	the isomorphisms.

\end{itemize}

\section{8/31 (Orbits, Quaternions)}

\begin{itemize}

	\item \textbf{Counting Rooks}: A chessboard has 8 columns by 8 rows.
	
	Q. How many ways are there to arrange 8 non-attacking rooks on a chessboard?
	
	A. For the first rook, you have the choice of 8 rows/cols. When you put it down, you eliminate that row and column, so you have 7 options. 
	This continues, so the answer is $8!$. Let the set of all non-attacking configurations be $A$, so $|A| = 8!$
	
	Q. How many ways are there up to symmetry? (What this means is that if one configuration differs from another up to a rotation/flip or 
	composition of the two, they are equivalent)
	
	A. The idea is to act the group $D_8$ on the set $A$. If $a\in A$ is a configuration, all the symmetries of $a$ are given the set $D_8\cdot a$, 
	which is the orbit of $a$ under $D_8$. So, to find all the configurations up to symmetry, we must count the number of orbits with Burnside's 
	Lemma; we compute $f(g)$, the number of fixed elements of $A$, for each $g\in G$. See the notes for how to do this; we end up with $5282 
	$ ways.
	
	\item \textbf{Quaternions}:
	
		\begin{itemize}
		
			\item The quaternions $\mathbb H$ "extend" $\mathbb C$. We can think of $\mathbb H$ as having the base set of $\mathbb 
			R^4$ and inheriting the addition and scalar multiplication, but having a different multiplication structure given by quaternion 
			multiplication. A general quaternion $z\in \mathbb H$ is of the form:
			$$
				z = a + bI + cJ + dK
			$$
			where $I, J,$ and $K$ obey the relations:
			$$
				IJ = K, JK = I, KI = J, I^2 = J^2 = K^2 = 1
			$$
			
			\item We can embed the quaternions into $M_{2x2}(\mathbb C)$ by identifying:
			$$
				1 = 
				\begin{pmatrix}
					1 & 0 \\ 0 & 1 \\
				\end{pmatrix}
				\;\;\;\;\;\;\;\;
				I = 
				\begin{pmatrix}
					i & 0 \\ 0 & -i \\
				\end{pmatrix}
				\;\;\;\;\;\;\;\;
				J = 
				\begin{pmatrix}
					0 & i \\ i & 0 \\
				\end{pmatrix}
				\;\;\;\;\;\;\;\;
				K = 
				\begin{pmatrix}
					0 & -1 \\ 1 & 0 \\
				\end{pmatrix}
			$$
			These are almost the Pauli matrices; notice that each Pauli matrix is just $i$ times a quaternion. You can see that as each of 
			these matrices is invertible, every quaternion has an inverse.
			
			\item We define \textbf{quaternion conjugation} as:
			$$
				\bar z := a - bI - cJ - dK
			$$
			You can then show from the relations that:
			$$
				z\bar z = a^2 + b^2 + c^2 + d^2
			$$
			This gives an easy way to show that every nonzero quaternion $z$ has an inverse $z^{-1} = \frac{\bar z}{a^2 + b^2 + c^2 + d^2}$, 
			and so $(\mathbb H^*, \times)$ forms a group. Likewise, defining the norm to be:
			$$
				|z| := z\bar z
			$$
			we see that 
			$$
				\psi : \mathbb H^*\rightarrow\mathbb R^*, z\mapsto |z|
			$$
			is a homomorphism from $\mathbb H^*$ to $\mathbb R^*$. Furthermore, the kernel of $\psi$ is:
			$$
				S^3 := ker(\psi) = \{a + bI + cJ + dK\in\mathbb H^*\}
			$$
			This is the \textbf{3-sphere}, which is the unit sphere in 4-dimensional Euclidean space.
			
			\item Quaternion rotations. \textbf{TODO}
		
		\end{itemize}
	
	\item \textbf{Burnside's Lemma}: Let a group $G$ act on a set $S$, and for each $g\in G$, let $f(g)$ be the number of elements of $S$ fixed 
	by $g$. Then:
	$$
		\#(orbits) = \frac{1}{|G|}\sum_{g\in G}f(g)
	$$
	
	\item \textbf{The Class Equation}: Let $G$ act on a group $S$. Let $\{s_i\}$ be representatives of the distinct orbits of $S$ under $G$. Then:
	$$
		|S| = \sum_i(G:G_{s_i})
	$$
	
	\item \textbf{Center of p-groups}: Any group of order $p^n$ with $p$ prime and $n\in\mathbb N$ has a nontrivial center.
	
	\item \textbf{Groups of order $p^2$}: Let $|G| = p^2$, for some prime $p$. Then, either $G\cong\mathbb Z/p^2\mathbb Z$ or 
	$G\cong(\mathbb Z/p\mathbb Z)^2$.
	
	\item \textbf{Nilpotent Groups}: A group $G$ is nilpotent if it can be reduced to $\{e\}$ by repeatedly quotienting out the center.
	
	All groups of order $p^n$ are nilpotent, and any finite nilpotent group is the product of p-groups.
	
	\item \textbf{Groups of order 2p}: If $|G|$ is a group of order $2p$ for some prime $p$, then $G$ is either cyclic or dihedral.
	
	By Cauchy, we can find an element $h\in G$ of order $p$ and $k\in G$ of order 2; let $H = \langle h\rangle$ and $K = \langle k\rangle$. Then, 
	$H\trianglelefteq G$ as $(G : H) = 2$, and $H\cap K = \{e\}$, so $HK = G$ and thus $G = H\rtimes K$. This semidirect product is characterized 
	by the homomorphisms $\psi : K\rightarrow Aut(H)$. As $H\cong \mathbb Z/p\mathbb Z$, $Aut(H) \cong (\mathbb Z / p\mathbb Z)^*$, which 
	is cyclic. There are only two elements of order 2 in $(\mathbb Z / p\mathbb Z)^*$, so we get two homomorphisms; the trivial one and the one 
	sending $\bar 2\mapsto -id$. The trivial one makes $H\rtimes K\cong H\times K\cong \mathbb Z/2p\mathbb Z$. The other homomorphism 
	makes $G\cong D_{2p}$, by the isomorphism $\phi: (h^i, k^j)\mapsto r^is^j$. To show this is a homomorphism:
	$$
		\phi((h^{i_1}, 1)(h^{i_2}, k^j)) = \phi(h^{i_1 + i_2}, k^j) = r^{i_1 + i_2}s^j = \phi(h^{i_1}, 1)\phi(h^{i_2}, k^j)
	$$
	$$
		\phi((h^{i_1}, k)(h^{i_2}, k^j)) = \phi(h^{i_1 - i_2}, k^{j + 1}) = r^{i_1 - i_2}ss^j = \phi(h^{i_1}, k)\phi(h^{i_2}, k^j)
	$$
	This is obviously a surjection and an injection as the generators have the same orders.

\end{itemize}

\section{9/5 (Sylow, Abelian)}

\begin{itemize}

	\item \textbf{Sylow Theorems}: Let $G$ be a finite group. A $p$-Sylow subgroup of $G$ is a subgroup of order $p^n$, where $p^n$ divides 
	$|G|$ but $p^{n + 1}$ does not divide $|G|$. Suppose $|G| = p^nm$, with $gcd(p^n, m) = 1$. Let $n_p$ be the number of p-Sylow subgroups 
	of $G$.
	
		\begin{enumerate}
		
			\item p-Sylow subgroups of $G$ exist, i.e. $n_p \neq 0$.
			
			\item $n_p \equiv 1 \mod p$, and $n_p | m$.
			
			\item Any p-subgroup of $G$ is contained within a p-Sylow subgroup.
			
			\item All the p-Sylow subgroups of $G$ are conjugate, and every conjugate of a p-Sylow subgroup is a p-Sylow.
			
			\item Let $P$ be a p-Sylow. Then $P\trianglelefteq G$ iff $n_p = 1$.
		
		\end{enumerate}
		
	\item \textbf{Solvable Groups}: A group $G$ is said to be \textbf{solvable} if either it is cyclic or it has a normal subgroup $N \trianglelefteq 
	G$ with $N$ and $G/N$ solvable.
	
	If $G$ has no normal subgroups, $G$ is called \textbf{simple}. The \textbf{Jordan Holder Theorem} states the choice of simple groups 
	in the chain does not depend on the choice of splitting; essentially, if we have two composition series for a group, then they are 
	equivalent.
	
	We can show that $A_5$ is simple by considering the rotations of the icosahedron, which is the group $A_5$. The order of the conjugacy 
	classes are $1, 12, 12, 15, 20$, and the only way that we can add these up to divide $60$ is $1$ or $60$, thus any union of conjugacy of 
	conjugacy classes must be trivial or the group itself.
	
	\item \textbf{Groups of order $pq$, $p < q$}: We have a normal subgroup of order $q$ by Cauchy, $H = \langle h\rangle\trianglelefteq G$ as $(G : 
	H) = p$. We also have a subgroup $K = \langle k\rangle\leq G$ with order $p$. It is easy to show $G = H\rtimes K$, so we can classify $G$ by 
	homomorphisms $\psi : K\rightarrow Aut(H)$. As $H\cong \mathbb Z/q\mathbb Z$ and $K\cong\mathbb Z/p\mathbb Z$, we need to find all 
	homomorphisms:
	$$
		\psi: \mathbb Z/p\mathbb Z\rightarrow Aut(\mathbb Z/q\mathbb Z)\cong (\mathbb Z/q\mathbb Z)^*
	$$
	$\mathbb Z/q\mathbb Z$ is cyclic of order $q - 1$, so if $p$ does not divide $q - 1$, we have only the trivial homomorphism and thus:
	$$
		p\not | \;(q - 1)\implies G\cong \mathbb Z/pq\mathbb Z
	$$
	If not, then we can still classify the groups by looking at the semidirect product.
	
	\item \textbf{Finitely Generated Abelian Groups}: Any finitely generated abelian group is isomorphic to the direct sum:
	$$
		G\cong \mathbb Z^r \bigoplus (\mathbb Z/m_1\mathbb Z)\bigoplus (\mathbb Z/m_2\mathbb Z) \bigoplus ... \bigoplus (\mathbb Z/m_n\mathbb 
		Z)
	$$
	where $r\geq 0$ and $m_1 | m_2 | ... | m_n$.
	
	\item \textbf{Abelian Groups}: TODO

\end{itemize}

\section{9/7 (Symmetric Groups)}

\begin{itemize}

	\item The \textbf{symmetric group} on $n$ letters is the group of all permutations of $n$ points, denoted $S_n$. It has $|S_n| = n!$.
	
	\item \textbf{Alternating Group, $A_n$}: Examine the action of $S_n$ on $\{x_1, ..., x_n\}$. We define:
	$$
		\Delta(x_1, ..., x_n) := \prod_{i < j}(x_i - x_j)
	$$
	Any element of $S_n$ maps $\Delta$ to $\pm\Delta$, and we define the \textbf{sign} of a permutation $\sigma\in S_n$, $\epsilon:S_n\rightarrow 
	\{\pm 1\}$ to satisfy:
	$$
		\sigma\Delta = \epsilon(\sigma)\Delta
	$$
	We define $A_n$ to be the kernel of the sign homomorphism,
	$$
		A_n := ker(\epsilon)\subset S_n
	$$
	We see $|A_n| = \frac{n!}{2}$.
	
	\item Recall some facts about $S_n$:
		
		\begin{enumerate}
		
			\item Each element in $S_n$ can be decomposed into a product of at most $\frac{n - 1}{2}$ disjoint cycles.
			
			\item A \textbf{transposition} is a 2-cycle. Every element in $S_n$ can be decomposed into a product of transpositions (not necessarily 
			disjoint)
			
			\item The order of a $k$-cycle is $k$.
			
			\item The sign of a transposition is $-1$. The sign of a product of cycles is the product of their signs.
			
			\item Conjugation is really nice on cycles. If $\gamma\in S_n$ and $(a_1 a_2 ... a_k)$ is a $k$-cycle, then:
			$$
				\gamma (a_1 \; a_2 \; ... \; a_k) \gamma^{-1} = (\gamma(a_1) \; \gamma(a_2) \; ... \; \gamma(a_k))
			$$
			
		\end{enumerate}
	
	\item \textbf{The Platonic Solids}: These are, with the number of rotations and total symmetries:
	
	\begin{table}[H]
	\centering
	\begin{tabular}{ | c | c | c | c | c | }
		\hline
		~ & Faces & Face Shape & Rotation Group & Symmetry Group \\
		\hline
		Tetrahedron & 4 & Triangle & $A_4$ & $S_4$\\
		\hline
		Cube & 6 & Square & $S_4$ & $S_4\times (\mathbb Z/2\mathbb Z)$ \\
		\hline
		Octahedron & 8 & Triangle & $S_4$ & $S_4\times (\mathbb Z/2\mathbb Z)$ \\
		\hline
		Dodecahedron & 12 & Pentagon & $A_5$ & $A_5\times (\mathbb Z/2\mathbb Z)$ \\
		\hline
		Icosahedron & 20 & Triangle & $A_5$ & $A_5\times (\mathbb Z/2\mathbb Z)$ \\
		\hline
	\end{tabular}
	\caption{Platonic Solids}
	\end{table}
	
	Note the symmetries of the cube and octahedron are the same and the dodecahedron and the icosahedron are the same (we can embed them 
	into the same rigid object). Also recall the number of rotations is $\#(faces)\#(edges / face)$ by Burnside's Lemma. We can see the symmetry 
	group of the cube is $S_4$ because any symmetry is a unique permutation of the 4 diagonals and hence it must be a subgroup of $S_4$. 
	
	\item \textbf{Cycle Shape}: Let $\sigma\in S_n$ be decomposed into unique cycles. If $\sigma$ is the product of $n_i$ $k_i$ cycles, we say 
	the cycle shape of $\sigma$ is:
	$$
		\prod_ik_i^{n_i}
	$$
	For example, the cycle $(194)(273)(68)(0)(5)$ has cycle shape $3^22^11^2$ in $S_{10}$.
	
	\item \textbf{Conjugacy Classes}: Two elements in $S_n$ are conjugate if and only if they have the same cycle shape.
	
	Q. Given $a, b\in S_n$ with the same cycle shape, how do we find $g\in S_n$ such that $a = gbg^{-1}$?
	
	A. Line their cycles up, and take $g$ to be the permutation between the corresponding symbols in their cycle shapes, starting with $b$ and 
	bubbling up to $a$.
	
	If $\sigma\in S_n$ has cycle shape $1^{n_1}2^{n_2}3^{n_3}...$, then the order of the centralizer of $\sigma$ is:
	$$
		|C_{S_n}(\sigma)| = 1^{n_1}(n_1)!2^{n_2}(n_2)!3^{n_3}(n_3)!...
	$$
	
	This enables us to find the conjugacy classes of $S_n$. We do an example with $S_4$:
	
	\begin{table}[H]
	\centering
		\begin{tabular}{| c | c | c | c | c |}
			\hline
			Partitions of 4 & Cycle shape & size(centralizer) & size(conjugacy class) & Rotation of cube \\
			\hline
			$1 + 1 + 1 + 1$ & $1^4$ & $4! = 24$ & $24 / 24 = 1$ & id \\
			\hline
			$2 + 1 + 1$ & $2^11^2$ & $2^11!1^22! = 4$ & $24 / 4 = 6$ & rotation by $\pi$ \\
			\hline
			$3 + 1$ & $3^11^1$ & $3$ & 24 / 3 = 8 & rotation by $2\pi / 3$ \\
			\hline
			$2 + 2$ & $2^2$ & $8$ & $24 / 3 = 8$ & rotation by $\pi$ \\
			\hline
			$4$ & $4^1$ & $4$ & $24 / 4 = 6$ & rotation by $\pi / 2$ \\
			\hline
		\end{tabular}
	\caption{Conjugacy classes of $S_4$ with geometric interpretation on cube.}
	\end{table}
	
	\item Problem: All normal subgroups of $S_n$: Obvious ones: $1, A_n, S_n$. Are there any others? Look for homomorphisms.
	
	$S_4$ is the group of symmetries of the cube, so it acts on the set of lines through opposite faces of the cube. There are $3$ of them, so we get 
	a nontrivial homomorphism $S_4\rightarrow S_3$ with the kernel being the identity and rotations by $\pi$. This has order $4$, and is thus a 
	normal subgroup of $S_4$ of order $4$. This generalizes, giving surjective homomorphisms from $S_2\rightarrow S_1$, $S_3\rightarrow S_2$, 
	and $S_4\rightarrow S_3$. 
	
	It does not work with $S_5\rightarrow S_4$ because $A_5\trianglelefteq S_5$ is simple. If $N\trianglelefteq S_5$, then $N\cap A_5\trianglelefteq 
	S_5$, so $N\cap A_5\trianglelefteq A_5$ and this means $N = 1 $ or $A_5$ or $S_5$, so \textbf{no} epimorphism $S_5\rightarrow S_4$.
	
	\item \textbf{Simplicity of $A_n$}: For $n \geq 5$, $A_n$ is simple.
	
	\textbf{TODO} read and understand a proof of this.
	
	\item Groups of order 120 containing $A_5$ and $\mathbb Z/2\mathbb Z$. We have:
	
		\begin{enumerate}
		
			\item $A_5\times \mathbb Z/2\mathbb Z$: Symmetries of the dodecahedron and icosahedron.
			
			\item $S_5$, subgroup $A_5$, quotient $S_5/A_5\cong\mathbb Z/2\mathbb Z$ ($S_5 = A_5\rtimes \mathbb Z/2\mathbb Z$).
		
			\item Binary icosahedral group: Use the homomorphism $S^3\rightarrow SO_3(\mathbb R)$ to lift $A_5$ (rotations of ico/dodeca, 
			so subgroup of $SO_3(\mathbb R)$) to $S^3$ for a group of twice the order. If $G$ is this group,the \textbf{Poincare 3-sphere} is the 
			quotient $S^3/G$.
		
		\end{enumerate}

	\item \textbf{Inner and Outer Automorphisms}: An \textbf{inner automorphism} of a group $G$ is of the form $x\mapsto gxg^{-1}$. We have the 
	exact sequence:
	$$
		1\rightarrow Z(G)\rightarrow Inn(G)\rightarrow Aut(G)\rightarrow Out(G)\rightarrow 1
	$$
	The map $Z(G)\rightarrow Inn(G)$, $g\mapsto\gamma_g$ is exact because it maps to the trivial automorphism, as conjugation by an element in 
	the center of $G$ is the identity. The \textbf{outer automorphisms} of $G$ is the group 
	$$
		Out(G) := Aut(G) / Inn(G)
	$$
	Except for $n = 6$, we have $Aut(S_n) \cong S_n \cong Aut(A_n)$, and all these automorphisms are inner. $S_6$ is really weird. $S_5$ has 
	a subgroup of order $20$ (index 6), so this gives us a homomorphism $S_5\rightarrow S_6$. $S_6$ has 12 subgroups isomorphic to $S_5$, 
	not 6 as one might expect. The "weird" ones (not $S_5$ or conjugates of $S_5$) give non-inner automorphisms.

\end{itemize}
	
\section{9/12 (Categories)}
	
\begin{itemize}

	\item \textbf{Definitions}: A \textbf{category} $C$ consists of a set of \textbf{objects}, $Obj(C)$, and given two objects $A, B\in Obj(C)$, a set 
	$Mor(A, B)$, called the set of \textbf{morphisms} of $A$ into $B$. For $A, B, D\in Obj(C)$, we have a law of composition, i.e. a map:
	$$
		Mor(B, D)\times Mor(A, B) \rightarrow Mor(A, D)
	$$
	
	\begin{enumerate}
	
		\item $Mor(A, B)$ and $Mor(A', B')$ are disjoint unless $A = A'$ and $B = B'$.
		
		\item For each object $A\in Obj(C)$ there is a morphism $id_A\in Mor(A, A)$ that is a right identity on $Mor(A, B)$ and a left identity on 
		$Mor(B, A)$.
		
		\item The law of composition is associative.
	
	\end{enumerate}
	
	\item Examples:
	
	\begin{itemize}
	
		\item A category which has a single object and whose morphisms are the elements of a fixed group $G$. Composition in this category is the 
		group product.
		
		\item Posets (partially ordered sets with $\leq$). We can let the objects of $C$ be the elements of the poset, and $Mor(a, b)$ have 1 element 
		if $a\leq b$ and no elements if not.
		
		\item $Grp$ is the category of Groups, $Ab$ is the category of Abelian Groups, and $Set$ is the category of Sets.
	
	\end{itemize}
	
	\item \textbf{Functors}: "Maps" between different categories. A \textbf{covariant functor} $F$ from a category $C$ to a category $D$ consists of a 
	a map from $Obj(C)\rightarrow Obj(D)$ and a map $Mor(C)\rightarrow Mor(D)$ such that:
		\begin{enumerate}
		
			\item For each object $A\in Obj(C)$, $F(id_A) = id_{F(A)}$
			
			\item $F(f\circ g) = F(f)\circ F(g)$, i.e. F preserves arrows
		
		\end{enumerate}
		
		A \textbf{contravariant functor} is like a covariant functor, but $F(f\circ g) = F(g)\circ F(f)$, i.e. it reverses all the arrows.
	
	\item Examples of Functors:
	
		\begin{enumerate}
		
			\item Forgetful functor: Map from an algebraic object to $Set$ that forgets about the structure. This is covariant.
			
			\item Homology groups $H_i$ provide a functor from topological spaces to abelian groups.
			
			\item Abelianization of a group $G$. Let $H$ be the commutator subgroup, i.e. $H = \langle\{xyx^{-1}y^{-1} \in G : x, y\in G\}\rangle$. 
			The quotient $G_{ab} := G/H$ is an Abelian group, and there is a functor $F$ taking $G$ to $G_{ab}$. Given a homomorphism $f : G
			\rightarrow G'$, we have an induced homomorphism $f_{ab} : G_{ab}\rightarrow G'_{ab}$. This is because if $G_C$ is the commutator 
			subgroup of $G$, then $G_C\leq ker(\pi_{G'}\circ f)$, and so the homomorphism $f$ factors through the quotient $G/G_C = G_{ab}$ 
			uniquely, giving a unique homomorphism $f_{ab} : G_{ab}\rightarrow G'_{ab}$.
			
			\item The free abelian group on a set $S$. This is a functor $F_{ab}(\cdot) : Set\rightarrow Ab$ which sends each set $S$ to 
			$F_{ab}(S)$ and each set map $\lambda : S\rightarrow S'$ to a map $F(\lambda) : F(S)\rightarrow F(S')$. 
			
			\item Category has one object (single point) with morphisms as members of a group $G$. We define a functor $F: C\rightarrow Set$ 
			where $F($point$) = $some set $S$ and $F(g)$ is a function from $S$ to $S$; then $g\cdot s := F(g)(s)$ is an action on $S$, as 
			$1\cdot s = F(1)(s) = id(s) = s$ and $g\cdot h\cdot s = g\cdot F(h)(s) = F(g)(F(h)(s)) = F(g)\circ F(h) (s) = F(gh)(s) = (gh)\cdot s$.
			
			\item Dual of a vector space. The functor $F(V) := V^*$ and for a morphism $f : V\rightarrow W$ gives a morphism $F(f) : F(W)
			\rightarrow F(V)$ is contravariant.
		
		\end{enumerate}
	
	\item $Hom(\cdot, \cdot)$ is a \textbf{bifunctor}, and it is covariant in one argument and contravariant in the other argument. If we fix $B$, then 
	$Hom(\cdot, B)$ is contravariant. This is because if we have $f : A_1\rightarrow A_2$, then a map $\phi\in Hom(A_2, B)$ has $\phi\circ f : A_1 
	\rightarrow B\in Hom(A_1, B)$ and so we have a natural map $Hom(A_2, B)\rightarrow Hom(A_1, B)$. Likewise, $Hom(A, \cdot)$ is covariant.
	
	\item \textbf{Natural transformations}: TODO
	
	\item \textbf{Universal objects}: Universal property means that any other object with this property factors uniquely through the object. Check notes 
	for examples of this with products, coproducts, pullbacks, pushouts, and equalizers. Universal objects are unique up to isomorphism, as a 
	universal object allows a unique morphism on itself which must be $id$, and it must admit a map into and out of the universal object.
	
	\item \textbf{Equilaser}: The equaliser of two maps $f, g : A\rightarrow B$ is an object $X$ equipped with a morphism $h : X\rightarrow A$ such 
	that $f\circ h= g\circ f$. Furthermore, if $\iota : Y\rightarrow A$ satisfies $f\circ\iota = g\circ\iota$, then this diagram factors through $X$, i.e. there 
	exists a unique morphism $\phi : Y\rightarrow X$ such that $\iota = h\circ\phi$.
	
	\item A \textbf{final object} in a category is universally repelling, i.e. it admits a unique morphism out of it into any other object in the category. 
	Examples are $\{e\}\in Obj(Grp)$ and the empty set in $Set$.
	
	\item \textbf{Limits}: A limit is an object $X$ with morphisms making an arbitrary diagram commute, and it is universal with respect to this property, 
	i.e. any other object that makes the diagram commute factors through $X$. Products are limits of the diagram $A, B$ with no arrows, pullouts 
	are limits of the diagram with morphisms $f : X\rightarrow Z$ and $g : Y\rightarrow Z$, and equalisers are limits of the diagram with $f : A
	\rightarrow B$ and $g : A\rightarrow B$. 
	
	\item \textbf{Duality}: Every construction has a co-construction which essentially just reverses all the arrows.
	
	\item Examples of Universal objects: Note how in general, \textbf{the pullback is some subobject of the product} and the \textbf{pushout is a 
	quotient of the coproduct}.
	
		\begin{itemize}
		
			\item Groups:
			
				\begin{itemize}
					
					\item Product: Direct Product with projection maps.
					
					\item Coproduct: Free Product with inclusion maps.
					
					\item Pullback (Fiber Product): Let $f: X\rightarrow Z$ and $g: Y\rightarrow Z$ be homomorphisms. The pullback is the 
					subgroup of the direct product $\{(x, y)\in X\times Y : f(x) = g(y)\}$.
					
					\item Pushout (Fiber Coproduct): Free Product with Amalgamation. This is the free product quotiented by the subgroup 
					generated by objects of the form $\{\iota_1(f(z))\iota_2(g(z))^{-1})\}$
					
				\end{itemize}
			
			\item Abelian Groups:
			
				\begin{itemize}
				
					\item Product: Direct product with projection maps.
					
					\item Coproduct: Direct sum with inclusion maps.
					
					\item Pullback: Let $f: X\rightarrow Z$ and $g: Y\rightarrow Z$ be homomorphisms. The pullback is the subgroup of the 
					direct product $\{(x, y)\in X\times Y : f(x) = g(y)\}$.
					
					\item Pushout: Let $f: Z\rightarrow X$ and $g: Z\rightarrow Y$ be homomorphisms. The pushout is the direct sum $X
					\bigoplus Y$ quotiented by elements of form $(f(z), -g(z))$.
				
				\end{itemize}
			
			\item Set:
			
				\begin{itemize}
				
					\item Product: Cartesian Product with projection maps.
					
					\item Coproduct: Disjoint union with inclusion maps.
					
					\item Pullback: Let $f: X\rightarrow Z$ and $g: Y\rightarrow Z$ be maps. The pullback is the subset of the Cartesian product 
					$\{(x, y)\in X\times Y : f(x) = g(y)\}$ .
					
					\item Pushout: Disjoint union quotiented by some equivalence relation.
				
				\end{itemize}
		
		\end{itemize}

\end{itemize}

\section{9/14 (Free Groups)}

\begin{itemize}

	\item \textbf{Free Abelian Group}: A free abelian group $F_{ab}(S)$ on a set $S$ can be thought of as the set of all commuting words of elements 
	of $S$. If $|S| = n$, then $F_{ab}(S)\cong \mathbb Z^n$, and $n$ is called the \textbf{rank} of $F_{ab}(S)$. The rank of any subgroup of $F_{ab}
	(S)$ is less than or equal to the rank of $F_{ab}(S)$. A \textbf{basis} of an abelian group $G$ is a subset $\{e_i\}_{i\in I}\subset G$ such that any 
	element of $G$ has a unique expression of a $\mathbb Z$-linear combination of the $e_i$. If an abelian group has a basis, it is called \textbf{free}.
	
	Furthermore, we have a universal property. Let $\iota : S\rightarrow F_{ab}(S)$ be the canonical inclusion map. If $\phi : S\rightarrow G$ is a map 
	from $S$ to \textbf{any abelian group $G$}, then $\phi$ factors through $F_{ab}(S)$, i.e. there exists a unique group homomorphism $\Phi : 
	F_{ab}(S)\rightarrow G$ such that $\phi = \Phi\circ\iota$. If $\lambda : S\rightarrow S'$ is any set map, then we get a unique group homomorphism 
	$\bar\lambda : F_{ab}(S)\rightarrow F_{ab}(S')$ where $\bar\lambda\circ\iota = \iota'\circ\lambda$.
	
	\item The \textbf{free group} on $\{g_1, ..., g_n\}$ is the "universal group" generated by these letters. It is the coproduct of $\mathbb Z$.
	
	\item Ex: Free group on $\{a, b, c\}$. It is the set of all reduced words on $\{1, a, b, c, a^{-1}, b^{-1}, c^{-1}\}$ where a reduced word takes out 
	any $aa^{-1}$, etc. and the operation is concatenation. Different reduced words correspond to different elements of the free group. We can show 
	this by considering a permutation argument if the size of the reduced word is $n$ on $S_n$.
	
	\item \textbf{Universal Property}: Same as the universal property on free abelian groups, but with an arbitrary $G$ rather than an abelian $G$. 
	
	\item \textbf{Subgroups of Free Groups}: Suppose that $G\leq F$ for a free group $F$ with $(F : G) = n$ finite. We can draw a picture with dots 
	representing the cosets and the action of the generators on the cosets as the edges. This graph must be connected because each coset must 
	be able to be multiplied into another coset, so there must be a generator that does this. Thus, \textbf{we get a correspondence between 
	subgroups of index n and connected graphs on n points with $g_1, ..., g_n$ colored cycles}.
	
	\item \textbf{Fundamental Group}: First, we need homotopy classes. Essentially, two graphs are in the same homotopy class if we can flatten out 
	a loop. The set of homotopy classes is a group, and the fundamental group is the group of homotopy classes of loops from the base point to 
	itself. 
	
	\item \textbf{Finding generators of subgroups}: Suppose $(F : G) = n$. Draw out the graph of $F/G$, where we have $n$ vertices and the number 
	of cycles is the number of generators of $F$. Then:
		
		\begin{enumerate}
		
			\item Contract the edges of the graph with distinct vertices down to a point. 
			
			\item Use these contractions to draw a maximal tree where you contracted the edges. 
			
			\item To find the generators, go along the edges of the maximal tree until you find a loop, do the loop, then go back along the edge. 
		
		\end{enumerate}
	
	We can see that a subgroup of infinite index can have infinite generators (even if $F$ has 2 generators), so subgroups of free groups can have 
	a larger rank than the original group.
	
	\item Free groups are \textbf{residually finite}: Nontrivial elements can be detected by maps into finite groups. More precisely, $G$ is residually 
	finite if for every nontrivial element $g\in G$, there is a map $f$ from $G$ into a finite group such $f(g) \neq 1$. This means that for any nontrivial 
	element of a free group, there is a subgroup of finite index not containing this element, for the kernel of this map $f$ gives the desired subgroup.
	
	\item \textbf{Number of Generators}: If $G\leq F$ has index $n$ and $G$ has $m$ generators, then $F$ has $n(m - 1) + 1$ generators.
	
	This follows because if $G$ has $m$ generators, then there are $mn$ edges on the graph (as each coset has an edge into and out of it) and 
	there are $n$ vertices for an index $n$ subgroup, so the Euler characteristic of the graph is $\chi = V - E = n - mn$. The number of generators 
	is 1 minus the Euler characteristic, so the number of generators is $1 - n(1 - m) = n(m - 1) + 1$.

\end{itemize}

\end{document}  