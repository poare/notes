\documentclass[12pt, oneside]{article}   	% use "amsart" instead of "article" for AMSLaTeX format
\usepackage[top=.5in, bottom=.5in, left = .5in, right=.5in, headheight=14.5pt, includeheadfoot]{geometry}
%\usepackage[margin = 1in]{geometry}                		% See geometry.pdf to learn the layout options. There are lots.
\geometry{letterpaper}                   		% ... or a4paper or a5paper or ... 
%\geometry{landscape}                		% Activate for rotated page geometry
%\usepackage[parfill]{parskip}    		% Activate to begin paragraphs with an empty line rather than an indent
\usepackage{graphicx}				% Use pdf, png, jpg, or eps§ with pdflatex; use eps in DVI mode
								% TeX will automatically convert eps --> pdf in pdflatex		
\usepackage{amssymb}
\usepackage{amsmath}
\usepackage[shortlabels]{enumitem}
\setlist{leftmargin=5.5mm}
\usepackage{float}
\usepackage{tikz-cd}
\usepackage{subcaption}
\usepackage{slashed}
\usepackage{mathrsfs}

% Packages from other template
\usepackage[final]{microtype}
\usepackage[USenglish]{babel}
\usepackage{hyperref}
\usepackage[T1]{fontenc}

%\usepackage{titlesec}
%\titlespacing{\section}{0pt}{12pt}{4pt}

\usepackage[compat=1.0.0]{tikz-feynman}

\usepackage{bm}
\usepackage{bbm}
\usepackage{bbold}

\usepackage{simpler-wick}

\usepackage{amsthm}
\theoremstyle{definition}
\newtheorem{definition}{Definition}[section]
\newtheorem{theorem}{Theorem}[section]
\newtheorem{corollary}{Corollary}[theorem]
\newtheorem{lemma}[theorem]{Lemma}

\newcommand{\N}{\mathbb{N}}
\newcommand{\R}{\mathbb{R}}
\newcommand{\Z}{\mathbb{Z}}
\newcommand{\Q}{\mathbb{Q}}

\newcommand{\RI}{\mathrm{RI}}
\newcommand{\Tr}{\text{Tr}}
\newcommand{\TrC}{\text{Tr}_{\text{C}}}
\newcommand{\TrD}{\text{Tr}_{\text{D}}}

\usepackage{simpler-wick}
\usepackage[compat=1.0.0]{tikz-feynman}   %note you need to compile this in LuaLaTeX for diagrams to render correctly

\usepackage{parskip}
    \setlength{\parindent}{0in}
    %\setlength{\parindent}{.25in}

\usepackage{fancyhdr}
    \renewcommand{\headrulewidth}{.85pt}
    \renewcommand{\footrulewidth}{.6pt}
    \pagestyle{fancy}
    \renewcommand{\sectionmark}[1]{\markboth{#1}{}}
    \fancyhf{}
    \fancyhead[R]{Patrick Oare}
    \fancyhead[C]{\fontsize{14}{16.8}\textbf{Recitation 6: Perturbation Theory}}
    \fancyhead[L]{8.323 S2022}
    \fancyfoot[C]{\vspace*{.15in}\thepage}

% PSet Sections
\iffalse
\usepackage[explicit]{titlesec}
    \titleformat{\section}{\vspace*{0pt}\fontsize{16}{19.2}\selectfont}{}{0in}{\textbf{#1}{\hrule height .7pt width .75\textwidth}}
    \titlespacing{\section}{.35in}{.5in}{\parskip}
    \titleformat{\subsection}{\fontsize{14}{16.8}\selectfont}{}{.5in}{\textbf{\uline{#1}}}
    \titlespacing{\subsection}{0pt}{.5in}{\parskip}
\fi

% make arrow superscripts
\DeclareFontFamily{OMS}{oasy}{\skewchar\font48 }
\DeclareFontShape{OMS}{oasy}{m}{n}{%
         <-5.5> oasy5     <5.5-6.5> oasy6
      <6.5-7.5> oasy7     <7.5-8.5> oasy8
      <8.5-9.5> oasy9     <9.5->  oasy10
      }{}
\DeclareFontShape{OMS}{oasy}{b}{n}{%
       <-6> oabsy5
      <6-8> oabsy7
      <8->  oabsy10
      }{}
\DeclareSymbolFont{oasy}{OMS}{oasy}{m}{n}
\SetSymbolFont{oasy}{bold}{OMS}{oasy}{b}{n}

\DeclareMathSymbol{\smallleftarrow}     {\mathrel}{oasy}{"20}
\DeclareMathSymbol{\smallrightarrow}    {\mathrel}{oasy}{"21}
\DeclareMathSymbol{\smallleftrightarrow}{\mathrel}{oasy}{"24}
%\newcommand{\cev}[1]{\reflectbox{\ensuremath{\vec{\reflectbox{\ensuremath{#1}}}}}}
\newcommand{\vecc}[1]{\overset{\scriptscriptstyle\smallrightarrow}{#1}}
\newcommand{\cev}[1]{\overset{\scriptscriptstyle\smallleftarrow}{#1}}
\newcommand{\cevvec}[1]{\overset{\scriptscriptstyle\smallleftrightarrow}{#1}}

\newcommand{\dbar}{d\hspace*{-0.08em}\bar{}\hspace*{0.1em}}

% to use a box environment, use \begin{answer} and \end{answer}
\usepackage{tcolorbox}
\tcbuselibrary{theorems}
\newtcolorbox{answerbox}{sharp corners=all, colframe=black, colback=black!5!white, boxrule=1.5pt, halign=flush center, width = 1\textwidth, valign=center}
\newenvironment{answer}{\begin{center}\begin{answerbox}}{\end{answerbox}\end{center}}

\begin{document}
%\maketitle

\section*{The Feynman Calculus}

\begin{itemize}

	\item The goal for the rest of the semester will be to obtain the correlation function,
	\begin{equation}
		G_n\equiv \langle\Omega | \mathrm{T} \{ \phi(x_1) ... \phi(x_n) \} | \Omega\rangle, \label{eq:corr_fn}
	\end{equation}
	in perturbation theory to an arbitrary order. The way that we will do this is with Wick's theorem, since we know how to evaluate free theory correlation functions. The essential way to do this is to split up the action into a free piece $S_0$ and an interaction piece $S_I$,
	\begin{align}
		S = S_0 + S_I && S_0 = \int d^4 x\, (-(\partial\phi)^2 + m^2\phi^2)
	\end{align}
	The \textbf{central equation for evaluating Eq.~(\ref{eq:corr_fn}) is}:
	\begin{align}
		G_n = \frac{\langle 0 | \mathrm{T} \{ \phi(x_1) ... \phi(x_n) e^{i S_I} \} |0\rangle }{\langle 0 | \mathrm{T} \{ e^{i S_I} \} |0\rangle} && S_I = -\int dt\, H_I\label{eq:hamiltonian_framework}
	\end{align}
	where the $|0\rangle$ means that these correlators are evaluated in the \textbf{free theory}. Note that the denominator is the vacuum normalization, since:
	\begin{equation}
	\mathcal Z[0] = \int D\phi(x) e^{i S} = \int D\phi(x) e^{i S_0} \left( e^{i S_I}\right) = \langle 0 | T\{ e^{i S_I} \} 0\rangle.
	\end{equation}
	This equation allows us to evaluate a correlation function in the \textit{interacting theory} in terms of a free theory correlation function, which we can compute by Taylor expanding $e^{i S_I}$ to a given order. 
	
	\item For this recitation, we'll consider a scalar theory with a $k$-point interaction,
	\begin{equation}
		\mathcal L_I = A \phi^k,
	\end{equation}
	and do an example with $k = 3$ soon. Note that the calculation is specified by three things:
	\begin{enumerate}
		\item $k$, which specifies the interaction $\phi^k$ in the theory. This will correspond to how many legs an internal vertex can have.
		\item $n$, which specifies the number of fields in the correlation function $G_n$. This will correspond to the number of external vertices a diagram has. 
		\item $r$, which is the order in perturbation theory we're calculating to. This will correspond to the number of internal vertices in each diagram. 
	\end{enumerate}
	
	\item Methods for perturbation theory. There are a few different frameworks to compute these correlation functions in. The two we'll be interested in are the \textbf{path integral} and the \textbf{Hamiltonian evolution} frameworks. 
	\begin{enumerate}
	    	\item Hamiltonian framework: This is the framework that we've been using in class, and that Eq.~(\ref{eq:hamiltonian_framework}) describes. 
		\item Path integral: The other option to evaluate these correlation functions is the path integral. This will boil down to the same thing as the Hamiltonian framework, and it boils down to what we did last week.
	\end{enumerate}
	For both of these frameworks, perturbation theory proceeds by expanding the exponential. It's easiest to write out an explicit expression for a correlation function using the path integral,
	\begin{align} \begin{split}
			G_n &= \frac{1}{\mathcal Z_0} \int D\phi(x) e^{iS_0 + i S_I} \phi(x_1) ... \phi(x_n) \\
			&= \sum_{r = 0}^\infty \frac{1}{r!}\frac{1}{\mathcal Z_0} \int D\phi(x) e^{i S_0} \phi(x_1) ... \phi(x_n) \underbrace{\left( i A \int d^4 z_1\, \phi(z_1)^k \right) ... \left( i A \int d^4 z_r\, \phi(z_r)^k \right)}_{\textnormal{r times}} \\
			&= \sum_{r = 0}^\infty \frac{(iA)^r}{r!} \int d^4 z_1 ... \int d^4 z_r \langle 0 | T\{ \phi(x_1) ... \phi(x_n) \phi(z_1)^k ... \phi(z_r)^k \} |0\rangle
	\end{split} \end{align}
	where we used the path integral in the middle equation to put the pieces back together into a free-field correlation function. Note that \textit{you never have to evaluate a time-ordered exponential explicitly, you can use Wick's theorem}! 
	
	\item \textbf{Feynman diagrams} will give us a concrete way to evaluate this sum to a given order in $A$. the idea is that order-by-order, we'll write down a set of diagrams which correspond to terms in the Taylor expansion of $e^{i S_I}$. The whole calculation will boil down to drawing all possible diagrams which contribute to the calculation, and evaluating the diagrams and the combinatorics. 

	\item \textbf{Components of a Feynman diagram}: To evaluate the correlation function $\langle \Omega | T\{ \phi(x_1) ... \phi(x_n) \} | \Omega \rangle$ in perturbation theory, we need 3 ingredients: {\color{red} DRAW THEM}
	\begin{itemize}
		\item External vertices: these correspond to the positions $x_1, ..., x_n$ of the fields in the correlator. There are $n$ external vertices, and each external vertex can only connect to one other vertex in a diagram (an edge from $x_j$ correspond Wick contractions of $\phi(x_j)$). An external vertex evaluates to 1. 
		\item Internal vertices: An internal vertex has $k$ legs, and evaluates to $A$ (assuming na\"ive Feynman rules). Each internal vertex corresponds to a position $z$ which is integrated over (it's from a power of $S_I = \int d^4 z ...$). To compute the correlator up to order $A^r$ in perturbation theory, you need $r$ internal vertices, since that term corresponds to $\frac{1}{r!} (S_I)^r$ in the Taylor expansion of the exponential. 
		\item Edges: An edge can connect any two vertices $x$ and $y$. Each edge contributes a Feynman propagator between its two vertices, $G_F(x, y)$.
	\end{itemize}
	
	\item \textbf{Feynman rules} give us prescriptions to do all of these things. We'll talk about the coordinate-space Feynman rules during this recitation, as I don't think we'll have time to hit the momentum space ones (although momentum space is generally simpler; I don't even remember the usual way to do coordinate-space rules, I had to look them up for recitation, since I'm so used to using momentum-space rules). 
	\begin{enumerate}
		\item Draw the internal and external vertices up to a given order in perturbation theory. You'll always the same number of external vertices for a given correlator, and the number of internal vertices is the degree in perturbation theory you're expanding up to. 
		\item Connect the vertices in all (topologically distinct) ways to get the total number of diagrams. Drop diagrams which have \textbf{vacuum bubbles}: a diagram has a vacuum bubble if there is a disconnected component made entirely of internal vertices and edges. You'll show in the problem set that these are cancelled out order by order with the $1 / \mathcal Z_0$ term. 
		\item Evaluate each diagram with the prescription above, multiplying all contributions by one another. 
		\item Do the combinatorics for each diagram. Each diagram will a priori represent a lot of the same terms in the Taylor expansion (more on this in the example later), and you need to count the number of each term carefully, or you'll miss factors of 2 and 4 (we'll see this in action in the example). Divide the sum by $r!$, which you get out of the Taylor expansion of the exponential. 
	\end{enumerate}
	
	\item \textbf{Modified Feynman rules}: I'm not sure if Hong made the distinction in class, but I think it's worth going over the different ways to evaluate Feynman diagrams. The modified Feynman rules are simpler ways to evaluate the combinatorics: instead of counting all distinct diagrams / contractions explicitly, you can make a modified counting scheme that's a little bit easier to evaluate and gives you a good first guess at the combinatorics. The idea is to find a heuristic way to count the number of Wick contractions each diagram contributes. At first guess, you can permute the internal vertices of each diagram, which will give you a $r!$ contractions corresponding to the same diagram, and you can also permute the legs of an internal vertex, leading to a $k!$ factor. Since there's a $1 / r!$ term in front of the sum of all $r$th order diagrams, this will cancel, and so in essence the modified Feynman rules change the rule associated with internal vertices:
	\begin{itemize}
		\item Internal vertices: Evaluate an internal vertex as $k! A$ (this is why we conventionally write the coupling as $A / k!$, so that the vertex rule is unity). 
	\end{itemize}
	However, there are certain situations when this na\"ive counting \textit{breaks down}, and we need to multiply the diagram by a factor of $1 / S$, where $S$ is called the \textbf{symmetry factor}. 
	
	\item Symmetry factors: Rules for symmetry factors (these multiply if the diagram satisfies 2 or more of them): {\color{red} DRAW EXAMPLES}
	\begin{enumerate}
		\item Propagators starting and ending at the same point (bubbles): Factor of 2. 
		\item $j$ propagators connecting two internal vertices: Factor of $j!$. 
		\item $j$ internal vertices which are interchangeable in how they connect to the diagram: Factor of $j!$. 
	\end{enumerate}
	Once you get comfortable with modified Feynman rules and symmetry factors, you'll basically exclusively use them, and forget the original rules for counting. 
	
	\item Example: $\phi^3$ theory, with:
	\begin{equation}
		S_I = -\int d^4 x\, \frac{g}{3!} \phi(x)^3
	\end{equation}

\end{itemize}

\end{document}