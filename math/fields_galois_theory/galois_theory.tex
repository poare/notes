\documentclass[11pt, oneside]{amsart}   	% use "amsart" instead of "article" for AMSLaTeX format
\usepackage{geometry}                		% See geometry.pdf to learn the layout options. There are lots.
\geometry{letterpaper}                   		% ... or a4paper or a5paper or ... 
%\geometry{landscape}                		% Activate for rotated page geometry
%\usepackage[parfill]{parskip}    		% Activate to begin paragraphs with an empty line rather than an indent
\usepackage{graphicx}				% Use pdf, png, jpg, or eps§ with pdflatex; use eps in DVI mode
								% TeX will automatically convert eps --> pdf in pdflatex		
\usepackage{amssymb}
\usepackage{amsthm}
\usepackage{mathtools}
\usepackage{float}

\theoremstyle{definition}
\newtheorem{definition}{Definition}[section]
\newtheorem{theorem}{Theorem}[section]
\newtheorem{corollary}{Corollary}[theorem]
\newtheorem{lemma}[theorem]{Lemma}
%SetFonts

\title{Math 250A Lecture Recaps (Galois Theory)}
\author{Patrick Oare}

\begin{document}
\maketitle
%\section{}
%\subsection{}

Unless otherwise specified, let $L / K$ be a Galois extension with Galois group $G$.

\section{11/7 (Galois Extensions, Fundamental Theorem)}

\begin{itemize}

	\item \textbf{Definitions}: An extension $L / K$ is called \textbf{Galois} if it is normal and separable. We define the \textbf{Galois group} of 
	the extension $L / K$ to be $Gal(L / K) := Aut(L / K) = \{\sigma\in Aut(L) : \forall k\in K, \sigma(k) = k\}$, i.e. $Gal(L / K)$ is the group of all 
	automorphisms of $L$ that fix $K$. If $\alpha\in L$, the \textbf{conjugates} of $\alpha$ under $Gal(L / K)$ are the set $\{\sigma(\alpha) : \sigma 
	\in Gal(L / K)\}$.
	
	\item \textbf{Galois Extensions}:
		
	\begin{theorem}
	
		For a finite extension $L / K$, the following are equivalent. Let $G = Gal(L / K)$.
		
		\begin{enumerate}
		
			\item $L$ is the splitting field of a separable polynomial over $K$.
			
			\item $L / K$ is Galois.
			
			\item $[L : K] = |G|$.
			
			\item $K = L^G$ is the fixed field of in $L$ by $G$.
		
		\end{enumerate}
	
	\end{theorem}
	
	Some of these are easy: $i\implies ii$ and $iii\implies iv$. For $ii\implies iii$, suppose $L / K$ is Galois. Let $M$ be the algebraic closure of $K$. We 
	have $\leq n = [L : K]$ maps $L\rightarrow M$ extending $id |_K$. But, $L / K$ separable implies we have $n$ such maps. For if $L = K(\alpha)$, 
	then the minimal polynomial of $\alpha$ is separable and so has $n$ distinct roots, so we have exactly $n$ maps, and if $L = K(\alpha_1, ..., 
	\alpha_n)$, then we proceed as in the proof above to get $n$ maps. But, $L / K$ normal implies the image of any map $L\rightarrow M$ lies in $L$ 
	(as then $L$ is a splitting field and uniquely determined), which gives us $n$ homomorphisms extending the identity on $K$, so $[L : K] = |G|$. 
	
	For $iv\implies ii$, let $\alpha\in L$. Look at all conjugates of $\alpha$ by $G$, and call them $\alpha_1, ..., \alpha_n$ ($\alpha_1 := \alpha$). Let 
	$f(x) := \prod_{i = 1}^n(x - \alpha_i)$. $f$ is fixed by $G$ (as in applying any $\sigma\in G$ we may reindex the product), so $f$ has coefficients in 
	$L^G = K$, and $f$ is the minimal polynomial of $\alpha$ over $K$ (really, take $f$ to be a product over distinct conjugates of $\alpha$).
	$L$ is. So, for any element in $L$, the minimal polynomial over $K$ is separable and has all its roots in $L$. Now, take a basis $\omega_1, ..., 
	\omega_k$ of $L / K$, and let $p_i(x)$ be the minimal polynomial of $\omega_i$ over $K$. Then, take all repeated factors out of 
	$\prod_{i = 1}^kp_i(x)$, and call it $g$. This makes this a separable polynomial, and then $L$ is the splitting field of $g$.
	
	\item \textbf{Minimal Polynomials under Galois Conjugates}: Let $\alpha\in L$ have minimal polynomial $p\in K[X]$. Then, any conjugate of $\alpha$ by 
	$G$ has minimal polynomial $p(x)$ as well.
	
	\item \textbf{Examples of Galois Extensions}:
	
	\begin{enumerate}
	
		\item $\mathbb Q(\sqrt[3]{2}, \omega)$ for $\omega := exp(2\pi i/3)$. This is the splitting field of $x^3 - 2$ over $\mathbb Q$ and has Galois group 
		$S_3$.
		
		\item $\mathbb C / \mathbb R$ is Galois with Galois group $\mathbb Z / 2\mathbb Z$-- the nontrivial element is complex conjugation.
		
		\item $\mathbb F_{16} / \mathbb F_2$ is the splitting field of $x^{16} - x$ over $\mathbb F_2$, and we have already shown this is separable. 
		Let $\phi$ be the Frobenius element of $\mathbb F_{16}$, i.e. $\phi(x) = x^2$. Then $Gal(\mathbb F_{16} / \mathbb F_2) \cong \mathbb Z /4 
		\mathbb Z$, and generated by $\phi$.
	
	\end{enumerate}
	
	\item \textbf{Galois Groups of Finite Fields}: Let $q = p^n$ for $n >= 1$. Then, the extension $\mathbb F_q / \mathbb F_p$ is Galois, and:
	$$
		Gal(\mathbb F_q / \mathbb F_p) \cong \mathbb Z / n\mathbb Z
	$$
	
	That it is Galois follows because $F_q$ is the splitting field of separable $x^q - x$ over $\mathbb F_p$. The Galois group is generated by the 
	\textbf{Frobenius element} (the Frobenius element for an extension of finite fields $L / K$ is $x\mapsto x^{|K|}$),
	$$
		\phi(x) := x^{char(\mathbb F_q)} = x^p
	$$
	
	The order of $\phi$ is n, as clearly if $m < n$ then $\phi^m\neq id$, but $\phi^n(a) = a^{np} = a^q = a$ and so $|\phi| = n$. But, $|Gal(\mathbb F_q / 
	\mathbb F_p)| = [\mathbb F_q / \mathbb F_p] = n$, and so in fact $Gal(\mathbb F_q / \mathbb F_p) = \langle \phi\rangle$.
	
	\item \textbf{Fundamental Theorem of Galois Theory}: Let $M / K$ be a Galois extension with $G = Gal(M / K)$. We have a bijection between the 
	intermediate extensions $L$ with $K \leq L \leq M$ and the subgroups $H\leq G$ given by sending $L$ to:
	$$
		L\mapsto Gal(M / L)
	$$
	where $Gal(M / L)$ is the group of $\sigma\in G$ fixing $L$. The inverse of this sends $H$ to:
	$$
		H\mapsto M^H
	$$
	which is all elements in $M$ fixed by $H$. This bijection \textbf{reverses inclusions}, so bigger subfields correspond to smaller subgroups.

\end{itemize}

\section{11 / 9 (Computing Galois Groups, Examples)}

\begin{itemize}

	\item \textbf{Seventh root of unity}: Let $\xi := exp(2\pi i / 7)$ be the 7th root of unity. Recall $\xi^7 - 1 = (\xi - 1)(\xi^6 + \xi^5 + \xi^4 + \xi^3 + \xi^2 + \xi 
	1) = (\xi - 1)\Phi_7(\xi) = 0$, so obviously $\Phi_7(\xi) = 0$ and is irreducible. We know the roots of $\Phi_7$, so:
	$$
		\Phi_7(x) = \sum_{i = 0}^6 x^i = (x - \xi)(x - \xi^2)...(x - \xi^6)
	$$
	and therefore the extension is Galois with degree $6$. Let $G := Gal(\mathbb Q(\xi) / \mathbb Q)$, so $|G| = 6$. Note that for $\sigma\in G$, $\sigma$ 
	is completely determined by its action on $\xi$ since all roots are powers of $\xi$. $\sigma$ may send $\xi$ to other root of $\Phi_7(x)$, so $\sigma(\xi) 
	= \xi^m$ for $m = 1, 2, ..., 6$. Thus we have found the $6$ elements of the Galois group, and we find $G \cong (\mathbb Z / 7\mathbb Z)^*$. 
	
	We can use this to determine the subfields of the extension-- As $(\mathbb Z / 7\mathbb Z)^*$ is cyclic of order 6, we have unique nontrivial subgroups 
	of orders $2$ and $3$-- these are $H := \{1, 2, 4\}$ and $J := \{1, 6\}$ (note $G$ is generated by 3). We find $\mathbb Q(\xi)^H$-- this will be a degree 
	2 extension of $\mathbb Q$. An obvious element is given by taking any element and summing its conjugates which are in the subgroup. In this case, 
	we take $a := \sigma(\xi) + \sigma^2(\xi) + \sigma^4(\xi) = \xi + \xi^2 + \xi^4$, which will be fixed under $H$. Note that $a^2 + a + 2 = 0$, so:
	$$
		a = \frac{-1 + \sqrt{-7}}{2} \implies \mathbb Q(a) = \mathbb Q(\sqrt{-7})
	$$
	and so our fixed field is $\mathbb Q(\sqrt{-7})$. For the degree 3 subfield over $\mathbb Q$, we take $b := \xi + \xi^6$ and find that the subfield is 
	$\mathbb Q(\cos (2\pi / 7))$. 
	
	\item \textbf{Roots of unity}: In general, if $\xi_n$ is a primitive $n$th root of unity, then:
	$$
		Gal(\mathbb Q(\xi_n) / \mathbb Q)\cong (\mathbb Z / n\mathbb Z)^*
	$$
	
	If $\gcd(n, m) = 1$, then:
	$$
		\mathbb Q(\xi_n)\cap\mathbb Q(\xi_m) = \mathbb Q
	$$
	
	\item \textbf{Normal Extensions vs. Subgroups}: Let $L / F$ be a Galois extension. Recall that if $K$ is an intermediate field, then $L / K$ is Galois 
	(as $L$ is a splitting field of $f(x)\in F[X]\subset K[X]$). Let $G = Gal(L / F)$, and let $H = Gal(L / K)\leq G$. Then, $H\trianglelefteq G$ iff $K / F$ 
	is a normal extension. If this is the case, then $K / F$ is Galois (as it is separable since $L / F$ is separable), and:
	$$
		Gal(K / F) \cong G / H
	$$
	
	This isomorphism follows because we can define a map $G\rightarrow Gal(K / F)$, $\sigma\mapsto \sigma |_K$, which has kernel $H$. 
	
	\item \textbf{Determining Galois groups by reduction modulo $p$}: Let $p$ be prime, $f\in\mathbb Z[X]$ monic with Galois group $G$. If $\bar f(x) := 
	f(x)\mod p$ has Galois group $\bar G$, then:
	$$
		\bar G\hookrightarrow G
	$$
	and so we may identify elements of $\bar G$ as elements of $G$. Combining these with the fact that $G\leq S_n$ is powerful; it is easy to find a 
	combination of cycles which are in $\bar G$, and we may put it together to show they generate a certain unique subgroup of $S_n$.
	
	In general, if we have a degree $n$ irreducible polynomial, the Galois group acts transitively on these $n$ roots. By orbit-stabilizer, this means that 
	$|G\alpha| = n = (G : G_\alpha)\implies n$ divides $|G|$.  If this is a degree $p$ irreducible polynomial, then the Galois group of the polynomial 
	contains a $p$-cycle. Since $p$ divides $|G|$, this implies $G$ has an element of order $p$ by Cauchy, which is a $p$-cycle. This is helpful: 
	http://www.math.uconn.edu/~kconrad/blurbs/galoistheory/galoisaspermgp.pdf
	
	\item \textbf{Condition for $G = S_p$}: Let $f$ be irreducible in $\mathbb Q[X]$ with $deg(f) = p$ prime. If $f$ has precisely two non-real roots in $
	\mathbb C$, then the Galois group of $f$ is $S_p$.
	
	For suppose this is the case. Then $G$ acts transitively on the $p$-roots of $f$, and hence contains a $p$-cycle by above. But since $f$ has 
	precisely two nonreal roots, these are complex conjugates of one another, and so complex conjugation induces an automorphism of the splitting 
	field fixing $\mathbb Q$. Since $S_p$ is generated by a $p$-cycle and a transposition, we are done.
	
	\item \textbf{Finding an extension with a given Galois group}: Let $G$ be a finite group. Then, we may find an extension $L / K$ with Galois group $G$. 
	We first consider $G = S_n$. Take $L = \mathbb Q(x_1, ..., x_n)$. $S_n$ acts on $L$ by permuting the variables $x_1, ..., x_n$, and so we may put 
	$K = L^G$. $K$ will be the set of symmetric functions in $n$ variables over $\mathbb Q$. One can show that \textbf{if $G$ is a finite group acting on a 
	field $L$, then $L / L^G$ is Galois with $G = Gal(L / L^G)$}, so this implies $Gal(L / K) = G$. 
	
	\item Ex: Galois group of $x^5 - 4x + 2$. This is irreducible by Eisenstein at $p = 2$, and hence 5 divides the order of the Galois group, so it contains 
	a 5-cycle. One can draw the graph to verify it has 2 complex roots, and so the Galois group must contain complex conjugation, a transposition. But any 
	transposition along with a $p$-cycle generate $S_p$, so the Galois group is $S_5$. We can do a similar thing for any prime $p$, so for any prime $p$, 
	we can find an extension $L / \mathbb Q$ with $Gal(L / \mathbb Q) = S_p$.

\end{itemize}

\section{11/14 (Cyclic Extensions)}

\begin{itemize}
	
	\item \textbf{3rd degree polynomials}: Let $f(x) = x^3 + ax^2 + bx + c\in K[X]$ be an irreducible and separable polynomial. Recall the discriminant of the 
	polynomial, if $\alpha_i$ are the roots, is:
	$$
		\Delta^2 = \prod_{i < j}(\alpha_i - \alpha_j)^2
	$$
	Let $G$ be the Galois group of $f(x)$. 3 divides $|G|$ as $G$ acts transitively on the roots, so since $G \leq S_3$, either $G\cong \mathbb Z / 
	3\mathbb Z$ or $G\cong S_3$. Note that if we examine $\Delta$, this is invariant under elements of $A_3$ and changes sign under elements of 
	$S_3\setminus A_3$. Thus, if the Galois group is $A_3$, $\Delta$ is invariant under the action $G$ and lies in the base field. If $G \cong S_3$, then 
	any $\sigma\in S_3\setminus A_3$ maps $\Delta\mapsto -\Delta$, so $\Delta$ is \textbf{not} in the base field. Thus, \textbf{if $\Delta^2$ has a square 
	root in the base field, $G = A_3$. If $\Delta^2$ has no square root in the base field, then $G = S_3$}. Note if $a = 0$, then:
	$$
		\Delta^2 = -4b^3 - 27c^2
	$$
	
	\item \textbf{Fundamental Theorem of Algebra}: $\mathbb C$ is algebraically closed.
	
	\begin{proof}
	
		We use the following facts about $\mathbb C$ and $\mathbb R$:
		
			\begin{enumerate}
			
				\item $char(\mathbb R) = 0$.
				
				\item Any polynomial in $\mathbb R$ of odd degree has a real root (can use IVT).
				
				\item $[\mathbb C : \mathbb R] = 2$, and every element of $\mathbb C$ has a square root in $\mathbb C$. 
			
			\end{enumerate}
			
		Let $L$ be a finite extension of $\mathbb C$-- we will show that $L = \mathbb C$. Since $char(\mathbb C) = 0$, $L / \mathbb C$ is separable, 
		and we can assume that $L / \mathbb R$ is Galois (just make it normal by making it a splitting field), so set $G = Gal(L / \mathbb R)$. By fact 
		$ii$, $\mathbb R$ has no algebraic extensions of odd degree, for we there are no irreducible polynomials of odd degree (can just strip off the real 
		root), which implies that $G$ has no subgroups of odd index $> 1$. Let $H = Gal(L / \mathbb C)$, so $(G : H) = 2$. But $\mathbb C$ has no 
		quadratic extensions by $iii$, so $H$ has no subgroups of index 2. Let $S$ be a 2-Sylow of $G$ (the order of $G$ is its index with $1$ and hence 
		is even). Then $S$ has odd index ($|S| = p^\alpha$ with $p$ not dividing $|G| / p^\alpha$), so $S = G$ as $G$ has no subgroups of odd index 
		other than $G$ itself. Thus, $G = S$ has order $2^n$ for some $n\implies |H| = 2^{n - 1}$. If $n - 1 > 0$, then $H$ would have a subgroup of 
		index 2, which we have shown is not possible, so $n - 1 = 0\implies n = 1\implies |G| = 2\implies\mathbb C$ is algebraically closed.
	
	\end{proof}
	
	\item \textbf{Lemma}: Suppose $V$ is a vector space over an infinite field $K$. Then, $V$ is not the union of a finite number of proper subspaces.
	
	\item \textbf{Theorem}: If $L / K$ is a finite separable extension, then $L = K(\alpha), \alpha\in K$ is a primitive extension.
	
	Let $M$ be a finite Galois extension containing $L$. Then there are finitely many intermediate extensions of $M / K$ as these correspond with 
	subgroups of the Galois group, and as $L\leq M$ there are only finitely many intermediate extensions of $L / K$. Each of these finitely many 
	extensions is a vector space over $K$, and so if $K$ is infinite, then $L$ is not the union of all of the finitely many subextensions by the above 
	lemma, so some $\alpha\in L$ is not in any smaller extension of $K$, and thus $L = K(\alpha)$. If $K$ is finite, then $\implies L$ is finite, so 
	$L^* = \langle\alpha\rangle$ and $L = K(\alpha)$. 
	
	\item \textbf{Purely inseparable extension}: An example of this is $\mathbb F_p(t, u) / \mathbb F_p(t^p, u^p)$. This has degree $p^2$, and every 
	element of $\mathbb F_p(t, u)$ generates an extension of degree $p$ or $1$. This implies this extension is not primitive as no element generates 
	an extension of degree $p^2$, and in fact this extension has an infinite number of subextensions.
	
	\item \textbf{Theorem}: Suppose that $L / K$ is a Galois extension such that:
	
		\begin{enumerate}
		
			\item $Gal(L / K)\cong \mathbb Z / p\mathbb Z$. 
			
			\item $K$ contains all the $p$th roots of unity.
			
			\item $char(K)\neq p$.
		
		\end{enumerate}
		
		Then $L = K(\sqrt[p]{a})$ for some $a\in K$.
	
	\begin{proof}
	
		To prove this, let $\sigma$ be a generator of the Galois group. We look at the eigenvectors of $\sigma$ as a linear transformation. Since $\sigma$ 
		generates the Galois group, $\sigma^p = 1$, so its eigenvalues are all the $p$th roots of unity and are in $K$. Pick any $v\in L$. Then the 
		element:
		$$
			v + \xi\sigma v + (\xi\sigma)^2v^2 + (\xi\sigma)^3v^3 + ... + (\xi\sigma)^{p - 1}v
		$$
		has eigenvalue $\xi^{-1}$, and similarly $v + \xi^2\sigma v + (\xi^2\sigma)^2v^2 + ...$ has eigenvalue $\xi^{-2}$, and so on. But $v$ is the 
		average of these as $1 + \xi + \xi^2 + ... + \xi^{p - 2} = 0$, so the eigenspaces sum to the entire space, and therefore:
		$$
			L = \bigoplus_{i = 0}^{p - 1}E_i
		$$
		where $E_i$ is the eigenspace of $\sigma$ with eigenvalue $\xi^i$-- each eigenspace is one dimensional. Now, pick $w$ to be any eigenvector 
		of $\sigma$ with $\sigma w = \xi w$, so $w\not\in K$ as $\sigma$ does not fix $w$. Then $\sigma w^p = \xi^p w^p = w^p$, so $w^p\in K$, 
		and if we put $a = w^p$ then $L = K(w) = K(\sqrt[p]{a})$ as $w$ has order $p$ under multiplication by $\xi$, so the elements $\{\sigma^i(w)\}$ 
		span each eigenspace and therefore generate $L$.
		
	\end{proof}
		
	\item \textbf{Artin-Schrier Equation}: The above proof breaks down if $char(K) = p$. Suppose $Gal(L / K) = \langle\sigma\rangle$. Then 
	$L$ cannot be of the form $K(\sqrt[p]{a})$ as $x^p - a$ is inseparable, so its splitting field is not a Galois extension. Now, since $|\sigma| = p$, we have 
	$\sigma^p = 1\implies (\sigma - 1)^p = 0$ by the Frobenius endomorphism, so $\sigma - 1$ is a nilpotent operator. Suppose $v$ is a rank 2 generalized 
	eigenvector, so $(\sigma - 1)^2 v = 0\implies \sigma(\sigma - 1)v = (\sigma - 1) v\implies (\sigma - 1) v\in K$ as it is fixed by a generator of the Galois 
	group. Thus, $\sigma v - v = a, a\in K$, and replacing $v$ with $v / a$ gives $\sigma v - v = 1\implies \sigma v = v + 1$, so $\sigma v^p = v^p + 1$. 
	Combining these, we have that $\sigma(v^p - v) = v^p - v\in K$ as $\sigma$ fixes it, so $v$ is a root of the \textbf{Artin-Schrier Equation}:
	$$
		x^p - x - b = 0\;\;\;\;\;\;\;\;\;\;\;\;\;\;\;\;\;\;\;\;\;\;\;\;\;\;\;\;\;\;\;\;\;\;\;\; (b\in K)
	$$
	
	This is the analog of $x^p - b = 0$ for characteristic $p$. Note that the polynomial $f(x) = x^p - x - b$ is separable in characteristic $p$ for any $b\in K$ 
	as it has derivative $-1$, so its splitting field is Galois. If $v$ is any root, then we see by inspection that $v + 1$ is a root, so the distinct roots are 
	$v, v + 1, ..., v + (p - 1)$. Thus, $K(v)$ is Galois, and $Gal(K(v) / K) = \{\sigma : v\mapsto v + i, i\in \mathbb Z / p\mathbb Z\}$. Thus the Galois group 
	of this equation is either trivial or is $\mathbb Z / \mathbb Z$. \textbf{If $x^p - x - b$ is irreducible in characteristic $p$, its Galois group is $\mathbb Z / 
	p\mathbb Z$}. If not, it splits into linear factors over $K$ and its Galois group is trivial.

\end{itemize}

\section{11/16 (Solvability, Cyclotomic Polynomials)}

\begin{itemize}

	\item A \textbf{cyclic (abelian)} extension is a Galois extension $L / K$ whose Galois group is cyclic (abelian).
	
	\item We say a polynomial equation is \textbf{solvable by radicals} if its roots can be expressed using only field operations and $n$th roots, or in 
	characteristic $p$ if it can also be expressed in roots of the Artin-Schrier equation. Equivalently, a field extension $L / K$ is \textbf{solvable by radicals} 
	if there is a tower of field extensions:
	$$
		K = K_0 \leq K_1 \leq K_2\leq ... \leq K_n = L
	$$
	such that for each $i$, there is $a_i\in K_i$ such that:
	$$
		K_{i + 1} = K_i(\sqrt[k_i]{a_i})
	$$
	
	\item A group $G$ is \textbf{solvable} if it admits a cyclic tower. This is equivalent to the group admitting an abelian tower, as any abelian tower may 
	be refined to a cyclic one.
	
	\item \textbf{A polynomial $f(x)\in K[X]$ is solvable by radicals iff its Galois group $G$ is solvable} (assuming the base field $K$ contains all the 
	relevant roots of unity).
	
	\begin{proof}
	
		Suppose that $f(x)$ is solvable in radicals with the tower $K_0\leq K_1\leq ... \leq K_n = L$. We look at the Galois groups $G_0\geq G_1\geq ... 
		\geq G_n = \{1\}$. Then $K_{i + 1} = K_i(\sqrt[k_i]{a_i})$ and so the extension $K_{i + 1} / K_i$ is Galois as the base field contains all the $k_i$th 
		roots of unity. Thus, $K_{i + 1} / K_i$ is normal, so $G_{i + 1}\trianglelefteq G_i$. We have already shown that if we have all the roots of unity, a 
		radical extension has a cyclic Galois group, so $Gal(K_{i + 1} / K_i) = G_i / G_{i + 1}$ is cyclic, so the group $G$ has a cyclic tower and is 
		solvable. Conversely, suppose $G$ is solvable with tower $G_0\geq G_1\geq ... \geq G_n$. Each $G_i / G_{i + 1}$ is cyclic and so the extension 
		$K_{i + 1} / K_i$ is either cyclic or generated by the Artin-Schrier polynomial (if the characteristic is $p$), and so the equation is solvable in 
		radicals.
	
	\end{proof}
	
	All polynomials of degree $\leq 4$ are solvable in radicals because the group $S_4$ admits a cyclic tower $\{1\}\trianglelefteq V_4\trianglelefteq 
	A_4\trianglelefteq S_4$, and any subgroup of a solvable group is solvable. 
	
	\item \textbf{Cyclotomic Polynomials}: The \textbf{$n$th roots of unity} are the roots of $x^n - 1$ over $\mathbb Q$. We call a $n$th root of unity 
	$\xi_n$ \textbf{primitive} if $\forall d | n, d < n$, $\xi_n$ is not a $d$th root of unity. We define the \textbf{$n$th cyclotomic polynomial} to be:
	$$
		\Phi_n(x) := \prod_{\xi_n}(x - \xi_n)
	$$
	The cyclotomic polynomials all have coefficients in $\mathbb Z$, and have degree $\phi(n)$, where $\phi$ is Euler's totient function. To compute 
	$\Phi_n(x)$, we divide $x^n - 1$ by all the cyclotomic polynomials less than $n$ dividing $n$. For an example of this, see notes.
	
	\item \textbf{$\Phi_n(x)$ is irreducible over $\mathbb Q$ with Galois group $(\mathbb Z / n\mathbb Z)^*$}
	
	\item \textbf{Example}: Suppose $n\in\mathbb Z$. Then there are infinitely many primes $p > 0$ such that $p\equiv 1\mod n$.
	
	TODO proof.
	
	\item \textbf{Theorem}: Given a finite abelian group $G$, there is an abelian extension $K / \mathbb Q$ such that $Gal(K / \mathbb Q) = G$.
	
	Put $G = (\mathbb Z / n_1\mathbb Z)\times (\mathbb Z / n_2\mathbb Z)\times ... \times (\mathbb Z / n_k\mathbb Z)$ with each $n_i$ coprime. By 
	above, we can choose distinct primes $p_i$ such that $p_i\equiv 1\mod n_i$. Then $\mathbb Z / n_i\mathbb Z$ is a quotient of $(\mathbb Z / 
	p_i\mathbb Z)^*$ as it is cyclic of order $p_i - 1$ and $n_i | p_i - 1$, so $G$ is a quotient of $(\mathbb Z / p_1\mathbb Z)^*\times (\mathbb Z / 
	p_2\mathbb Z)^*\times ...\times (\mathbb Z / p_k\mathbb Z)^*\cong (\mathbb Z / p_1p_2...p_k)^*$ by the Chinese remainder theorem. But the group 
	$(\mathbb Z / p_1p_2...p_k)^*$ is the Galois group of $\Phi_{p_1...p_k}(x)$, and so $G$ is a quotient of a Galois group and hence a Galois group.
	
	\item \textbf{Kroenecker-Weber-Hilbert Theorem}: If $K / \mathbb Q$ is Galois with $Gal(K / \mathbb Q)$ abelian, then $K$ is contained in a cyclotomic 
	extension of $\mathbb Q$, i.e. $K\leq \mathbb Q(\xi)$ for some primitive $n$th root of unity $\xi$.
	
	\item \textbf{Wedderburn's Theorem}: Any finite division algebra is a field.
	
	Recall any group $G$ is a union of its conjugacy classes, and the order of a conjugacy class is the index of its stabilizer, i.e. $|Gx| = (G : G_x)$ for 
	$G_x := \{g\in G : gxg^{-1} = x\}$. Let $L$ be a finite division algebra with center $K$. We induct on the size of the division algebra. 
	$K$ is obviously a field, so $K = \mathbb F_q$ for some prime power $q$, and $L$ is a $K$-vector space of dimension $n$ for some $n$. 
	Look at $G = K^*$ with $|G| = q - 1$. Suppose $a\in G$. The stabilizer of $a$ in $L$ under conjugation is a subalgebra of $L$ and therefore a $K$
	-vector space, so the size is $q^k$. This includes $0$, so the size is really $q^k - 1$. By the class equation on $L^*$: 
	$$
		|L^*| = q^n - 1 = |Z(G)| + \sum_i(G : C_G(a_i)) = (q - 1) + \sum_i\frac{q^n - 1}{q^{k_i} - 1}
	$$
	with each $k_i < n$. Note that $q^n - 1$ and $\frac{q^n - 1}{q^{k_i} - 1}$ are divisible by $\Phi_n(q)$ as $k_i | n$, so this implies $q - 1$ is divisible 
	by $\Phi_n(q)$ as well and thus $\Phi_n(q) = \prod_{i\in(\mathbb Z / n\mathbb Z)^*}(q - \xi_i) \leq q - 1$. But $|q - \xi | > |q - 1|$ unless $\xi = 1$, so this 
	implies $n = 1$ and thus $L = K$.

\end{itemize}

\section{11/21 (Norm and Trace)}

\begin{itemize}

	\item \textbf{Definitions}: Let $l / K$ be a finite extension, and choose some $a\in L$. The map $m_a: L\rightarrow L, x\mapsto ax$ is a linear 
	transformation of $L$ as a $K$-vector space. We define the \textbf{trace} and the \textbf{norm} of $a$ to be:
	$$
		tr : L\rightarrow K \;\;\;\;\;\;\;\;\;\;\;\;\;\;\;\;\;\;\;\;\;\;\;\;\;\;\;tr(a) := tr(m_a)
	$$
	$$
		N : L^*\rightarrow K^*\;\;\;\;\;\;\;\;\;\;\;\;\;\;\;\;\;\;\;\;\;\;\;\; N(a) := det(m_a)
	$$
	
	The norm and the trace are homomorphisms, i.e. $N(ab) = N(a)N(b)$ and $tr(a + b) = tr(a) + tr(b)$. 
	
	\item \textbf{Norm, Trace as Galois conjugates}: Suppose $L = K(a)$. Then $a$ is the root of an irreducible $p(x) := x^n + b_{n - 1}x^{n - 1} + ...+ b_0 = 
	0$, and we can pick a basis of $L / K$ to be $\{1, a, a^2, ..., a^{n - 1}\}$, and the matrix of $a$ in this basis is upper triangular except for the last column 
	(i.e. must express $a^n$ in terms of this basis to get last column). We note the trace of this matrix is just $-b_{n - 1}$ and its determinant is $\pm b_0$. 
	If the roots of $p(x)$ are $a_1, ..., a_n$ with $a = a_1$, then $b_{n - 1} = \sum_i a_i$ and $b_0$ = $\pm\prod_i a_i$. This gives us a formula for the 
	trace and norm, as we note that the Galois group acts transitively on the roots.
	
	If $L / K$ is Galois and $G = Gal(L / K)$, this gives the following formula for the norm and trace of $a\in L$:
	$$
		tr(a) = \sum_{\sigma\in G}\sigma a
	$$
	$$
		N(a) = \prod_{\sigma\in G}\sigma a
	$$
	
	\item \textbf{Algebraic integers}: An \textbf{algebraic integer} $\alpha$ is any number which is the root of a monic polynomial in $\mathbb Z[X]$. For 
	example, $\omega := exp(2\pi i / 3)$ is an algebraic integer because it is a root of $\Phi_3(x) = x^2 + x + 1 = 0$. Algebraic integers form a ring under 
	the usual addition and multiplication.
	
	\textbf{Theorem}: Let $L / \mathbb Q$ be a finite extension, and $\alpha\in L$. TFAE:
	\begin{enumerate}
		\item $\alpha$ is an algebraic integer.
		\item We can find a finitely generated $\mathbb Z$-module $A$ in $L$ such that $\alpha A\subset A$ (note Borcherds says we may also pick $A$ 
		such that $L = span_\mathbb{Q}(A)$, but I'm not sure if this is the case).
	\end{enumerate}
	
	To prove $i\implies ii$, just take $A = span_\mathbb{Z}\{1, \alpha, \alpha^2, ..., \alpha^{n - 1}\}$, where $n$ is the degree of the minimal polynomial of 
	$\alpha$. Then evidently this is a finitely generated $\mathbb Z$-module which satisfies $\alpha A\subset A$ as $\alpha^n$ is a linear combination of 
	its lower powers. For the converse, view $\alpha$ as a linear map $T : x\mapsto \alpha x\in End(A)$. $\alpha$ is obviously an eigenvalue of this, so 
	$char_T(\alpha) = 0$, and $char_T(x)\in \mathbb Z[X]$ as we are working over $\mathbb Z$, so $\alpha$ is an algebraic integer.
	
	\item \textbf{Quadratic Fields}: Suppose $N$ is squarefree and $L = \mathbb Q(\sqrt{N})$-- we will determine the algebraic integers in $L$. The 
	obvious examples are $m + n\sqrt{N}$, since $\sqrt{N}$ is an algebraic integer and they form a ring. The key here is that if $\alpha$ is an algebraic 
	integer, then so are $tr(\alpha)$ and $N(\alpha)$, as $\sigma\alpha$ will be an algebraic integer since it will satisfy the same polynomial as $\alpha$. 
	Since algebraic integers form a ring, $tr(\alpha)$ and $N(\alpha)$ will be algebraic integers, and will be in $\mathbb Z$ because the only degree 1 
	algebraic integers over $\mathbb Q$ are elements of $\mathbb Z$. We pick a basis $\beta := \{1, \sqrt{N}\}$ of $L / \mathbb Q$, and compute the trace 
	and norm of $m + n\sqrt{N}$. Let $T_{m, n}$ be the linear transformation $x\mapsto (m + n\sqrt{N})x$. Then:
	$$
		[T_{m, n}]_\beta = 
		\begin{pmatrix}
			m & nN \\ n & m
		\end{pmatrix}
	$$
	
	We see that, for $m + n\sqrt{N}\in L$:
	$$
		N(m + n\sqrt{N}) = det(T_{m ,n}) = m^2 - n^2 N
	$$
	$$
		tr(m + n\sqrt{N}) = tr(T_{m, n}) = 2m
	$$
	
	Since the trace and norm are in $\mathbb Z$, this implies that either $m\in\mathbb Z$ or $m\in\mathbb Z + \frac{1}{2}$. If $m\in \mathbb Z$, then 
	$n^2 N\in \mathbb Z$, so $n\in\mathbb Z$ as $N$ is squarefree (for if $n = \frac{c}{d}$ with $gcd(c, d) = 1$, then $c^2 N = d^2 k\implies d^2 | N\implies 
	d = 1$). This therefore reduces to the first case of $m + n\sqrt{N}$ for $m, n\in\mathbb Z$. Suppose $m\in\mathbb Z + \frac{1}{2}$. Then $m^2 = k + 
	\frac{1}{4}\implies \frac{1}{4} - n^2 N \in\mathbb Z\implies (2n)^2 N\equiv 1\mod 4$. For $N\equiv 2, 4\mod 4$ this has no solutions, and for $N\equiv 1
	\mod 4$ this has solutions $2n$ odd. Thus, the algebraic integers of $\mathbb Q(\sqrt{N})$ are:
	$$
	\begin{cases} 
		\mathbb Z[\sqrt{N}] & n\equiv 2, 3\mod 4 \\
		\mathbb Z[\frac{1 + \sqrt{N}}{2}] & n\equiv 1\mod 4
	\end{cases}
	$$
	
	\item \textbf{Theorem (Artin)}: Let $G$ be a group or monoid, and $K$ a field. A \textbf{character} of $G$ with values in $K$ is a homomorphism $\chi : 
	G\rightarrow K^*$. If $\chi_1, ..., \chi_n$ are distinct characters, then they are linearly independent, i.e. $\forall g\in G$ $a_1\chi_1(g) + ... + a_n
	\chi_n(g) = 0$ implies $a_1 = ... = a_n = 0$. 
	
	\item \textbf{Trace as a bilinear form}: The trace gives us a bilinear form $(\cdot, \cdot): L\times L\rightarrow K$ given by:
	$$
		(a, b) := tr(ab)
	$$
	i.e. this form is linear in each argument. We say a bilinear form is \textbf{degenerate} if the map $b\mapsto (a\mapsto (a, b))$ is not an isomorphism of 
	$L$ with its dual space. Equivalently, a bilinear form is degenerate if there is a nonzero $x\in L$ such that $\forall y\in L$, $(x, y) = 0$, so this form is 
	degenerate if $tr(a) = 0$ for every $a\in L$. For example, take $L = \mathbb F_p(t)$ and $K = \mathbb F_p(t^p)$. Then $tr : L\rightarrow K$ is 
	identically zero on $L$ because every element of $L$ has minimal polynomial of the form $x^p - a$, $a\in\mathbb F_p(t^p)$ and so the coefficient on 
	$x^{p - 1}$, which is the trace, is 0. 
	
	We note that \textbf{for separable extensions, the trace is not identically $0$}, so $(\cdot, \cdot)$ is nondegenerate. In characteristic $0$, this is easy 
	as $tr(1) = \sum_{\sigma\in G}\sigma(1) = |G| = [L : K]\neq 0$. 
	
	\textbf{For any Galois extension $L / K$, the form $(\cdot, \cdot)$ is nondegenerate} (equivalently, the trace does not vanish completely on $L$). This is 
	because $tr(a) = \sigma_1(a) + ... + \sigma_n(a)$, and we may view each $\sigma\in Gal(L / K)$ as a character $L^*\rightarrow L^*$. So, if the trace 
	vanishes for every element of $L$, then this contradicts Artin's theorem on independence of characters, and thus the trace is not identically zero.

	\item \textbf{Discriminant of a Field Extension}: Let $L / K$ be a field extension. We define the \textbf{discriminant} of $L / K$ to be the discriminant of 
	the bilinear form $(a, b) = tr(ab)$ on $L$ as a $K$-vector space. If $a_1, ..., a_n$ is a basis for $L / K$, then this is:
	$$
		Disc_{L / K}(a_1, ..., a_n) = det
		\begin{pmatrix}
			(a_1, a_1) & (a_1, a_2) & \cdots \\
			(a_2, a_1) & (a_2, a_2) & \cdots \\
			\vdots & \vdots & \ddots
		\end{pmatrix}
	$$
	
	Note the discriminant is not independent of basis: if $b_1, ..., b_n$ is another basis and $b_i = \sum_jA_{ij}a_j$, then:
	$$
		Disc_{L / K}(a_1, ..., a_n) = det(A)^2 Disc_{L / K}(b_1, ..., b_n)
	$$
	However, it is defined up to multiplication by a square, and thus $Disc_{L / K}\in K^* / K^{*2}$
	
	Suppose $L = K(a)$ is Galois. Let $p(x)$ be the minimal polynomial of $a$ in $K[X]$, and pick the basis $\{1, a, a^2, ..., a^{n - 1}\}$ of $L / K$. Then 
	the traces reduce to $tr(a^k) = \sum_{\sigma\in G}\sigma a^k$, and we may plug these in so simplify the discriminant to the product of two 
	Vandermonde determinants. This ends up simplifying to:
	$$
		Disc_{L / K}(1, a, ... a^{n - 1}) = \prod_{i < j}(\sigma_i a - \sigma_j a)^2 = \Delta^2
	$$
	where $\Delta^2$ is the discriminant of the polynomial $p(x)$.
	
	\textbf{Discriminant applications}: Which of the following fields are isomorphic? 
	\begin{enumerate}
		\item $\mathbb L = Q[X] / (x^3 + x + 1), Disc(L / \mathbb Q) = -31$.
		\item $\mathbb L = Q[X] / (x^3 + x - 1), Disc(L / \mathbb Q) = -31$
		\item $\mathbb L = Q[X] / (x^3 - x + 1), Disc(L / \mathbb Q) = -23$
	\end{enumerate}
	The first two have equal discriminants and are thus isomorphic; it is possible for two non-isomorphic extensions to have the same discriminant, but this 
	is quite rare. Note that $-23$ and $-31$ are not equal modulo a square as $\frac{-31}{-23}$ is not a square in $\mathbb Q$, so these discriminants are 
	not equal.
	
	Another example is that of finding algebraic integers in $L = \mathbb Q(\alpha)$ with $\alpha^3 + \alpha + 1 = 0$. The discriminant of the basis $\{1, 
	\alpha, \alpha^2\}$ in this extension is $-31$. Let $A$ be the $\mathbb Z$-linear span of this basis, and let $B$ be all algebraic integers in $L$. Clearly 
	$A\subset B$ as $\alpha$ is an algebraic integer, and we wish to show $A = B$. If $X$ is the change of basis from $A$ to $B$, then $Disc{L / 
	\mathbb Q}(B) = det(X)^2Disc_{L / \mathbb Q}(A)$, and $det(X) = |B / A|$. Since $-31$ is square-free, $det(X)^2 = 1$, so $|B / A| = 1\implies A = B$. 
	This generalizes to any square-free discriminant, so if the discriminant is square-free we can easily identify the ring of algebraic integers in $L/ \mathbb 
	Q$. 

	\item \textbf{Theorem}: If $L / K$ is a finite Galois extension of finite fields, then $N : L^*\rightarrow K^*$ and $tr : L\rightarrow K$ are surjective.
	
	Essentially, take $q = |K|$ and $n =[L : K]$. Then $Gal(L / K) = \langle F\rangle$ where $F : x\mapsto x^q$ is the Frobenius element, so:
	$$
		N(a) = \prod_{i = 0}^{n - 1}F^i(a) = a\cdot a^q\cdot a^{q^2}\cdot...\cdot a^{q^{n - 1}} = a^{\frac{q^{n - 1}}{q - 1}}
	$$
	
	As the polynomial $x^{\frac{q^{n - 1}}{q - 1}} - 1$ has degree $\frac{q^{n - 1}}{q - 1}$, it has $\leq \frac{q^{n - 1}}{q - 1}$ roots and therefore $|ker(N)| 
	\leq \frac{q^{n - 1}}{q - 1}$. The order of $L^*$ is $q^n - 1$, and since $L^* / ker(N)\cong im(N)$, we have:
	$$
		q^n - 1 = |L^*| = |im(N)|\times |ker(N)| \leq \frac{q^{n - 1}}{q - 1}|im(N)|\implies q - 1\leq |im(N)|
	$$
	which implies that $im(N) = K^*$ as this is the order of $K^*$. 

\end{itemize}

\section{11/28 (Solving Equations, Galois Cohomology)}

\begin{itemize}

	\item \textbf{Lemma}: This is a simple and useful lemma that we will use often in this lecture. Suppose $G$ is a finite group acting on a $K$-vector 
	space $V$. Let $g\in G$ have order $n$. Then, for any $v\in V$, the vector:
	$$
		w := \sum_{i = 0}^{n - 1}g^i(v)
	$$
	is fixed under the action of $g$, i.e. $g(w) = w$.
	
	\begin{proof}
		$$
		g(w) = g(\sum_{i = 0}^{n - 1} g^i(v)) = \sum_{i = 1}^n g^i(v) = g^n(v) + \sum_{i = 1}^{n - 1}g^i(v) = v + \sum_{i = 1}^{n - 1}g^i(v) = 
		\sum_{i = 1}^{n - 1}g^i(v) = w
		$$
	\end{proof}

	\item \textbf{Hilbert's Theorem 90}: Suppose $L / K$ is a cyclic Galois extension with generator $\sigma$ and degree $n = [L : K]$. Then: 
	$$
	N(a) = 1 \iff a = \frac{b}{\sigma b}
	$$
	for some $b\in L^*$.
	
	\begin{proof}
	
		If $a = \frac{b}{\sigma b}$, then we have $N(a) = 1$ because $N(\sigma b) = N(b)$ by reindexing the finite sum over the group. Conversely, 
		suppose $N(a) = 1$. We wish to find a fixed vector $b\in L^*$ under the linear map $a\sigma$, i.e. a vector $b$ with $a\sigma b = b$. By the 
		above lemma, if $a\sigma$ has finite order, we may just average over it acting on an arbitrary $v\in L$ to find $b$. Note that $(a\sigma)^2(v) = 
		a\sigma(a\sigma(v)) = a\sigma(a)\sigma^2(v)$, and in general:
		$$
			(a\sigma)^i = a\sigma(a)\sigma^2(a)...\sigma^{i - 1}(a)\sigma^i
		$$
		Then since $\sigma^n = id$, $(a\sigma)^n = a\sigma(a)\sigma^2(a)...\sigma^{n - 1}(a)\sigma^n = \prod_{i = 0}^{n - 1}\sigma^i(a) = N(a) = 1$, 
		so $a\sigma$ has finite order. Thus we may take an arbitrary $\theta\in L$ and find a fixed vector to be:
		$$
			b = \sum_{i\in\mathbb Z / n\mathbb Z}(a\sigma)^i(\theta)
		$$
		We must show that $\theta$ can be picked to make $b$ nonzero, and then we will be done. This follows from independence of the characters 
		$\{1, \sigma, ..., \sigma^{n - 1}\}$, as we have $b = (c_0 \sigma^0 + c_1\sigma + ... + c_{n - 1}\sigma^{n - 1})\theta$ for $c_i = a\sigma(a)
		\sigma^2(a)...\sigma^{i - 1}(a)\in L$, so if $b$ was identically 0 for every $\theta\in L$, this would contradict Artin's theorem. 
	
	\end{proof}
	
	\item \textbf{Remember relations between roots}: Suppose $f(x)$ is separable with degree $n$ and roots $\alpha_1, ..., \alpha_n$. Then recall:
	$$
		f(x) = \prod_{i = 1}^n(x - \alpha_i) = x^n - e_1x^{n - 1} + e_2x^{n - 2} + ...
	$$
	where $e_1 = \sum_{i = 1}^n\alpha_i$, ..., are the elementary symmetric functions in variables $\alpha_i$. In particular, this allows one to easily 
	determine $\sum_{i = 1}^n\alpha_i$ and $\prod_{i = 1}^n\alpha_i$ by looking at the coefficient on the $x^{n - 1}$ term and the constant term. 

	\item \textbf{Solving $x^3 + x + 1 = 0$}: Let $L$ be the splitting field, and we will work over $\mathbb Q(\omega)$, for $\omega = exp(2\pi i / 3)$, a 
	primitive 3rd root of unity. This has discriminant $-31$ which is not a square in $\mathbb Q(\omega)$, so $Gal(L /\mathbb Q) = S_3$. $S_3$ is 
	solvable by the cyclic tower:
	$$
		1\trianglelefteq \mathbb Z / 3\mathbb Z\trianglelefteq S_3
	$$
	We may use the Galois correspondence to get the corresponding tower of fixed fields:
	$$
		L\geq K\geq \mathbb Q(\omega)
	$$
	The degree $[K : \mathbb Q(\omega)] = (S_3 : A_3) = 2$, and so $K / \mathbb Q(\omega)$ is a quadratic extension. Let the roots of $f$ be $\alpha_1, 
	\alpha_2, \alpha_3$, and let $\sigma = (123)$. $S_3$ acts on the roots by permuting them, and we want to find $K = L^{A_3}$, and $A_3 = \langle
	\sigma\rangle$. Note that $\Delta = \sqrt{-31} = (\alpha_1 - \alpha_2)(\alpha_1 - \alpha_3)(\alpha_2 - \alpha_3)$ is fixed by $\sigma$ but not by 
	transpositions, so $K = \mathbb Q(\omega)(\Delta) = \mathbb Q(\omega)(\sqrt{-31})$. Now, $[L : K] = 3$ so $Gal(L / K)\cong\mathbb Z / 3\mathbb Z$, 
	 and since $K$ contains all the 3rd roots of unity, this implies $L = K(\sqrt[3]{b})$ for some $b$. But (from the proof above with cyclic Galois group 
	 and ground field containing roots of unity) we have $\sqrt[3]{b} = w$ with $\sigma w = \omega w$. Note that for any $c\in L$, $c + 
	 \omega^{-1}\sigma(c) + \omega^{-2}\sigma^2(c)$ has eigenvalue $\omega$ under $\sigma$, and so we may take any linear combination like this. So, 
	 pick $c = \alpha_1$ and take $y := \alpha_1 +  \omega^{-1}\sigma(\alpha_1) + \omega^{-2}\sigma^2(\alpha_1) = \alpha_1 +  \omega^{-1}\alpha_2 + 
	 \omega^{-2}\alpha_3$, and so $L = K(y)$, and $y$ is a cube root of an element of $K$. Similarly, let $z = \alpha_1 + \omega\alpha_2 + \omega^2
	 \alpha_3$, which has eigenvalue $\omega^{-1} = \omega^2$.  Furthermore, $0 = \alpha_1 + \alpha_2 + \alpha_3$ as the coefficient on $x^2$ is $0$, 
	 and this has eigenvalue $1$-- if we can find $y^3, z^3$, we can solve for the roots by linear algebra. We know both $y^3$ and $z^3$ are in $K$ and 
	 therefore are fixed by $\sigma$. We can expand out $y^3 + z^3$ in terms of the $\alpha_i$ to get that $y^3 + z^3 = -27c$ and $y^3b^3 = -27b^3$, 
	 so $y^3$ and $z^3$ are roots of $x^2 + 27x - 27 = 0$, and we may solve for $y^3, z^3$, then solve for $y, z$, and finally solve for the roots $\alpha_i$.
	
	\item \textbf{Solving 4th degree polynomials}: TODO 
	
	\item \textbf{Galois Cohomology}: Suppose $G$ acts on a module $M$. We can define the \textbf{invariants} of $M$ under $G$ by:
	$$
		M^G = \{m\in M : gm = m, \forall g\in G\}
	$$
	This is the largest submodule of $M$ upon which $G$ acts trivially. We can define the dual notion to be the largest quotient of $M$ upon which $G$ 
	acts trivially:
	$$
		M_G = M / \{m - gm : g\in G, m\in M\}
	$$
	Now, the functors $M\mapsto M^G$ and $M\mapsto M_G$ are \textbf{not exact}. They are both covariant functors, but $M\mapsto M^G$ is \textbf{left 
	exact} and $M\mapsto M_G$ is \textbf{right exact}, i.e. if $0\rightarrow A\rightarrow B\rightarrow C\rightarrow 0$ is exact, then the following are as well:
	$$
		0\rightarrow A^G\rightarrow B^G\rightarrow C^G
	$$
	$$
		\;\;\;\;\;\;\;\;\;\;\;\;\;\;A_G\rightarrow B_G\rightarrow C_G\rightarrow 0
	$$
	
	We often want to know how these fail to be exact. Let $\mathbb Z G$ be the group ring of $G$ over $\mathbb Z$. We note that:
	$$
		M^G \cong Hom_{\mathbb Z G}(\mathbb Z, M)
	$$
	by the bijection sending $m\in M^G$ to the map $\phi_m : z\mapsto zm$. Note we view $\mathbb Z$ is a $\mathbb ZG$ module with $g$ acting trivially 
	on $\mathbb Z$, i.e. $gz = z$. This will be a $\mathbb ZG$ homomorphism, as it clearly respects $+$ and $\phi_m((\sum_{g\in G} c_gg)z) = \phi_m
	(\sum_{g\in G}c_gz) = (\sum_{g\in G}c_gz)m = (\sum_{g\in G}c_gg)zm = (\sum_{g\in G}c_gg)\phi_m(z)$ as $gm = m$ for $m\in M^G$. So, we may 
	view $\cdot^G$ as the functor $Hom_{\mathbb ZG}(\mathbb Z, \cdot)$, which we recall is not exact. The failure for this to be exact is controlled by the 
	\textbf{Ext functor}, so we put $H^0(G, M) := M^G$ and:
	$$
		H^i(G, M) := Ext_{\mathbb ZG}^i(\mathbb Z, M)
	$$
	
	Similarly, we have:
	$$
		M_G\cong \mathbb Z\otimes_{\mathbb ZG} M
	$$
	and the failure for $\otimes$ to preserve exactness is measured by $Tor$. We then define $H_0(G, M) := M_G$, and:
	$$
		H_i(G, M) := Tor_i^{\mathbb ZG}(\mathbb Z, M)
	$$
	
	These are the \textbf{ith cohomology groups}. They measure how inexact a sequence is-- for an exact sequence $0\rightarrow A\rightarrow B 
	\rightarrow C \rightarrow 0$, the sequence $0\rightarrow H^0(A)\rightarrow H^0(B)\rightarrow H^0(C)\rightarrow H^1(A)\rightarrow H^1(B)\rightarrow 
	H^1(C)\rightarrow H^2(A)\rightarrow...$ is exact.

\end{itemize}

\section{11/30 (Galois Cohomology, Infinite Extensions)}

\begin{itemize}

	\item \textbf{Crossed Homomorphisms}: Let $G$ act on a abelian group $A$ by means of a homomorphism $G\rightarrow Aut(A)$ (for example, $G$ a 
	Galois group acting on a $L^*$). We define a \textbf{crossed homomorphism}, also called a \textbf{1-cocycle}, to be a map $G\rightarrow A$ with 
	$\sigma\mapsto a_\sigma\in M$ satisfying:
	$$
		a_{\sigma\tau} = a\sigma + \sigma a_\tau
	$$
	We may equivalently view a 1-cocycle as a family of elements $\{a_\sigma\}_{\sigma\in G}$ satisfying this relation. If $\{a_\sigma\}_{\sigma\in G}$ and 
	$\{b_\sigma\}_{\sigma\in G}$ are 1-cocycles, then $\{a_\sigma + b_\sigma\}_{\sigma\in G}$ is also a 1-cocycle, and so 1-cocycles form a group, which 
	we write as $Z^1(G, A)$. By a \textbf{principal crossed homomorphism}, also called a \textbf{1-coboundary}, we mean a 1-cocycle $\{a_\sigma\}
	_{\sigma\in G}$ such $\exists\beta\in A$ such that 
	$$
		a_\sigma = \beta - \sigma(\beta), \forall \sigma\in G
	$$
	Note we use $\beta - \sigma \beta$ here, but we may use $\beta / \sigma\beta$ if the group law is multiplicative. These similarly form a group, which 
	we write as $B^1(G, A)$. Lang's definition of the \textbf{first cohomology group} is:
	$$
		H^1(G, A) := Z^1(G, A) / B^1(G, A)
	$$
	
	\item \textbf{Hilbert's Theorem 90, Generalized}: Let $L / K$ be a Galois extension with $G = Gal(L / K)$. Then:
	$$
		H^1(G, L^*) = \{1\}
	$$
	and:
	$$
		H^1(G, L) = \{0\}
	$$
	
	\begin{proof}
	
		We must show that every 1-cocycle is a 1-coboundary. Let $\{a_\sigma\}_{\sigma\in G}$ be a 1-cocycle. Note the map $a_\sigma\sigma : L
		\rightarrow L$ is a linear map on $L$, and so we get a map $\phi: G\rightarrow End(L), \sigma\rightarrow a_\sigma\sigma$. This is in fact a 
		homomorphism: note that $(a_\sigma\sigma)(a_\tau\tau)(v) = a_\sigma\cdot\sigma(a_\tau\tau(v)) = a\sigma\cdot\sigma(a_\tau)\sigma(\tau(v)) = 
		(a_\sigma\sigma a_\tau\sigma\tau)(v)$, so $\phi(\sigma\tau) = a_{\sigma\tau}\sigma\tau = a_\sigma\sigma a_\tau\sigma\tau = (a_\sigma\sigma)
		(a_\tau\tau) = \phi(\sigma)\phi(\tau)$. Now, we wish to show there is a $b$ that is fixed under this map $a_\sigma\sigma$, i.e. $a\sigma\sigma b = 
		b$. $G$ still acts on $L^*$ by the twisted action $\sigma\mapsto a_\sigma\sigma$ as this is a homomorphism, and so we can use our usual 
		technique of averaging elements. That is, for each $v\in L^*$:
		$$
			b := \sum_{\sigma\in G}a_\sigma\sigma(v)
		$$
		is fixed under the action. But, the elements $a_\sigma\sigma$ are still characters on $L^*$, and so by linear independence of characters we may 
		find a $v$ making $b$ nonzero, and thus $\forall\sigma\in G$, $a_\sigma = b / \sigma b$, and so $\{a_\sigma\}_{\sigma\in G}$ is a 1-coboundary.
	
	\end{proof}
	
	Note this is stronger than the earlier statement of the theorem. Suppose that $G$ is cyclic and $G = \langle\sigma\rangle$. Then we may define $a_1 = 
	1, a_\sigma = a, a_{\sigma^2} = a\sigma(a) = a\sigma a, ..., a_{\sigma^i} = a\sigma(a)\sigma^2(a)...\sigma^{i - 1}(a)$. We have $a_{\sigma^n} = N(a)$, 
	so if $N(a) = 1\implies a_{\sigma^n} = 1$ and $\{a_{\sigma^i}\}$ is a 1-cocycle, which implies it is a 1-coboundary. Thus, there is some $b\in L^*$ with 
	$a_{\sigma^i} = b / \sigma^ib$ for every $i$, and in particular for $i = 1$ this gives $a_\sigma = a = b / \sigma b$.
	
	\item \textbf{Normal Basis Theorem}: Let $L / K$ be a Galois extension of degree $n$, and let $Gal(L / K) = \{\sigma_1, ..., \sigma_n\}$. Then, there is 
	an element $w\in L$ such that $\{\sigma_1w, ..., \sigma_nw\}$ form a basis of $L / K$. 
	
	\item \textbf{Equivalence of $H^1$ Definitions}: \textbf{TODO}
	
	\item \textbf{Infinite Galois Extensions}: We define an \textbf{infinite Galois extension} to be an algebraic, normal, and separable extension. Let $L / K$ 
	be an infinite Galois extension. How can we compute $Gal(L / K)$? The idea is to look at all finite Galois subextensions $L_i / K$. We can induce a 
	map from $G$ into the inverse limit of this family, and this will end up being an isomorphism. So, if we let $i$ range over finite normal subextensions 
	$L_i$ of $L / K$, then:
	$$
		Gal(L / K) = \varprojlim_i Gal(L_i / K)
	$$
	
	\item \textbf{Example: Algebraic closure of $\mathbb F_p$}: Let $L = \bar{\mathbb F}_p$ be the algebraic closure of $K = \mathbb F_p$. Then:
	$$
		L = \bigcup_{k\geq 1}\mathbb F_{p^k}
	$$
	Recall that $Gal(\mathbb F_p^k / \mathbb F_p)\cong \mathbb Z / k\mathbb Z$, so we get:
	$$
		Gal(\bar{\mathbb F}_p / \mathbb F_p) = \varprojlim_n\mathbb Z / n\mathbb Z\cong \prod_p\mathbb Z_p
	$$
	where $\mathbb Z_p = \varprojlim_k \mathbb Z / p^k Z$ is the $p$-adic integers. 
	
	This group $\varprojlim_n\mathbb Z / n\mathbb Z$ is called the \textbf{profinite completion} of $\mathbb Z$. 
	
	\item \textbf{Profinite Groups}: A group is \textbf{profinite} if it is the inverse limit of a directed system of finite groups. The \textbf{profinite completion} 
	of $G$ is the group:
	$$
		\varprojlim_i G / G_i\subset \prod_i G / G_i
	$$
	where $i$ ranges over all normal $G_i\trianglelefteq G$ with $(G : G_i)$ finite. We get a homomorphism $G\rightarrow \varprojlim_i G / G_i$, and the 
	image of $G$ is dense in the Krull topology. 
	
	\item \textbf{The Krull Topology}: Recall that to give a set $S$ the \textbf{discrete topology} means to let each subset of $S$ be open. Given a 
	collection $\{X_i\}_{i\in I}$ of topological spaces, we may give this the \textbf{product topology} by defining a base for the open sets of $\prod_iX_i$ 
	to be the open sets of each $X_i$ times $X_j$ for all $j\neq i$. In other words, the open sets of $\prod_i X_i$ are:
	$$
		\prod_iU_i
	$$
	where $U_i$ is open in $X_i$ and $U_i\neq X_i$ for all but finitely many $i$. 
	
	\item \textbf{Cyclotomic Extension of $\mathbb Q$}: We take 
	$$
		L = \bigcup_n\mathbb Q(\xi_n)
	$$
	and $K = \mathbb Q$, where $\xi_n$ is a primitive $n$th root. We have that $Gal(\mathbb Q(\xi_n) / \mathbb Q)\cong (\mathbb Z / n\mathbb Z)^*$, so:
	$$
		Gal(L / \mathbb Q) = \varprojlim_n (\mathbb Z / n\mathbb Z)^*\cong \prod_p\mathbb Z_p^*
	$$
	
	\textbf{Kummer Theory}: The problem is to find all abelian extensions of $K$, given that $K$ has "enough" roots of unity. Let $\bar K$ be the 
	separable algebraic closure of $K$, so the largest separable extension of $K$ in the algebraic closure, and $\mu_n\subset \bar K^*$. We examine:
	$$
		1\rightarrow\mu_n\rightarrow\bar K^*\xrightarrow{x\mapsto x^n}\bar K^*\rightarrow 1
	$$
	
	These groups are acted on by $G := Gal(\bar K / K)$, and we assume $\mu_n\subset K$. We look at the invariants under $Gal(\bar K / K)$. The 
	invariants of $\bar K^*$ will be $K^*$ as $K$ is fixed, and $\mu_n$ is contained in $K$, and so will be invariant. Since $\cdot^G$ is left exact, we get:
	$$
		1\rightarrow\mu_n\rightarrow K^*\xrightarrow{x\mapsto x^n} K^*\rightarrow H^1(G, \mu_n)\rightarrow H^1(G, \bar K^*)\rightarrow ...
	$$
	
	By Hilbert's theorem 90, $H^1(G, \bar K^*) = 1$ is trivial, and $H^1(G, \mu_n)\cong Hom(G, \mu_n)$ because $G$ acts trivially on $\mu_n$, so we get 
	the exact sequence:
	$$
		K^*\xrightarrow{x\mapsto x^n} K^*\rightarrow Hom(G, \mu_n)\rightarrow 1
	$$
	
	We have that $Hom(G, \mu_n)\cong K^* / (K^*)^n$, and the kernel of elements in $Hom(G, \mu_n)$ is the subgroups $H\trianglelefteq G$ with 
	$G / H$ cyclic of order $n$, which is the same as the cyclic Galois extensions $L / K$. 

\end{itemize}

\end{document}  