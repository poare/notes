\documentclass[11pt, oneside]{article}   	% use "amsart" instead of "article" for AMSLaTeX format
\usepackage[margin = .5in]{geometry}                		% See geometry.pdf to learn the layout options. There are lots.
\geometry{letterpaper}                   		% ... or a4paper or a5paper or ... 
%\geometry{landscape}                		% Activate for rotated page geometry
%\usepackage[parfill]{parskip}    		% Activate to begin paragraphs with an empty line rather than an indent
\usepackage{graphicx}				% Use pdf, png, jpg, or eps§ with pdflatex; use eps in DVI mode
								% TeX will automatically convert eps --> pdf in pdflatex		
\usepackage{amssymb}
\usepackage{amsmath}
\usepackage[shortlabels]{enumitem}
\usepackage{float}
\usepackage{tikz-cd}

\usepackage{slashed}
\usepackage{amsthm}
\theoremstyle{definition}
\newtheorem{definition}{Definition}[section]
\newtheorem{theorem}{Theorem}[section]
\newtheorem{corollary}{Corollary}[theorem]
\newtheorem{lemma}[theorem]{Lemma}

\newcommand{\N}{\mathbb{N}}
\newcommand{\R}{\mathbb{R}}
\newcommand{\Z}{\mathbb{Z}}
\newcommand{\Q}{\mathbb{Q}}

%SetFonts

%SetFonts


\title{8.513 Term Paper: Symmetry and Topology in Lattice QCD}
\author{Patrick Oare}
\date{December 10th, 2019}							% Activate to display a given date or no date

\begin{document}
\maketitle
\section{Introduction}

Quantum Chromodynamics (QCD) is the theory of the strong nuclear force. It is a $SU(3)$ gauge theory coupled to 6 flavors of 
fermions, which are known as the quarks. At low energies, the coupling of QCD is reasonably large and perturbation theory cannot 
be applied to the theory, which means physics must be extracted by non-perturbative means. One such way to do this is to formulate 
QCD as a \textbf{lattice gauge theory} by discretizing spacetime-- the advantage of this is that the path integral becomes a finite 
(albeit large) dimensional integral which can be evaluated numerically using computers. Although lattice QCD is used primarily 
as a calculational tool for full QCD observables, the theory by itself has many interesting properties. In this paper, I will discuss 
the details of such a theory and consider the role of topology in studying lattice gauge theories. 

We will begin by making some initial definitions: denote our spacetime lattice with spacing $a$ by $\Lambda$, and Wick rotate to 
imaginary time so that $\Lambda$ is a Euclidean lattice\footnote{We must Wick rotate so that the Boltzmann factor of $e^{iS}$ in the 
path integral becomes a valid probability density $e^{-S}$, in order to perform any computations at all.}. In the full theory 
of QCD, the dynamical fields in the path integral are the quark fields $\psi_f$ and gluon fields $A_\mu$. When we 
discretize QCD, we will still work with the quark fields, but instead of directly working with the gauge fields $A_\mu$ we will work 
\textbf{link fields} $U_\mu(n)$ which transform in the following way under a gauge transformation $\Omega : \Lambda\rightarrow 
SU(3)$:
\begin{equation}
	U_\mu(n)\xrightarrow{\Omega(n)}\Omega(n) U_\mu(n)\Omega(n + \hat{\mu})^\dagger
\end{equation}
Here $\hat{\mu}$ is the unit vector in the $\mu$ direction, and $n\in\Lambda$ denotes a site in the lattice. The link fields can be taken 
to be $U_\mu(n) = \exp(ia A_\mu(n))$, so are intimately related to the gauge field, and take values (as they must) in $SU(3)$. 
With this transformation law, the link fields act as a connection between the fibers at different points in $\Lambda$:
\begin{equation}
	U_\mu(n)\psi(n + \hat{\mu})\xrightarrow{\Omega} \Omega(n) U_\mu(n)\psi(n + \hat{\mu})
\end{equation}
This transformation means that $U_\mu(n)\psi(n + \hat\mu)$ is valued in the fiber at point $n$, and so can directly be compared with 
$\psi(n)$. This allows us to add and subtract fermion fields at different points in a gauge invariant way, and so define a covariant derivative. 

We may now write down a first pass at a fermion action, which will be equivalent to $\slashed D + m$ upon taking the continuum limit. 
The direct discretization of this action is thus:
\begin{align}
	S_f^0[\psi_f, \bar\psi_f, U] &= a^4\sum_{n\in\Lambda}\sum_f \bar\psi_f(n)\left(\gamma^\mu\frac{U_\mu(n)\psi_f(n + \hat\mu) - U_{-\mu}(n)\psi_f(n - \hat\mu)}{2a} - m_f\psi_f(n)\right) 
	\\
	&= a^4\sum_{n, m\in\Lambda}\sum_f \bar\psi_f(n)_{\alpha}^a D_{f}^0(n | m)_{\alpha\beta}^{ab}\psi_f(m)_{\beta}^b~
	\label{eq:naive_action}
\end{align}
where we have defined the \textbf{Dirac operator} $D_{\alpha\beta}^{ab}(n | m)$ to be:
\begin{equation}
	D_f^0(n | m)_{\alpha\beta}^{ab} := (\gamma^\mu)_{\alpha\beta} \left(\frac{U_\mu(n)^{ab} \delta_{n + \hat\mu, m} - U_{-\mu}(n)^{ab} \delta_{n - \hat{\mu}, m}}{2a}\right) 
	+ m_f\delta_{\alpha\beta}\delta^{ab}\delta_{nm}
\end{equation}
Note the Greek indices $\alpha, \beta$ are in Dirac space, and the Latin indices $a, b$ are in color space. The Dirac operator 
is one of the fundamental objects we will study later when considering the role of topology in lattice QCD, and is the discretized 
version of $i\slashed D - m$ in Euclidean space. 

However, there is a slight problem with the action in Equation~\ref{eq:naive_action}. Because we are working on a lattice, the 
Fourier transform $\tilde D^0(p)$ of the Dirac operator $D^0(n | m)$ has extra unphysical poles at the edges of the Brillioun zone. 
These extra poles are known in the literature as \textbf{doublers}, and must be eliminated by adjusting the action. We do this by 
adding a corresponding Wilson term to the Dirac operator, so that the full Dirac operator and action now become:
\begin{equation}
	D_f^W(n | m)_{\alpha\beta}^{ab} := \left(m_f + \frac{4}{a}\right)\delta_{\alpha\beta}\delta^{ab}\delta_{nm} - \frac{1}{2a}\sum_{\mu = \pm 1}^{\pm 4} (1 - \gamma^\mu)_{\alpha
	\beta} U_\mu(n)^{ab}\delta_{n + \hat{\mu}, m}~
	\label{eq:dirac_operator}
\end{equation}
\begin{equation}
	S_f[\psi, \bar\psi, U] = a^4\sum_{n, m\in\Lambda}\sum_f \bar\psi_f(n)_{\alpha}^a D_f^W(n | m)_{\alpha\beta}^{ab}\psi_f(m)_{\beta}^b
\end{equation}
We will soon see that the Wilson term makes it much more difficult to deal with chiral symmetry on the lattice than in the continuum, 
and is responsible for many of the interesting topological properties of lattice gauge theories as compared to their continuum counterparts. 

For completeness we will record the glue action as well:
\begin{equation}
	S_{g}[U] = \frac{2}{g^2}\sum_{n\in\Lambda}\sum_{\mu < \nu}\mathbb Re\; tr\{1 - U_{\mu\nu}(n)\}
\end{equation}
where $U_{\mu\nu}(n) := U_\mu(n) U_\nu(n + \hat\mu) U_{-\mu}(n + \hat\mu + \hat\nu) U_{-\nu}(n + \hat\nu)$ is known as a \textbf{plaquette}, 
and performs parallel transport around a closed loop. The partition function for the full theory of lattice QCD is thus:
\begin{equation}
	Z = \int D\psi D\bar\psi D U e^{-S_f - S_g}
\end{equation}
where the measures $D\psi, D\bar\psi, DU$ now contain only a finite amount of of sites to be integrated over. 

\section{Chiral Symmetry and the Ginsparg-Wilson Relation}

Chiral symmetry in standard QCD is realized by taking the approximations that the light quarks are massless. Let $D_{cont} := \gamma^\mu D_\mu + m$ be the continuum 
Dirac operator in Euclidean space. Then for a massless fermion described by $\mathcal L = \bar\psi D_{cont}\psi$, the Lagrangian is invariant under:
\begin{equation}
	\psi\mapsto \exp(i\theta\gamma_5)\psi
\end{equation}
because of the anticommutation relation $\{D_{cont}, \gamma_5\}$. This is the core of chiral symmetry in the continuum because it allows us to split our Lagrangian 
into two pieces with definite chirality:
\begin{equation}
	\mathcal L\supset \bar\psi_L D_{cont}\psi_L + \bar\psi_R D_{cont}\psi_R
\end{equation}
and rotate $\psi_L$ and $\psi_R$ separately, giving the chiral symmetry $SU(N_f)_L\times SU(N_f)_R$, where $N_f$ is the number of (approximately) massless quarks. 

Upon discretization, the relation $\{D, \gamma_5\} = 0$ falls apart because of the Wilson term. Even if we assume our lattice quarks are massless, the piece proportional to 
$1_{\alpha\beta}$ does not anticommute with $\gamma_5$ (just as a mass term does not because it is proportional to $1_{\alpha\beta}$). More generally, it was shown by 
Nielson and Ninomiya \cite{nogo} that \textit{any attempt} to remove the doublers from a lattice regularized theory would result in such a breaking of this anticommutation 
relation, and thus the essence of chiral symmetry in lattice theories must be reformulated. 

In 1998, Luscher \cite{luscher} proposed an alternative symmetry on the lattice that acts as chiral symmetry, and indeed goes into chiral symmetry in the continuum 
limit $a\rightarrow 0$. The modified anticommutation relation is known as the \textbf{Ginsparg-Wilson equation}:
\begin{equation}
	\gamma_5 D + D\gamma_5 = aD\gamma_5 D
\end{equation}
Although this is not satisfied by the Wilson-Dirac operator in Equation~\ref{eq:dirac_operator}, it is satisfied by a different class of Dirac operators on the lattice. One in 
particular was defined by Neuberger \cite{neuberger} and is known as the \textbf{overlap operator}:
\begin{equation}
	D := 
\end{equation}

\section{The Index Theorem and Topological Charge}

%next section is up for grabs

\newpage
\begin{thebibliography}{99}

	\bibitem{nogo}
	A No-Go Theorem For Regularizing Chiral Fermions TODO
	
	\bibitem{luscher}
	Luscher
	
	\bibitem{neuberger}
	Neuberger

\end{thebibliography}

\end{document}