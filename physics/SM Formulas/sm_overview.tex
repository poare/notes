\documentclass[11pt, oneside]{article}   	% use "amsart" instead of "article" for AMSLaTeX format
\usepackage[margin = 1in]{geometry}                		% See geometry.pdf to learn the layout options. There are lots.
\geometry{letterpaper}                   		% ... or a4paper or a5paper or ... 
%\geometry{landscape}                		% Activate for rotated page geometry
%\usepackage[parfill]{parskip}    		% Activate to begin paragraphs with an empty line rather than an indent
\usepackage{graphicx}				% Use pdf, png, jpg, or eps§ with pdflatex; use eps in DVI mode
								% TeX will automatically convert eps --> pdf in pdflatex		
\usepackage{amssymb}
\usepackage{amsmath}
\usepackage[shortlabels]{enumitem}
\usepackage{float}
\usepackage{tikz-cd}
\usepackage{subcaption}
\usepackage{slashed}

\usepackage{amsthm}
\theoremstyle{definition}
\newtheorem{definition}{Definition}[section]
\newtheorem{theorem}{Theorem}[section]
\newtheorem{corollary}{Corollary}[theorem]
\newtheorem{lemma}[theorem]{Lemma}

\newcommand{\N}{\mathbb{N}}
\newcommand{\R}{\mathbb{R}}
\newcommand{\Z}{\mathbb{Z}}
\newcommand{\Q}{\mathbb{Q}}

\usepackage{simpler-wick}

% make arrow superscripts
\DeclareFontFamily{OMS}{oasy}{\skewchar\font48 }
\DeclareFontShape{OMS}{oasy}{m}{n}{%
         <-5.5> oasy5     <5.5-6.5> oasy6
      <6.5-7.5> oasy7     <7.5-8.5> oasy8
      <8.5-9.5> oasy9     <9.5->  oasy10
      }{}
\DeclareFontShape{OMS}{oasy}{b}{n}{%
       <-6> oabsy5
      <6-8> oabsy7
      <8->  oabsy10
      }{}
\DeclareSymbolFont{oasy}{OMS}{oasy}{m}{n}
\SetSymbolFont{oasy}{bold}{OMS}{oasy}{b}{n}

\DeclareMathSymbol{\smallleftarrow}     {\mathrel}{oasy}{"20}
\DeclareMathSymbol{\smallrightarrow}    {\mathrel}{oasy}{"21}
\DeclareMathSymbol{\smallleftrightarrow}{\mathrel}{oasy}{"24}
%\newcommand{\cev}[1]{\reflectbox{\ensuremath{\vec{\reflectbox{\ensuremath{#1}}}}}}
\newcommand{\vecc}[1]{\overset{\scriptscriptstyle\smallrightarrow}{#1}}
\newcommand{\cev}[1]{\overset{\scriptscriptstyle\smallleftarrow}{#1}}
\newcommand{\cevvec}[1]{\overset{\scriptscriptstyle\smallleftrightarrow}{#1}}

\newcommand{\dbar}{d\hspace*{-0.08em}\bar{}\hspace*{0.1em}}

%SetFonts

%SetFonts


\title{Standard Model Overview}
\author{Patrick Oare}
\date{}							% Activate to display a given date or no date

\begin{document}
\maketitle

The Standard Model is a $SU(3)_C\times SU(2)_L\times U(1)_Y$ gauge theory which describes our physical world. It is currently 
our best approximation to the physics that describes our universe, although we know that it does not encapsulate all of physics. 
The SM is (up to a global symmetry, i.e. modulo $\mathbb Z / 6\mathbb Z$) completely characterized by its gauge symmetry 
and matter fields. The SM is typically split up into a few sectors when it is studied.

\textbf{Electroweak theory} is the sector of the Standard Model (SM) that deals with the gauge group $SU(2)_L\times U(1)_Y$. The 
$SU(2)$ piece acts on the left handed fermion fields in the SM, and the $U(1)_Y$ factor is the hypercharge. The unique 
physics in this sector primarily comes from the spontaneous symmetry breaking of $SU(2)_L\times U(1)_Y
\rightarrow U(1)_\mathrm{EM}$. This gives rise to masses for fermions and the gauge bosons of the broken symmetry, 
which are the $W_\mu^\pm$ and $Z_\mu$ bosons. The unbroken symmetry is electromagnetism and manifests at low 
energies (less than the vev of the Higgs), and it is what we 
manifestly see in our everyday life. This symmetry breaking also provides constraints between the masses of the electroweak 
gauge bosons, the vev of the Higgs, and the mass of the Higgs. 

\textbf{Quantum chromodynamics} (QCD) is the SM sector which describes the $SU(3)_C$ factor of the gauge group. 
The SM's fermion content coupled with the nature of the $SU(3)$ gauge theory which makes up QCD gives it some 
rather strange properties that are not seen in other sectors or in QED. First, QCD had \textit{dimensional transmutation}, 
in which a scale $\Lambda_\mathrm{QCD}$ is generated by the theory seemingly out of dimensionless couplings and numbers. 
$\Lambda_\mathrm{QCD}$ is defined by being the scale at which the running coupling $\alpha(\mu)$ diverges, i.e. where the Landau 
pole in $N_f$ flavor QCD is. Secondly, QCD has \textit{asymptotic freedom}; it is non-perturbative at low energies ($\alpha(\mu)$ is 
too large to have a well defined perturbative expansion), but at high energies the $\alpha(\mu)$ flows to zero and becomes small, which 
allows one to compute QCD observables in perturbation theory. The final major feature QCD contains is \textit{confinement}; 
its potential scales as $V(r)\sim r$, and so particles will clump together to minimize this potential: a lone quark can never be found, 
it will always be confined into a bound state with other quarks. These bound states are called \textit{hadrons}, and studying QCD 
reveals a rich spectrum of such particles.

The \textbf{flavor sector} of the SM describes how the different copies of fermion fields interact. Flavor is the quantum number 
of the SM which distinguishes the different species of particles, i.e. which distinguishes the $d$ quark from the $s$ quark and the 
electron from the muon. The interesting physics in the flavor sector comes from quantifying the difference between these 
flavor eigenstates, which the SM couplings are built up from, between the mass eigenstates, which are the physical states which 
propagate from point to point. This manifests itself as a unitary rotation between the mass and flavor bases, and the 
\textit{CKM} matrix $V_\mathrm{CKM}$ describes how different this rotation is for up-type quarks vs. down-type quarks. There 
is also a corresponding analogue for the lepton sector, called the \textit{PMNS} matrix, yet that is not well understood because 
it is directly related to neutrino oscillations. This mixing between the flavor and mass eigenstates allows for flavor-changing decays 
in the SM, and the irremovable phase in the CKM matrix directly leads to CP violation in the electroweak interactions. 

The SM has three generations of fermions, as follows:
\begin{table}[H]
	\centering
	\begin{tabular}{ | c | c | c | c | c | c | }
		\hline
		Generation & $u$-type quark & $d$-type quark & $e$-type lepton &$\nu$-type lepton \\
		\hline
		1 & Up quark, $u$ & Down quark, $d$ & Electron, $e$ & Electron neutrino, $\nu_e$ \\
		\hline
		2 & Charm quark, $c$ & Strange quark, $s$ & Muon, $\mu$ & Muon neutrino, $\nu_\mu$ \\
		\hline
		3 & Top quark, $t$ & Bottom quark, $b$ & Tau, $\tau$ & Tau neutrino, $\nu_\tau$ \\
		\hline
	\end{tabular}
	\caption{Standard Model fermions.}
	\label{table:lattice_details}
\end{table}

Note the mass hierarchy in the SM is more complicated than this generational picture suggests; although $m_u < m_d$, 
we instead have $m_s < m_c$ and $m_b < m_t$ in the second and third generations. The mass hierarchy for all SM particles 
and some of the most common hadrons is:
\begin{table}[H]
	\centering
	\begin{tabular}{ | c | c | c | c | c | c | c | c | }
		\hline
		$\nu$ & e & u & d & s & $\mu$ & $\pi^0$ & $\pi^\pm$ \\
		\hline
		$\approx 0$ & 0.511 MeV & 2.2 MeV & 4.7 MeV & 93 MeV & 110 MeV & 134 MeV & 139 MeV \\
		\hline
	\end{tabular}
		\begin{tabular}{ | c | c | c | c | c | c | c | c | c | }
		\hline
		$p^+$ & $n^0$ & c & $\tau$ & b & W & Z & H & t \\
		\hline
		938 MeV & 939 MeV & 1.3 GeV & 1.8 GeV & 4.7 GeV & 80 GeV & 91 GeV & 125 GeV & 172 GeV \\
		\hline
	\end{tabular}
	\caption{Mass hierarchy of the SM and some light QCD bound states. The pion is $\pi$, neutron is $n^0$, 
	and proton is $p^+$. The units to the left of the $c$ quark are MeV, and the units to the right are GeV. Note that 
	$\Lambda_\mathrm{QCD}\approx 150 - 200\;\mathrm{MeV}$; the particles in the upper row are lighter than $\Lambda_\mathrm{QCD}$, 
	while the particles in the lower row are heavier than it. Masses are sourced from the Particle Data Group's Review of 
	Particle Physics.}
\end{table}

The mass of the pion is a good number to keep in mind for hadronic decay; as the lightest hadron, a process can only decay into hadrons 
if the incoming kinematics is sufficient for pion creation: since quarks must be confined, it is not enough for a process to occur if 
the kinematics simply allows $u$, $d$, or $s$ quark creation.

The structure of the Standard Model is completely determined by the irreps of $SU(3)_c\times SU(2)_L\times 
U(1)_Y$ which the particles transform under. Here, the $N$-dimensional fundamental representation of $SU(N)$ is denoted by 
\textbf{N}, and the irreps of the Lorentz group are denoted by $(j_L, j_R)$ as an irrep of $SO(1, 3)\cong SU(2)\times SU(2)$. 

\begin{table}[H]
	\centering
	\begin{tabular}{ | c | c | c | c | c | }
		\hline
		Particle & SU(3) & SU(2) & U(1) & Lorentz \\
		\hline
		$Q_L = \begin{pmatrix} u_L \\ d_L \end{pmatrix}$ & \textbf{3} & \textbf{2} & $1/6$ & $(1/2, 0)$ \\
		\hline
		$u_R$ & \textbf{3} & 1 & $2/3$ & $(0, 1/2)$ \\
		\hline
		$d_R$ & \textbf{3} & 1 & $-1/3$ & $(0, 1/2)$ \\
		\hline
		$\ell_L = \begin{pmatrix} \nu_L \\ e_L \end{pmatrix}$ & 1 & \textbf{2} & $-\frac{1}{2}$ & $(1/2, 0)$ \\
		\hline
		$e_R$ & 1 & 1 & -1 &  $(0, 1/2)$ \\
		\hline
		$\nu_R$ & 1 & 1 & 0 & $(0, 1/2)$ \\
		\hline
		H & 1 & \textbf{2} & $1/2$ & $(0, 0)$ \\ 
		\hline
	\end{tabular}
	\caption{Charges of the particles in the SM (not including gauge bosons). All of the particles have been 
	seen in nature except for the sterile right-handed neutrino, which may or may not exist. }~
	\label{table:charges}
\end{table}

The gauge pieces of the SM are simply from Yang-Mills theory. Let $B_{\mu}$, $W_{\mu\nu}^a$, and $G_{\mu\nu}^A$ be the 
gauge fields for $U(1)_Y$, $SU(2)_L$, and $SU(3)$ respectively, with $a \in \{1, 2, 3\}$ and $A\in \{1, ..., 8\}$. Denote their 
corresponding field strengths by the same letter, i.e. 
\begin{align}
	W_{\mu\nu} &= \partial_\mu W_{\nu} - \partial_\nu W_\mu - i g [W_\mu, W_\nu] \\
	W_{\mu\nu}^a &= \partial_\mu W_\nu^a - \partial_\nu W_\mu^a + g f^{abc} W_\mu^b W_\nu^c
\end{align}
and let the covariant derivative be $D_\mu$, which acts on a field $\phi$ as:
\begin{equation}
	D_\mu\phi = \partial_\mu\phi - ig' B_\mu \phi - ig W_\mu^a t^a \phi - ig_3 G_\mu^A T^A\phi
\end{equation}
where $A$ sums over the gauge fields. Note that $t^a \phi$ and $T^A\phi$ will change based on what representation $\phi$ 
is in; if $\phi = \phi^i$ lives in the fundamental representation, then $t^a\phi = (t^a)^{ij}\phi^j$ where $t^a$ ($T^A$) is 
represented by half the Pauli (Gell-Mann) matrices, but if $\phi = \phi^a$ lives in the adjoint, then $T^a \phi = [T^a, \phi^b T^b] 
= if^{abc} \phi^bT^c$, i.e. $D_\mu\phi^a = \partial_\mu\phi^a + g f^{abc} A_\mu^b \phi^c$. 

Given this setup, the Standard Model Lagrangian is typically split up into four parts:
\begin{equation}
	\mathcal L_\mathrm{SM} = \mathcal{L}_\mathrm{Gauge} + \mathcal{L}_\mathrm{Fermi} + \mathcal{L}_\mathrm{Higgs} + 
	\mathcal{L}_\mathrm{Yukawa} + \mathcal{L}_{\nu_R}
\end{equation}
Each of these sectors are relatively self-explanatory. We have:
\begin{align}
	\mathcal{L}_\mathrm{Gauge} &= -\frac{1}{4} B_{\mu\nu} B^{\mu\nu} - \frac{1}{4} W_{\mu\nu}^a W^{\mu\nu a} - \frac{1}{4} 
	G_{\mu\nu}^A G^{\mu\nu A} + \theta_\mathrm{QCD} \epsilon^{\mu\nu\alpha\beta} G_{\mu\nu} G_{\alpha\beta} \\
	\mathcal{L}_\mathrm{Fermi} &= i\sum_\psi\overline\psi \slashed D \psi = i\sum_{\psi_L} \overline\psi_L 
	\overline\sigma^\mu D_\mu\psi_L + i\sum_{\psi_R} \sigma^\mu D_\mu \psi_R \\
	\mathcal{L}_\mathrm{Higgs} &= D_\mu H D^\mu H^\dagger + \mu^2 H^\dagger H - \lambda (H^\dagger H)^2 \\
	\mathcal{L}_\mathrm{Yukawa} &= - Y_{ij}^d \overline Q_L^i H d_R^j - Y_{ij}^u \overline Q_L^i\epsilon H^* u_R^j - 
	Y_{ij}^e \overline\ell_L^i H e_R^j~\label{eq:yukawa} \\
	\mathcal{L}_{\nu_R} &= -Y_{ij}^\nu \overline \ell_L\epsilon H^* \nu_R^j - i M_{ij} (\nu_R^i)^c \nu_R^j + h.c.
\end{align}
Here $\slashed D\psi = \overline\sigma^\mu D_\mu \psi_L + \sigma^\mu D_\mu \psi_R$ with $\sigma^\mu = (1, 
\sigma^i)$ and $\overline\sigma^\mu = (1, -\sigma^i)$, $\epsilon^{ab} = i\sigma^2 = \begin{pmatrix} 0 & 1 \\ -1 & 0 \end{pmatrix}$, 
and 
\begin{equation}
	H = \frac{1}{\sqrt{2}}\begin{pmatrix} \phi^+ \\ \phi^0 \end{pmatrix}
\end{equation}
We will later gauge transform $H$ to unitary gauge to make more apparent where the physical Higgs boson is. A few 
comments about this Lagrangian:
\begin{enumerate}
	\item \textbf{The SM Lagrangian does not have any explicit mass terms for the fermion}. A Dirac mass term of the form 
	$m_q \overline q q = m_q(\overline q_R q_L + \overline q_L q_R)$ violates $SU(2)_L$ symmetry since $q_L$ and $q_R$ 
	transform differently, and a Majorana mass term which goes as $mqq = m(\epsilon^{ab} q_{L, a} q_{L, a} + \epsilon_{\dot a\dot b} 
	q_R^{\dot a} q_R^{\dot b})$ also violates $SU(2)_L$ symmetry. 
	\item \textbf{Transformation properties of the Higgs}: The Yukawa couplings must be singlets under $SU(2)$ and $U(1)$. 
	To verify the $U(1)$ properties, one can simply add the hypercharges. The $SU(2)$ properties are harder; for the 
	up quark term, the hypercharge cancellation means we need the antiparticle field for $Q_L$ and $H$. To make an 
	$SU(2)$ singlet, one must notice the transformation properties of the $\epsilon$ tensor under $U\in SU(2)$:
	\begin{equation}
		\epsilon U\epsilon = - U^*\implies U\epsilon = \epsilon U^*
	\end{equation}
	as $\epsilon^2 = -1$. Now if $A, B$ are both fields in \textbf{2} of SU(2), then:
	\begin{equation}
		A^\dagger \epsilon B^* \mapsto A^\dagger U^\dagger \epsilon U^*  B^* = A^\dagger U^\dagger U \epsilon B^* = 
		A^\dagger \epsilon B^*
	\end{equation}
	hence we see that $A^\dagger \epsilon B^*$ is a singlet under SU(2). Equivalently, we can use $\epsilon^{ab}$ or $\epsilon_{ab}$ to 
	contract the color indices in $Q_L^*$ and $H^*$ into a singlet.
	\item \textbf{Parameters in the SM}: Once the gauge symmetry is specified and the charges of the 
	fermions are set, the theory is not yet complete. It needs experimental input in the form of input parameters like the coupling of 
	each force and masses of some of the particles. There are 19 independent parameters in the Standard Model:
\begin{enumerate}
	\item Gauge couplings (3 parameters).
	\item Fermion masses (9 parameters).
	\item Higgs vev $v$ and mass $m_H$ (2 parameters). 
	\item Angles $\theta_{12}, \theta_{13}, \theta_{23}$ in the CKM matrix (3 parameters).
	\item Irremovable phase $\delta$ in the CKM matrix (1 parameter). 
	\item $\theta_\mathrm{QCD}\approx 0$, which is the \textbf{strong CP problem} (1 parameter).
\end{enumerate}
\end{enumerate}
Here are some of the current problems with the Standard model. 
\begin{itemize}
	\item Neutrino masses: We know that neutrino masses exist because neutrino oscillations have been discovered, but we don't know 
	the nature of the neutrino. There are two main ways to incorporate neutrino masses into the SM, and they depend on the nature of the 
	neutrino. If the neutrino is a Majorana particle, then neutrino masses can be added via a dimension-5 operator:
	\begin{equation}
		\Delta\mathcal L^{(1)}_\mathrm{mass} = \frac{c_5}{\Lambda} \epsilon^{ij}(\epsilon^{ab} \ell_{ia} H_b) (\epsilon^{cd} \ell_{jc} H_d)
	\end{equation}
	where the color and spinor indices are contracted with $\epsilon^{ij}$. On the other hand, if the neutrino is a Dirac particle, we can 
	add in right-handed neutrinos as a field $\nu_R$ (see Table~\ref{table:charges}), and the neutrino will gain a Dirac mass via a new 
	Yukawa coupling:
	\begin{equation}
		\Delta\mathcal L^{(2)}_\mathrm{mass} = (Y_\nu)_{ij} H\ell_i \nu_j^\dagger + h.c. 
	\end{equation}
	Neutrinoless double $\beta$ decay is a process which may be able to tell us the nature of the neutrino; if this process is observed, 
	the neutrino must be able to annihilate itself and will therefore be its own antiparticle, i.e. we would have evidence the neutrino is 
	a Majorana particle. 
	\item Strong CP: The strong CP problem is the question of why (to the precision of current experiments) the CP violating term 
	in the QCD Lagrangian, $\theta G\wedge G$, vanishes. There is no reason this term should not be present in 
	$\mathcal L_\mathrm{SM}$ and it would generate CP violating interactions, yet no one has seen CP violation in nature in the QCD 
	sector. A common solution to this is the \textbf{QCD axion}, which adds in a field to dynamically set the CP violating coupling to zero, 
	ensuring that $\theta_\mathrm{QCD}$ is zero. 
	\item Hierarchy (fine tuning): Fine-tuning problems are related to relative sizes of quantities. They often appear when discussing the 
	size of loop effects: typically there is no reason to assume that loop effects are small, and when loops are taken into account to 
	compute quantities that we have measured to be small, there must be some precise calculation of loops that allow for this. 
	See https://en.wikipedia.org/wiki/Hierarchy\_problem for more detail.
\end{itemize}

\end{document}