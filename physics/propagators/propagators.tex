\documentclass[11pt, oneside]{article}   	% use "amsart" instead of "article" for AMSLaTeX format
\usepackage[margin = 1in]{geometry}                		% See geometry.pdf to learn the layout options. There are lots.
\geometry{letterpaper}                   		% ... or a4paper or a5paper or ... 
%\geometry{landscape}                		% Activate for rotated page geometry
%\usepackage[parfill]{parskip}    		% Activate to begin paragraphs with an empty line rather than an indent
\usepackage{graphicx}				% Use pdf, png, jpg, or eps§ with pdflatex; use eps in DVI mode
								% TeX will automatically convert eps --> pdf in pdflatex		
\usepackage{amssymb}
\usepackage{amsmath}
\usepackage[shortlabels]{enumitem}
\usepackage{float}
\usepackage{tikz-cd}

\usepackage{amsthm}
\theoremstyle{definition}
\newtheorem{definition}{Definition}[section]
\newtheorem{theorem}{Theorem}[section]
\newtheorem{corollary}{Corollary}[theorem]
\newtheorem{lemma}[theorem]{Lemma}

\newcommand{\N}{\mathbb{N}}
\newcommand{\R}{\mathbb{R}}
\newcommand{\Z}{\mathbb{Z}}
\newcommand{\Q}{\mathbb{Q}}

%SetFonts

%SetFonts


\title{Propagators}
\author{Patrick Oare}
\date{}							% Activate to display a given date or no date

\begin{document}
\maketitle

The propagator is simply the inverse of the quadratic term in the Lagrangian. Let's do a few examples.

\section{Abelian case (photon propagator)}

After applying Fadeev-Popov to fix the gauge, we have the gauge-fixed Lagrangian:
\begin{equation}
	\mathcal L = -\frac{1}{4} F_{\mu\nu}F^{\mu\nu} + \overline\psi (i\gamma^\mu D_\mu - m)\psi -\frac{1}{2\xi} (\partial_\mu 
	A^\mu)^2
\end{equation}
We consider the parts of $\mathcal L$ which are pure gauge, and we expand them out, integrating by parts to make this more 
compact:
$$
	\mathcal L_{gauge} = -\frac{1}{4} (\partial_\mu A_\nu - \partial_\nu A_\mu) (\partial^\mu A^\nu - \partial^\nu A^\mu) - 
	\frac{1}{2\xi}(\partial_\mu A^\mu)(\partial_\nu A^\nu)
$$
$$
	= \frac{1}{2} A^\mu \left(g_{\mu\nu}\partial^2 - \left(1 - \frac{1}{\xi}\right) \partial_\mu\partial_\nu\right) A^\nu
$$
\begin{equation}
	= \frac{1}{2}A^\mu D_{\mu\nu} A^\nu
\end{equation}
where we have defined the operator $D_{\mu\nu}$ by:
\begin{equation}
	D_{\mu\nu} := g_{\mu\nu}\partial^2 - \left(1 - \frac{1}{\xi}\right)\partial_\mu\partial_\nu
\end{equation}

The differential operator $D_{\mu\nu}$ is the term which we will need to invert (i.e. we need to find a Green's function for 
$D_{\mu\nu}$) by solving the equation:
\begin{equation}
	D_{\mu\nu}\Pi^{\nu\alpha}(x) = \delta^\alpha_\mu \delta^4(x)
\end{equation}
The Green's function $\Pi^{\mu\nu}(x)$ is the \textbf{propagator}. We solve this by taking it to momentum space:
\begin{equation}
	\left(g_{\mu\nu}k^2 - \left(1 - \frac{1}{\xi}\right) k_\mu k_\nu\right)\tilde\Pi^{\nu\alpha} = \delta^\alpha_\mu
\end{equation}
One can then verify the result in any QFT textbook works for $\tilde\Pi^{\mu\nu}$. The easiest way to invert this equation is to 
write it as a $4\times 4$ matrix in Lorentz space, then find the inverse of the matrix. 

\section{Non-abelian case}

\end{document}