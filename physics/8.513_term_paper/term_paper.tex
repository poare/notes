\documentclass[11pt, oneside]{article}   	% use "amsart" instead of "article" for AMSLaTeX format
\usepackage[margin = .5in]{geometry}                		% See geometry.pdf to learn the layout options. There are lots.
\geometry{letterpaper}                   		% ... or a4paper or a5paper or ... 
%\geometry{landscape}                		% Activate for rotated page geometry
%\usepackage[parfill]{parskip}    		% Activate to begin paragraphs with an empty line rather than an indent
\usepackage{graphicx}				% Use pdf, png, jpg, or eps§ with pdflatex; use eps in DVI mode
								% TeX will automatically convert eps --> pdf in pdflatex		
\usepackage{amssymb}
\usepackage{amsmath}
\usepackage[shortlabels]{enumitem}
\usepackage{float}
\usepackage{tikz-cd}

\usepackage{slashed}
\usepackage{amsthm}
\theoremstyle{definition}
\newtheorem{definition}{Definition}[section]
\newtheorem{theorem}{Theorem}[section]
\newtheorem{corollary}{Corollary}[theorem]
\newtheorem{lemma}[theorem]{Lemma}

\newcommand{\N}{\mathbb{N}}
\newcommand{\R}{\mathbb{R}}
\newcommand{\Z}{\mathbb{Z}}
\newcommand{\Q}{\mathbb{Q}}

%SetFonts

%SetFonts


\title{8.513 Term Paper: The Index Theorem in Lattice QCD}
\author{Patrick Oare}
\date{December 10th, 2019}							% Activate to display a given date or no date

\begin{document}
\maketitle

\section{Introduction}
Quantum Chromodynamics (QCD) is the theory of the strong nuclear force. It is a $SU(3)$ gauge theory coupled to 6 flavors of 
fermions-- the quarks. At low energies, the coupling of QCD is large and perturbation theory cannot be applied to the theory, 
which means physics must be extracted by non-perturbative means. One such way to do this is to formulate QCD as a 
\textbf{lattice gauge theory} by discretizing spacetime-- the advantage of this is that the path integral becomes a finite (albeit 
large) dimensional integral which can be evaluated on a computer. 

One of the key properties of QCD in the continuum is an approximate chiral symmetry; when we take the light quarks ($u, d$, 
sometimes $s$) to be massless, the QCD Lagrangian decouples into left and right handed fields of definite chirality. This 
allows us to study QCD in many ways which do not require perturbation theory, and allows for a definition of topological charge 
in QCD. However, when QCD is put on the lattice, chiral symmetry appears broken and the original continuum definition of a 
topological charge is no longer a good quantity to consider. In this paper, I will discuss how chiral symmetry can be modified to 
generate a new definition of topological charge that is valid on the lattice. 

\section{QCD on the Lattice}
This section will outline the background we need to study symmetries on the lattice; specifically, we will define our theory and 
the operators of relevance to us. We begin by making some initial definitions: denote our spacetime lattice with spacing $a$ by 
$\Lambda$, and Wick rotate to imaginary time so that $\Lambda$ is a Euclidean lattice\footnote{We must Wick rotate so that 
the Boltzmann factor of $e^{iS}$ in the path integral becomes a valid probability density $e^{-S}$, in order to perform any 
computations at all.}. In the full theory of QCD, the dynamical fields in the path integral are the quark fields $\psi_f$ and gluon 
fields $A_\mu$. When we discretize QCD, we will still work with the quark fields, but instead of directly working with the gauge 
fields $A_\mu$ we will work \textbf{link fields} $U_\mu(n)$ which transform in the following way under a gauge transformation 
$\Omega : \Lambda\rightarrow SU(3)$:
\begin{equation}
	U_\mu(n)\xrightarrow{\Omega(n)}\Omega(n) U_\mu(n)\Omega(n + \hat{\mu})^\dagger
\end{equation}
Here $\hat{\mu}$ is the unit vector in the $\mu$ direction, and $n\in\Lambda$ denotes a site in the lattice. The link fields can be 
taken to be $U_\mu(n) = \exp(ia A_\mu(n))$, so are intimately related to the gauge field, and take values (as they must) in 
$SU(3)$. With this transformation law, the link fields act as a connection between the fibers at different points in $\Lambda$:
\begin{equation}
	U_\mu(n)\psi(n + \hat{\mu})\xrightarrow{\Omega} \Omega(n) U_\mu(n)\psi(n + \hat{\mu})
\end{equation}
This transformation means that $U_\mu(n)\psi(n + \hat\mu)$ is valued in the fiber at point $n$, and so can directly be 
compared with $\psi(n)$. This allows us to add and subtract fermion fields at different points in a gauge invariant way, and so 
define a covariant derivative. 

We may now write down a first pass at a fermion action, which will be equivalent to $\slashed D + m$ upon taking the 
continuum limit. The direct discretization of this action is thus:
\begin{align}
	S_f^0[\psi_f, \bar\psi_f, U] &= a^4\sum_{n\in\Lambda}\sum_f \bar\psi_f(n)\left(\gamma^\mu\frac{U_\mu(n)\psi_f(n + \hat\mu) - U_{-\mu}(n)\psi_f(n - \hat\mu)}{2a} - m_f\psi_f(n)\right) 
	\\
	&= a^4\sum_{n, m\in\Lambda}\sum_f \bar\psi_f(n)_{\alpha}^a D_{f}^0(n | m)_{\alpha\beta}^{ab}\psi_f(m)_{\beta}^b~
	\label{eq:naive_action}
\end{align}
where we have defined the \textbf{Dirac operator} $D_{\alpha\beta}^{ab}(n | m)$ to be:
\begin{equation}
	D_f^0(n | m)_{\alpha\beta}^{ab} := (\gamma^\mu)_{\alpha\beta} \left(\frac{U_\mu(n)^{ab} \delta_{n + \hat\mu, m} - U_{-\mu}(n)^{ab} \delta_{n - \hat{\mu}, m}}{2a}\right) 
	+ m_f\delta_{\alpha\beta}\delta^{ab}\delta_{nm}
\end{equation}
Note the Greek indices $\alpha, \beta$ are in Dirac space, and the Latin indices $a, b$ are in color space. The Dirac operator 
is one of the fundamental objects we will study later when considering the role of topology in lattice QCD, and is the discretized 
version of $i\slashed D - m$ in Euclidean space. 

However, there is a slight problem with the action in Equation~\ref{eq:naive_action}. Because we are working on a lattice, the 
Fourier transform $\tilde D^0(p)$ of the Dirac operator $D^0(n | m)$ has extra unphysical poles at the edges of the Brillioun 
zone. These extra poles are known in the literature as \textbf{doublers}, and must be eliminated by adjusting the action. We do 
this by adding a corresponding Wilson term to the Dirac operator, so that the full Dirac operator and action now become:
\begin{equation}
	D_f^W(n | m)_{\alpha\beta}^{ab} := \left(m_f + \frac{4}{a}\right)\delta_{\alpha\beta}\delta^{ab}\delta_{nm} - 
	\frac{1}{2a}\sum_{\mu = \pm 1}^{\pm 4} (1 - \gamma^\mu)_{\alpha\beta} U_\mu(n)^{ab}\delta_{n + \hat{\mu}, m}~
	\label{eq:dirac_operator}
\end{equation}
\begin{equation}
	S_f[\psi, \bar\psi, U] = a^4\sum_{n, m\in\Lambda}\sum_f \bar\psi_f(n)_{\alpha}^a D_f^W(n | m)_{\alpha\beta}^{ab}\psi_f(m)_{\beta}^b
\end{equation}
We will soon see that the Wilson term makes it much more difficult to deal with chiral symmetry on the lattice than in the 
continuum, and is responsible for many of the interesting topological properties of lattice gauge theories as compared to their 
continuum counterparts. 

For completeness we will record the glue action as well:
\begin{equation}
	S_{g}[U] = \frac{2}{g^2}\sum_{n\in\Lambda}\sum_{\mu < \nu}\mathbb Re\; tr\{1 - U_{\mu\nu}(n)\}
\end{equation}
where $U_{\mu\nu}(n) := U_\mu(n) U_\nu(n + \hat\mu) U_{-\mu}(n + \hat\mu + \hat\nu) U_{-\nu}(n + \hat\nu)$ is known as a 
\textbf{plaquette}, and performs parallel transport around a closed loop. The partition function for the full theory of lattice QCD 
is thus:
\begin{equation}
	Z = \int D\psi D\bar\psi D U e^{-S_f - S_g}
\end{equation}
where the measures $D\psi, D\bar\psi, DU$ now contain only a finite amount of of sites to be integrated over. 

\section{Chiral Symmetry and the Ginsparg-Wilson Relation}
Chiral symmetry in standard QCD is realized by taking the approximations that the light quarks are massless. 
Let $D_{cont} := \gamma^\mu (\partial_\mu + i A_\mu)$ be the continuum Dirac operator in Euclidean space for a massless 
field. The key equation for chiral symmetry in the continuum is the anticommutation of $D_{cont}$ with $\gamma_5$:
\begin{equation}
	\{D, \gamma_5\} = 0~
	\label{eq:cont_anticommutation}
\end{equation}
Using Equation~\ref{eq:cont_anticommutation}, it is immediate to show that for a fermion described by $\mathcal L = 
\bar\psi D_{cont}\psi$, the Lagrangian is invariant under:
\begin{equation}
	\psi\mapsto \exp(i\theta\gamma_5)\psi
\end{equation}
Furthermore, using the projector $P_\pm = \frac{1\pm\gamma_5}{2}$, we can split a fermion field $\psi$ into two pieces with 
definite chirality $\psi_L = P_-\psi$, $\psi_R = P_+\psi$ which rotate into themselves under chiral rotations. Note $P_\pm 
P_\mp = 0$, and that $DP_\pm = P_\mp D$ due to Equation~\ref{eq:cont_anticommutation}. Using these relations, we 
see that the chiral fields decouple in the Dirac Lagrangian and make chiral symmetry manifest:
\begin{equation}
	\mathcal L = \bar\psi_L D_{cont}\psi_L + \bar\psi_R D_{cont}\psi_R~
	\label{eq:cont_chiral_lagrangian}
\end{equation}
Because of this decoupling, rotating $\psi_L$ and $\psi_R$ separately in flavor space\footnote{Flavor space is where we view 
$\psi$ as a vector of different species of quarks. For example, we will generally either view $\psi_i = (u_i\; d_i)^T$ or 
$\psi_i = (u_i\; d_i\; s_i)^T$ when we examine the massless quarks in the theory, depending on whether or not we approximate 
the strange quark as massless.} leaves the Lagrangian invariant and gives us the chiral symmetry $SU(N_f)_L\times 
SU(N_f)_R$, where $N_f$ is the number of (approximately) massless quarks. 

Upon discretization, the relation $\{D, \gamma_5\} = 0$ falls apart because of the Wilson term. Even if we assume our lattice 
quarks are massless, the piece proportional to $\delta_{\alpha\beta}$ does not anticommute with $\gamma_5$. More generally, 
it was shown by Nielson and Ninomiya \cite{nogo} that \textit{any attempt} to remove the doublers from a lattice regularized 
theory would result in such a breaking of this anticommutation relation, and thus the essence of chiral symmetry in lattice 
theories must be reformulated. 

Ginsparg and Wilson \cite{ginsparg} proposed an alternative symmetry on the lattice that acts as chiral symmetry, and indeed 
goes into chiral symmetry in the continuum limit $a\rightarrow 0$. The modified anticommutation relation is known as the 
\textbf{Ginsparg-Wilson equation}:
\begin{equation}
	\gamma_5 D + D\gamma_5 = aD\gamma_5 D~
	\label{eq:ginsparg_wilson}
\end{equation}
Although this is not satisfied by the Wilson-Dirac operator in Equation~\ref{eq:dirac_operator}, it is satisfied by a different class 
of Dirac operators on the lattice. One in particular was defined by Neuberger \cite{neuberger} and is known as the 
\textbf{overlap operator}:
\begin{equation}
	D_{over} = \frac{1}{a}\left(1 - A (A^\dagger A)^{-1/2}\right)\;\;\;\;\;\;\;\;\;\;\;\;\;\;\;\;\;\;\;\;\;\;\;\;\;\;\;\;\;\;\;\;\;\;\;\;\;\;\;\;\;\;  
	A = 1 - aD^W
\end{equation}
For the duration of this paper, we will take $D_f(n | m)_{\alpha\beta}^{ab}$ to be a lattice Dirac operator which satisfies 
Equation~\ref{eq:ginsparg_wilson}.

L�scher \cite{L�scher} showed that because $D$ satisfies the Ginsparg-Wilson equation, the Lagrangian density is invariant 
under a modified chiral rotation of the field $\psi$:
\begin{equation}
	\psi\mapsto \exp\left(ia\gamma_5 \left(1 - \frac{1}{2}a D\right)\right)\psi
	\;\;\;\;\;\;\;\;\;\;\;\;\;\;\;\;\;\;\;\;\;\;\;\;\;\;\;\;\;\;\;\;\;\;\;\;\;\;\;\;\;\; \bar\psi\mapsto\bar\psi 
	\exp\left(ia\left(1 - \frac{1}{2}aD\right)\gamma_5\right)~
	\label{eq:L�scher_transformation}
\end{equation}
because taking $a$ to be an small parameter and expanding in powers of $a$, we find that the extra terms obey the 
Ginsparg-Wilson equation order by order and cancel:
\begin{equation}
	\mathcal L \mapsto \bar\psi D\psi + ia\bar\psi\left[\gamma_5 D + D\gamma_5 - a D\gamma_5 D\right]\psi + O(a^3) = 
	\mathcal L + O(a^3)
\end{equation}
hence we can take the Noether current generated by this symmetry to be our new definition of a chiral current. Note that the 
transformation in Equation~\ref{eq:L�scher_transformation} reduces to the standard chiral transformation in the limit 
$a\rightarrow 0$, and so this symmetry become chiral symmetry in the continuum limit. 

To make manifest the modified chiral symmetry of the Lagrangian, we follow an argument by Niedermayer \cite{niedermayer} 
and introduce modified chiral projectors which act as the standard projectors $\frac{1\pm\gamma_5}{2}$ in the continuum limit:
\begin{equation}
	\hat\gamma_5 := \gamma_5 (1 - a D)\;\;\;\;\;\;\;\;\;\;\;\;\;\;\;\;\;\;\;\;\;\;\;\;\;\;\;\;\;\;\;\;\;\;\;\;\;\;\;\;\;\; \hat P_\pm := 
	\frac{1\pm\hat\gamma_5}{2}
\end{equation}
Because $D$ satisfies Equation~\ref{eq:ginsparg_wilson}, these new projectors satisfy $D\hat P_\pm = P_\mp D$, which is a 
very similar algebra to the standard chiral projectors. This means that if we introduce modified chiral fields:
\begin{equation}
	\psi_L = \hat P_-\psi ;\;\;\;\;\;\;\;\;\;\;\;\;\;\;\;\;\;\;\;\;\; \psi_L = \hat P_+\psi ;\;\;\;\;\;\;\;\;\;\;\;\;\;\;\;\;\;\;\;\;\; \bar\psi_L = \bar\psi 
	P_+ ;\;\;\;\;\;\;\;\;\;\;\;\;\;\;\;\;\;\;\;\;\; \bar\psi_R = \bar\psi P_-
\end{equation}
then our Lagrangian splits into chiral components $\mathcal L = \bar\psi_L D\psi_L + \bar\psi_R D\psi_R$ as in the continuum 
case. 

While this composition successfully decouples the Lagrangian into chiral components, there is a major difference between 
this decomposition and the continuum case in Equation~\ref{eq:cont_chiral_lagrangian}. In the continuum, \textit{chirality is 
local} since chiral fields at one point are direct projections of the original fields at the same point. In other words, $\psi_L(x)$ 
only depends on the value of $\psi(x)$. On the lattice, this is no longer true-- \textit{chirality is non-local} because the projectors 
$\hat P_\pm$ contain a copy of $D$; this means that $\psi_L(x)$ will contain information from \textbf{other spacetime points} 
$\psi_L(y)$ for which $D(x | y)$ is nonzero. Thus, we have chiral symmetry on the lattice, but it comes at a cost. 

% TODO talk about chiral anomaly on the lattice? In L�scher's paper

\section{The Index Theorem and Topological Charge}
\label{sec:charge}

We now turn to other implications the Ginsparg-Wilson equation has for fermions on the lattice-- namely, a consequence of this 
relation is that there is a well defined topological charge. To define a topological charge, we must first consider the spectrum of 
the Dirac operator $D$. Neidermayer's paper shows a few important results that we will need:
\begin{enumerate}
	\item $D$ is a normal operator, and hence its eigenvectors form an orthonormal basis of our Hilbert space.
	\item The spectrum of $D$ is a circle in the complex plane of radius $\frac{1}{a}$ centered at $\lambda = \frac{1}{a}$. 
	\item If $\lambda\notin\mathbb R$ is an eigenvalue with eigenvector $u(\lambda)$, then $\langle u(\lambda), \gamma_5 
	u(\lambda)\rangle = 0$. 
	\item An eigenstate has definite chirality (i.e. is an eigenstate of $\gamma_5$) iff its corresponding eigenvalue is real. 
\end{enumerate}
All of these arguments are detailed in Appendix~\ref{subsec:spectrum}. Note we adopt the notation here that $\langle\cdot, 
\cdot\rangle$ is the inner product on Dirac spinors, i.e. $\langle u, v\rangle = u^\dagger v$ where the transpose is taken in 
Dirac space. This implies that we can describe the left and right movers in our lattice regularized theory by describing the 
eigenspaces of $D$ which have real eigenvalues. Because the distribution of eigenvalues lies on a circle, we therefore know 
the left and right movers must have exactly eigenvalues $\lambda = 0$ or $\lambda = \frac{2}{a}$. 

We can use this to define a topological charge on the lattice, following Hasenfratz's paper \cite{hasenfratz}. We define a 
topological charge by:
\begin{equation}
	Q_{lat} := \frac{1}{2}a\; tr\{\gamma_5 D\}
\end{equation}
where the trace is taken over color and Dirac indices, and over lattice sites as well\footnote{Note that given a Dirac operator $D$ 
satisfying $\{\gamma_5, D\} = aD\gamma_5 D$, we get a topological charge. This implies there are many different definitions of 
topological charge on the lattice, one for each such Dirac operator. In the continuum, we only have one definition of 
charge, as $q(x) = \epsilon_{\mu\nu\rho\sigma} F^{\mu\nu a}F^{\rho\sigma a}$. Evidently, these different definitions of charge from 
different values of $D$ must be similar enough that they go into the same continuum limit.}. This must have an integral value, which 
is equal to the difference in the number of left movers and right movers in our theory. To show this, we have:
\begin{equation}
	Q_{lat} = \frac{a}{2}\; tr\{\gamma_5 D\} = -\frac{1}{2}\; \left[2tr\{\gamma_5\} - tr\{a\gamma_5 D\}\right] = 
	-\frac{1}{2}\; tr\{\gamma_5 (2 - aD)\}
\end{equation}
as $\gamma_5$ is traceless. Because $D$ is normal, we can evaluate the trace by summing over its eigenbasis 
$\{u(\lambda)\}_\lambda$ for $\lambda$ an eigenvalue of $D$ as follows:
\begin{equation}
	Q_{lat} = -\frac{1}{2}\sum_\lambda\langle u(\lambda), \gamma_5(2 - aD) u(\lambda)\rangle = 
	-\frac{1}{2}\sum_\lambda (2 - a\lambda) \langle u(\lambda), \gamma_5 u(\lambda)\rangle
\end{equation}
The sum here restricts to $\lambda = 0, \frac{2}{a}$ by property $4$ listed above. Furthermore, because the $\lambda = 
\frac{2}{a}$ term cancels in the sum, we can express this directly in terms of the number of zero modes with left and right 
chirality. If we let $n_L, n_R\in\mathbb Z$ be the number of such modes, then the sum evaluates directly to:
\begin{equation}
	Q_{lat} = n_L - n_R
\end{equation}
because $\gamma_5 u(0) = \pm u(0)$, with $+$ for right chirality and $-$ for left chirality. Because $n_L - n_R$ must take 
integral values, it cannot be varied smoothly, and must be a topological invariant of the field configuration we are using. 
Thus, we have shown the existence of a well defined topological charge on the lattice!

\section{Conclusion}

This term paper has detailed the formulation of chiral symmetry on the lattice, and how it can lead to the definition of a 
topologically invariant charge. We first defined our quantities of interest on the lattice. We saw that the discretization of 
spacetime led to the extra unwanted particles called doublers, and we fixed this problem by adding a gauge invariant 
term to the lattice Dirac operator. 

Unfortunately, adding a term to fix the doublers broke the continuum notion of chiral symmetry, and so we studied a 
new formulation of chiral symmetry on the lattice. In particular, although the Wilson term forced the anticommutation 
$\{\gamma_5, D\}$ to be nonzero, a particular class of operators satisfying the Ginsparg-Wilson equation $\{\gamma_5, D\} = 
aD\gamma_5 D$ allow for a remnant of chiral symmetry to be seen in the lattice gauge theory. 

Finally, we proceeded to study the spectrum of Dirac operators which satisfy this relation. We saw that the spectrum of such 
operators formed a circle in the complex plane. The spectrum allows for zero modes, and these zero modes can be chosen 
to be eigenstates of $\gamma_5$ with definite chirality. As a result of this, an invariant topological charge was 
defined, and this charge reproduces the standard topological charge in continuum QCD as the lattice spacing is removed. 

\newpage
\section{Appendix}

\subsection{Spectrum of the Dirac Operator}
\label{subsec:spectrum}

Here we detail the claims made in Section~\ref{sec:charge} about the spectrum of the Dirac operator. We will show that:
\begin{enumerate}
	\item $D$ is a normal operator, and hence its eigenvectors form an orthonormal basis of our Hilbert space.
	\item The spectrum of $D$ is a circle in the complex plane of radius $\frac{1}{a}$ centered at $\lambda = \frac{1}{a}$. 
	\item If $\lambda\notin\mathbb R$ is an eigenvalue with eigenvector $u(\lambda)$, then $\langle u(\lambda), \gamma_5 
	u(\lambda)\rangle = 0$. 
	\item An eigenstate has definite chirality (i.e. is an eigenstate of $\gamma_5$) iff its corresponding eigenvalue is real. 
\end{enumerate}

First, note that $D$ is $\gamma_5$-hermitian, which means that $\gamma_5 D\gamma_5 = 
D^\dagger$. We can multiply Equation~\ref{eq:ginsparg_wilson} on the right by $\gamma_5$ and use $\gamma_5$-hermicity to show that 
$D^\dagger + D = aDD^\dagger$, and multiplying on the left with $\gamma_5$ gives us $D + D^\dagger = aD^\dagger D$. Equating 
these two equations, we see that:
\begin{equation}
	[D, D^\dagger] = 0
\end{equation}
which is the definition of a \textit{normal operator}, and thus the eigenspaces of $D$ are orthogonal. 

Now, introduce the operator $V := 1 - a D$. Then direct computation shows $V^\dagger V = 1 - a(D + D^\dagger - a D^\dagger 
D) = 1$ because of the equations we just derived, hence $V$ is unitary. This implies that the eigenvalues of $V$ have unit 
norm and can be parameterized as $e^{i\theta}$ for $\theta\in [-\pi, \pi)$. So, the eigenvalues of $D$ must be of the form:
\begin{equation}
	\lambda = \frac{1}{a}\left(1 - e^{i\theta}\right)
\end{equation}
and must therefore lie on the circle in the complex plane of radius $\frac{1}{a}$ centered at $\frac{1}{a}$. In particular, notice 
that there are only two real eigenvalues, one at $\lambda = 0$ and one at $\lambda = \frac{2}{a}$. It turns out that the real 
eigenvalues are the only ones whose eigenvectors may have definite chirality. Suppose we have an eigenvalue $\lambda$ of 
$D$ with corresponding nonzero eigenvector $u = u(\lambda)$. Then:
\begin{equation}
	\lambda\langle u, \gamma_5 u\rangle = \langle u, \gamma_5 Du\rangle = \langle u, D^\dagger\gamma_5 u\rangle = \langle Du, \gamma_5 u\rangle = \lambda^*
	\langle u, \gamma_5 u\rangle\iff \mathbb{I}m\{\lambda\} = 0\textnormal{ or } \langle u, \gamma_5 u\rangle = 0
\end{equation}
Hence we see that if $\lambda\notin\mathbb R$, $u(\lambda)$ \textbf{cannot} have definite chirality, and we have shown 
claim $4$ as well. 

On the other hand, we can also show the real modes have definite chirality. Let $u(0)$ be a 0-mode, so $Du(0) = 0$. Then 
we also have $\gamma_5 Du(0) = D\gamma_5 D u(0) = 0$, hence by the Ginsparg Wilson equation we also have 
$D\gamma_5 u(0) = 0$. So, we see that on the $\lambda = 0$ eigenspace of $D$, $[\gamma_5, D] = 0$ and hence we 
can simultaneously diagonalize $\gamma_5$ and $D$. Therefore, we can choose modes with $\lambda = 0$ to have 
definite chirality, i.e. $\gamma_5 u(0) = \pm u(0)$ depending on whether it is left or right handed. For the $\lambda = 
\frac{2}{a}$ subspace, we can make a very similar argument, thus we can choose the eigenvectors corresponding to 
real modes to have definite chirality. 

%next section is up for grabs

\newpage
\begin{thebibliography}{99}

	\bibitem{nogo}
	Nielsen, H. B., \& Ninomiya, M. (1981). A no-go theorem for regularizing chiral fermions. 
	In Physics Letters B (Vol. 105). https://doi.org/10.1016/0370-2693(81)91026-1
	
	\bibitem{ginsparg}
	Ginsparg, P. H., \& Wilson, K. G. (1982). A remnant of chiral symmetry on the lattice (Vol. 25). 
	https://journals.aps.org/prd/pdf/10.1103/PhysRevD.25.2649
	
	\bibitem{neuberger}
	Neuberger, H. (1999). The Overlap Dirac Operator. hep-lat/9910040v1
	
	\bibitem{L�scher}
	L�scher, M. (1998). Exact chiral symmetry on the lattice and the Ginsparg-Wilson relation. hep-lat/9802011
	
	\bibitem{niedermayer}
	Niedermayer, F. (1998). Exact chiral symmetry, topological charge and related topics. hep-lat/9810026
	
	\bibitem{hasenfratz}
	Hasenfratz, P., Laliena, V., \& Niedermayer, F. (1998). The index theorem in QCD with a finite cut-off. hep-lat/9801021

\end{thebibliography}

\end{document}