\documentclass[12pt, oneside]{article}   	% use "amsart" instead of "article" for AMSLaTeX format
\usepackage[top=.5in, bottom=.5in, left = .5in, right=.5in, headheight=14.5pt, includeheadfoot]{geometry}
%\usepackage[margin = 1in]{geometry}                		% See geometry.pdf to learn the layout options. There are lots.
\geometry{letterpaper}                   		% ... or a4paper or a5paper or ... 
%\geometry{landscape}                		% Activate for rotated page geometry
%\usepackage[parfill]{parskip}    		% Activate to begin paragraphs with an empty line rather than an indent
\usepackage{graphicx}				% Use pdf, png, jpg, or eps§ with pdflatex; use eps in DVI mode
								% TeX will automatically convert eps --> pdf in pdflatex		
\usepackage{amssymb}
\usepackage{amsmath}
\usepackage[shortlabels]{enumitem}
\setlist{leftmargin=5.5mm}
\usepackage{float}
\usepackage{tikz-cd}
\usepackage{subcaption}
\usepackage{slashed}
\usepackage{mathrsfs}

% Packages from other template
\usepackage[final]{microtype}
\usepackage[USenglish]{babel}
\usepackage{hyperref}
\usepackage[T1]{fontenc}

%\usepackage{titlesec}
%\titlespacing{\section}{0pt}{12pt}{4pt}

\usepackage[compat=1.0.0]{tikz-feynman}

\usepackage{bm}
\usepackage{bbm}
\usepackage{bbold}

\usepackage{amsthm}
\theoremstyle{definition}
\newtheorem{definition}{Definition}[section]
\newtheorem{theorem}{Theorem}[section]
\newtheorem{corollary}{Corollary}[theorem]
\newtheorem{lemma}[theorem]{Lemma}

\newcommand{\N}{\mathbb{N}}
\newcommand{\R}{\mathbb{R}}
\newcommand{\Z}{\mathbb{Z}}
\newcommand{\Q}{\mathbb{Q}}

\newcommand{\RI}{\mathrm{RI}}
\newcommand{\Tr}{\text{Tr}}
\newcommand{\TrC}{\text{Tr}_{\text{C}}}
\newcommand{\TrD}{\text{Tr}_{\text{D}}}

\usepackage{simpler-wick}
\usepackage[compat=1.0.0]{tikz-feynman}   %note you need to compile this in LuaLaTeX for diagrams to render correctly

\usepackage{parskip}
    \setlength{\parindent}{0in}
    %\setlength{\parindent}{.25in}

\usepackage{fancyhdr}
    \renewcommand{\headrulewidth}{.85pt}
    \renewcommand{\footrulewidth}{.6pt}
    \pagestyle{fancy}
    \renewcommand{\sectionmark}[1]{\markboth{#1}{}}
    \fancyhf{}
    \fancyhead[R]{Patrick Oare}
    \fancyhead[C]{\fontsize{14}{16.8}\textbf{Recitation 3: Contour integration, symmetries}}
    \fancyhead[L]{8.323 S2022}
    \fancyfoot[C]{\vspace*{.15in}\thepage}

% PSet Sections
\iffalse
\usepackage[explicit]{titlesec}
    \titleformat{\section}{\vspace*{0pt}\fontsize{16}{19.2}\selectfont}{}{0in}{\textbf{#1}{\hrule height .7pt width .75\textwidth}}
    \titlespacing{\section}{.35in}{.5in}{\parskip}
    \titleformat{\subsection}{\fontsize{14}{16.8}\selectfont}{}{.5in}{\textbf{\uline{#1}}}
    \titlespacing{\subsection}{0pt}{.5in}{\parskip}
\fi

% make arrow superscripts
\DeclareFontFamily{OMS}{oasy}{\skewchar\font48 }
\DeclareFontShape{OMS}{oasy}{m}{n}{%
         <-5.5> oasy5     <5.5-6.5> oasy6
      <6.5-7.5> oasy7     <7.5-8.5> oasy8
      <8.5-9.5> oasy9     <9.5->  oasy10
      }{}
\DeclareFontShape{OMS}{oasy}{b}{n}{%
       <-6> oabsy5
      <6-8> oabsy7
      <8->  oabsy10
      }{}
\DeclareSymbolFont{oasy}{OMS}{oasy}{m}{n}
\SetSymbolFont{oasy}{bold}{OMS}{oasy}{b}{n}

\DeclareMathSymbol{\smallleftarrow}     {\mathrel}{oasy}{"20}
\DeclareMathSymbol{\smallrightarrow}    {\mathrel}{oasy}{"21}
\DeclareMathSymbol{\smallleftrightarrow}{\mathrel}{oasy}{"24}
%\newcommand{\cev}[1]{\reflectbox{\ensuremath{\vec{\reflectbox{\ensuremath{#1}}}}}}
\newcommand{\vecc}[1]{\overset{\scriptscriptstyle\smallrightarrow}{#1}}
\newcommand{\cev}[1]{\overset{\scriptscriptstyle\smallleftarrow}{#1}}
\newcommand{\cevvec}[1]{\overset{\scriptscriptstyle\smallleftrightarrow}{#1}}

\newcommand{\dbar}{d\hspace*{-0.08em}\bar{}\hspace*{0.1em}}

% to use a box environment, use \begin{answer} and \end{answer}
\usepackage{tcolorbox}
\tcbuselibrary{theorems}
\newtcolorbox{answerbox}{sharp corners=all, colframe=black, colback=black!5!white, boxrule=1.5pt, halign=flush center, width = 1\textwidth, valign=center}
\newenvironment{answer}{\begin{center}\begin{answerbox}}{\end{answerbox}\end{center}}

\begin{document}
%\maketitle

% Possible topics: should check out what's on the pset as well
% - Contour integration *
% - Propagators *
% - Symmetry operations *
% - Quantizing complex fields

\section*{Contour integration}

See handwritten notes.

\section*{Symmetries}

\begin{itemize}
	
	\item A \textbf{Lie group} $G$ is a manifold with a group structure. A \textbf{Lie algebra} $\mathfrak g$ is a vector space equipped with a Lie bracket $[\cdot, \cdot]$ (a commutator). The Lie algebra $\mathfrak g$ implements infinitesimal symmetry transformations, and the Lie group $G$ implements finite symmetry transformations. 
	
	\item The \textbf{exponential map} relates elements of the Lie algebra to the Lie group. If I have an element $A\in \mathfrak g$ of the Lie algebra, I can get a corresponding element of the Lie group by:
	\begin{equation}
		g = e^{iA}\in G\label{eq:lie_group_expansion}
	\end{equation}
	The nice part about this is that the Lie algebra \textit{parameterizes} the Lie group. Since $\mathfrak g$ is a vector space, we can write out $A = A^a t^a$ where $\{t^a\}$ is a basis for the Lie algebra. Then if I specify the coordinates $A^a$ (just an $n$-tuple of numbers), I can write down any symmetry operator I want. 
	
	\item Ex: Rotations in quantum mechanics. For a spin $1/2$ system, rotations are implemented with the exponential of the angular momentum. The rotation operator is:
	\begin{equation}
		U_{1/2}(\hat n, \theta) = \exp(-i\theta\hat n\cdot \bm \sigma / 2)
	\end{equation}
	were $\bm \sigma$ is the vector of Pauli matrices, $\bm \sigma = (\sigma_x, \sigma_y, \sigma_z)$. Here the Lie algebra is $\mathfrak{su}(2)$, and spanned by the Pauli matrices-- a basis for $\mathfrak{su}(2)$ is $\{\sigma_x, \sigma_y, \sigma_z\}$, and as such we can write \textit{any} element of $\mathfrak{su}(2)$ as $-\theta \hat n\cdot \bm \sigma / 2$. 
	
	\item Ex: Lorentz group. The Lorentz algebra $\mathfrak{so}(1, 3)$ is spanned by the tensor $\mathcal J^{\mu\nu}$, or equivalently the boost or rotation generators $K_i, J_i$. We saw in the first recitation that we can write down any Lorentz transformation as:
	\begin{equation}
		\Lambda = \exp\left(\frac{i}{2}\omega_{\mu\nu} \mathcal J^{\mu\nu}\right)
	\end{equation}
	where $\omega_{\mu\nu}$ contains 6 independent real parameters (boost and rotation angles) that parameterize the Lie algebra. In the context of what we're doing here, we see that $\mathfrak{so}(1, 3)$ is spanned by $\{\mathcal J^{01}, \mathcal J^{02}, \mathcal J^{03}, \mathcal J^{12}, \mathcal J^{13}, \mathcal J^{23}\}$, and so any element of the Lie algebra is written as a linear combination of these $\mathcal J^{\mu\nu}$, i.e. $\omega_{\mu\nu} \mathcal J^{\mu\nu}$. 
	
	\item Another example: Conserved charges $H$ and $P^i$ in QFT. These are \textbf{generators} of time and spatial translation, respectively. They generate the Lie algebra for translation, and an arbitrary element of the algebra can be written as $Ht - P^i x^i = - P_\mu x^\mu$. We exponentiate this to get finite symmetry transformations:
	\begin{equation}
		U_x = \exp\left(iHt - i P^i x^i \right) = e^{-iP_\mu x^\mu}
	\end{equation}
	
	% Something to talk about: operators live in a representation which gets conjugated by $U_\Lambda$. 
	\item A Lie group (algebra) is typically pretty abstractly defined. To act it on a vector space, we need a \textbf{representation} of the group or algebra. For a vector space $V$, a representation of $G$ is a map $D : G\rightarrow GL(V)$, where $GL(V)$ is the space of invertible linear transformations on $V$. Likewise, a representation of $\mathfrak g$ is a map $d : \mathfrak g\rightarrow gl(V)$, where $gl(V)$ is the space of \textit{all} linear transformations on $V$. All a representation does is it gives us a way to act a symmetry group on a vector space. 
	
	\item Given a representation of the algebra, we can \textit{induce} a representation of the group by:
	\begin{equation}
		D(g) = e^{i d(A)}
	\end{equation}
	where $g = e^{iA}$ as in Eq.~(\ref{eq:lie_group_expansion}). 
	
	% Go through a geometric interpretation for $[Q^\alpha, \phi_a] = -i f_a^\alpha$. 
	\item Let's consider our Fock space:
	\begin{equation}
		H = |0\rangle\oplus \{a_{\vec k}^\dagger |0\rangle : \vec k\in\mathbb{R}^3\}\oplus \{a_{\vec k}^\dagger a_{\vec k'}^\dagger |0\rangle : \vec k, \vec k'\in\mathbb R^3\} \oplus ...
	\end{equation}
	To act a Lorentz transformation $\Lambda\in\mathrm{SO}(1, 3)$ on an element $|\psi\rangle\in H$, we need a representation of $SO(1, 3)$ on $H$. To specify the representation, we can simply show us where it sends the generators to, i.e. we can specify $d(\mathcal J^{\mu\nu})$, since:
	\begin{equation}
		U_\Lambda = D(\Lambda) = \exp \left(\frac{i}{2} \omega_{\mu\nu}\; d(\mathcal J^{\mu\nu})\right).
	\end{equation}
	This equation is important! Fields of different spin in QFT all correspond to different representations $d(\mathcal J^{\mu\nu})$. 
	
	\item Conserved charges $M_{\mu\nu}$: The representation that we use to act symmetries on our Fock space $H$ is:
	\begin{equation}
		d_\mathrm{Fock}(J^{\mu\nu}) = M^{\mu\nu} = -\frac{i}{2} \int \frac{d^3\vec k}{(2\pi)^3} k^\mu \left(a_{\vec k}^\dagger\partial_{k_\nu} a_{\vec k} - (\partial_{k_\nu} a_{\vec k}^\dagger ) a_{\vec k}\right) - (\mu\leftrightarrow\nu)
	\end{equation}
	This means that we want to act a Lorentz transformation on a state, let's say a one-particle state $|\vec k\rangle$ for concreteness, we just need to act $U_\Lambda$ on it:
	\begin{equation}
		|\vec k\rangle\mapsto U_\Lambda |\vec k\rangle = \exp\left(\frac{i}{2} \omega_{\mu\nu} M^{\mu\nu}\right)|\vec k\rangle.
	\end{equation}
	
	\item Invariance of a state under symmetries. For a state $|\Omega\rangle$ which is \textbf{conserved} under a symmetry $U$, we require that it is unchanged by the symmetry, $U |\Omega\rangle = |\Omega\rangle$. In the language we've been using, since $U = e^{i d(A)}$ for generators $A$, invariance of $|\Omega\rangle$ implies that \textbf{generators map the state to 0}. For example, the vacuum in QFT is invariant under time-translation, spatial translation, and Lorentz transformations. This implies that:
	\begin{equation}
		H|0\rangle = P^i|0\rangle = M^{\mu\nu}|0\rangle = 0
	\end{equation}
	Note that $|0\rangle$ is the vacuum state, and $0$ is the zero vector-- these are different things!
	
\end{itemize}

\section*{Complex fields}

\begin{itemize}

	\item Complex fields: For a scalar field theory with complex fields, $\phi$ and $\phi^*$ are now independent degrees of freedoms. They can both be expanded with two types of creation and annihilation operators:
	\begin{align}
		\phi(\vec x, t) &= \int \frac{d^3\vec k}{(2\pi)^3} \frac{1}{\sqrt{2\omega_{\vec k}}} (a_{\vec k} e^{-i\omega_{\vec k} t + i\vec k \cdot\vec x} + b_{\vec k}^\dagger e^{i\omega_{\vec k} t - i\vec k \cdot\vec x}) \\
		\phi^*(\vec x, t) &= \int \frac{d^3\vec k}{(2\pi)^3} \frac{1}{\sqrt{2\omega_{\vec k}}} (b_{\vec k} e^{-i\omega_{\vec k} t + i\vec k \cdot\vec x} + a_{\vec k}^\dagger e^{i\omega_{\vec k} t - i\vec k \cdot\vec x}).
	\end{align}
	Here $a_{\vec k}, a_{\vec k}^\dagger$ create and destroy \textit{particles}, and $b_{\vec k}, b_{\vec k}^\dagger$ create and destroy \textit{anti-particles}. When $\phi$ was a real field, the reality constraint $\phi = \phi^*$ removed the second set of creation / annihilation operators from the equation, but when $\phi$ is complex you must include the second set of operators $b_{\vec k}$. Note that:
	\begin{equation}
		[a_{\vec k}, b_{\vec k}] = [a_{\vec k}^\dagger, b_{\vec k}] = [a_{\vec k}, b_{\vec k}^\dagger] = 0.
	\end{equation}

	\item Physical intuition: $a_{\vec k}^\dagger |0\rangle$ creates a particle from the vacuum with momentum $\vec k$. How do you create a particle localized at position $\vec x$? Let's see what $\phi(\vec x, t)$ does when it acts on the vacuum:
	\begin{equation}
		\phi(\vec x, 0)|0\rangle = \int \frac{d^3\vec k}{(2\pi)^3} \frac{1}{\sqrt{2\omega_{\vec k}}} e^{-i\vec k\cdot\vec x} b_{\vec k}^\dagger |0\rangle = 
		\underbrace{\int \frac{d^3\vec k}{(2\pi)^3} \frac{1}{2\omega_{\vec k}} e^{-i\vec k\cdot\vec x} |\vec k\rangle_{b}}_{\textnormal{Antiparticle wave packet at $\vec x$}}
	\end{equation}
	So, this creates an antiparticle with momentum $\vec k$. If we want to create a particle, we instead need to act with $\phi^*$:
	\begin{equation}
		\phi^*(\vec x, 0) |0\rangle = \int \frac{d^3\vec k}{(2\pi)^3} \frac{1}{2\omega_{\vec k}} e^{-i\vec k\cdot\vec x} |\vec k\rangle_{a}
	\end{equation}

\end{itemize}

%\section*{Propagators}

%\begin{itemize}

%      	\item Green's functions are inverses of differential operators. What this means is that a Green's function $G(z, y)$ for a differential operator $K(x, z)$ satisfies:
%        	\begin{equation}
%        		\int d^4 z\; K(x, z) G(z, y) = \delta^{(4)}(x - y).
%        	\end{equation}
%	This is of the form of a matrix multiplication over a ``continuous index" on the matrix, $z$. We'll see stuff like this in QFT all the time-- in QM we would have had an honest to goodness matrix equation, without any of the integrals; however, since we have continuous degrees of freedom, the matrix multiplication turns into this. 
	
%	\item General form of a propagator: A \textbf{propagator} is just an inverse of the free 
	
%	\item Why time-ordering? We'll see later there are two big things that come out of this:
%	\begin{itemize}
%		\item Wick's theorem: This is what we'll use to evaluate correlation functions in the free theory.
%		\item The path integral: When we move to the path integral formalism, we'll see that differentiating the generating functional naturally gives us time-ordered correlation functions. 
%	\end{itemize}
%\end{itemize}

\end{document}