\documentclass[11pt, oneside]{amsart}   	% use "amsart" instead of "article" for AMSLaTeX format
\usepackage[margin = 1in]{geometry}                		% See geometry.pdf to learn the layout options. There are lots.
\geometry{letterpaper}                   		% ... or a4paper or a5paper or ... 
%\geometry{landscape}                		% Activate for rotated page geometry
%\usepackage[parfill]{parskip}    		% Activate to begin paragraphs with an empty line rather than an indent
\usepackage{graphicx}				% Use pdf, png, jpg, or eps§ with pdflatex; use eps in DVI mode
								% TeX will automatically convert eps --> pdf in pdflatex		
\usepackage{amssymb}
\usepackage{amsmath}
\usepackage{amsthm}
\usepackage[shortlabels]{enumitem}
\usepackage{float}

\theoremstyle{definition}
\newtheorem{definition}{Definition}[section]
\newtheorem{theorem}{Theorem}[section]
\newtheorem{corollary}{Corollary}[theorem]
\newtheorem{lemma}[theorem]{Lemma}
\newtheorem{example}{Example}[section]

\graphicspath{{/Users/theoares/Desktop/Math\ 142/notes}}

%SetFonts

%SetFonts


\title{Topological Manifolds}
\author{Patrick Oare}
\date{}							% Activate to display a given date or no date

\begin{document}
\maketitle

Topological manifolds are an interesting topic of study. In these notes, we focus on the algebraic properties of these manifolds rather than the differential properties 
of the manifolds. The main goal will be to classify all $0$, $1$, and $2$ manifolds up to homeomorphism.

\section{Definitions}

	We begin with the definition of a manifold; however, we need to define the notion of a second-countable topological space first:
	
	\begin{definition}[Second-Countable]
		A space $X$ is second-countable if it has a countable base.
	\end{definition}
	
	Now we may define a manifold:
	
	\begin{definition}[Manifold]
		A \textbf{manifold of dimension $n$} is a second countable Hausdorff space $X$ such that $\forall x\in X$, there is an open neighborhood $U_x\ni x$ 
		and a homeomorphism $\phi : U_x\rightarrow \mathbb R^n$. Such a subset and map $(U_x, \phi)$ is called a \textbf{chart}. 
	\end{definition}
	
	We include second-countable and Hausdorff in the definition to exclude strange spaces from being manifolds, including the ``long line" and the line with two 
	origins. 
	
	This definition of a manifold only includes spaces which we would considered open, i.e. they have no boundary. We now generalize this to include spaces with 
	boundary.
	
	\begin{definition}[Manifold with Boundary]
		An \textbf{$n$-manifold with boundary} is a topological space $X$ such that:
		\begin{enumerate}
			\item $X$ is Hausdorff and second-countable.
			\item $\forall x\in X$, there is an open neighborhood $U_x\ni x$ and a homeomorphism $\phi : U_x\rightarrow\mathbb R^n$ or a homeomorphism 
			$\phi : U_x\rightarrow\mathbb R^n_+$, where $\mathbb R^n_+ = \{(x_1, ..., x_n) : x_n\geq 0\}$.
		\end{enumerate}
	\end{definition}
	
	Now we may talk about two subsets of a manifold $X$, the \textbf{interior} and the \textbf{boundary} of $X$.
	
	\begin{definition}[Interior]
		The \textbf{interior} of $X$ is:
		$$
			int(X) = \{x\in X : \exists\textnormal{ a chart $(U_x, \phi)$ around $x$ with $U_x\cong\mathbb R^n$.}\}
		$$
	\end{definition}
	
	\begin{definition}[Boundary]
		The \textbf{boundary} of $X$ is:
		$$
			\partial X = X\setminus int(X) = \{x\in X : \exists\textnormal{ a chart $(U_x, \phi)$ around $x$ and $\phi(x)\in\mathbb R^n_+$.}\}
		$$
	\end{definition}
	
	\begin{definition}[Closed Manifold]
		A manifold $X$ is \textbf{closed} if it is compact and $\partial X = \emptyset$.
	\end{definition}
	
	For some examples of manifolds:
	
		\begin{itemize}
		
		\item $\mathbb R^n$ and $S^n$ are $n$-manifolds.
		
		\item Any open subset $U\subseteq X$ of an $n$-manifold is an $n$-manifold.
		
		\item If $X$ and $Y$ are $n$-manifolds, then $X\times Y$ is an $n + m$ manifold.
		
		\item If $X$ is an $n$-manifold with boundary, then $\partial X$ is an $(n - 1)$-manifold. For example, $B^n$ is an $n$-manifold with boundary $S^{n - 1} 
		= \partial B^n$, which is an $n - 1$ manifold.
		
		\item We can attach manifolds along a boundary. If $X, Y$ are $n$-manifolds with boundary and $f : \partial X\rightarrow\partial Y$ is a homeomorphism, 
		then $X\cup_f Y$ is an $n$-manifold.
		
	\end{itemize}
	
	This next example is quite important, so we give it a definition. It is an operation that joins two manifolds together called the \textbf{connected sum}.
	
	\begin{definition}[Connected Sum]
		Let $X, Y$ be two $n$-manifolds. Choose $x\in X, y\in Y$ and charts $(U_x, \phi)$ and $(V_y, \psi)$ of $x$ and $y$, and pick $U\subseteq U_x$, $V
		\subseteq V_y$ such that $U\cong_\phi B^n$ and $V\cong_\psi B^n$. Let $X' = X\setminus int(U)$ and $Y' = Y\setminus int(V)$, and choose a 
		homeomorphism $f : \partial Y'\rightarrow\partial X'$. We define the \textbf{connected sum} of $X$ and $Y$ to be the manifold $X\# Y$ given by:
		$$
			X\# Y := X'\cup_f Y'
		$$
	\end{definition}
	
	If $X$ and $Y$ are path-connected, then the connected sum is a well defined manifold, i.e. it does not depend on the choice of $x$ and $y$.
	
	Next, we will turn to the classification of $0$, $1$, and $2$ manifolds. We will aim to classify all closed and connected 0, 1, and 2 manifolds up to 
	homeomorphism. This will ultimately allow us to prove the following theorem for $n = 0, 1, 2$ dimensions:
	
	\begin{theorem}[Generalized Poincare Conjecture]
		If $X$ is a closed, connected $n$-manifold and $X$ is homotopy equivalent to $S^n$, then $X$ is homeomorphic to $S^n$.
	\end{theorem}

\section{0-Manifolds}

	If $X$ is a 0 manifold, there is not much interesting we can say about it. Given $x\in X$, we have a chart $(\phi, U_x)$ where $\phi : U_x\rightarrow \mathbb 
	R^0 = \{0\}$ is a homeomorphism, so because $\phi$ is a bijection, $U_x = \{x\}$. But then every subset of $X$ is open, so $X$ must be a discrete space. 
	Conversely, every discrete space which is second countable (discrete spaces are automatically Hausdorff) is a 0 manifold, so \textbf{$X$ is a 0 manifold if 
	and only if it is a second countable discrete topological space}. This implies that connected $0$ manifolds are homeomorphic to the one-point space $\{*\}$, 
	which can never be homotopy equivalent to $S^0$, and thus proves the Generalized Poincare Conjecture in $n = 0$ dimensions.

\section{1-Manifolds}

	We will begin with the classification, and then cite the tools used to prove the theorem.
	
	\begin{theorem}[Classification of connected 1-manifolds]
		If $X$ is a connected 1-manifold (perhaps with boundary), then $X$ is homeomorphic to one of:
		\begin{enumerate}
			\item $\mathbb R$
			\item $S^1$
			\item $[0, 1]$
			\item $[0, 1)$
		\end{enumerate}
	\end{theorem}
	
	\begin{corollary}[Generalized Poincare, $n$ = 1]
		If $X$ is a connected and closed 1-manifold, then $X\cong S^1$.
	\end{corollary}
	
	The corollary is immediate from the classification theorem. The easiest way to prove this classification is to prove a lemma, which we will state.
	
	\begin{lemma}
		Let $X$ be a connected 1-manifold. If $(U, \phi)$ and $(V, \psi)$ are charts on $X$ and $U, V\cong\mathbb R$, then $U\cap V$ has at most 2 connected 
		components.
	\end{lemma}
	
\section{Free Products and Abelianization}

	Before we proceed with the classification of 2-manifolds, it is important to develop some group theory which will help us to classify 2-manifolds (especially 
	when we consider the uniqueness of the decomposition). Let $G = \langle a_1, ..., a_n | r_1 = 1, ..., r_m = 1\rangle$ and $G' = \langle b_1, ..., b_{n'} | s_1 = 
	1, ..., s_{m'} = 1\rangle$ be two groups. We define:
	
	\begin{definition}[Free Product]
		The \textbf{free product} of $G$ and $G'$ is the group $G*G'$ given by:
		$$
			G*G' := \langle a_1, ..., a_n, b_1, ..., b_{n'} | r_1 = 1, ..., r_m = 1, s_1 = 1, ..., s_{m'} = 1\rangle
		$$
	\end{definition}
	
	The best way to visualize the free product is to be the group on all letters from $G$ and $G'$ in which the group operation is concatenation and all words 
	are reduced via inverses and the relations on the group. We don't impose any additional relations on the structure of $G*G'$ by taking the free product; 
	in particular, we cannot commute letters from different groups because we don't have any relations that say they commute. 
	
	\begin{definition}[Free Group]
		Let $n\in\mathbb N$. The \textbf{free group on $n$ generators} is the group:
		$$
			F_n := \langle a_1, ..., a_n\rangle
		$$
	\end{definition}
	
	This is the group on $n$ generators with no relations, and it can also be constructed by $F_1 = \mathbb Z$ and $F_{n + 1} = F_n * \mathbb Z$ for all $n\in 
	\mathbb N$. 
	
	Free groups will be very helpful for describing fundamental groups of surfaces, but they are difficult to work with, primarily because they and their quotients 
	are not necessarily Abelian. In particular, it is hard to tell when two presentations of non-Abelian groups actually correspond to different groups, so we wish 
	to force these groups to be Abelian. This process is known as \textbf{Abelianization}.
	
	\begin{definition}[Abelianization]
		Let $G$ be a group, and let $N$ be its \textbf{commutator subgroup}, i.e. $N$ is the normal closure of $\{ghg^{-1}h^{-1} : g, h\in G\}$. Then we 
		define its \textbf{Abelianization} to be the group $Ab(G) := G / N$.
	\end{definition} 
	
	By quotienting out the commutator subgroup of $G$, we have necessarily forced an additional relation on it; that of $gh = hg$. This is best seen as what 
	Abelianizing does to a presentation of $G$. If we have a presentation of $G$ as before, then a presentation of $Ab(G)$ is given by:
	$$
		Ab(G) = \langle a_1, ..., a_n | r_1 = 1, ..., r_m = 1, a_1a_2 = a_2a_1, ..., a_1a_n = a_na_1, ..., a_{n - 1}a_n = a_na_{n - 1}\rangle
	$$
	
	We essentially keep the original presentation but add additional relations to the group forcing all its elements to commute. There is a nice relation between 
	free products and direct products when groups are Abelianized; it is easiest to think of direct products as the ``abelian" version of free products. This is 
	reflected in the Abelianization of the free group on $n$ generators, which is:
	$$
		Ab(F_n)\cong\mathbb Z^n
	$$
	 
	 It is also evident for two arbitrary groups $G, H$, in that:
	 $$
	 	Ab(G*H) \cong Ab(G)\times Ab(H)
	 $$
	
	As one final fact for this section, if $G \cong G'$, then $Ab(G)\cong Ab(G')$. However, the converse is not true at all; this is apparent because $Ab(F_2)\cong 
	\mathbb Z^2 = Ab(\mathbb Z^2)$, but obviously $F_2$ and $\mathbb Z^2$ are not isomorphic groups.
	

\section{2-Manifolds (Surfaces)}

	2-manifolds are the most interesting types that we will study; the classification is long and quite involved, but the proofs are not as tedius as those for the 
	1-manifolds, so I will include them whenever possible. 
	
	\begin{definition}[Surface]
		A \textbf{surface} is a 2-manifold with or without boundary.
	\end{definition}
	
	Note that if $S$ is a compact surface with non-zero boundary, then $\partial S$ is a compact 1-manifold without boundary, and thus by our previous 
	classification $\partial S\cong S^1\sqcup ...\sqcup S^1$. So, given any compact surface $S$, we may glue on copies of $D^2$ via an attaching map to each 
	copy of $S^1$ in the boundary of $S$, and we may form a 2-manifold $\tilde S$ which is compact and has no boundary. Thus, we will restrict our study of 
	2-manifolds to closed manifolds, i.e. 2-manifolds which are compact and connected. We will now begin our classification, and we will start with the theorem:
	
	\begin{theorem}[Classification of 2-manifolds]
		Let $S$ be a closed, connected surface. Then:
		$$
			S\cong S^2\# T^2\# ...\# T^2\#\mathbb RP^2\# ...\#\mathbb RP^2
		$$
		where there are $n, m\in\mathbb Z_{\geq 0}$ copies of $T^2$ and $\mathbb RP^2$, respectively.
	\end{theorem}
	
	\subsection{Existence and Cellular Decompositions}
	
	We will now attempt to simplify our study of surfaces by restricting it to studying polygons which correspond to the surface in some way, called a \textbf{cellular 
	decomposition} of $S$. For a rigorous definition, check the class notes. It can be shown that every cellular decomposition is equivalent to one with a single 
	polygon, so we may restrict our cellular decomposition to one polygon with identified edges. Given a cellular decomposition of a surface, we may describe it by 
	a \textbf{word} with letters representing the orientation of each edge. We will say the surface is equal to this word by abuse of notation. \textbf{Words are 
	equivalent up to cyclic permutation}, i.e. if a surface $S$ is described by $abcb^{-1}a^{-1}c^{-1}$, then it can also be described by the word $cb^{-1}a^{-1}c^{-1}
	ab$. Some simple examples of this are:
	\begin{itemize}
		\item $S^2 = aa^{-1}$.
		\item $\mathbb RP^2 = aa$.
		\item $T^2 = aba^{-1}b^{-1}$.
	\end{itemize}
	
	These become apparent upon drawing out the cellular decompositions of these surfaces, which I will include when I have time. We will now state a collection 
	of lemmas which will allow us to reduce a cellular decomposition into something we recognize. These lemmas will essentially allow us to reduce a word by 
	representing the corresponding space as a connect sum of two simpler spaces which are described by subwords of the original.
	
	\begin{lemma}[S]
		If $S\cong Xaa^{-1}$ and $X\neq\emptyset$, then $S\cong X$.
	\end{lemma}
	
	This lemma allows us to simplify inverse letters that are next to one another, assuming the rest of the word is non-trivial. If there are no other letters in the word, 
	then we cannot do this, and the surface is instead homeomorphic to $S^2$. 
	
	\begin{lemma}[P1]
		If $S = Xaa$, then $S\cong\mathbb RP^2\#X$. 
	\end{lemma}
	
	\begin{lemma}[T1]
		If $S = Xaba^{-1}b^{-1}$, then $S\cong T^2\#X$. 
	\end{lemma}
	
	\begin{lemma}[P2]
		If $S = XaYa$ and $X, Y\neq\emptyset$, then $S = bbXY^{-1}$. 
	\end{lemma}
	
	\begin{lemma}[T2]
		If $S = WaXbYa^{-1}Zb^{-1}$, then $S\cong T^2\# S_1$, where $S_1 = ZYXW$. 
	\end{lemma}
	
	A brief note on inverse words: you can take an inverse of a word by flipping the order and taking individual inverses of letters. This is also equivalent to reading 
	the word in the other direction starting at the same base point; either way will give you the same inverse word. We may use these lemmas to prove that 
	$$
		T^2\#\mathbb RP^2\cong \mathbb RP^2\#\mathbb RP^2\#\mathbb RP^2
	$$
	and that:
	$$
		K\cong\mathbb RP^2\#\mathbb RP^2
	$$
	where $K$ is the Klein bottle. Now, the existence of a decomposition of $S$ into a connect sum of a sphere, projective planes, and tori can be proven by 
	just stripping off the correct words from its cellular decomposition. 
	
	
	\subsection{Uniqueness and Seifert-van Kampen}
	
	The next question to ask is if the $n$ and $m$ values in the decomposition are unique, which we will eventually prove. We define:
	
	\begin{definition}[Genus]
		Let:
		$$
			S_g := T^2\# ... \# T^2 \;\;\;\;\;\;\;\;\;\;\;\;\;\;\;\;\;\;\;\;\;\;\;\;\;\;\;\;\;\;\;\;\;\;\;\;\;\;\;\;\;\;\;\;\;\;\;\;\;\;\;\;\;\;\;\;\;\;\;\;\;\;\;\;
			N_g := \mathbb RP^2\# ... \#\mathbb RP^2
		$$
		where in each case, $g$ copies of each surface are connect summed together. $g$ is called the \textbf{genus} of the surface.
	\end{definition}
	
	We will now state the Seifert-van Kampen theorem, which we will use to finish the classification of surfaces theorem.
	
	\begin{theorem}[Seifert-van Kampen]
		Let $X = A\cup B$ with $A, B$ open and path-connected, $A\cap B$ path-connected, and $p\in A\cap B$. Let $\iota_A : A\cap B\rightarrow A$ and 
		$\iota_B : A\cap B\rightarrow B$ be the inclusion maps. Then:
		$$
			\pi_1(X, p)\cong\pi_1(A, p) * \pi_1(B, p) / N
		$$
		where $N\triangleleft\pi_1(A, p) * \pi_1(B, p)$ is the normal closure of $\{(\iota_A)_*(g)[(\iota_B)_*(g)]^{-1} : g\in \pi_1(A\cap B, p)\}$.
	\end{theorem}
	
	We need to quotient by $N$ to enforce relations on the new fundamental group and avoid double counting loops in the intersection $A\cap B$. The Seifert-van 
	Kampen theorem can be used to compute the fundamental group of $X_n$, the one-point union on $n$ circles. It is easiest to use when $A\cap B$ is 
	contractible, because then we don't need to enforce any relations on $\pi_1(A, p) * \pi_1(B, p)$. We can do something similar to compute the fundamental 
	group of $S_g$ and $N_g$. These are:
	$$
		\pi_1(S_g)\cong\langle a_1, ..., a_{2g} | a_1a_2a_1^{-1}a_2^{-1}...a_{2g - 1}a_{2g}a_{2g-1}^{-1}a_{2g}^{-1} = 1\rangle
	$$
	$$
		\pi_1(N_g)\cong\langle a_1, ..., a_g | a_1^2a_2^2...a_g^2 = 1\rangle
	$$
	
	To show these are distinct, we can Abelianize them. For $S_g$, this allows us to commute the $a_i$ past one another in the single relation, clearing that 
	relation and leaving us with an abelian group on $2g$ generators:
	$$
		H_1(S_g) = Ab(\pi_1(S_g))= \mathbb Z^{2g}
	$$
	
	For $N_g$, we can change the presentation to (where we assume the presentation includes commutation relations) $Ab(N_g) = \langle a_1, ..., a_1 ... a_g | 
	(a_1 ... a_g)^2 = 1\rangle = \langle a_1, ..., a_{g - 1}, b | b^2 = 1\rangle$, and so we see that:
	$$
		H_1(N_g) = Ab(\pi_1(N_g)) = \mathbb Z^{g - 1}\times (\mathbb Z / 2\mathbb Z)
	$$
	
	Thus $S_g$ and $N_{g'}$ have different abelianized fundamental groups and therefore different fundamental groups, so they are not homotopy equivalent. 

\end{document}