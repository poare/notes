\documentclass[11pt, oneside]{article}   	% use "amsart" instead of "article" for AMSLaTeX format
\usepackage[margin = 1in]{geometry}                		% See geometry.pdf to learn the layout options. There are lots.
\geometry{letterpaper}                   		% ... or a4paper or a5paper or ... 
%\geometry{landscape}                		% Activate for rotated page geometry
%\usepackage[parfill]{parskip}    		% Activate to begin paragraphs with an empty line rather than an indent
\usepackage{graphicx}				% Use pdf, png, jpg, or eps§ with pdflatex; use eps in DVI mode
								% TeX will automatically convert eps --> pdf in pdflatex		
\usepackage{amssymb}
\usepackage{amsmath}
\usepackage[shortlabels]{enumitem}
\usepackage{float}
\usepackage{tikz-cd}

\usepackage{amsthm}
\theoremstyle{definition}
\newtheorem{definition}{Definition}[section]
\newtheorem{theorem}{Theorem}[section]
\newtheorem{corollary}{Corollary}[theorem]
\newtheorem{lemma}[theorem]{Lemma}
\newtheorem{prop}[theorem]{Proposition}
\newtheorem{axiom}[theorem]{Axiom}

\newcommand{\N}{\mathbb{N}}
\newcommand{\R}{\mathbb{R}}
\newcommand{\Z}{\mathbb{Z}}
\newcommand{\Q}{\mathbb{Q}}

%SetFonts

%SetFonts


\title{Set Theory}
\author{Patrick Oare}
\date{}							% Activate to display a given date or no date

\begin{document}
\maketitle
\section{Introduction}

Set theory provides the foundations for essentially all of modern mathematics. It is used in every mathematical field, often 
implicitly. In these notes, we will study how to make this foundation rigorous. We will study the axiomatization of set theory 
in such a way that properties of sets can be deduced from \textbf{only} the axioms which we decide upon. We will build 
upon these axioms and study how to construct different mathematical objects, including many objects that you may have 
studied before. Various number systems will also be constructed, so you can be at ease the next time you want to verify 
\textit{why} $2 + 3 = 5$.

\section{First Order Logic}

Many of the axioms of set theory will be written down precisely in \textbf{first order logic}. This is a system which makes precise 
notions like ``membership in" a set, as well as rules of logic like the symbols $\implies$, $\wedge$, $\vee$, and more. It is 
the \textit{language} in which we make statements in set theory. We will treat first order logic briefly, as it is only needed in certain 
statements we will make later. A statement in first order logic contains the following \textbf{symbols}\footnote{When asked to 
provide a first order formula, you must \textbf{only} use these symbols.}:

\begin{itemize}
	\item \textbf{Variables}: Letters like $x, y, z, a, b, ...$
	\item The symbols $\in$ and $=$. 
	\item \textbf{Logical connectors}: $\wedge$ or $\&$ (``and"), $\vee$ (``or"), $\neg$ (``not"), $\implies$ (``implies"), $\iff$ 
	(``equivalent").
	\item \textbf{Quantifiers}: $\exists$ (``exists"), $\forall$ (``for all").
\end{itemize}

Using symbols, we can write formulas, which are essentially just statements of truth. Formulas can have different types of 
variables, which may give different conditions of truth depending on context. As an example, consider the formula:
$$
	\phi(x) = \exists y\exists z ((y\in z) \wedge (z\in x))
$$
This is a first order formula because it only makes use of the symbols presented above. The variables $y$ and $z$ are called 
\textbf{bound variables}; they are bound by quantifiers to make it clear what context they are being used in for the formula. 
The variable $x$ is a \textbf{free variable}; you are allowed to input a value for $x$ yourself, and the formula will be either true 
or false. For example, $\phi(\emptyset)$ is true, yet $\phi(\{\{\emptyset\}\})$ is false. First order logic can be used to generate 
more complicated formulas or symbols to use, which can then be used to simplify different formulas. For example, we may 
write the symbol $a\subseteq b$ as shorthand for the formula $\forall t (t\in a\implies t\in b)$. Note that to specify exactly the 
elements of a set $a$, we will often write $\forall x (x\in a \iff ...)$. 

\section{Axiomatization of Set Theory}

When we study the properties of sets, we need to decide upon a common language: what exactly do we mean by a set? One 
will have an intuitive understanding of this question, but a rigorous set of axioms must be decided on before propositions may 
be proved. We will state the axioms of set theory here, and develop them further as we use them more. The first 9 axioms 
are the \textbf{Zermelo-Fraenkel (ZF) Axioms} of set theory, and they are widely accepted to be the foundations most of 
mathematics is built off of. The tenth axiom listed is the axiom of choice, and its acceptance is more controversial than the 
other 9. The system of all ten axioms is denoted \textbf{ZFC}, standing for ZF with Choice.

\begin{enumerate}

	\item \textbf{Extensionality axiom}: If two sets have the same members, then they are equal. 
	\begin{equation}
		\forall a\forall b (\forall x (x\in a\iff x\in b)\implies a = b)
	\end{equation}
	
	\item \textbf{Empty set axiom}: There is a set $\emptyset$ which has no members.
	\begin{equation}
		\exists a \forall x \neg(x\in a)
	\end{equation}
	
	\item \textbf{Pairing axiom}: For any sets $u$ and $v$, there is a set $\{u, v\}$ whose members are exactly $u$ and $v$.
	\begin{equation}
		\forall u\forall v \exists x (\forall z (z\in x\implies z = u\vee z = v))
	\end{equation}
	
	\item \textbf{Union axiom}: For any set $a$, there is a set $b$ (denoted $\bigcup a$) whose members are precisely the 
	members of members of $a$. 
	\begin{equation}
		\forall a\exists b (\forall x (x\in b)\iff \exists c (c\in a\wedge x\in c))
	\end{equation}
	
	\item \textbf{Subset axiom}: For any set $c$ and any first order formula $\Phi(x, t_1, ..., t_k)$, there is a set $b$ whose 
	elements are precisely the elements of $c$ that satisfy $\Phi(x, t_1, ..., t_k)$. 
	\begin{equation}
		\forall c\forall t_1, ..., t_k\exists b (\forall x (x\in b)\iff x\in c\wedge \Phi(b, t_1, ..., t_k))
	\end{equation}
	
	\item \textbf{Power set axiom}: For any set $a$, there is a set $b$ (denoted $\mathcal P a$) whose elements are 
	precisely the subsets of $a$.
	\begin{equation}
		\forall a\exists b (\forall x (x\in b)\iff \forall y (y\in x\implies y\in a))
	\end{equation}
	
	\item \textbf{Infinity axiom}: There exists an inductive set.
	\begin{equation}
		\exists a (\emptyset\in a\wedge \forall b\in a (b^+\in a))
	\end{equation}
	
	\item \textbf{Replacement axiom}: 
	
	\item \textbf{Regularity axiom}: 
	
	\item \textbf{Axiom of choice}: For any relation $R$, there is a function $G\subseteq R$ such that $dom(G) = dom(R)$. 
	\begin{equation}
		\forall a\forall R ((R\subset a\times a)\iff (\exists G\subseteq R) (dom(G) = dom(R)))
	\end{equation}

\end{enumerate}

We will now discuss the importance of the first six axioms listed here. We will eventually have the tools needed to discuss 
Axioms 7 - 10 in depth, but at the moment these axioms are best studied after some additional knowledge is built up.

\subsection{Extensionality and empty set axioms}

These are the most basic axioms of set theory, yet give us a surprising amount of power. Extensionality allows us to know 
when two sets are equal; this is if they have exactly the same members, which is quite an obvious statement. However, using 
this axiom allows us to make well-defined definitions of types of sets; in other words, we can\footnote{Almost; we must also 
prove the existence of the set which has been defined.} specify a set using \textbf{set builder} notation. This is a completely 
unambiguous notation which allows one to specify a set's elements exactly. However, we will see later that we need the 
subset axiom to do this correctly, as simply defining the elements of a set does not prove that such a set \textbf{exists}. 

The \textbf{empty set} axiom gives us our first element to work with in set theory. At the time this axiom is introduced, no sets 
actually exist; it is the first of its kind, which gives the empty set a special role in set theory. Without other axioms, the only 
set which exists is the empty set. 
\begin{theorem}
	There is a unique set which has no members
\end{theorem}
This follows immediately from its definition and the extensionality axiom. 

\subsection{Pairing and union axioms}

The pairing axiom can be used in conjunction with the empty set axiom to start constructing sets, although these sets are rather 
limited. First, we can use the pairing axiom with $u = v = \emptyset$ to create the set $\{\emptyset\}$. We can then join together 
copies of these sets to create 2-element sets. However, it is severely limited; pairing can only be used to produce sets with 
two elements, and we should be able to produce sets with arbitrary amounts of elements. This is done precisely by the union 
axiom. The union of a set is unique, because as with all other sets, specifying exactly the elements as we have done means 
we may use the extensionality axiom. We denote the union of the set $\{c, d\}$ (which we note exists by pairing) by $c\cup d$. 
Given $n$ sets $\{c_j\}_{j = 1}^n$, we can inductively construct the union of this set by:
\begin{equation}
	\bigcup_{j = 1}^n c_j := \left(\bigcup_{j = 1}^{n - 1} c_j\right)\cup c_n
\end{equation}

We can now build sets which have more than 2 elements. Suppose we wish to construct $\{a, b, c\}$. Then we can use the 
pairing axiom to build $\{a, b\}$ and the set $\{c\}$, then union these to form $\{a, b, c\}$, i.e. $\{a, b, c\} = \{a, b\}\cup\{c\}$. 

As one final note on this subject, in proofs it is often easiest to use the following fact: we have $x\in\bigcup a$ if there is $b\in a$ 
such that $x\in b$. 

\subsection{Subset axiom}

The subset axiom is necessary to prevent paradoxes which may be encountered when studying certain types of sets. The 
most famous example of this is \textbf{Russel's paradox}, which states that if the set $a = \{x : x\notin x\}$ exists, we have a 
contradiction. This leads to the following theorem. 
\begin{theorem}
	There is no set of all sets, i.e. there is no set which contains all other sets as elements.
\end{theorem}
\begin{proof}
	Suppose that $M$ was such a set. Then we can construct the set $\{x\in M : x\notin x\}$ and obtain Russell's paradox, 
	a contradiction. 
\end{proof}

\textbf{The easiest way to show a set exists is to exhibit that it is a subset of an existing set}. This will be our method of proof 
when trying to construct sets. In other words, if you can use the set builder notation:
\begin{equation}
	\{x\in a : \Phi(x, t_1, ..., t_k)\}
\end{equation}
where $a$ is a set which we have previously shown exists, then \textbf{this set exists}. 

We can use the subset axiom to define the intersection of sets, which we will do in the following way. 
\begin{prop}
	Let $a$ and $b$ be two sets. Then there is a unique set $c$ whose elements are precisely the members of $a$ and $b$, 
	and $c$ is unique.
\end{prop}
\begin{proof}
	Apply the subset axiom with the formula $\Phi(x, a, b) = x\in a\wedge x\in b$ to the set $a$, i.e. the set $c$ is the 
	following:
	$$
		c = \{x\in a : x\in b\}
	$$
	That $c$ is unique follows from extensionality.
\end{proof}
\begin{definition}[Intersection]
	The set $c$ is denoted $a\cap b$, and called the \textbf{intersection} of $a$ and $b$. 
\end{definition}

\begin{prop}
	Let $A\neq\emptyset$. Then there is a set $B$, denoted $\bigcap A$ whose elements are precisely the sets which are 
	members of \textbf{all} members of $A$.
\end{prop}
\begin{proof}
	$A\neq\emptyset$, so pick $c\in A$. Then we can construct the following set:
	$$
		B = \{x\in c : \forall C (C\in A\iff x\in C)\}
	$$
\end{proof}

\subsection{Power set axiom and set algebra}

The power set axiom is only needed for infinite sets; it would be possible to construct the power set of a finite set with 
the union and subset axioms we have already seen. Now that we can construct all subsets of any set, we can worry about 
how to deal with subsets and what operations we can use on them. Namely, we will discuss \textbf{set algebra} and how 
different operations relate to one another.

\begin{definition}[Relative Complement]
	For sets $A, B$, the \textbf{relative complement} of $B$ in $A$ is the set:
	\begin{equation}
		A\setminus B = \{x\in A : x\notin B\}
	\end{equation}
\end{definition}

Now that we have all the major set operations, we can enumerate some important laws when dealing with set algebra. The first 
are the \textbf{distributive laws}:
$$
	A\cup (B\cap C) = (A\cup B)\cap (A\cup C)\;\;\;\;\;\;\;\;\;\;\;\;\;\;\;\;\;\;\;\;\;\;\;\;\;\;\;\;\;\; A\cap (B\cup C) = (A\cap B)\cup (A\cap C)
$$
$$
	A\cup (\bigcap B) = \bigcap \{A\cup b : b\in B\}\;\;\;\;\;\;\;\;\;\;\;\;\;\;\;\;\;\;\;\;\;\;\;\;\;\;\;\;\;\;\;\;\; A\cap (\bigcup B) = \bigcup \{A\cap 
	b : b\in B\}
$$
As important as these are \textbf{DeMorgan's laws}:
$$
	C\setminus (A\cup B) = (C\setminus A)\cap (C\setminus B) \;\;\;\;\;\;\;\;\;\;\;\;\;\;\;\;\;\;\;\;\;\;\;\;\;\;\;\;\;\; C\setminus (A\cap B)
	= (C\setminus A)\cup (C\setminus B)
$$
$$
	C\setminus (\bigcup A) = \bigcap \{C\setminus a : a\in A\} \;\;\;\;\;\;\;\;\;\;\;\;\;\;\;\;\;\;\;\;\;\;\;\;\;\;\;\;\;\;\; C\setminus (\bigcap A) 
	= \bigcup\{C\setminus a : a\in A\}
$$

\subsection{Axiom of Choice}

The axiom of choice is perhaps the most controversial axiom in set theory. Although it is quite intuitive, assuming this axiom 
often leads to many non-intuitive results, while at the same time gives one a very useful tool for proofs dealing with infinite sets. 

TODO

\newpage
\section{Ordered Pairs, Relations, and Functions}

We now turn to the discussion of ordered pairs, and how these will give us a way to define relations and functions. We will 
formalize the notion of an ordered pair and discuss how to write one explicitly as a set, then develop relations and functions 
as specific types of sets of ordered pairs. 

\subsection{Ordered Pairs}

As you have most likely seen before, an ordered pair is an object of the form $(a, b)$. We will seek to encode this pair 
specifically as a set. Note that $(a, b)$ is a different object than $\{a, b\}$; to see this, $(a, b)\neq (b, a)$ unless $a = b$, while 
the sets $\{a, b\}$ and $\{b, a\}$ are equal. Ordered pairs also do not collapse under repeated elements, i.e. the ordered 
pair $(a, a)$ does not equal the tuple $(a)$. We will use the notation $\langle a, b\rangle$ interchangeably with $(a, b)$. 

\begin{definition}[Ordered Pair]
	Let $x$ and $y$ be sets. Then the \textbf{ordered pair} $(x, y)$ is defined to be:
	\begin{equation}
		(x, y) := \{\{x\}, \{x, y\}\}
	\end{equation}
\end{definition}

\begin{prop}
	If $(a, b) = (c, d)$, then $a = c$ and $b = d$. 
\end{prop}

\begin{proof}
	Suppose that $(a, b) = (c, d)$. If $a = b$, then $(a, b) = \{\{a\}\}$ contains 1 element, so $\{c\} = \{c, d\}\implies c = d$, 
	and so $\{\{a\}\} = \{\{c\}\}$, hence $a = b = c = d$. Suppose $a\neq b$. Then $\{\{a\}, \{a, b\}\} = \{\{c\}, \{c, d\}\}$, hence 
	$\{a\} = \{c\}$ and $\{a, b\} = \{c, d\}$ because the number of the elements in each of these sets are the same. Thus 
	$a = c$ and $b = d$, so we are done.
\end{proof}

We see that this definition of an ordered pair indeed has the properties that we want; namely, that the order in the pair matters. 
We will now define the product of sets and prove that it exists.

\begin{definition}[Cartesian Product]
	If $A$ and $B$ are sets, the \textbf{Cartesian product} of $A$ and $B$, denoted $A\times B$, is the set of pairs of the 
	form $(a, b)$ for $a\in A$ and $b\in B$.
\end{definition}

To show that $A\times B$ exists for sets $A$ and $B$, the easiest way to approach this is to use the subset axiom. Namely, we 
want to find a set which contains $A\times B$ as a subset.

\begin{lemma}
	If $x\in A$ and $y\in B$, then $(x, y)\in \mathcal P\mathcal P (A\cup B)$. 
\end{lemma}

The proof of this lemma is immediate from the definition of $(x, y)$. It also proves the existence of $A\times B$ for given sets 
$A$ and $B$, because we can see that:
$$
	A\times B = \{(a, b)\in \mathcal P\mathcal P (A\cup B) : a\in A, b\in B\}
$$

For notation's sake, we will denote $A\times A$ by $A^2$. As another quick lemma, we have the following statement which is 
often useful to remember when we deal with relations in the next subsection.

\begin{lemma}
	If $(x, y)\in A$, then $x$ and $y$ are in $\bigcup\bigcup A$. ~
	\label{lemma:3d}
\end{lemma}

Finally, note that now we have defined pairs, we can define $n$-tuples for $n \geq 1$. We will give an inductive definition as to 
not clutter the definitions.

\begin{definition}[$n$-tuple]
	For sets $x_1, ..., x_n$, we define the \textbf{$n$-tuple} to be:
	\begin{equation}
		(x_1, ..., x_n) := ((x_1, ..., x_{n - 1}), x_n)
	\end{equation}
\end{definition}

\subsection{Relations and Functions}

We begin with the definition of a relation, as well as some basic terminology we will need.

\begin{definition}[Relation]
	A \textbf{relation} is a set of ordered pairs. If $R$ is a relation, we use the notation $xRy$ to denote $(x, y)\in R$. 
\end{definition}

\begin{definition}
	Let $R$ be a relation. Then the \textbf{domain} of $R$ is:
	\begin{equation}
		dom(R) := \{x : \exists y (x, y)\in R\}
	\end{equation}
	The \textbf{range} of $R$ is:
	\begin{equation}
		ran(R) := \{y : \exists x (x, y)\in R\}
	\end{equation}
	The \textbf{field} of $R$ is:
	\begin{equation}
		fld(R) := dom(R)\cup ran(R)
	\end{equation}
\end{definition}

To prove that these sets exists, it suffices to show they are a subset of an already constructed set. Note that by 
Lemma~\ref{lemma:3d}, we know that $dom(R)$ and $ran(R)$ are subsets of $\bigcup\bigcup R$. Thus these sets do in 
fact exist. We now define the notion of a function.

\begin{definition}[Function]
	A \textbf{function} $F$ is a relation such that for each $x\in dom(F)$, there is exactly one $y$ with $(x, y)\in F$. This 
	$y$ is called the \textbf{image} of $x$ under $F$, and denoted $y = F(x)$. 
\end{definition}

\begin{definition}
	We say $F$ is a function from $A$ to $B$ if $A = dom(F)$ and $ran(F) \subseteq B$. If $ran(F) = B$, we say that $F$ is 
	\textbf{surjective}. If for each $y\in ran(F)$ we have a unique $x$ with $y = F(x)$, then we say $F$ is \textbf{injective}. 
\end{definition}

This definition of a function is probably very different than what one would typically think of as a function; usually a function 
is thought of as an abstract rule which associates an output with any input, yet we just defined it as what is usually considered 
to be the \textbf{graph} of a function. However, this definition allows us to encode all the information from a function into a 
single set $F$, and so we now can use all of the axioms we have stated to deal with functions. 

We now turn our attention to creating new relations from existing ones. We will apply these ideas to functions as well, as we will 
be able to specify things more precisely with functions.

\begin{definition}
	Let $R, S$ be relations.
	\begin{enumerate}
		\item The \textbf{inverse} of $R$ is the relation:
		\begin{equation}
			R^{-1} := \{(v, u) : (u, v)\in R\}
		\end{equation}
		\item The \textbf{composition} of $R$ and $S$ is the set:
		\begin{equation}
			R\circ S := \{(u, v) : \exists t (uSt\wedge tRv)\}
		\end{equation}
		\item The \textbf{restriction} of $R$ to a set $A$ is the set:
		\begin{equation}
			R|_A := \{(u, v) : u\in A\wedge uRv\}
		\end{equation}
		\item The \textbf{image} of a set $A$ under $R$ is:
		\begin{equation}
			R[A] := ran(R|_A)
		\end{equation}
	\end{enumerate}
\end{definition}

To remember the definition $R\circ S$, think of these as functions $F\circ G$ acting on $x$. If $(x, z)\in F\circ G$, then $z = 
F(G(x))$, i.e. we can find $y$ with $y = G(x)$ and $z = F(y)$, equivalently $x G y$ and $y F z$. We will study composition 
in the specific case of functions.

\begin{prop}
	If $F$ and $G$ are functions, then $F\circ G$ is a function whose domain is:
	\begin{equation}
		dom(F\circ G) := \{x\in dom(G) : G(x)\in dom(F)\}
	\end{equation}
\end{prop}
\begin{proof}
	TODO.
\end{proof}

\begin{definition}[Identity]
	Let $A$ be a set. Then we define the \textbf{identity function} of $A$ to be:
	\begin{equation}
		I_A := \{(a, a) : a\in A\}
	\end{equation}
	If $F : A\rightarrow B$ is a bijection, then $F^{-1}$ is a function and $F\circ F^{-1} = I_B$, $F^{-1}\circ F = I_A$. 
\end{definition}

\begin{theorem}
	Assume $F : A\rightarrow B$ and $A\neq\emptyset$. 
	\begin{enumerate}
		\item There is a function $G : B\rightarrow A$ such that $G\circ F = I_A$ (a \textbf{left inverse}) iff $F$ is injective.
		\item There is a function $G : B\rightarrow A$ such that $F\circ G = I_B$ (a \textbf{right inverse}) iff $F$ is onto $B$. 
	\end{enumerate}
\end{theorem}

\begin{proof}
	TODO. 
\end{proof}

Note that to prove one direction of the surjective part of the theorem, we need to use the axiom of choice. Our original 
formulation of this axiom states that we can choose one element of the range of each element of a relation to turn it into a 
function.

\begin{definition}[Single-rooted]
	A relation $R$ is \textbf{single-rooted} if for each $y\in ran(R)$, there is a unique $x\in dom(R)$ such that $x R y$. 
\end{definition}

Single rooted relations are the analogs of injective functions, and they are useful because they preserve certain equalities 
among the set algebra.

\begin{theorem}
	Let $R$ be a relation. Then for any set $\mathcal A$, we have the following.
	\begin{enumerate}
		\item \begin{equation}R\left[\bigcup\mathcal A\right] = \bigcup_{A\in\mathcal A}R[A]\end{equation}
		\item \begin{equation}R\left[\bigcap\mathcal A\right]\subseteq \bigcap_{A\in\mathcal A} R\left[A\right]\end{equation} 
		with equality iff $R$ is single-rooted.
		\item \begin{equation}R\left[A\setminus B\right]\supseteq R\left[A\right]\setminus R\left[B\right]\end{equation} with 
		equality iff $R$ is single-rooted.
	\end{enumerate}
\end{theorem}
\begin{proof} TODO \end{proof}

These theorems give rise to immediate corollaries about the inverse relations of functions. These are extremely useful in 
areas of math like topology, where the preimage of open sets is used to determined if a function is continuous or not.

\begin{lemma}
	For any function $G$ and sets $A, B$, and $\mathcal A$, we have:
	\begin{enumerate}
		\item \begin{equation}G^{-1}\left[\bigcup\mathcal A\right] = \bigcup\{G^{-1}[A] : A\in\mathcal A\}\end{equation}
		\item If $\mathcal A\neq\emptyset$, we have:
		\begin{equation}
			G^{-1}\left[\bigcap\mathcal A\right] = \bigcap\{G^{-1}[A] : A\in\mathcal A\}
		\end{equation}
		\item \begin{equation}G^{-1}[A\setminus B] = G^{-1}[A]\setminus G^{-1}[B]\end{equation}
	\end{enumerate}
\end{lemma}

\subsection{Equivalence Relations}

\begin{definition}
	Let $R$ be a relation on a set $A$. Then:
	\begin{itemize}
		\item $R$ is \textbf{reflexive} if for each $x\in A$, $x R x$. 
		\item $R$ is \textbf{symmetric} if for all $x, y$, $x R y$ implies $y R x$. 
		\item $R$ is \textbf{transitive} if for all $x, y, z$, $x R y$ and $y R z$ implies that $x R z$. 
	\end{itemize}
\end{definition}

\begin{definition}[Equivalence Relation]
	$R$ is an \textbf{equivalence relation} on $A$ if $R$ is a relation on $A$ and $R$ is reflexive, symmetric, and transitive. 
	If $R$ is an equivalence relation on $A$ and $x\in A$, the set:
	\begin{equation}
		[x]_R := \{t : x R t\}
	\end{equation}
	is called the \textbf{equivalence class} of $x$ modulo $R$. We denote the set of equivalence classes on $A$ modulo 
	$R$ by $A / R$, called the \textbf{quotient} of $A$ mod $R$. 
\end{definition}

We will show that the equivalence classes of an equivalence relation form a partition of a set, and that one can equally well 
describe such a relation with the relation itself, the equivalence classes, or the canonical projection map it induces.

\begin{definition}[Partition]
	A partition $\Pi$ of a non-empty set $A$ is a set of subsets of $A$ such that:
	\begin{itemize}
		\item $\emptyset\in\Pi$. 
		\item $\forall B, C\in\Pi, B\neq C\implies B\cap C = \emptyset$.
		\item $A = \bigcup\Pi$. 
	\end{itemize}
\end{definition}

\begin{lemma}
	Let $R$ be an equivalence relation on $A$. Let $x, y\in A$. Then:
	\begin{itemize}
		\item $x\in [x]_R$.
		\item If $xRy$, then $[x]_R = [y]_R$. 
		\item If $(x, y)\notin R$, then $[x]_R\cap [y]_R = \emptyset$. 
	\end{itemize}
	In other words, the equivalence classes of $R$ form a \textbf{partition} of $A$.
\end{lemma} 

Suppose that $f : A\rightarrow B$ is a function. We may define a relation $E_f$ by $x E_f y$ iff $f(x) = f(y)$. This is in fact 
an equivalence relation. Given a set of equivalence classes $\{[x]_R : x\in A\}$ on a set $A$, we can construct a function 
$f : A\rightarrow A / R, x\mapsto [x]_R$. Then the equivalence clases $[x]_R$ will be exactly the equivalence classes of the 
relation $E_f$. 

As in algebra, every equivalence relation induces a canonical map into the quotient, $\pi_R : A\rightarrow A / R, x\mapsto 
[x]_R$. Given the projection $\pi_R$, we have $R = E_{\pi_R}$. This is because $\pi_R(x) = \pi_R(y)$ iff $x R y$, so $x 
E_{\pi_R} y$ iff $x R y$. 

Finally, given a partition $\Pi$, we can construct an equivalence relation $R_\Pi$ by $x R_\Pi z$ iff $\exists z\in Pi$ such that 
$x\in z$ and $y\in z$. This relation satisfies $\Pi = A / R_\Pi$. We see that given one of the following:
\begin{itemize}
	\item A relation on $A$.
	\item The partition $A / R$. 
	\item The canonical projection $\pi : A\rightarrow A / R$. 
\end{itemize}
we can construct the other two descriptions of a corresponding equivalence relation without any ambiguity. Thus, any of these 
descriptions equivalently identifies an equivalence relation on $A$. 

As one last topic in this section, we will discuss maps between quotients. Let $R$ be an equivalence relation and $G : 
A / R\rightarrow B$ a function. Then we can lift $G$ to $A$ by defining $\hat G = G\circ\pi_R$, i.e. the following diagram 
commutes.
\[
	\begin{tikzcd}
		A \arrow[r, "\hat G"] \arrow[d, "\pi_R"] & B \\
		A / R \arrow[ru, "G"] & {} \\
	\end{tikzcd}
\]

However, the interesting question is to consider the other direction: when can we factor a map through the quotient?
\begin{lemma}
	Let $R$ be an equivalence relation on $A$ and $F : A\rightarrow B$ a function. Then we can factor $F$ through $A / R$ 
	iff $\forall x, y\in A, x R y\implies F(x) = F(y)$ (i.e. $F$ sends each member of each equivalence class to the same image, 
	$F$ is \textbf{constant} on each equivalence class). In other words, we can find $\tilde F : A / R\rightarrow B$ such that 
	the diagram commutes.
\[
	\begin{tikzcd}
		A \arrow[d, "\pi"] \arrow[dr, "F"] & {} \\
		A / R \arrow[r, "\tilde F"] & B \\
	\end{tikzcd}
\]
\end{lemma}

There is one more notion of a function respecting an equivalence relation. Let $F : A\rightarrow A$ be a function, $R$ an 
equivalence relation on $A$. We say that $F$ is \textbf{compatible with $R$} if $xRy\implies F(x) R F(y)$. If this is the case, 
then we can find $\tilde F : A / R\rightarrow A / R$ such that $[F(x)]_R = \tilde F([x]_R)$, i.e. we can factor $F$ through the 
following commutative square.
\[
	\begin{tikzcd}
		A \arrow[r, "F"] \arrow[d, "\pi"] & A \arrow[d, "\pi"] \\
		A / R \arrow[r, "\tilde F"] & A / R \\
	\end{tikzcd}
\]

\newpage
\section{Number Systems}

In this section, we will explicitly construct the most common number systems used in mathematics: the naturals, the integers, 
the rationals, and the reals. Although you have most likely already seen each of these number systems used in different areas 
in math, this section will show that these numbers \textit{actually exist} as concrete mathematical entities. We will build 
them up from scratch using the set theory that we have already established (introducing one new axiom on the way), and 
exhibit that the structures we construct share the important properties of each of these number systems.

\subsection{The Natural Numbers $\mathbb N$}

The natural numbers are the building blocks of all mathematics; they give us a way to quantify objects by counting, and from 
this number system we can construct many other more intricate systems. \textit{In these notes, we will denote the set of natural 
numbers by the symbol $\omega$}. This notation will be useful later as we study ordinals later.As you are probably well aware, 
the set $\omega = \{0, 1, 2, 3, ...\}$\footnote{Although in many areas of math $\omega$ starts at 1, when constructing the 
number system we generally take it to start at 0 and associate this with the empty set. For the positive natural numbers, we 
will use the symbol $\omega^+$.} and is equipped with two canonical operations
addition $+_\omega : \omega\times\omega\rightarrow\omega$ and multiplication $\cdot_\omega : 
\omega\times\omega\rightarrow\omega$. We also require a relation $<_\omega\subset\omega\times\omega$ which describes 
exactly how the natural numbers are ordered, and satisfies some intuitive properties.
We will soon see precisely how to define the system $(\omega, +_\omega, \cdot_\omega, <_\omega)$, and then state and 
prove some statements that may be familiar to you from your discrete math class. Note that I will often drop the subscripts on 
the $+, \cdot$, and $<$ symbols.

In constructing $\omega$, we are looking to associate each number with a set that we will specify, similar to how we 
encoded the definition of a tuple into a set definition. We will define the $+$ and 
$\cdot$ operations on these sets, and we will show that this association of sets with the number system forms a Peano 
system\footnote{The Peano axioms are the basic properties that the set $\omega$ must satisfy.}. The easiest way to 
do this is to first make the assignment $0 := \emptyset$ and count up from this definition. The most obvious choice for 
1 is to let it be the set containing 0 and so on, i.e. we let $1 := \{0\}$, $2 := \{1\}$, $3 = \{2\}$, continuing forever. However, 
this assignment makes it difficult to order the natural numbers. We must have $0 < 1 < 2 < 3 < ...$, but there is no clear 
definition of $<_\omega$ in this case. Instead, we make the following definition.

\begin{definition}[Successor]
	Let $a$ be a set. The \textbf{successor} of $a$, denoted $a^+$, is the set:
	\begin{equation}
		a^+ := a\cup\{a\}
	\end{equation}
\end{definition}

We will represent each natural number as the successor of the preceding number, with 0 being defined as the empty set. 
This is known as the \textbf{von-Neumann representation} of the natural numbers. For the first few numbers, this looks like the 
following:
\begin{equation}
\begin{cases}
	0 := \emptyset \\
	1 := 0^+ = \{\emptyset\} = \{0\} \\
	2 := 1^+ = \{\emptyset, \{\emptyset\}\} = \{0, 1\} \\
	3 := 2^+ = \{0, 1, 2\} \\
	\cdots
\end{cases}
\end{equation}
Note that for each of these small numbers, we may write $n = \{0, 1, 2, ..., n - 1\}$. This is no accident, and in fact will hold for 
a general $n\in\omega$, once we complete the definition of $\omega$. 

\begin{definition}[Inductive set]
	A set $a$ is \textbf{inductive} if $\emptyset\in a$ and if $b\in a$ implies $b^+\in a$. 
\end{definition}

In other words, an inductive set contains 0 and contains the successor of every element in it. Notice that the natural numbers 
will end up being the smallest possible inductive set, as they contain $0, 1, 2, 3$, and so on. However, under the axioms of set 
theory we have currently discussed (axioms 1 - 6) we cannot prove the existence of or construct an inductive set. Hence 
we require axiom 7 to postulate the existence of the natural numbers, which we will restate here for convenience.

\begin{axiom}[Infinity axiom]
	There exists an inductive set.
\end{axiom}
\begin{definition}[Natural number]
	A \textbf{natural number} is a set which belongs to all inductive sets. We let $\omega$ denote the set of natural numbers.
\end{definition}
Note that we can define $\omega$ now by the subset axiom. Pick an inductive set $a$ by the infinity axiom, then define 
$\omega$ to be:
\begin{equation}
	\omega := \{n\in a : \forall b (b\textnormal{ inductive}\implies n\in b)\}
\end{equation}
By examining this construction of $\omega$, it is clear that $\omega$ is an inductive set. Indeed, $\emptyset$ is contained 
in every inductive set, and if $n\in\omega$, then $n$ is in every inductive set, hence $n^+$ is contained in every set, and 
hence $n^+\in\omega$. We state this as a proposition.

\begin{prop}
	$\omega$ is inductive.
\end{prop}	

This proposition, although obvious, will now allow us to state the induction principle, which is a key property of the natural 
numbers. 
\begin{theorem}[Induction Principle]
	Any inductive subset of $\omega$ is $\omega$.
\end{theorem}
\begin{proof}
	If $a\subseteq\omega$ is inductive, then $\omega\subseteq a$ by its definition as the smallest inductive set, hence 
	$a = \omega$.
\end{proof}

The inductive principle is the induction that you know and love, just wrapped in layers of abstraction. To use it like induction, 
we must prove that a subset $a\subseteq\omega$ is inductive. We do this by showing:
\begin{enumerate}
	\item $0\in a$. This is equivalent to showing a base case for the induction.
	\item $n\in a$ implies $n^+\in a$. This is equivalent to the inductive step; If you assume something holds for case $n$ 
	(i.e. $n\in a$), then you must show that it holds for $n^+$ (i.e. $n^+\in a$). 
\end{enumerate}
Many of the following proofs about the set $\omega$ will use induction as a tool. Upon studying them carefully, it will become 
intuitive that the induction principle is the same as the induction you are used to.

\begin{prop}
	Every nonzero natural number $n$ is the successor of another natural number. 
\end{prop}
\begin{proof}
	Let $P = \{n\in\omega : n = 0\vee\exists m\in\omega (n = m^+)\}$. We show that $P$ is inductive. Clearly $0\in P$, 
	so assume that $n\in P$. Then $n\in\omega$, so $n^+\in P$ by definition. Hence $P$ is inductive, so $P = \omega$.
\end{proof}

Along with a set being inductive, there is another important quality a set may have that will be of use to us as we continue 
to study $\omega$; it is called transitivity, not to be confused with transitivity of a relation.

\begin{definition}[Transitive]
	A set $a$ is transitive if:
	\begin{equation}
		\forall x\in a\forall y\in x (y\in a)
	\end{equation}
\end{definition}

An easy way to remember this (and why it is called a transitive set) is that $a$ is transitive if $y\in x\in a$ implies that 
$y\in a$, i.e. if the symbol $\in$ is ``transitive". Note that there are three equivalent definitions of a set being transitive. 
$a$ is transitive if and only if:
\begin{itemize}
	\item $y\in x\in a\implies y\in a$
	\item $a\subseteq Pa$
	\item $\forall b\in a, b\subseteq a$
\end{itemize}

\begin{lemma}
	If $a$ is transitive, then $a = \bigcup a^+$.
\end{lemma}
\begin{proof}
	$\bigcup a^+ = \bigcup a\cup\{a\} = (\bigcup a)\cup a = a$ since $\bigcup a\subseteq a$ because $a$ is transitive.
\end{proof}

\begin{prop}
	Every natural number $n$ is a transitive set.
\end{prop}
\begin{proof}
	Let $P = \{n\in\omega : n\textnormal{ is transitive}\}$. Then $\emptyset\in P$ as it has no members. Suppose that 
	$n\in P$. Then $\bigcup n^+ = n\subseteq n$, so $n^+$ is transitive, hence $n^+\in P$, which shows that $P = \omega$.
\end{proof}

\begin{prop}
	$\omega$ is a transitive set.
\end{prop}
\begin{proof}
	Again by induction. Let $P = \{n\in\omega : n\subseteq\omega\}$. Then $\emptyset\in P$ and if $n\in P$, $n^+ = n\cup
	\{n\}\subseteq\omega$ because $n\subseteq\omega$ and $\{n\}\subseteq\omega$, so $P$ is inductive.
\end{proof}

Notice that although we do not define the ``predecessor" operation, for nonzero $n$ the previous natural number is 
$\bigcup n$, as $n = \bigcup n^+$. This proves the following claim.
\begin{prop}
	The function $\omega\rightarrow\omega, n\mapsto n^+$ is injective. 
\end{prop}

Our next goal is to build up the operations of addition and multiplication on $\omega$. We will do this by recursively defining 
these operations, but first we need to define what we mean by recursion. We will use the following theorem.

\begin{theorem}[Recursion]
	Let $A$ be a set, $a\in A$ and $F : A\rightarrow A$. Then there exists a unique function $h : \omega\rightarrow A$ 
	such that:
	\begin{enumerate}
		\item $h(0) = a$
		\item $\forall n\in\omega, h(n^+) = F(h(n))$
	\end{enumerate}
\end{theorem}
For notation which may be a little more intuitive, notice that this theorem allows us to define a sequence by defining $a_0$ and 
$a_{n^+} = F(a_n)$, i.e. the function $h$ represents a sequence of elements of $A$, and $F$ tells us how to get from 
one element in the sequence to the next. The proof of this is involved, but can be found in any textbook on the subject or the 
internet. Using recursion, we may now define addition and multiplication.

\begin{definition}[Addition]
	For $m\in\omega$, use recursion to define a function $A_m : \omega\rightarrow\omega$ with $A_m(0) := m$ 
	and $A_m(n^+) := A_m(n)^+$ (here $F : \omega\rightarrow\omega$ with $F(k) = k$). We define:
	\begin{equation}
		m + n := A_m(n)
	\end{equation}
\end{definition}

\begin{definition}[Multiplication]
	For $m\in\omega$, define $M_n : \omega\rightarrow\omega$ with $M_m(0) := 0$ and $M_m(n^+) := M_m(n) + m$. 
	Define:
	\begin{equation}
		m\cdot n := M_m(n)
	\end{equation}
\end{definition}

Without the clutter of $A_m$ and $M_m$, note that these definitions are equivalent to letting $m + 0 = m$, $m + n^+ = (m + 
n)^+$, $m\cdot 0 = 0$, and $m\cdot n^+ = m\cdot n + m$, so these operations do exactly what we expect. We must now 
prove that they have additional properties that we expect, so that we do not have to wonder whether an expression like $1 + 2$ 
is the same as $2 + 1$. 

\begin{theorem}
	Addition and multiplication are associative and commutative. Addition distributes over multiplication, i.e. $m\cdot (n + p) 
	= m\cdot n + m\cdot p$. 
\end{theorem}

\begin{proof}
Fix a number in the equation you are trying to prove and induct on the other number. You will need 
to prove the following lemma as well:
\begin{lemma}
	$\forall m\in\omega, 0 + m = m$
\end{lemma}
\end{proof}

The last piece of the puzzle for $\omega$ is to define the ordering $<_\omega$. As each natural number has the form 
$n = \{0, 1, 2, ..., n - 1\}$, we see that each $n\in\omega$ contains all smaller natural numbers. Indeed, we make the 
definition of $<_\omega$ to be the membership operation.
\begin{definition}[$<_\omega$]
	For $m, n\in\omega$, define $m <_\omega n$ if $m\in n$. We write $m\leq_\omega n$ if $m <_\omega n$ or $m = n$. 
\end{definition}

\begin{lemma}
	For $m, n\in\omega$, the following hold.
	\begin{enumerate}
		\item $m\in n$ iff $m^+\in n^+$. 
		\item $m\not\in m$.
	\end{enumerate}
\end{lemma}

\begin{definition}[Linear ordering]
	Let $<$ be a relation on a set $a$. We say that $<$ is a \textbf{linear ordering} on $a$ if:
	\begin{enumerate}
		\item $<$ is transitive.
		\item The relation $<$ is a \textbf{trichotomy}, i.e. for any $x, y\in a$, \textit{precisely one of the following holds.}
			\begin{itemize}[-]
				\item $x < y$
				\item $y < x$
				\item $x = y$
			\end{itemize}
	\end{enumerate}
\end{definition}

\begin{theorem}
	The relation $<_\omega$ is a linear ordering on $\omega$.
\end{theorem}

\begin{proof}
	We have already shown that $<_\omega$ is a transitive relation by showing that each $n\in\omega$ is a transitive set. 
	To show $<_\omega$ is a trichotomy, let $T = \{n\in\omega : \forall m\in\omega (n <_\omega m\wedge n = m\wedge 
	m <_\omega n)\}$, and we show $T$ is inductive. Clearly $\emptyset\in T$, so let $n\in T$. We show that $n^+\in T$ 
	as well. Let $m\in\omega$. Then either we have $m <_\omega n$, $m = n$, or $n <_\omega m$. Suppose that 
	$m <_\omega n$. Then we have $m <_\omega n <_\omega n^+$, since $n\in n^+$. If $m = n$, then again $m <_\omega
	n^+$. Finally, suppose that $m >_\omega n$. If $m = n^+$, we are done. If not, then $n^+\in m$, hence $n^+ <_\omega 
	m$. In any case, we see that $n^+\in T$< so $T$ is a transitive set and we are done.
\end{proof}

Linear orderings will come up later in the course when we discuss another type of ordering called a well-ordering, so keep the 
intuition in mind from the linearly ordered system $(\omega, <_\omega)$. To end this section, we will state a few simple 
propositions and theorems about $(\omega, <_\omega)$. 

\begin{corollary}
	For all $m, n, p\in\omega$, we have:
	\begin{enumerate}
		\item $m <_\omega n\iff m\subset n$ (strictly).
		\item $m\leq_\omega n\iff m\subseteq n$.
		\item $m < n\iff m + p <_\omega n + p$
		\item If $p\neq 0$, then $m < n\iff m\cdot p <_\omega n\cdot p$. 
	\end{enumerate}
\end{corollary}

\begin{theorem}[Well-ordering property for $\omega$]
	Let $A\subseteq\omega$ be nonempty. Then $A$ has a least element, that is, $\exists a\in A$ such that $a\leq b\forall 
	b\in A$. 
\end{theorem}
\begin{proof}
	Assume that $A$ does not have a least element. Let $B = \{n\in\omega : \forall m <_\omega n, m\notin A\}$ and show 
	that $B$ is inductive. 
\end{proof}

\begin{theorem}[Strong induction principle]
	Let $A\subseteq\omega$ and assume that $\forall a\in \omega$, if $\forall b < a, b\in A$, then $a\in A$. Then $A = 
	\omega$.
\end{theorem}
\begin{proof}
	Suppose that $\omega\neq A$, so $\omega\setminus A$ is nonempty. Then the set $\omega\setminus A$ has a least 
	element, and show that this leads towards a contradiction.
\end{proof}

\newpage
\subsection{The Integers $\mathbb Z$ and the Rational Numbers $\mathbb Q$.}

We will now construct the integers and the rational numbers. The construction of both of these systems is quite similar; we 
will take an existing structure ($\omega$ for the case of the integers, and $\mathbb Z$ for the case of the rationals) and 
manipulate it via set products and quotients until we have an object that behaves as we wish it to. 

We begin with the integers. As we have already constructed $\omega$, we have the numbers $0, 1, 2, 3, ...$ equipped with a 
canonical addition and multiplication. We wish to include additive inverses in addition to the numbers in $\omega$ in the 
form of $-1, -2, -3, ...$, such that the entire set forms a commutative ring under $+$ and $\cdot$. 

\subsection{The Real Numbers $\mathbb R$}

\section{Acknowledgements}

These notes are primarily based on two sources. First, Pierre Simon's Math 135 lectures at the \textit{University of 
California, Berkeley}. Second, the text \textit{Elements of Set Theory} by Herbert Enderton.

\end{document}