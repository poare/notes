\def \root {../../..}			% path to root (/notes)
%\def \root {/Users/theoares/Dropbox\ (MIT)/notes}
\input{\root/template/preamble.tex}

\usepackage[toc,page]{appendix}
\usepackage{listings}

\newcommand{\poarecomment}[1]{\textcolor{red}{#1}}

% Sort bibliography in correct order
\bibliographystyle{unsrtnat}
\usepackage[numbers,sort&compress]{natbib}

\renewcommand{\im}{\mathrm{Im}}
\newcommand{\re}{\mathrm{Re}}
\newcommand{\esssup}{\mathrm{ess}\,\mathrm{sup}}

\begin{document}

\title{Notes on Matrix-Valued Spectral Function Reconstructions}
\author{Patrick Oare}
%\affiliation{Center for Theoretical Physics, Massachusetts Institute of Technology, Boston, MA 02139, USA}

\date{\today}
% \preprint{MIT-CTP/...}

\maketitle

\begin{abstract}
Notes on solving the spectral function inversion problem in lattice QCD for matrix-valued correlators. 
\end{abstract}

\section{Introduction}

TODO

abc

\section{The GEVP}

\section{Formalism}

The beauty of the matrix-valued reconstruction problem is that the interpolation problem maps onto another well-studied function class. For the matrix-valued case, instead of Schur functions, we'll be studying \textbf{Carath\'eodory functions}. 


\bibliography{spectral_matrix_corrs}

\newpage
\begin{appendices}

\newpage
\section{Blaschke Products}
\label{app:blaschke}

TODO
\end{appendices}

\end{document}