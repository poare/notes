\documentclass[11pt, oneside]{article}   	% use "amsart" instead of "article" for AMSLaTeX format
\usepackage[margin = 1in]{geometry}                		% See geometry.pdf to learn the layout options. There are lots.
\geometry{letterpaper}                   		% ... or a4paper or a5paper or ... 
%\geometry{landscape}                		% Activate for rotated page geometry
%\usepackage[parfill]{parskip}    		% Activate to begin paragraphs with an empty line rather than an indent
\usepackage{graphicx}				% Use pdf, png, jpg, or eps§ with pdflatex; use eps in DVI mode
								% TeX will automatically convert eps --> pdf in pdflatex		
\usepackage{amssymb}
\usepackage{amsmath}
\usepackage[shortlabels]{enumitem}
\usepackage{float}
\usepackage{tikz-cd}

\usepackage{amsthm}
\theoremstyle{definition}
\newtheorem{definition}{Definition}[section]
\newtheorem{theorem}{Theorem}[section]
\newtheorem{corollary}{Corollary}[theorem]
\newtheorem{lemma}[theorem]{Lemma}

\newcommand{\N}{\mathbb{N}}
\newcommand{\R}{\mathbb{R}}
\newcommand{\Z}{\mathbb{Z}}
\newcommand{\Q}{\mathbb{Q}}

%SetFonts

%SetFonts


\title{Differential Forms}
\author{Patrick Oare}
\date{}							% Activate to display a given date or no date

\begin{document}
\maketitle

Differential forms generalize the notion of integration over Euclidean space to an arbitrary smooth manifold. Although they take 
a bit of manifold theory to even get to the definition, they are very useful for all areas of math and physics. In particular, they 
provide the ultimate generalization of multivariable calculus. We will see that there is a general analog of Stokes' theorem which 
is the origin of all the fundamental theorems of calculus you have studied (i.e. the one-dimensional version, divergence theorem, 
Green's theorem, etc.). 

Once we work up to the definition of a form and how to integrate such an object, we will turn to cohomology theories which 
stem from the existence of forms. Namely, we will study the de Rham complex on a manifold, which is the cochain complex 
generated by the exterior derivative. We will then turn to \v{C}ech cohomology and its similarities with the de Rham 
cohomology. 

\section{Background}

I will most likely write this up into a separate notes on differential manifolds later, but for now I will state the main definitions 
and theorems that we will need to study forms. I will assume the reader is familiar with the definition of a differentiable 
manifold $M$, but we will briefly recap the definitions for notations sake. 

Note in these notes unless otherwise specified we will use summation convention, so any repeated index is summed over. 
We will also try to be accurate with upper and lower indices. 

\subsection{Smooth Manifolds}

A \textbf{chart} on a space is a pair $(U, \phi)$ 
such that $U\subseteq M$ is open and $\phi : U\xrightarrow{\sim}\mathbb R^n$ is a homeomorphism onto an open subset 
$\phi(U)\subseteq\mathbb R^n$. We call $n$ the 
\textbf{dimension} of $M$. A collection $\{(U_\alpha, \phi_\alpha)\}_{\alpha\in A}$ of charts which cover $M$ is called an 
\textbf{atlas} if the overlap functions $\phi_\alpha\circ\phi_\beta^{-1}$ are smooth on $\mathbb R^n$ whenever their images 
overlap; this condition on the charts is known as being \textbf{$C^\infty$ compatible}. A \textbf{smooth manifold} is a topological 
manifold with a \textbf{maximal atlas}, namely an atlas $\mathfrak M$ such that if $\mathfrak A$ is another atlas of $M$ which 
contains $\mathfrak M$, then $\mathfrak M = \mathfrak A$. 

These definitions are technical, and most likely will never be used in these notes. The important property that a smooth manifold 
$M$ has is it can be covered by charts, which put a coordinate system around every point $p\in M$. We will often write a chart 
$(U, \phi)$ as $(U, x^1, ..., x^n)$, where $x^i$ are the coordinates on the chart $U\subset M$. Formally, if we let $r^i : \mathbb 
R^n\rightarrow\mathbb R$ be the coordinate functions on $\mathbb R^n$, then the coordinates $x^i$ are defined as 
$x^i := r^i\circ\phi$. 

A function $f : M\rightarrow\mathbb R$ is \textbf{smooth}, or $C^\infty$, at a point $p\in M$ if there is a chart $(U, \phi)$ about 
$p$ such that $f\circ \phi^{-1} : \mathbb R^n\rightarrow\mathbb R$ is smooth as a map between Euclidean spaces. $f$ is a 
\textbf{smooth function} if it is smooth at each $p\in M$. We use the notation:
\begin{equation}
	C^\infty(M) := \{ f : M\rightarrow\mathbb R \; | \; f \textnormal{ is smooth.}\}
\end{equation}
A map $f : M\rightarrow N$ between two manifolds $M$ and $N$ (of dimensions $m$ and $n$, respectively) is said to be 
\textbf{smooth} at $p\in M$ if there are charts $(U, \phi)$ about $p\in M$ and $(V, \psi)$ about $f(p)\in N$ such that the 
composition $\psi\circ f\circ\phi^{-1} : \mathbb R^m\rightarrow\mathbb R^n$ is smooth. $f$ is smooth if it is smooth at each point 
in $M$. A \textbf{diffeomorphism} is an isomorphism in the category of smooth manifolds, and it is a bijective $C^\infty$ map $M
\rightarrow N$ which has a smooth inverse. If $F : M\rightarrow N$ is a map and $(V, y^1, ..., y^n)$ is a chart about $\phi(p)\in 
N$, then the components of $F$ are functions $F^i := y^i\circ F : F^{-1}(V)\subset M\rightarrow\mathbb R$, and $F$ is 
smooth iff its component functions are smooth on each chart (by the overlap condition, if $F^i$ is smooth on one chart then it 
is smooth on every chart). 

We can extend the notion of differentiation immediately to $C^\infty$ functions given a chart. Let $p\in M$ be contained by 
a chart $(U, \phi)$. Let $r^i$ be the standard coordinates on $\mathbb R^n$ and push these to coordinates $x^i$ on $M$. For
$f\in C^\infty(M)$ the \textbf{partial derivative} of $f$ at $p\in M$ is:
\begin{equation}
	\frac{\partial}{\partial x^i}\bigg|_p f := \frac{\partial f}{\partial x^i}(p) := \frac{\partial}{\partial r^i}\bigg|_{\phi(p)} (f\circ\phi^{-1})
\end{equation}
where the last expression is defined as a derivate on $\mathbb R^n$. Note that $\partial x^i / \partial x^j = \delta^i_j$, as it 
should. 

We conclude this section by stating the Inverse Function Theorem. A map $f : M\rightarrow N$ is called \textbf{locally 
invertible} at a point $p\in M$ if there is a neighborhood $U$ about $p$ such that $f|_U : U\rightarrow f(U)$ is a 
diffeomorphism. 
\begin{theorem}[Inverse Function Theorem]
	Let $f : M\rightarrow N$ be a map between two manifolds of the same dimension, and $p\in M$. Let $(U, x^i)$ be a 
	chart about $p$, and suppose that there is a chart $(V, y^i)$ about $f(p)$ such that $f(U)\subseteq V$. Then $f$ is 
	locally invertible iff the Jacobian determinant is nonzero, i.e. iff
	\begin{equation}
		det\left(\frac{\partial f^i}{\partial x^j}(p)\right)\neq 0
	\end{equation}
	where $f^i = y^i\circ f$. 
\end{theorem}

\subsection{The Tangent Bundle}

To each point $p$ on a manifold $M$, we can associate a space of tangent vectors, called the \textbf{tangent space} 
and denoted $T_p M$. Visualizing this in your head provides good intuition, but it is not perfect; namely, when you 
visualize tangent vectors to a point, you tend to think of the manifold being embedded into a higher dimensional Euclidean 
space. Although manifolds often come with such an embedding, we want a way to define tangent vectors independent of 
embedding in a larger space-- a way to define a tangent space that is intrinsic to the manifold itself. To do this, we note 
that for each tangent vector $v\in T_p M$, we can define a curve $\alpha : (-\epsilon, \epsilon)\rightarrow M$ such that 
$\alpha(0) = p$ and $\alpha'(0) = v$. In other words, we can find a curve running through $p$ whose derivative is $v$. 

One way to make this definition is to define the tangent space as the tangent vectors of curves running through the point, but 
we will take a slightly different approach. For us, a tangent vector will be defined as an operator which acts like a directional 
derivative, since for each vector we can associate a such a directional derivative. For a function $f\in C^\infty(M)$, we define 
the \textbf{germ} of $f$ at $p$ to be the class of functions which equal $f$ on some open neighborhood about $p$. This is 
an equivalence relation $\sim$ on $C^\infty(M)$, and we denote by $C_p^\infty(M)$ the set of germs of $C^\infty(M)$ functions 
at $p$. The notion of a derivative can be shown to be in one-to-one correspondence with any operator that satisfies the 
Leibniz rule, called a derivation:
\begin{definition}[Derivation, tangent vector]
	A \textbf{derivation} at a point $p\in M$ is a linear map $D : C_p^\infty(M)\rightarrow\mathbb R$ such that:
	\begin{equation}
		D(fg) = D(f) g(p) + f(p) D(g)
	\end{equation}
	A \textbf{tangent vector} at a point $p\in M$ is a derivation at $p$, and we denote the space of tangent vectors by 
	$T_pM$. 
\end{definition}

Given a chart $(U, x^i)$, the set of partial derivatives evaluated at $p$ are derivations at $p$, and in fact form a basis for the tangent space:
\begin{equation}
	\textnormal{span}\left\{\frac{\partial}{\partial x^i}\bigg|_p\right\}_{i = 1}^n = T_p M
\end{equation}
which is easy to prove. The tangent space of an $n$-dimensional manifold is thus an $n$-dimensional vector space. So, given 
a tangent vector $X_p\in T_p M$, we define $X_p$ by its action $X_p f$ on functions $f\in C^\infty(M)$ (since we can drop 
each function down to the germs at $p$, $C_p^\infty(M)$). We can similarly expand it about the derivatives:
\begin{equation}
	X_p = \sum_{i = 1}^n X^i\partial_i
\end{equation}
where $X^i$ are the coordinates of the vector $X$. 

Given a map $f : M\rightarrow N$, we get a natural map $f_* : T_p M\rightarrow T_{f(p)} N$ for each $p\in M$ defined by:
\begin{equation}
	(f_*X_p)(g) := X_p(g\circ f)
\end{equation}
where here $g$ is a function in $C^\infty(N)$, and so $g\circ f\in C^\infty(M)$. This map is generally called the 
\textbf{push-forward}, or the \textbf{differential}. When we dualize to study covectors, we will see its analog is a natural chain 
map between the de Rham cochain complexes. The push forward can be thought of as a sort of ``derivative" since it 
approximates a smooth map as a linearized map between tangent spaces. It is often denoted as $df$ instead of $f_*$, but 
we will reserve the use of $df$ for discussing differential forms\footnote{We will see that when the exterior derivative acts on a 
map $h : M\rightarrow\mathbb R$, the one-form $dh$ is naturally the map $h_* : T_p M\rightarrow\mathbb R\in T_p^* M$, and 
so the notation $df$ and $f_*$ is really one and the same for real-valued maps in $C^\infty(M)$.} later. Indeed, it even 
satisfies a functorial chain rule.
\begin{theorem}
	Let $f : M\rightarrow N$ and $g : N\rightarrow P$ be smooth maps between manifolds. Then:
	\begin{equation}
		(g\circ f)_* = g_*\circ f_* : T_p M\rightarrow T_{(g\circ f)(p)}P
	\end{equation}
\end{theorem}
Let $U\subset M$ be open and $f : M\rightarrow N$ be a smooth map. If $f_* : T_p M\rightarrow T_{f(p)}N$ is an injective map 
as $p$ ranges over $U$, then $f$ is called an \textbf{immersion} on $U$, and if $f_*$ is surjective on $U$, then $f$ is 
called a \textbf{submersion}. 

The set of tangent spaces to a manifold naturally has a fiber bundle structure. We define the \textbf{tangent bundle} $TM$ to 
be the set:
\begin{equation}
	TM := \coprod_{p\in M} T_p M
\end{equation}
There is a natural projection map sending each vector to its base point:
\begin{align}
	\pi : TM &\rightarrow M \\
		X_p &\mapsto p
\end{align}

We can define a topology on the total space $TM$. To define this topology, consider a chart $(U, \phi) = (U, x^i)$, and let $n = 
\dim(M)$. On $TU$, we can represent a general element by $(p, v^i\partial_i) = (x^1, ..., x^n, v^1, ..., v^n)$, where $p = 
(x^1, ..., x^n)$ and the derivatives $\partial_i$ are evaluated at $p$. So, we have a natural map:
\begin{align}
	\tilde \phi : TU &\rightarrow \mathbb R^{2n} \\
	(p, v^i\partial_i) &\mapsto (x^1, ..., x^n, v^1, ..., v^n)
\end{align}
This is a bijection between $TU$ and $U\times R^n$, and so we can use this to transfer the topology of $R^{2n}$ to $TU$. 
This makes $TU$ into a topological space. Furthermore, given a maximal atlas $\{(U_\alpha, \phi_\alpha)\}$ of $M$, the 
set $\{(TU_\alpha, \tilde \phi_{\alpha *})\}$ is a maximal atlas for $TM$. So, the bundle \textbf{$TM$ is naturally a $2n$ 
dimensional manifold}. As a fiber bundle, the total space is $TM$ and the base space is $M$. 

Given a map $f : M\rightarrow N$, we can use the differential at each tangent space to define a push forward between the two 
bundles:
\begin{align}
	f_* : TM &\rightarrow TN \\
		(p, X_p) &\mapsto (f(p), f_*(X_p))
\end{align}
which is naturally a morphism of bundles. 

Recall the definition of a section of a fiber bundle.
\begin{definition}[Section]
	Let $\pi : E\rightarrow B$ be a fiber bundle. A \textbf{section} of $(\pi, E, B)$ is a continuous map $s : B\rightarrow E$ 
	such that $\pi\circ s = id_B$. A section is \textbf{smooth} if $s$ is a smooth map. We denote the set of all $C^\infty$ 
	sections of of $E$ by $\Gamma(E)$. 
\end{definition}
\begin{definition}[Frame]
	A \textbf{frame} of a vector bundle $\pi : E\rightarrow B$ over an open set $U\subseteq B$ is a collection of sections 
	$\{s_1, ..., s_r\}$ such that at each $p\in B$, $\{s_1(p), ..., s_r(p)\}$ is a basis for the fiber $\pi^{-1}(p)$. 
\end{definition}
A section\footnote{In physics, whenever we think about a ``field", we are really thinking about sections of a bundle (or we are 
just thinking about standard functions). For example, a QCD gauge field is a section of a fiber bundle with fibers $SU(3)$ over 
the base space $\mathbb R^4$. Sections are thus a concept that you've almost certainly seen before, albeit in a much less 
rigorous way.} of the tangent bundle is called a \textbf{vector field}, which is perhaps the easiest way to visualize the definition 
of a section; it is simply an assignment of each base point to an element in its fiber which is continuous. A frame on the bundle 
$TM$ is then a set of pointwise linearly independent vector fields, and the requirement that the tangent bundle has a global 
frame means that there are $n$ nowhere vanishing, linearly independent vector fields on the entire manifold. As an 
example of a frame, the vector fields $\{\partial_x, \partial_y, \partial_z\}$ form a global frame on $\mathbb R^3$. 

\subsection{Partitions of Unity}

\section{Differential Forms}

At their most basic level, differential form generalize differential notions of length, area, volume, etc. in Euclidean space to 
arbitrary manifolds. These generalize the notion of integration to manifolds, and in the next section we will see how to integrate 
a form. If I give you a vector (a one-dimensional shape) and ask you its length, what do you say? You give me a real number! 
The length $\ell$ of a vector in a space $V$ is simply a map $V\rightarrow\mathbb R$, i.e. it is a dual vector $\ell\in Hom(V, 
\mathbb R)$. 

This generalizes to higher dimensional objects, which we will take to be ``oriented products" of vectors. 
For example, in two dimensions, suppose I represent a parallelogram with sides $v$ and $w$ by some object $v\wedge w$. 
Then to have an oriented area, I must have $v\wedge w = -w\wedge v$. There is a natural construction called the 
\textbf{exterior power} of a vector space which takes in a space $V$ and spits an object with such an antisymmetric product 
of copies of $V$, and later we will use this to define a $k$-form, our generalized notion of volume. 

\subsection{1-forms}

Before working with $k$-forms, we must define 1-forms. 
\begin{definition}[Cotangent space]
	Let $p\in M$. The \textbf{cotangent space} at $p$ is the dual of the tangent space at $p$, i.e. is:
	\begin{equation}
		T_p^* M := (T_p M)^* = Hom(T_p M, \mathbb R)
	\end{equation}
	We call elements of the cotangent space \textbf{covectors}.
\end{definition}

So, a covector takes in a vector and spits out a number, i.e. if $\omega_p\in T_p^* M$, then $\omega_p(X_p)\in\mathbb R$ 
for any vector $X_p\in T_p M$. Let $(U, x^i)$ be a chart about $p\in M$. Then as $\{\partial_i\}_{i = 1}^n$ is a basis for 
$T^p M$, we have a corresponding dual basis $\{dx^i\}_{i = 1}^n$. The dual basis satisfies:
\begin{equation}
	dx^i(\partial_j) = \delta^i_j
\end{equation}
We will soon see that $dx^i$ can be determined by acting the exterior derivative on the coordinate function $x^i$, and so this 
notation is quite suggestive. Note that we can use the cotangent spaces to define a fiber bundle:
\begin{definition}[Cotangent bundle]
	The \textbf{cotangent bundle} $T^*M$ of a manifold $M$ is:
	\begin{equation}
		T^* M := \coprod_{p\in M} T_p^* M
	\end{equation}
\end{definition}
We can define a bijection with $\mathbb R^{2n}$ by sending $(x^1, ..., x^n, \omega_i dx^i)$ to the corresponding element 
$(x^1, ..., x^n, \omega_1, ..., \omega_n)$, and use this to transfer the topology to $T^*M$, making it into a smooth manifold 
just like $TM$. We can now define a 1-form (for good measure we also define a 0-form):
\begin{definition}[1-form]
	A \textbf{differential 1-form} $\omega$ is a section of the cotangent bundle, i.e. a map $\omega : M\rightarrow T^* M$ with 
	$\pi\circ \omega = id_M$. A 1-form is \textbf{smooth} if it is smooth as a map $M\rightarrow T^*M$. The set of all 
	$C^\infty$ 1-forms will be denoted by $\Omega^1(M)$, so we have:
	\begin{equation}
		\Omega^1(M) := \Gamma(T^*M)
	\end{equation}
\end{definition}
\begin{definition}[0-form]
	We define a 0-form on $M$ to be a smooth function:
	\begin{equation}
		\Omega^0(M) := C^\infty(M)
	\end{equation}
\end{definition}
We already have seen some examples of 1-forms: consider a function $f\in C^\infty(M)$. The induced push forward map 
is $f_* : T_p M\rightarrow T_{f(p)}\mathbb R$. However, there is a canonical isomorphism $\iota : T_{q}\mathbb \mathbb R
\rightarrow\mathbb R$, $a\frac{d}{dt}|_q\mapsto a\in\mathbb R$. Hence, we consider the \textbf{differential map} $df|_p 
:= \iota\circ f_*$, which is a map $T_p M\rightarrow \mathbb R$. This has the exact same action as the push-forward, we 
just ``forget" the basis vector and consider the corresponding scalar. Thus, $df|_p$ is a linear map from $T_p M$ into 
$\mathbb R$, hence is a covector, $df|_p\in T_p^* M$. As we vary $p$ across the entire manifold, we see that the map $df$ is 
in fact a section of the cotangent bundle, and thus $df$ is a 1-form, which we will soon see is a smooth 1-form. 

The explicit action of $df\in\Omega^1(M)$ on a vector $X_p\in T_pM$ is:
\begin{equation}
	df(X_p) = X_p(f)
\end{equation}
which is valued in $\mathbb R$, as $X_p$ maps $C^\infty(M)$ into $\mathbb R$. Note the difference with $f_*$; the 
push forward $f_*$ maps $X_p$ to an element of the tangent space of the codomain of $f$, while $df$ maps $X_p$ to 
a real number. For $f\in C^\infty(M)$ these are isomorphic, but it is an important distinction between the two. For example, 
$f_* X_p$ acts on functions $\mathbb R\rightarrow\mathbb R$, and its action on $h : \mathbb R\rightarrow\mathbb R$ is 
$(f_* X_p)(h) = X_p(h\circ f)$, although when we consider $df (X_p)$ we need not act it on the a function $h$ to get a 
real number; it just spits out a real number immediately.

%We're going to overload our notation 
%a bit, and we will write either $df$ or $f_*$\footnote{Technically, here we are taking $f_* : T_pM\rightarrow T_{f(p)}\mathbb R$ 
%and $df : T_p M\rightarrow\mathbb R$. This is not too important, except that technically we have the correspondence 
%$f_*(X_p) = (df|_p)(X_p) \frac{d}{dt}|_{f(p)}$ because $f_*(X_p)$ is a vector in the tangent space $T_{f(p)}\mathbb R$, while 
%$df|_p X_p$ is a scalar, and so must be multiplied by the basis vector $\frac{d}{dt}|_{f(p)}$ to be pulled into the tangent 
%space that $f_*$ maps into. These spaces are isomorphic, so there is no need to really worry about this.} for the induced 
%map $TM\rightarrow T\mathbb R$. The induced map $df : T_p M
%\rightarrow T_{f(p)}\mathbb R\cong\mathbb R$ sends an element of $T_p M$ to a scalar, and hence $df|_p$ is a covector. 
%As we vary $p$ across the entire manifold, we see that the map $df$ is in fact a section of the cotangent bundle, and 
%thus $df\in\Omega^1(M)$. We will soon generalize this to the exterior derivative. 

Consider what happens when we apply this differential to the coordinate 
functions $dx^i$. Then acting on the basis elements, by definition of the induced map we have:
\begin{equation}
	dx^i(\partial_j) = \partial_j x^i = \delta^i_j
\end{equation}
and so the original notation of $\{dx^i\}$ as the dual basis was suggestive of this actually being the differential 
of the coordinate charts! 

Now, suppose we have a function $f\in C^\infty(M)$. We can expand out $df = a_i dx^i$ in a chart $(U, x^i)$, since at 
each point $\{dx^i|_p\}$ is a basis for $T_p^* M$. To determine the coefficients $a_i$, act each side on $\partial_j$:
\begin{equation}
	df(\partial_j) = a_i dx^i(\partial_j)\implies \partial_j f = a_i\delta^i_j = a_j
\end{equation}
This implies that for any function $f\in C^\infty(M)$, the differential of a function can be written as:
\begin{equation}
	df = \frac{\partial f}{\partial x^i}dx^i
\end{equation}

Note that any 1-form $\omega = a_i dx^i$ is determined completely by the coordinate functions $a_i(p)$ for $p\in M$. It is 
a reasonably simple proof to show that $\omega$ is $C^\infty$ iff each of its coordinate functions $a_i(p)$ are $C^\infty$ as 
well. Thus we have the following corollary:
\begin{corollary}
	If $f\in C^\infty(M)$, then $df\in\Omega^1(M)$, so $df$ is a smooth 1-form. 
\end{corollary}
\begin{lemma}
	Let $\omega$ be a 1-form. If $f$ is a function and $X$ is a vector field on $M$, then $\omega(fX) = f\omega(X)$. 
\end{lemma}
Finally, note that we can scalar multiply 1-forms by functions. For $g\in C^\infty(M)$ and $\omega\in\Omega^1(M)$, we can 
define a one form $g\omega\in\Omega^1(M)$ by scalar multiplication:
\begin{equation}
	(g\omega)|_p := g(p)\omega_p
\end{equation}
Because this is scalar multiplication, $g\omega = \omega g$ since $g(p)$ can simply pull out of 1-forms, since the 
action of a 1-form $g\omega$ on a vector field $X$ at $p\in M$ is $(g\omega) X = (g\omega)|_p X_p = g(p)\omega_p X_p
= \omega_p g(p) X_p$. 

Since we have dualized the tangent bundle to define 1-forms, a map between manifolds induces a pullback map 
of forms, because the direction of the push forward map must reverse itself. 
\begin{definition}[Pullback]
Let $f : M\rightarrow N$ be a smooth map. Then at each $p\in M$, we induce the \textbf{pullback map} $f^* : T_{f(p)}^* N 
\rightarrow T_p^* M$ defined pointwise by its action on vectors:
\begin{equation}
	f^*(\omega)|_p X_p := \omega_{f(p)}(f_* X_p)
\end{equation}
We will often write these as fields and instead write this relation between sections of the tangent and cotangent bundles:
\begin{equation}
	f^*(\omega) X = \omega(f_* X)
\end{equation}
Pictorially, the following diagram commutes:
\begin{equation}\begin{tikzcd}
	T_p M\arrow[r, "f_*"] \arrow[dr, bend right, "f^*\omega"'] & T_p N \arrow[d, "\omega"] \\
	& \mathbb R
\end{tikzcd}\end{equation}
Let $\omega\in\Omega^1(N)$ be a 1-form. Then $f^*$ extends to a pullback on 1-forms:
\begin{equation}
	f^* : \Omega^1(N) \rightarrow\Omega^1(M)
\end{equation}
by pulling back each form pointwise, $f^*(\omega)|_p := f^*(\omega_p)$. 
\end{definition}

Any map can also pull back a function $g\in C^\infty(M)$ by precomposition, i.e. we get an induced map $f^* : C^\infty(N)
\rightarrow C^\infty(M)$, $f^*(g) := g\circ f$, and so we have pullbacks $f^* : \Omega^k(N)\rightarrow\Omega^k(M)$ for $k\in 
\{0, 1\}$. There are a few particularly important properties of the pullback. These will later generalize and show us that $f^*$ is 
a chain map on the de Rham complex, but for now we will show how this works with the differential $d : \Omega^0(M)
\rightarrow \Omega^1(M)$. 
\begin{theorem}
	Let $f : M\rightarrow N$ be a $C^\infty$ map between manifolds. Then the induced map $f^* : \Omega^k(N)\rightarrow 
	\Omega^k(M)$ for $k = 0, 1$ satisfies:
	\begin{enumerate}
		\item $f^* d = df^*$, i.e. the following diagram commutes:
		\begin{equation}\begin{tikzcd}
			C^\infty(N)\arrow[r, "f^*"]\arrow[d, "d"] & C^\infty(M) \arrow[d, "d"] \\
			\Omega^1(N)\arrow[r, "f^*"] & \Omega^1(M)
		\end{tikzcd}\end{equation}
		\item $f^*(\omega + \tau) = f^*\omega + f^*\tau$.
		\item For $g\in C^\infty(N)$, $f^*(g\omega) = (f^*g)(f^*\omega)$. 
	\end{enumerate}
\end{theorem}
\begin{proof}
	We prove the first by picking $g\in C^\infty(N)$, and act $df^*(g)$ and $f^*(dg)$ (both in $\Omega^1(M)$ on $X_p\in T_p 
	M$, as if these agree for arbitrary $X_p$ then they are equal as linear functionals). We have:
	\begin{equation}
		df^*(g) X_p = d(g\circ f) X_p = (g\circ f)_* X_p = X_p(g\circ f)
	\end{equation}
	\begin{equation}
		f^*(dg) X_p = dg(f_* X_p) = (f_*X_p)(g) = X_p(g\circ f)
	\end{equation}
	For 2, we $\omega, \tau\in\Omega^1(N)$ and must these on vectors $X\in TM$. This gives:
	\begin{equation}
		f^*(\omega + \tau)(X) = (\omega + \tau)(f_* X) = \omega f_* X + \tau f_* X = f^*(\omega)X + f^*(\tau) X
	\end{equation}
	For 3, we act this on $X_p\in T_p M$ and have:
	\begin{equation}
		f^*(g\omega) X_p = (g\omega)_{f(p)} (f_* X_p) = g(f(p)) \omega_{f(p)}(f_* X_p) = (f^* g)(f^*\omega) X_p
	\end{equation}
\end{proof}
This quick proof is a nice example to recall what the different types of maps do, and to work through it, and the best 
way to work through it is to draw out all the maps and induced maps. Note that although relations between 1-forms, vector 
fields, functions, and other objects are simplest to write out as sections of a bundle (i.e. vector / covector valued fields), when 
we do computations with them we typically pick a point $p\in M$ to do the computation over. 

\subsection{$k$-forms}

At this point, we turn towards differential $k$-forms. If you are not familiar with the idea of the exterior algebra to a vector 
space, it would be a good idea to read my notes on module theory and familiarize yourself with it. Differential forms are 
exactly sections of the $k$th exterior power of the cotangent bundle; by taking the exterior power, we make forms 
implicitly antisymmetric, which makes them useful to define orientations. 

Recall that for a vector space $V$, $\Lambda^k V$ is the quotient of $\bigotimes_k V$ by the ideal generated by 
$\{x_1\otimes ...\otimes x_k : x_i = x_j, i\neq j\}$. The image of $\otimes$ in this algebra is denoted by the wedge product 
$\wedge$, and the exterior algebra of $V$ is $\Lambda^* V := \bigoplus_{k = 0}^\infty\Lambda^k V$. The exterior 
algebra $(\Lambda^* V, +, \times, \wedge)$ is the ``best antisymmetric product" of $V$, as any alternating map from 
$V^k$ to $\mathbb R$ factors to a linear map $\Lambda^* V\rightarrow\mathbb R$. 

\begin{definition}[Alternating map]
	A map $f : V^k\rightarrow K$ for a field $K$ and a $K$-vector space $V$ is called \textbf{alternating} if for each $\sigma
	\in S_k$, we have:
	\begin{equation}
		f(v_{\sigma (1)}, ..., v_{\sigma (k)}) = \textnormal{sgn}(\sigma) f(v_1, ..., v_k)
	\end{equation}
	where $\{v_1, ..., v_k\}\subseteq V$. 
\end{definition}

To study differential forms, let $p\in M$ and consider the $k$th exterior power of the cotangent space $\Lambda^k(T_p^* M)$. 
We can vary $p\in M$ and take a union to form a fiber bundle over $M$, called the $k$th exterior power of the cotangent 
bundle, and a section of this bundle is called a differential form.
\begin{definition}[Differential form]
	The \textbf{$k$th exterior power of the cotangent bundle} is the fiber bundle:
	\begin{equation}
		\Lambda^k(T^* M) := \coprod_{p\in M}\Lambda^k(T_p^* M)
	\end{equation}
	A \textbf{$k$-covector} is an element of this set. A smooth section of $\Lambda^k(T^* M)$ is called a \textbf{differential 
	form}, and we call $k$ its \textbf{rank}. We denote the set of all $k$-forms on a manifold $M$ by $\Omega^k(M)$, so:
	\begin{equation}
		\Omega^k(M) := \Gamma\left(\Lambda^k(T^* M)\right)
	\end{equation}
\end{definition}
Note $\Lambda^k(T_p^* M)$ is in bijection with the space of $k$-linear, alternating maps $(T_p^* M)^k\rightarrow\mathbb R$, 
so we may equivalently view a differential form as an alternating map on the tangent space at each point $p\in M$:
\begin{align}
	\omega\in\Omega^k(T^*M) &\iff\forall p\in M, \;\omega_p\in\Lambda^k(T^*M) \\
	&\iff\forall p\in M, \;\omega_p : (T_p M)^k\rightarrow \mathbb R\textnormal{ is linear and alternating}
\end{align}
We will typically view a differential form as an alternating map, as working with the abstract definition of a $k$-covector 
is very difficult to deal with and does not give us much. 

A differential form is a specific example of a tensor field.
\begin{definition}[Tensor]
	Let $p\in M$. A \textbf{tensor} at $p$ is a multilinear map:
	\begin{equation}
		F : (T_p M)^k\times (T_p^* M)^\ell\rightarrow\mathbb R
	\end{equation}
	where $k, \ell\in\mathbb N_{\geq 0}$. We say the tensor has \textbf{type} $(k, \ell)$. 
\end{definition}
A differential form of rank $k$ may thus equivalently be categorized as an \textbf{alternating tensor of type} $\bf{(0, k)}$ (where 
by an alternating tensor we mean it is an alternating map), and this definition often appears in literature. 

Consider $k$ vector fields $\{X_1, ..., X_k\}$ on $M$. Then we 
get a function $\omega(X_1, ..., X_k) : M\rightarrow\mathbb R$, $p\mapsto \omega_p((X_1)_p, ..., (X_k)_p)$. Because 
$\omega$ is an alternating function, note that for example $\omega(X_1, X_2, ..., X_k) = -\omega(X_2, X_1, ..., X_k)$ and 
so on. This function is linear with respect to functions $h : M\rightarrow\mathbb R$ in the sense that:
\begin{equation}
	\omega(X_1, ..., hX_i, ..., X_k) = h\omega(X_1, ..., X_i, ..., X_k)
\end{equation}
for each $i$. 

We now consider what a general differential form looks like for a chart $(U, x^i)$. At each $p\in M$, $\{dx^i|_p\}_{i = 1}^n$ is a 
basis for the cotangent space. Because we are taking the $k$th exterior power, a basis is thus formed by taking all the 
combinations:
\begin{equation}
	\beta = \{dx^{i_1}\wedge ...\wedge dx^{i_k} : 1\leq i_1 < i_2 < ... < i_k \leq n\}
\end{equation}
Note that this implies \textbf{the dimension of this exterior power is $dim(\Lambda^k(T_p^* M)) = {n\choose k}$, and 
$\Lambda^k(T_p^* M)$ vanishes for $k > n$}. A succinct way to write this is with a multi-index:
\begin{definition}[Multi-index]
	Let $k, n\in\mathbb N_{\geq 0}$. A \textbf{multi-index} between $1$ and $n$ of length $k$ is an ordered $k$-tuple 
	of the form:
	\begin{equation}
		I = (i_1, ..., i_k)
	\end{equation}
	such that $1\leq i_1< ...< i_k\leq n$. We let $\mathfrak I_k^n$ denote the set of all multi-indices between 1 and $n$ of 
	length $k$. 
\end{definition}
With this notation, we may write a basis for $\Lambda^k(T_p^*M)$ as $\{dx^I : I\in\mathfrak I_k^n\}$, where by $dx^I$ we 
mean $I = (i_1, ..., i_k)$ and $dx^I = dx^{i_1}\wedge ...\wedge dx^{i_k}$. So, we may expand each $k$-covector $\omega_p
\in\Lambda^k(T_p^* M)$ as a linear combination:
\begin{equation}
	\omega_p = a_{i_1, ..., i_k}(p)dx^{i_1}|_p\wedge ...\wedge dx^{i_k}|_p = \sum_{I\in\mathfrak I_k^n} a_I(p) dx^I|_p
\end{equation}
This implies a $k$-form $\omega\in\Omega^k(M)$ can be expanded as:
\begin{equation}
	\omega = a_{i_1, ..., i_k}dx^{i_1}\wedge ...\wedge dx^{i_k} = \sum_{I\in\mathfrak I_k^n} a_I dx^I~
	\label{eq:k_form}
\end{equation}
A $k$-form $\omega$ represented by Equation~\ref{eq:k_form} is \textbf{smooth} if and only if $a_I : M\rightarrow\mathbb R$ 
is smooth as a function on $M$. 

We can now define a pullback of $k$-forms. 
\begin{definition}[Pullback]
	Let $f : M\rightarrow N$ be $C^\infty$. Then there is an induced map:
	\begin{equation}
		f^* : \Lambda^k\left(T_{f(p)}^* N\right)\rightarrow\Lambda^k\left(T_p^* M\right)
	\end{equation}
	which sends a $k$-covector $\omega_{f(p)}$ to $f^*(\omega_{f(p)})$ which acts on $\left(T_p^M\right)^k$ as follows:
	\begin{equation}
		f^*(\omega_{f(p)})(x_1, ..., x_k) := \omega_{f(p)}(f_* x_1, ..., f_* x_k)
	\end{equation}
	for $x_i\in T_p M$. This extends to a \textbf{pullback map} of forms $f^* : \Omega^k(N)\rightarrow\Omega^k(M)$ which 
	acts on vector fields:
	\begin{equation}
		f^*(\omega)(X_1, ..., X_k) := \omega(f_* X_1, ..., f_* X_k)
	\end{equation}
\end{definition}
\begin{lemma}
	Let $f : M\rightarrow N$ be $C^\infty$. If $\omega, \tau\in\Omega^k(N)$ and $a\in\mathbb R$, then:
	\begin{enumerate}
		\item $f^*(\omega + \tau) = f^*\omega + f^*\tau$. 
		\item $f^*(a\omega) = af^*\omega$. 
		\item If $\omega$ is $C^\infty$, then $f^*\omega$ is $C^\infty$ as well. 
	\end{enumerate}
\end{lemma}

\subsection{The wedge product}

We now turn to the wedge product, which will give the space of differential $k$-forms an algebra product. Abstractly, this 
is the same wedge product that is studied when the exterior power $\Lambda^k(T_p^* M)$ is considered, and extends 
to sections $\Omega^k(M)$ pointwise. We have already used this wedge product in the previous section. However, we will 
define this in a more concrete way than as ``the image of the tensor product operation in the quotient $\Lambda^k(\cdot)$". We 
first define the wedge product generally on a vector space $V$. 

\begin{definition}[Wedge product]
	Let $V$ be a vector space, $\alpha\in\Lambda^k(V^*)$, and $\beta\in\Lambda^\ell(V^*)$ be two alternating 
	tensors\footnote{Another common notation is to denote the space of alternating tensors $\Lambda^k(V^*)$ by $A^k(V)$.}. 
	Then we define their \textbf{wedge product} to be an alternating tensor $\alpha\wedge\beta\in\Lambda^{k + \ell}(V^*)$ 
	such that:
	\begin{equation}
		(\alpha\wedge\beta)(v_1, ..., v_{k + \ell}) := \frac{1}{k!\ell!}\sum_{\sigma\in S_{k + \ell}}sgn(\sigma)
		\alpha(v_{\sigma(1)}, ..., v_{\sigma(k)})\beta(v_{\sigma(k + 1), ..., v_{\sigma(k + \ell)}})~
		\label{eq:wedge}
	\end{equation}
	where $S_n$ is the symmetric group on $n$ letters. 
\end{definition}

The wedge product may be rewritten as a slightly simpler sum without the extra factorial factors. We define a 
$\bf{(k, \ell)}$-\textbf{shuffle} to be a permutation $\tau\in S_{k + \ell}$ such that:
\begin{equation}
	\tau(1) < ... < \tau(k) \& \tau(k + 1) < ... < \tau(k + \ell)
\end{equation}
Essentially, $\tau$ mixes the sets $\{1, ..., k\}$ and $\{k + 1, ..., \ell\}$ in increasing order. Let $\mathfrak S_{k, \ell}\subseteq
S_{k + \ell}$ be the set of all $(k, \ell)$ shuffles. Then Equation~\ref{eq:wedge} may be rewritten as:
\begin{equation}
	(\alpha\wedge\beta)(v_1, ..., v_{k + \ell}) = \sum_{\tau\in\mathfrak S_{k, \ell}}sgn(\tau)\alpha(v_{\tau(1)}, ..., v_{\tau(k)})
	\beta(v_{\tau(k + 1)}, ..., v_{\tau(k + \ell)})
\end{equation}
The wedge product is totally anticommutative in the graded sense, in that if $\alpha\in\Lambda^k(V^*)$ and $\beta\in
\Lambda^\ell(V^*)$, then:
\begin{equation}
	\alpha\wedge\beta = (-1)^{k\ell}\beta\wedge\alpha
\end{equation}
This implies that for any $k$-covector of odd degree $\alpha\in\Lambda^{2k}(V^*)$, $\alpha\wedge\alpha = 0$. The wedge 
product is also associative, and this associativity will give the spaces $\Omega^k(M)$ an algebra structure. One final nice 
property of the wedge product is that for $\alpha^1, ..., \alpha^k\in V^*$ and $v_1, ..., v_k\in V$, then:
\begin{equation}
	(\alpha^1\wedge ...\wedge\alpha^k)(v_1, ..., v_k) = \det\{a^i(v_j)\}
\end{equation}

Now, we turn to using the wedge product as an operation on forms. 
\begin{definition}
	Let $\omega\in\Omega^k(M)$ and $\tau\in\Omega^\ell(M)$ be two differential forms. Then we define their \textbf{wedge 
	product} to be $\omega\wedge\tau\in\Omega^{k + \ell}(M)$ defined pointwise as a wedge product on covectors:
	\begin{equation}
		(\omega\wedge\tau)_p :=\omega_p\wedge\tau_p
	\end{equation}
\end{definition}
It is easy to show that $\omega\wedge\tau$ is $C^\infty$ if both $\omega$ and $\tau$ are. This structure makes 
$(\Omega^k(M), +, \cdot, \wedge)$ into algebra. We can further define the \textbf{algebra of $C^\infty$ differential forms} 
$\Omega^*(M)$ as:
\begin{equation}
	\Omega^*(M) := \bigoplus_{k = 0}^\infty\Omega^k(M)
\end{equation}
Then $(\Omega^*, +, \cdot, \wedge)$ is naturally a graded algebra over $\mathbb R$, and we will soon associate a differential 
with it to make it into a cochain complex. The pullback $f^*$ is also an algebra homorphism.
\begin{theorem}
	Let $f : M\rightarrow N$ be smooth. Then the pullback respects $\wedge$, i.e. if $\omega\in\Omega^k(N), 
	\tau\in\Omega^\ell(N)$ are differential forms on $N$, we have:
	\begin{equation}
		f^*(\omega\wedge\tau) = f^*(\omega)\wedge f^*(\tau)
	\end{equation}
\end{theorem}
\begin{proof}
	This is simply by working through the definitions. Act $f^*(\omega\wedge\tau)$ on vectors $X_1, ..., X_{k + \ell}\in T_p M$:
	\begin{align}
		f^*(\omega\wedge\tau)_p(X_1, ..., X_{k + \ell}) &= (\omega_p\wedge\tau_p)(f_* X_1, ..., f_* X_{k + \ell})\\
		&= \sum_{\sigma\in\mathfrak S_{k, \ell}}sgn(\sigma) \omega_p(f_* X_{\sigma(1)}, ..., f_* X_{\sigma(k)})
		\tau_p(f_* X_{\sigma(k + 1)}, ..., f_* X_{\sigma(k + \ell)}) \\
		&= \sum_{\sigma\in\mathfrak S_{k, \ell}}sgn(\sigma) f^*(\omega_p)(X_{\sigma(1)}, ..., X_{\sigma(k)})f^*(\tau_p)
		(X_{\sigma(k + 1)}, ..., X_{\sigma(k + \ell)}) \\
		&= (f^*(\omega_p)\wedge f^*(\tau_p))(X_1, ..., X_{k + \ell})
	\end{align}
\end{proof}

\subsection{The exterior derivative}

\subsection{Summary}

There were a lot of definitions introduced in this section, so I will summarize the major definitions and results here for a 
manifold $M$ of dimension $n$.
\begin{itemize}
	\item The \textbf{cotangent space} at $p\in M$ is the dual to the tangent space $T_p M$:
	\begin{equation}
		T_p^* M := Hom(T_p M, \mathbb R)
	\end{equation}
	A basis for this space is $\{dx^i\}_{i = 1}^n$, and the dual vectors obey $dx^i(\partial_j) = \delta^i_j$. 
	
	\item A \textbf{$k$-covector} $\omega_p$ at a point $p\in M$ is an element of the $k$th exterior power of the cotangent 
	space at $p$ is: 
	\begin{equation}
		\omega_p\in\Lambda^k\left(T_p^* M\right) = span\left\{dx^{i_1}\wedge ...\wedge dx^{i_k} : 1\leq i_1 < ... < i_k\leq n 
		\right\}
	\end{equation}
	A basis for this space is $\{dx^{i_1}\wedge ...\wedge dx^{i_k}\}_{(i_j)\in\mathfrak I_k^n}$, and the dimension is $n\choose 
	k$. 
	
	\item The \textbf{$k$th exterior power of the cotangent bundle} is:
	\begin{equation}
		\Lambda^k\left(T^* M\right) :=\coprod_{p\in M}\Lambda^k\left(T_p^* M\right)
	\end{equation}
	
	\item A \textbf{differential form} $\omega$ of rank $k$ is a section of the $k$th exterior power of the cotangent bundle:
	\begin{equation}
		\omega\in\Omega^k(M) := \Gamma(\Lambda^k(T^* M))
	\end{equation}
	and can be expressed as $\omega = a_{i_1, ..., i_k} dx^{i_1}\wedge ...\wedge dx^{i_k}$. 
	
	\item The \textbf{wedge product} of two differential forms $\omega\in\Omega^k(M), \tau\in\Omega^\ell(M)$ is defined 
	pointwise as:
	\begin{equation}
		(\omega_p\wedge\tau_p)(v_1, ..., v_{k + \ell}) := \sum_{\sigma\in\mathfrak S_{k, \ell}} sgn(\sigma)
		\omega_p(v_{\sigma(1)}, ..., v_{\sigma(k)})\tau_p(v_{\sigma(k + 1)}, ..., v_{\sigma(k + \ell)})
	\end{equation}
	This naturally gives $(\Omega^k(M), +, \cdot, \wedge)$ the structure of a $\mathbb R$-algebra. 
	
	\item Given a smooth map $f : M\rightarrow N$, we induce a \textbf{pullback map} of differential forms $f^* : \Omega^k(N)
	\rightarrow\Omega^k(M)$ which acts on vector fields as:
	\begin{equation}
		f^*(\omega)(X_1, ..., X_k) = \omega(f_* X_1, ..., f_* X_k)
	\end{equation}
	Furthermore, $f^*$ is a homomorphism of algebras. 
	
	\item The \textbf{exterior derivative} is a natural map $d : \Omega^k(M)\rightarrow\Omega^{k + 1}(M)$ defined as 
	$df = (\partial f / \partial x^i)dx^i$ on $f\in\Omega^0(M)$ and:
	\begin{equation}
		d\omega := dg_{i_1, ..., i_k}\wedge dx^{i_1}\wedge ...\wedge 
		dx^{i_k}
	\end{equation}
	on $\omega = g_{i_1, ..., i_k} dx^{i_1}\wedge...\wedge dx^{i_k}\in\Omega^k(M)$. 
	
	\item The pullback is a chain map with respect to the de Rham complex $(\Omega^*(M), d)$
\end{itemize}

\section{Integration}

\subsection{Orientation}

\subsection{Integration}

\end{document}