\documentclass[11pt, oneside]{amsart}   	% use "amsart" instead of "article" for AMSLaTeX format
\usepackage[margin = 1in]{geometry}                		% See geometry.pdf to learn the layout options. There are lots.
\geometry{letterpaper}                   		% ... or a4paper or a5paper or ... 
%\geometry{landscape}                		% Activate for rotated page geometry
%\usepackage[parfill]{parskip}    		% Activate to begin paragraphs with an empty line rather than an indent
\usepackage{graphicx}				% Use pdf, png, jpg, or eps§ with pdflatex; use eps in DVI mode
								% TeX will automatically convert eps --> pdf in pdflatex		
\usepackage{amssymb}
\usepackage{amsmath}
\usepackage{amsthm}
\usepackage[shortlabels]{enumitem}
\usepackage{float}

\theoremstyle{definition}
\newtheorem{definition}{Definition}[section]
\newtheorem{theorem}{Theorem}[section]
\newtheorem{corollary}{Corollary}[theorem]
\newtheorem{lemma}[theorem]{Lemma}
\newtheorem{example}{Example}[section]

\graphicspath{{/Users/theoares/Desktop/Math\ 142/notes}}

%SetFonts

%SetFonts


\title{Homotopy}
\author{Patrick Oare}
\date{}							% Activate to display a given date or no date

\begin{document}
\maketitle

	Homotopy is a method that we use to study the deformation of continuous maps. We will see that it provides another notion of topological equivalence, 
	called \textbf{homotopy equivalence}, which is weaker than homeomorphism but provides a large amount of topological information nonetheless. We will 
	define the notion of a \textbf{fundamental group} which can be used to easily calculate the homotopy type of a space, and develop different ways to calculate 
	this group. Finally, we will discuss homotopy lifting and covering spaces, which provide a useful tool for calculating homotopy information about a specific 
	space.
	
	For these notes, let $X$ and $Y$ be topological spaces. Recall that a \textbf{path} is a continuous map $\gamma : [0, 1]\rightarrow X$. 

\section{Definitions}

	We begin by defining the notion of homotopic maps, which will give us a method to define homotopy equivalence. 
	
	\begin{definition}[Homotopy]
		Let $f, g : X\rightarrow Y$ be continuous maps. Then a \textbf{homotopy} $F$ from $f$ to $g$ is a continuous map $F : X\times [0, 1]\rightarrow Y$ such 
		that $F(x, 0) = f(x)$ and $F(x, 1) = g(x)$ for all $x\in X$, which we denote by $f\simeq_F g$. If $A\subset X$ and $f(a) = F(a, t)$ for all $t\in [0, 1]$ and $a\in 
		A$, then we say that $F$ is a homotopy from $f$ to $g$ \textbf{relative to $A$}, and we write this as $f\simeq_F g\textnormal{ rel }A$.
	\end{definition}
	
	To illustrate the notion of homotopy, we now consider an important example.
	
	\begin{example}[Straight-Line Homotopy]
	
		Let $Y\subseteq\mathbb R^n$. We say that $Y$ is \textbf{convex} if any two points in $Y$ can be connected by a line segment lying in $Y$. Let $f, g : 
		X\rightarrow Y$ be continuous maps from an arbitrary topological space $X$ to $Y$. Then we can construct a homotopy $F$ from $f$ to $g$, called 
		the \textbf{straight-line homotopy}, by defining:
		$$
			F(x, t) := (1 - t)f(x) + g(x)
		$$
		Thus any two maps into a convex space are homotopic, which will be quite useful later.
	
	\end{example}
	
	To make the notion of homotopy more useful, we now show that homotopy relative to a set $A$ is an equivalence relation. This will allow us to quotient the 
	set of maps by this relation and define the notion of a homotopy class, which is essential to defining the fundamental group of a space.
	
	\begin{theorem}
		The relation $f\simeq g$ on the set $\{f : X\rightarrow Y | f\textnormal{ continuous}\}$ is an equivalence relation. Similarly, $f\simeq g$ rel $A$ is an 
		equivalence relation.
	\end{theorem}
	
	\begin{proof}
		Reflexivity and symmetry are easy ($F(x, t) = f(x)$ and $G(x, t) = F(x, 1 - t)$). We show transitivity. Suppose that $f\simeq_F g$ and $g\simeq_G h$. 
		We can define a homotopy $H : X\times [0, 1]\rightarrow Y$ by:
		$$
			H(x, t) := \begin{cases}
				F(x, 2t) & t\in [0, \frac{1}{2}] \\
				G(x, 2t - 1) & t\in [\frac{1}{2}, 1]
			\end{cases}
		$$
		and it is easy to verify that this is a homotopy from $f$ to $h$. 
	\end{proof}
	
	Homotopies behave well with respect to compositions of continuous maps as well. If $f, g : X\rightarrow Y$ are homotopic via $F$, then for any continuous 
	$h : Y\rightarrow Z$, $h\circ f\simeq h\circ g$ via the homotopy $G : X\times [0, 1]\rightarrow Z, G(x, t) := h(F(x, t))$. We will now define a new notion of 
	equivalence for topological spaces using homotopy.
	
	\begin{definition}[Homotopy Equivalence]
		If $X$ and $Y$ are topological spaces, we say that they are \textbf{homotopy equivalent} if there are continuous functions $f : X\rightarrow Y$ and $g : 
		Y\rightarrow X$ such that $g\circ f\simeq id|_X$ and $f\circ g\simeq id|_Y$. We call $f$ a \textbf{homotopy equivalence} from $X$ to $Y$, and $g$ its 
		\textbf{homotopy inverse}. We write $X\simeq Y$ to denote that $X$ and $Y$ are homotopy equivalent. 
	\end{definition}
	
	Note that homotopy equivalence is weaker than homeomorphism. This follows because if $f : X\rightarrow Y$ is a homeomorphism, then $f^{-1}$ is a 
	homotopy inverse as well, since $id\simeq id$. Thus any two spaces which are homeomorphic are homotopy equivalent, but not necessarily vice versa. 
	We prove a lemma.
	
	\begin{lemma}
		Homotopy equivalence is an equivalence relation on topological spaces.
	\end{lemma}
	
	The proof of this is easy; the hardest part is transitivity, which follows because homotopy and compositions behave nicely with one another. The last part of 
	this section will deal with retracting spaces. We begin with the notion of a deformation retract.
	
	\begin{definition}[Deformation Retract]
		Let $A\subseteq X$ be a subspace. A \textbf{deformation retract} of $X$ onto $A$ is a homotopy $G$ relative to $A$ from $id_X$ to a continuous function 
		$g : X\rightarrow A$. In other words, it is a map $G : X\times [0, 1]\rightarrow X$ relative to $A$ such that $G(x, 0) = x$ and $G(x, 1)\in A$. 
	\end{definition}
	
	If $X$ deformation retracts onto $A$, then the two spaces are homotopy equivalent. An equivalent definition of a deformation retract is the map $\iota\circ f : 
	X\rightarrow X$, where $\iota : A\rightarrow X$ is the inclusion and a homotopy equivalence with homotopy inverse $f$. These two definitions are equivalent 
	via $f(x) = \pi(G(x, 1))$, where $\pi$ projects $X$ onto $A$. Finally, we introduce what a contractible space is.
	
	\begin{definition}[Contractible Space]
		We say that $X$ is \textbf{contractible} if there is a deformation retract of $X$ onto the one point space $\{x_0\}$, where $x_0\in X$ is a point. 
	\end{definition}
	
	That a space is contractible will eventually show us its fundamental group is trivial. In this case, there are not many interesting things we can say about its 
	structure with respect to homotopy. An example is $\mathbb R^n$, which is contractible because of the straight-line homotopy.

\section{The Fundamental Group}

	\begin{definition}[Fundamental Group]
		Let $p\in X$ be a point. We define the \textbf{fundamental group of $X$ based at $p$} to be the set $\pi_1(X, p)$ of homotopy classes relative to $\{0, 1\}$ 
		of continuous paths from $p$ to $p$, i.e:
		$$
			\pi_1(X, p) = \{[\gamma] | \gamma : [0, 1]\rightarrow X\textnormal{ is continuous, }\gamma(0) = \gamma(1) = p\}
		$$
	\end{definition}
	
	\begin{theorem}
		$\pi_1(X, p)$ is a group under the operation:
		$$
			[\alpha][\beta] := [\alpha\cdot\beta]
		$$
		where $\alpha\cdot\beta : [0, 1]\rightarrow X$ is the path:
		$$
			(\alpha\cdot\beta)(t) = \begin{cases}
				\alpha(2t) & t\in [0, \frac{1}{2}] \\
				\beta(2t - 1) & t\in [\frac{1}{2}, 1]
			\end{cases}
		$$
	\end{theorem}
	
	\begin{proof}
		You can show this is a well defined operation. We will list out the important steps of the proof.
		\begin{itemize}
			\item \textbf{Identity} The identity loop is $e : [0, 1]\rightarrow X$ given by:
			$$
				e(x) = p
			$$
			If we wish to show that this gives an identity with the group operation, we can pick a loop $\alpha$ based at $p$ and show it is homotopic to $\alpha
			\cdot e$. Indeed, this homotopy is given by $F : [0, 1]\times [0, 1]\rightarrow X$ defined as:
			$$
				F(x, t) = \begin{cases}
					\alpha(\frac{2x}{2 - t}) & x\in [0, 1 - \frac{t}{2}] \\
					p & x\in [1 - \frac{t}{2}, 1]
				\end{cases}
			$$
		
			\item \textbf{Inverse} Given $[\alpha]\in\pi_1(X, p)$, we can define its inverse by:
			$$
				\alpha^{-1}(t) = \alpha(1 - t)
			$$
			
			\item \textbf{Associativity}: This is not too interesting, so refer to the notes.
		\end{itemize}
	\end{proof}
	
	Note that the base point $p$ of $\pi_1(X, p)$ does matter; we will show that if $X$ is path-connected, it does not, but two base points which lie in different 
	components of $X$ can give different fundamental groups. First, we will show how we can induce homomorphisms of different fundamental groups. 
	
	The first way to do this is by changing the base point of the fundamental group. Suppose that there is a path $\gamma : [0, 1]\rightarrow X$ from $p$ to $q$. 
	Then we may use $\gamma$ to give an isomorphism of fundamental groups. We define $\gamma_* : \pi_1(X, p)\rightarrow\pi_1(X, q)$ to be the map:
	$$
		\gamma_*([\alpha]) = [\gamma^{-1}\cdot\alpha\cdot\gamma]
	$$
	
	This map will be an isomorphism of fundamental groups. In particular, this shows that in a path-connected space, the fundamental groups based at each point 
	are isomorphic. However, this is not the only way to induce maps between fundamental groups. Suppose that we have a continuous map $f : X\rightarrow Y$ 
	such that $f(p) = q$. Then we get a homomorphism of fundamental groups:
	$$
		f_* : \pi_1(X, p)\rightarrow\pi_1(X, q), \;\; f_*([\alpha]) = [f\circ\alpha]
	$$
	This is well defined because homotopy respects composition, and will be a homomorphism. Because it is a homomorphism, this implies that for loops $\alpha$ 
	and $\beta$ based at $p$ and a continuous function $f : X\rightarrow Y$, we have:
	$$
		f\circ (\alpha\cdot\beta)\simeq (f\circ\alpha)\cdot (f\circ\beta)
	$$
	Furthermore, one can show that if $f : X\rightarrow Y$ is a homeomorphism, then $f_*$ is a group isomorphism. The map $f\mapsto f_*$ is actually a sort of 
	covariant functor, as shown in the next theorem:
	
	\begin{theorem}
		Let $f : X\rightarrow Y$ and $g : Y\rightarrow Z$ be continuous. Then:
		$$
			(g\circ f)_* = g_*\circ f_*
		$$
	\end{theorem}
	
	Now, we consider path-components of $X$.
	
	\begin{definition}[Path Component]
		A \textbf{path component} is a subset $A\subseteq X$ which is path-connected, and such that if $A\subset B$ is strict, then $B$ is not path-connected.
	\end{definition}
	
	\begin{theorem}
		If $X$ is path-connected, then for any $p, q\in X$, $\pi_1(X, p)\cong\pi_1(X, q)$. 
	\end{theorem}
	
	\begin{proof}
		Since $X$ is path-connected, pick a path $\gamma : [0, 1]\rightarrow X$ from $p$ to $q$. We will show that $\gamma_* : \pi_1(X, p)\rightarrow\pi_1(X, 
		q)$ given by $\gamma([\alpha]) = \gamma^{-1}\cdot\alpha\cdot\gamma$. This will be a group isomorphism.
	\end{proof}
	
	If $X$ is a path-connected space, then this allows us to write $\pi_1(X)$ without loss of generality. We now state one last definition.
	
	\begin{definition}[Simply Connected]
		A path-connected space $X$ is \textbf{simply connected} if its fundamental group is trivial.
	\end{definition}

\section{Orbit Spaces}

	We now discuss the construction of a new space given an action of a group on an existing space. This construction will be called an \textbf{orbit space}, and 
	allow us to compute the fundamental group of some spaces of interest.
	
	\begin{definition}[Group Action]
		Let $G$ be a group. We say $G$ acts on a space $X$ if:
		\begin{enumerate}
			\item $\forall g\in G$, $g$ defines a homeomorphism $f_g : X\rightarrow X$.
			\item $\forall g, h\in G$, $f_{gh} = f_g\circ f_h$. 
			\item If $e\in G$ is the identity, then $f_e = id_X$
		\end{enumerate}
		Furthermore, we say that $G$ \textbf{acts nicely} (properly discontinuously) on $X$ if $G$ acts on $X$ and:
		\begin{enumerate}[4.]
			\item For each $x\in X$ and $g\in G\setminus\{e\}$, there is an open neighborhood $U$ of $x$ such that $U\cap f_g(U) = \emptyset$.
		\end{enumerate}
	\end{definition}
	
	The last condition essentially just says that the action permutes the pieces of $X$; in other words, for any $x$ we can find a small enough neighborhood 
	such that the action moves the neighborhood entirely to another part of the space. This also implies that if $G$ acts nicely on $X$ and if $g\neq e$, then 
	$f_g$ has no fixed points. Now, we can define the notion of an orbit space.
	
	\begin{definition}[Orbit Space]
		Let $G$ act on $X$. We define the \textbf{orbit space} of this action to be the space $X / G$ defined by the partition $P$, where $x$ and $y$ are in the 
		same subset of $P$ iff $\exists g\in G$ such that $f_g(x) = y$. 
	\end{definition}
	
	Note that this construction essentially just carves $X$ up into its orbits under $G$ and identifies each of these orbits as a point in the orbit space. Some 
	examples of orbit spaces include:
	\begin{itemize}
		\item $\mathbb Z$ acts on $\mathbb R$ nicely via $f_n(x) = x + n$. The orbit space $\mathbb R / \mathbb Z$ is homeomorphic to $S^1$.
		
		\item Similarly, $\mathbb Z^2$ acts on $\mathbb R^2$ via $f_{m, n}(x, y) = (x + m, y + n)$ and its orbit space is $\mathbb R^2 / \mathbb Z^2\cong 
		T^2$.
		
		\item The Mobius Strip $M$. This is homeomorphic to the orbit space formed by acting $\mathbb Z$ on $\mathbb R\times [0, 1]$ given by:
		$$
			f_1(x, y) = (x + 1, 1 - y)
		$$
		and $f_n = f_{n - 1}\circ f_1$. 
		
		\item The Klein Bottle $K$. This is homeomorphic to an orbit space $\mathbb R^2 / G$, where we define $G$ by its action on $\mathbb R^2$. We take 
		$G = \langle r, u | rur = u\rangle$, where $r$ denotes a right translation of $x\in\mathbb R^2$ and $u$ is an upward translation, and where we identify 
		$\mathbb R^2$ as being broken up into a grid with the correct identification of edges to form $K$. 
		
		\item The Projective Plane $\mathbb RP^2$. This is the orbit space $S^2 / (\mathbb Z / 2\mathbb Z)$, where the action of $\mathbb Z / 2\mathbb Z$ is 
		given by $0\cdot x = x$ and $1\cdot x = -x$. 
	\end{itemize}

\section{Calculations}

	This section will provide us with practical reason to study fundamental groups. We will develop the techniques necessary to calculate the fundamental groups 
	of some simple surfaces. 

	We begin with the circle $S^1$. Before we start, we will show how $\mathbb R$ is a covering space of $S^1$ (which we will define later), which gives us a way 
	to lift paths and homotopies from the circle to $\mathbb R$. For this section, we will define a map $f : \mathbb R\rightarrow S^1$ by:
	$$
		f(x) := e^{2\pi i x}
	$$
	This map allows us to prove two lemmas which will be necessary to understanding $\pi_1(S^1)$. We will prove them in full generality for an arbitrary 
	covering space (which behaves much like the map $f : \mathbb R\rightarrow S^1$), which we will now define.
	
	\begin{definition}[Covering Space]
		We call a continuous function $\pi : \tilde X\rightarrow X$ a \textbf{covering space map} and $\tilde X$ a \textbf{covering space} of $X$ if $\forall x\in X$, 
		there is an open neighborhood $U_x$ of $x$ such that 
		$$
			\pi^{-1}(U_x) = \bigcup_\alpha\tilde U_\alpha
		$$
		where each $\tilde U_\alpha$ is open, $\tilde U_\alpha\cap \tilde U_\alpha' = \emptyset$ for $\alpha\neq \alpha'$, and $\pi|_{\tilde U_\alpha}\rightarrow 
		U_x$ is a homeomorphism.
	\end{definition}
	
	Essentially all a covering space is is a space and a map that is large enough to nicely cover our original space. The map $f$ that we already constructed is a 
	covering space map, as if we lift any open subset $U\subseteq S^1$ to $\mathbb R$ we will have an infinite number of components, each of which are 
	homeomorphic to $U$. A covering space map gives us the tools to lift paths and homotopies, which we now describe.
	
	\begin{lemma}[Path-Lifting]
		Let $p\in X$ and $q\in\pi^{-1}(\{p\})$. Then every path such that $\sigma(0) = p$ has a unique lift $\tilde\sigma$ to $\tilde X$ such that $\tilde\sigma(0) = 
		q$. 
	\end{lemma}
	
	\begin{lemma}[Homotopy Lifting]
		Let $\sigma, \sigma'$ be loops in $X$ based at $p$ that are homotopic via $F$. Then there is a unique lift $\tilde F$ of $F$ to $\tilde X$ such that 
		$\tilde\sigma\simeq_{\tilde f}\tilde\sigma'$ rel $\{0, 1\}$, where $\tilde\sigma, \tilde\sigma'$ are the unique lifts of $\sigma$ and $\sigma'$ to $\tilde X$. 
	\end{lemma}
	
	Notice that there is a specific version of these lemmas for the covering space map $f$; we will need this to prove the result about the fundamental group of 
	$S^1$. Here is one more definition, and you can show that it is well defined.
	
	\begin{definition}
		Let $\pi : \tilde X\rightarrow X$ be a covering space map. If $\pi^{-1}(\{x\})$ is finite for all $x\in X$ and $|\pi^{-1}(\{x\})| = n\in\mathbb N$, then we say 
		that $\tilde X$ is an \textbf{n-sheeted covering space}.
	\end{definition}
	
	Finally, there is a connection between some orbit spaces and covering spaces, which we now state.
	
	\begin{theorem}
		Let $G$ act nicely on a space $X$. Then the natural projection $\pi : X\rightarrow X / G$ is a covering space map.
	\end{theorem}
	
	Covering spaces are important in their own right, and they give us helpful tools to compute fundamental groups. Many proofs involving $\pi_1(S^1)$ involve 
	path and homotopy lifting, and so it is important to understand what this means. For the rest of this section, we will limit our discussion of covering spaces 
	to the circle equipped with covering space map $f : \mathbb R\rightarrow S^1$. 
	
	\begin{theorem}
		The fundamental group of the circle $\pi_1(S^1)$ is isomorphic to the integers $\mathbb Z$. 
	\end{theorem}
	
	The proof is tedious, so I won't repeat it here. It will use the isomorphism $\mathbb Z\rightarrow\pi_1(S^1)$ given by $n\mapsto [f\circ\gamma_n]$, 
	where $\gamma_n : [0, 1]\rightarrow \mathbb R$, $x\mapsto nx$. Now that we have one fundamental group, we can use this to compute others. We will 
	first state a special case of the Seifert-van Kampen theorem:
	
	\begin{theorem}
		Let $X$ be connected such that $X = U\cup V$ with $U, V$ both open and simply connected, and such that $U\cap V$ is path-connected. Let $p\in 
		U\cap V$. Then $\pi_1(X, p)\cong\{1\}$. 
	\end{theorem}
	
	\begin{corollary}
		For $n\geq 2$:
		$$
			\pi_1(S^n)\cong\{1\}
		$$
	\end{corollary}
	
	The corollary follows immediately from the theorem, because $S^n\setminus\{p\}\cong\mathbb R^n$ for any point $p$, and we may subtract out the north and 
	south poles to give us a path-connected intersection for $n\geq 2$. Another theorem that is useful is:
	
	\begin{theorem}
		For spaces $X, Y$ and $x_0\in X$, $y_0\in Y$, we have:
		$$
			\pi_1(X\times Y, (x_0, y_0))\cong\pi_1(X, x_0)\times \pi_1(Y, y_0)
		$$
	\end{theorem}
	
	\begin{proof}
		We have continuous maps $p_1 : X\times Y\rightarrow X$ and $p_2 : X\times Y\rightarrow Y$, so we map take the natural map $\psi : \pi_1(X\times Y, 
		(x_0, y_0)\rightarrow\pi_1(X, x_0)\times \pi_1(Y, y_0)$ defined by $[\alpha]\mapsto ((p_1)_*([\alpha]), (p_2)_*([\alpha])) = ([p_1\circ\alpha], [p_2\circ
		\alpha])$, which will be an isomorphism.
	\end{proof}
	
	This gives the immediate corollary that $\pi_1(T^n)\cong\mathbb Z^n$. The next theorem deals with orbit spaces, and will make it very easy for us to compute 
	the fundamental groups of some spaces we have already studied.
	
	\begin{theorem}
		Let $G$ act nicely on a space $X$ with $X$ simply connected and path-connected. Then:
		$$
			\pi_1(X / G)\cong G
		$$
	\end{theorem}
	
	This proof will use homotopy and path lifting lemmas, which are true for orbit spaces if the action is nice. This allows us to compute $\pi_1$ for the Klein 
	bottle and for the Mobius band, which are listed in the last section of this document. Now, we state two theorems which will allow us to show that 
	two topological spaces are distinct. 
	
	\begin{theorem}
		Let $f, g : X\rightarrow Y$ be continuous and $f\simeq_F g$. Let $\gamma : [0, 1]\rightarrow Y$, $\gamma(t) = F(p, t)$. Then $g_* = \gamma_*\circ
		f_*$, i.e. $g_*$ can be described by the sequence of maps:
		$$
			\pi_1(X, p)\xrightarrow{f_*}\pi_1(Y, f(p))\xrightarrow{\gamma_*}\pi_1(Y, g(p))
		$$
	\end{theorem}
	
	\begin{theorem}
		If $X$ and $Y$ are path-connected and $X\simeq Y$, then:
		$$
			\pi_1(X)\cong\pi_1(Y)
		$$
	\end{theorem}
	
	This last theorem uses the previous one in its proof. It is very useful; if we wish to show two spaces are not homotopy equivalent, we can compute their 
	fundamental groups and show they are non-isomorphic. Note that the converse of this is not true; if two spaces have the same fundamental group, we cannot 
	say if they are homotopy equivalent or not. More generally, this proof uses the following theorem:
	
	\begin{theorem}
		Let $f : X\rightarrow Y$ be a homotopy equivalence. Then $f_* : \pi_1(X, p)\rightarrow\pi_1(Y, f(p))$ is a group isomorphism.
	\end{theorem}

\section{Brouwer Fixed Point Theorem}
	
	\begin{theorem}[Brouwer Fixed Point]
		If $f : B^n\rightarrow B^n$ is continuous, then $f$ has a fixed point.
	\end{theorem}

\newpage

\section{List of Fundamental Groups}

\begin{itemize}
	\item The circle $S^1$. 
	$$
		\pi_1(S^1)\cong\mathbb Z
	$$
	
	\item The $n$-sphere for $n\geq 2$.
	$$
		\pi_1(S^n)\cong\{1\}
	$$
	
	\item The $n$-torus $T^n$. 
	$$
		\pi_1(T^n)\cong\mathbb Z^n
	$$
	
	\item Any contractible space $X$ (includes Euclidean space $\mathbb R^n$).
	$$
		\pi_1(X)\cong\{1\}
	$$
	
	\item The Mobius strip $M$. 
	$$
		\pi_1(M)\cong\mathbb Z
	$$
	
	\item The Klein Bottle $K$.
	$$
		\pi_1(K)\cong\langle r, u | rur = u\rangle
	$$
	
	\item The Projective Plane $\mathbb RP^2$.
	$$
		\pi_1(\mathbb RP^2)\cong\mathbb Z / 2\mathbb Z
	$$
	
	\item The one-point union of $n$ circles, $X_n$ at $p$.
	$$
		\pi_1(X_n, p)\cong F_n
	$$
	
	\item The connected sum of $g$ tori, $S_g = T^2\# ...\# T^2$.
	$$
		\pi_1(S_g)\cong\langle a_1, ..., a_{2g} | a_1a_2a_1^{-1}a_2^{-1}...a_{2g - 1}a_{2g}a_{2g-1}^{-1}a_{2g}^{-1} = 1\rangle
	$$
	
	\item The connected sum of $g$ projective planes, $N_g = \mathbb RP^2\# ...\#\mathbb RP^2$.
	$$
		\pi_1(N_g)\cong\langle a_1, ..., a_g | a_1^2a_2^2...a_g^2 = 1\rangle
	$$
	
\end{itemize}

\end{document}