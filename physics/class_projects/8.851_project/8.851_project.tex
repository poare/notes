\documentclass[11pt, oneside]{article}   	% use "amsart" instead of "article" for AMSLaTeX format
\usepackage[margin = 1in]{geometry}                		% See geometry.pdf to learn the layout options. There are lots.
\geometry{letterpaper}                   		% ... or a4paper or a5paper or ... 
%\geometry{landscape}                		% Activate for rotated page geometry
%\usepackage[parfill]{parskip}    		% Activate to begin paragraphs with an empty line rather than an indent
\usepackage{graphicx}				% Use pdf, png, jpg, or eps§ with pdflatex; use eps in DVI mode
								% TeX will automatically convert eps --> pdf in pdflatex		
\usepackage{amssymb}
\usepackage{amsmath}
\usepackage[shortlabels]{enumitem}
\usepackage{float}
\usepackage{tikz-cd}
\usepackage{subcaption}
\usepackage{slashed}

\usepackage{amsthm}
\theoremstyle{definition}
\newtheorem{definition}{Definition}[section]
\newtheorem{theorem}{Theorem}[section]
\newtheorem{corollary}{Corollary}[theorem]
\newtheorem{lemma}[theorem]{Lemma}

\newcommand{\N}{\mathbb{N}}
\newcommand{\R}{\mathbb{R}}
\newcommand{\Z}{\mathbb{Z}}
\newcommand{\Q}{\mathbb{Q}}

\usepackage{simpler-wick}
\usepackage[compat=1.0.0]{tikz-feynman}   %note you need to compile this in LuaLaTeX for diagrams to render correctly

% make arrow superscripts
\DeclareFontFamily{OMS}{oasy}{\skewchar\font48 }
\DeclareFontShape{OMS}{oasy}{m}{n}{%
         <-5.5> oasy5     <5.5-6.5> oasy6
      <6.5-7.5> oasy7     <7.5-8.5> oasy8
      <8.5-9.5> oasy9     <9.5->  oasy10
      }{}
\DeclareFontShape{OMS}{oasy}{b}{n}{%
       <-6> oabsy5
      <6-8> oabsy7
      <8->  oabsy10
      }{}
\DeclareSymbolFont{oasy}{OMS}{oasy}{m}{n}
\SetSymbolFont{oasy}{bold}{OMS}{oasy}{b}{n}

\DeclareMathSymbol{\smallleftarrow}     {\mathrel}{oasy}{"20}
\DeclareMathSymbol{\smallrightarrow}    {\mathrel}{oasy}{"21}
\DeclareMathSymbol{\smallleftrightarrow}{\mathrel}{oasy}{"24}
%\newcommand{\cev}[1]{\reflectbox{\ensuremath{\vec{\reflectbox{\ensuremath{#1}}}}}}
\newcommand{\vecc}[1]{\overset{\scriptscriptstyle\smallrightarrow}{#1}}
\newcommand{\cev}[1]{\overset{\scriptscriptstyle\smallleftarrow}{#1}}
\newcommand{\cevvec}[1]{\overset{\scriptscriptstyle\smallleftrightarrow}{#1}}

\newcommand{\dbar}{d\hspace*{-0.08em}\bar{}\hspace*{0.1em}}

%SetFonts

%SetFonts


\title{8.851 Final Project: Chiral Lagrangians in Lattice QCD}
\author{Patrick Oare}
\date{}							% Activate to display a given date or no date

\begin{document}
\maketitle

\thispagestyle{empty}

\section*{Abstract}

This project will discuss how to implement the chiral Lagrangian in lattice QCD. We will begin with a brief introduction to 
lattice QCD and introduce Wilson fermions, before turning to a discussion of chiral symmetry on the lattice and introducing 
Ginsparg-Wilson fermions. We will show how to construct the chiral Lagrangian for these types of fermions by regarding the 
finite lattice-spacing corrections to the QCD action as external operators. A spurion analysis similar to that for the QCD mass 
term can be conducted to add these corrections to the chiral Lagrangian, and we shall compute these spurions and present 
the chiral Lagrangian for Wilson and Ginsparg-Wilson fermions in lattice QCD up to order $a^2$.

\newpage
\setcounter{page}{1}

\section{Introduction}

% TODO this is too trivial of an introduction paragraph probably
Quantum Chromodynamics (QCD) is the theory of the strong nuclear force. It is a $SU(3)$ gauge theory coupled to 6 
flavors of fermions-- the quarks. At low energies, the coupling of QCD is large and perturbation theory cannot be applied to 
the theory, which means physics must be extracted by non-perturbative means. One such way to do this is to formulate 
QCD as a \textbf{lattice gauge theory} by discretizing spacetime-- the advantage of this is that the path integral becomes a 
finite (albeit large) dimensional integral which can be evaluated on a computer via a Monte Carlo simulation. 

Unfortunately due to the complexity of QCD, it is often difficult to approximate the path integral with a low degree of 
uncertainty. A major limitation is inverting the discretized Dirac operator $\slashed D + m_q$ as a $N^2$ dimensional matrix, 
where $N$ is the number of lattice sites. When $m$ is small this operator is near singular, and inversions are much noisier 
and more expensive. To reduce this cost and error, lattice computations are often done where the light quarks are taken to 
have a larger-than-physical mass, and then extrapolated to the limit of physical quark mass.

These quark mass extrapolations all rely on chiral perturbation theory ($\chi$PT), which offers us valuable insight into the 
structure of low-energy QCD. Looking at QCD in the massless quark limit allows us to study the spontaneous breaking of 
chiral symmetry, which yields the chiral Lagrangian. Using the chiral Lagrangian, we can analytically study the light 
spectrum of QCD, and make predictions about how changing the light quark masses will change the physics of QCD.

The chiral Lagrangian was developed for continuum studies of QCD, and as a result it does not contain any discretization 
artifacts which are generated by discretizing spacetime and putting it on a finite lattice. As a result, it is of interest to 
study how to put the chiral Lagrangian on the lattice. We will see that it is more difficult than in the continuum case because 
lattice actions have a plethora of terms which break chiral symmetry. However, we can deal with these terms by treating 
them as spurions, much like how the mass term is added to the continuum chiral Lagrangian. 

We will begin with a review of chiral perturbation theory and the spurion technique. Next, we will give an overview of lattice 
QCD and discuss its degrees of freedom. We will then use the Symanzik action as an example and show how to add 
discretization artifacts from the action to the chiral Lagrangian. Finally, we will end by studying Ginsparg-Wilson 
fermions in chiral perturbation theory, which are a specific type of lattice fermion that behaves nicely under chiral symmetry.

\section{Preliminaries}

\subsection{The chiral Lagrangian}

The Lagrangian for massless QCD in the Minkowski continuum is
\begin{equation}
	\mathcal L = \overline\psi i\slashed D \psi = \overline\psi_L i\slashed D \psi_L + \overline\psi_R i\slashed D\psi_R
\end{equation}
where the left and right handed components of $\psi$ are defined as $\psi_L = \frac{1 - \gamma_5}{2}\psi$ and $\psi_R 
= \frac{1 + \gamma_5}{2}\psi$. This decomposition makes it evident that massless QCD is invariant under the 
symmetries $(\psi_L, \psi_R)\mapsto (L\psi_L, R\psi_R)$, where $(L, R)\in SU(N_f)_L\times SU(N_f)_R$\footnote{The 
symmetry group here could be taken to be $U(N_f)_L\times U(N_f)_R$, but the axial $U(1)$ pieces is broken by anomalies 
and $U(1)_V$ symmetry leads to $B - L$ conservation.} are flavor rotations and $N_f$ is the number of massless quarks. 
The chiral symmetry of $\mathcal L$ under $G := SU(N_f)_L\times SU(N_f)_R$ is spontaneously broken by the quark 
condensate $\langle q\overline q\rangle\neq 0$ down to the subgroup $H := SU(N_f)_V$, where $SU(N_f)_V$ is the vector 
subgroup of $G$ given by $H = \{(V, V) \in G : V\in SU(N_f)\}$. 

The chiral Lagrangian is a theory of the Goldstone bosons which result from the spontaneous symmetry breaking 
pattern $G\rightarrow H$. It can be matched to QCD at low energies, and used to make phenomenological predictions 
about the light degrees of freedom of QCD. To construct the chiral Lagrangian, we begin with a field $\Sigma$ for the 
Goldstone bosons. Since Goldstone bosons live in the quotient space $G / H$, $\Sigma$ must parameterize this quotient, 
and by exhibiting such a parameterization we can determine how $\Sigma$ must transform under a chiral rotation. 
Let $(g_L, g_R)\;mod\; H$ be an arbitrary element in $G / H$. Then we may write:
\begin{equation}
	(g_L, g_R)\;mod\;H = (g_L g_R^\dagger, 1) (g_R, g_R)\;mod\; H = (g_L g_R^\dagger, 1) \;mod\; H
\end{equation}
as $(g_R, g_R)$ lives in the vector subgroup $H$. This means that we can parameterize $G / H$ with $\Sigma := g_L 
g_R^\dagger\in SU(N_f)$, and hence we see that $\Sigma$ must transform under $(L, R)\in G$ as:
\begin{equation}
	\Sigma\mapsto L\Sigma R^\dagger
\end{equation}

We can expand $\Sigma(x) := \exp\left(\frac{2i\Pi^a(x) \tau^a}{f}\right)$, where $\tau^a$ are the generators of $SU(N_f)$, 
and the $\Pi^a$ will be either a triplet or octet of mesons, depending on whether we take $N_f = 2$ or $N_f = 3$. The 
lowest order chiral Lagrangian can now be written down with the most general set of terms consistent with $\Sigma
\mapsto L\Sigma R^\dagger$, which is:
\begin{equation}
	\mathcal L_\chi = \frac{f^2}{4}\, tr\left\{\partial_\mu\Sigma \partial^\mu\Sigma^\dagger\right\}
\end{equation}
This can be expanded in the meson fields $\Pi^a$ to compute their dynamics and interactions. 

We can add other interactions to this Lagrangian via a \textbf{spurion} analysis, which we will later generalize for the 
case of lattice field theory. To do this, we will take an operator not invariant under chiral symmetry and add it to the QCD 
Lagrangian with a Wilson coefficient which we will take to be a dynamical field instead of simply a coupling. This field is 
called the spurion, and taken to transform under $G$ in such a way that the entire term is invariant under $G$. Once 
this is done, this term is added to the chiral Lagrangian and set equal to its initial value. 

An explicit example of this method is done by adding the QCD mass term to $\mathcal L_\chi$. The mass term $m\overline 
\psi \psi = m(\overline\psi_L\psi_R \overline\psi_R\psi_L)$ couples together the right and left handed components of $\psi$, 
and thus is only invariant under the vector subgroup $SU(N_f)_V$ and not the entire group $G$. To make this invariant 
under $G$, we promote $m$ to a field $M$. We take $M$ to transform under $(L, R)\in G$ as $M\mapsto L M R^\dagger$, 
which makes $\overline\psi_L M\psi_R + \overline\psi_R M^\dagger \psi_L$ invariant under chiral symmetry. This term 
can now be added to $\mathcal L_\chi$ at lowest order:
\begin{equation}
	\Delta\mathcal L_\chi = \mu\, tr\left\{M\Sigma^\dagger + M^\dagger\Sigma\right\}
\end{equation}
where $\mu$ is a coupling. Once this term is added, we set it to its constant value $M_0 = diag(m_1, m_2, ...)$ and obtain:
\begin{equation}
	\mathcal L_\chi = \frac{f^2}{4} tr\left\{\partial_\mu\Sigma\partial^\mu\Sigma^\dagger\right\} - \mu\,tr\left\{
	M_0\left(\Sigma + \Sigma^\dagger\right)\right\}
\end{equation}
When we consider the chiral Lagrangian on the lattice, we will add in discretization artifacts to the Lagrangian as spurions.
Before we can do this, we must discuss some background from lattice gauge theory.

\subsection{Lattice QCD}

This section will outline the background we need to study chiral Lagrangians on the lattice; specifically, we will define our 
theory and the operators of relevance to us. We begin by making some initial definitions: denote our spacetime lattice with 
spacing $a$ by $\Lambda$, and Wick rotate to imaginary time so that $\Lambda$ is a Euclidean lattice\footnote{We must 
Wick rotate so that the Boltzmann factor of $e^{iS}$ in the path integral becomes a valid probability density $e^{-S}$, in 
order to perform any computations at all.}. In the full theory of QCD, the dynamical fields in the path integral are the quark 
fields $\psi$ and gluon fields $A_\mu$. When we discretize QCD, we will still work with the quark fields, but instead of 
directly working with the gauge fields $A_\mu$ we will work with \textbf{link fields} $U_\mu(x)$ which transform in the 
following way under a gauge transformation $\Omega : \Lambda\rightarrow SU(3)$:
\begin{equation}
	U_\mu(x)\xrightarrow{\Omega(x)}\Omega(x) U_\mu(x)\Omega(x + a\hat{\mu})^\dagger
\end{equation}
Here $\hat{\mu}$ is the unit vector in the $\mu$ direction, and $x\in\Lambda$ denotes a site in the lattice. These link fields 
are the discrete counterpart of a Wilson line in continuum QFT, and may be expanded as $U_\mu(x) = \exp(ia A_\mu(x))$. 
They serve the role of a connection between fibers at different points in $\Lambda$, and hence they allow us to define a 
\textbf{discrete covariant derivative} $\nabla_\mu$ as:
\begin{equation}
	\nabla_\mu\psi(x) := \frac{U_\mu(x)\psi(x + a\hat{\mu}) - \psi(x)}{a}
\end{equation}
$\nabla_\mu$ will take the place of $D_\mu = \partial_\mu + ig A_\mu$ in our lattice theory. 

Using this, we can discretize the Euclidean Lagrangian $\overline\psi(\slashed D + m)\psi$. Note that discretization is not 
unique, as different choices of lattice actions can yield the same continuum limit. The simplest choice of discretization of 
the Dirac operator $\slashed D$ is known as the \textbf{Wilson operator} $D_W$:
\begin{equation}
	D_W := \frac{1}{2}\left\{\gamma^\mu (\nabla_\mu + \nabla_\mu^* - a r \nabla_\mu^*\nabla^\mu)\right\}~
	\label{eq:D_wilson}
\end{equation}
where $r$ is a parameter which is typically set to 1 in lattice calculations. There are two parts to the Wilson operator: the 
first is the naive discretization of $\slashed D$ as $\frac{1}{2}\gamma^\mu(\nabla_\mu + \nabla_\mu^*)$. The second 
term proportional to $a\nabla_\mu^*\nabla^\mu$ is added to deal with the so-called \textbf{doubling problem}. The discrete 
nature of the lattice gives rise to a periodic Brillouin zone in momentum space, of period $2\pi / a$. The naive discretization 
of $\slashed D$ has a pole at $p^2 = 0$, but because of this periodicity also has poles at each edge of the Brillouin zone, 
which correspond to unphysical particles which are called doublers. The second term in $D_W$ is added to cancel out 
these poles and remove the doublers, leaving only the physical pole in $D_W$ at $p^2 = 0$. Adding this term to 
$D_W$ is completely valid because it vanishes as we take the continuum limit. The Wilson operator thus gives us an 
action for \textbf{Wilson fermions}:
\begin{equation}
	\mathcal S_W = a^4\sum_x \overline\psi(D_W + m_q)\psi~
	\label{eq:wilson_fermions}
\end{equation}

% Symmetries that we allow: H(4), C, and P
Upon discretization of our theory, the Euclidean symmetry group $O(4)$ breaks down to its finite subgroup $H(4)$. If we are 
given a 4-vector $X^\mu$, this reduction in symmetry means that we can form more invariants under $H(4)$ from $X^\mu$ 
than we originally had under $O(4)$. In the continuum case, the only Lorentz invariant is $X^2 = X_\mu X^\mu$, and any 
multiple or power of this quantity. However, under $H(4)$, we can form the quantities:
\begin{equation}
	X^{[2n]} := \sum_\mu X_\mu^{2n}
\end{equation}
which will be invariant under all hypercubic symmetries for $n\in\mathbb Z$. Effectively, this means that when we examine 
the operators that we can add to the chiral Lagrangian via a spurion analysis, we can also consider operators of this form 
which are \textit{not} Lorentz invariants but \textit{are} $H(4)$ invariants. 

% Discretization artifacts: what they look like and the Symanzik EFT
Equation~\ref{eq:wilson_fermions} provides a simple lattice action which is quite useful in practice and allows for efficient 
computation of observables. However, because we have mutilated our original theory of QCD by putting it on a finite lattice, 
we also must deal with discretization artifacts. To see where these come from, let us examine a toy model of discretization 
in 1D. We will approximate the derivative of a function $f'(x)$ with a discrete difference operator $\delta$ on a lattice 
with spacing $a$. We may Taylor expand $f(x \pm a)$ as:
\begin{equation}
	f(x\pm a) = f(x) \pm a f'(x) + \frac{a^2}{2}f''(x) \pm \frac{a^3}{6} f^{(6)}(x) + \mathcal O(a^4)
\end{equation}
The discrete difference of $f$ thus approximates $f'(x)$ as:
\begin{equation}
	\delta f(x) = \frac{f(x + a) - f(x - a)}{2a} = f'(x) + \frac{a^2}{6} f^{(3)}(x) + \mathcal O(a^3)~
	\label{eq:difference_expansion}
\end{equation}
Note that the $a^2$ correction is written as the continuum operator $f^{(3)}(x)$, not a discrete one. 

We see that this allows us to quantify the error we impose from discretization in powers of $a$. We can also think about 
ways to \textit{reduce} this error that we impose by adding extra terms to the difference operator in such a way 
that correction terms come in at a higher order in $a$. For example, we can eliminate the term of order $a^2$ in the 
difference expansion in Equation~\ref{eq:difference_expansion} by adding in a term $D^{(3)}[f]$ at order $a^2$ in the power 
counting which proportional to $f^{(3)}(x)$ in the continuum limit. Here, we will take:
\begin{equation}
	D^{(3)}[f](x) = \frac{f(x + 2a) - 2f(x + a) + 2f(x - a) - f(x - 2a)}{2a^3} = f^{(3)}(x) + \mathcal O(a^2)
\end{equation}
It is now evident that using the modified difference operator $\Delta := \delta - \frac{a^2}{6}D^{(3)}$ will eliminate the 
discretization artifacts in Equation~\ref{eq:difference_expansion} at order $a^2$:
\begin{equation}
	\Delta f(x) = f'(x) + \mathcal O(a^4)
\end{equation}
In such a manner, we can in principle continue to remove discetization artifacts order by order. 

On the lattice, this procedure is called \textbf{Symanzik improvement}, and leads to a power counting EFT formulation 
of corrections to the Wilson action. Modifying our difference operators amounts to adding higher dimensional continuum 
operators to $\mathcal L_W$ and tuning the Wilson coefficients to eliminate these discretization artifacts. We will thus 
generally expand the lattice action as a power series in $a$:
\begin{equation}
	S_S = a^4\sum_x\left(a^{-1} \mathcal L_{-1} + a^0 \mathcal L_0 + a^1 \mathcal L_1 + a^2 \mathcal L_2 + ...\right)
\end{equation}
where each $\mathcal L_k$ contains continuum operators of dimension $4 + k$. Note that $a^{-1} \mathcal L_{-1} + a^0 
\mathcal L_0$ will be the usual Wilson Lagrangian. 

\subsection{Ginsparg-Wilson fermions}

Wilson fermions are cheap computationally and the simplest type of lattice fermion which are used in real computations, but 
unfortunately they play rather poorly with chiral symmetry. To see this, note that the key relation for chiral symmetry in the 
continuum is:
\begin{equation}
	\{\gamma_5, \slashed D\} = 0
\end{equation}
as this allows the massless QCD Lagrangian to split into a sum $\overline\psi_L \slashed D\psi_L + \overline\psi_R 
\slashed D\psi_R$ to have independent pieces with definite chirality. 

Unfortunately on the lattice, we do not have this property. For Wilson fermions governed by the Dirac operator in 
Equation~\ref{eq:D_wilson}, we see the Wilson Dirac operator $D_W$ contains the term $a \overline\psi\nabla^2\psi$ which 
connects the different chiral components of $\psi$. Even in the massless limit, this term will break chiral symmetry. 
More generally, it was shown by Nielson and Ninomiya \cite{nogo} that \textit{any attempt} to remove the doublers from a 
lattice regularized theory would result in such a breaking of this anticommutation relation, and thus the essence of chiral 
symmetry in lattice theories must be reformulated. 

Ginsparg and Wilson \cite{ginsparg} proposed an alternative symmetry on the lattice that acts as chiral symmetry, and 
indeed goes into chiral symmetry in the continuum limit $a\rightarrow 0$. The modified anticommutation relation is known as 
the \textbf{Ginsparg-Wilson equation}:
\begin{equation}
	\{\gamma_5,  D_G\} = aD_G\gamma_5 D_G~
	\label{eq:ginsparg_wilson}
\end{equation}
where we will denote by $D_G$ any lattice Dirac operator which satisfies this property. The Ginsparg-Wilson equation is not 
satisfied by Wilson Dirac operator $D_W$, but solutions of it are known to exist and can be implemented\footnote{See, for 
example, Neuberger's \textbf{overlap operator} in Reference~\cite{neuberger}.}.

Suppose that $D_G$ satisfies the Ginsparg-Wilson equation. Then $\overline\psi D_G \psi$ is invariant under a 
modified chiral rotation of the fields defined by:
\begin{equation}
	\psi\mapsto \exp\left(i\epsilon\gamma_5 \left(1 - \frac{1}{2}a D\right)\right)\psi
	\;\;\;\;\;\;\;\;\;\;\;\;\;\;\;\;\;\;\;\;\;\;\;\;\;\;\;\;\;\;\;\;\;\;\;\;\;\;\;\;\;\; \bar\psi\mapsto\bar\psi 
	\exp\left(i\epsilon\left(1 - \frac{1}{2}aD\right)\gamma_5\right)~
	\label{eq:luscher_transformation}
\end{equation}
hence we can take the Noether current generated by this symmetry to be our new definition of a chiral current. 

To make manifest the modified chiral symmetry of the Lagrangian, we follow an argument by 
Niedermayer \cite{niedermayer} and introduce modified chiral projectors which act as the standard projectors 
$\frac{1\pm\gamma_5}{2}$ in the continuum limit:
\begin{equation}
	\hat\gamma_5 := \gamma_5 (1 - a D)\;\;\;\;\;\;\;\;\;\;\;\;\;\;\;\;\;\;\;\;\;\;\;\;\;\;\;\;\;\;\;\;\;\;\;\;\;\;\;\;\;\; \hat P_\pm := 
	\frac{1\pm\hat\gamma_5}{2}
\end{equation}
Because $D$ satisfies Equation~\ref{eq:ginsparg_wilson}, these new projectors satisfy $D\hat P_\pm = P_\mp D$, which is a 
very similar algebra to the standard chiral projectors. This means that if we introduce modified chiral fields:
\begin{equation}
	\psi_L = \hat P_-\psi ;\;\;\;\;\;\;\;\;\;\;\;\;\;\;\;\;\;\;\;\;\; \psi_R = \hat P_+\psi ;\;\;\;\;\;\;\;\;\;\;\;\;\;\;\;\;\;\;\;\;\; \bar\psi_L = \bar\psi 
	P_+ ;\;\;\;\;\;\;\;\;\;\;\;\;\;\;\;\;\;\;\;\;\; \bar\psi_R = \bar\psi P_-
\end{equation}
then our Lagrangian splits into chiral components $\mathcal L = \bar\psi_L D\psi_L + \bar\psi_R D\psi_R$ as in the 
continuum case, and has chiral symmetry $SU(N_f)_L\times SU(N_f)_R$ as in the continuum case. 

\section{Chiral Lagrangians in Lattice QCD}

\subsection{Wilson fermions}

We are now in a position to put the chiral Lagrangian on the lattice. We will begin by adding Wilson fermions to the 
chiral Lagrangian, as they are often cheaper computationally and simpler to use. We will consider discretization artifacts in 
the Symanzik EFT as external fields, and add them to the chiral Lagrangian as spurions. We begin by examining the 
operators which we may add to our theory. The first two terms are those making up the usual QCD Lagrangian:
\begin{align}
	\mathcal O(a^{-1})\;\;&:\;\;\overline\psi\psi \\
	\mathcal O(a^{0})\;\;&:\;\; m_0\overline\psi\psi, \psi \slashed D \psi \\
\end{align}
Note that terms which are equal modulo a mass coefficient can be taken to be the same, as we can simply absorb these 
extra terms into our definitions of the couplings by redefining them, i.e. above we can take $m'\overline\psi \psi := (a^{-1} + 
m_0)\overline\psi \psi$. 

The operators that contribute at order $a$ are dimension 5 operators which are invariant under $C$, $P$, and $H(4)$ 
rotations. There are two such operators:
\begin{equation}
\mathcal O^{(5)}_1 = \overline\psi i\sigma_{\mu\nu} F^{\mu\nu} \psi\;\;\;\;\;\;\;\;\;\;\;\;\;\;\;\;\;\;\;\;\;\;\;\;\;\;\;\;\;\;\;\;\;\;\;\;\;\;\;\;\;\;\;\;
\mathcal O^{(5)}_2 = \overline\psi \vec D^2\psi + h.c. 
\end{equation}
$\mathcal O_2^{(5)}$ can be eliminated using the equations of motion in favor of $m^2\overline \psi \psi$ and 
$\mathcal O_1^{(5)}$, which can be shown to hold to all orders in perturbation theory~\cite{oa_improvement}. Therefore, 
we need only consider $\mathcal O_1^{(5)}$, which we will call the \textbf{Pauli term}. This transforms like the mass term 
$m\overline\psi \psi$ under chiral transformations, and we will need a corresponding spurion field for it in $\mathcal L_\chi$. 

At dimension 6, we must consider two types of operators: those which break chiral symmetry, and those which do not. 
There are six different dimension 6 operators which break chiral symmetry:
\begin{align}
        \mathcal O_1^{(6)} = (\overline\psi\psi)^2 \;\;\;\;\;\;\;\;\;\;\;\;\;\;\;\;\;\;\;\; \mathcal O_2^{(6)} &= 
        (\overline\psi\gamma_5\psi)^2\;\;\;\;\;\;\;\;\;\;\;\;\;\;\;\;\;\; \mathcal O_3^{(6)} = (\overline\psi\sigma_{\mu\nu}\psi)^2 
        \nonumber \\ 
        \mathcal O_4^{(6)} = (\overline\psi t^a \psi)^2 \;\;\;\;\;\;\;\;\;\;\;\;\;\;\;\;\;  \mathcal O_5^{(6)} &= (\overline\psi 
        t^a\gamma_5\psi)^2 \;\;\;\;\;\;\;\;\;\;\;\;\;\;\;  \mathcal O_6^{(6)} = (\overline\psi t^a\sigma_{\mu\nu}\psi)^2 ~
        \label{eq:dimension_six}
\end{align}
Each of these symmetry breaking structures must be added to $\mathcal L_\chi$ with a spurion field. Of the operators 
which preserve chiral symmetry, all but one can be incorporated into the chiral Lagrangian by a redefinition of couplings or 
renormalization coefficients. The exceptional operator is:
\begin{equation}
	\mathcal O_7^{(6)} = \sum_\mu\overline\psi\gamma_\mu D_\mu D_\mu D_\mu\psi
\end{equation}
$\mathcal O_7^{(6)}$ is special because it is the first operator we have seen which breaks $O(4)$ symmetry, and this 
symmetry breaking should be accounted for in the chiral Lagrangian at the appropriate order. 

$\mathcal L_\chi$ will power count in the parameters $p^2$, $m$, and $a$. The leading order (LO) and next to leading 
order (NLO) parts of the Lagrangian are $\mathcal L_\chi = \mathcal L_\chi^{(0)} + \mathcal L_\chi^{(1)} + ...$, where these 
terms are:
\begin{align}
    \mathcal L_\chi^{(0)}&\in \mathcal O(p^2, m, a)\nonumber \\
    \mathcal L_\chi^{(1)}&\in \mathcal O(p^4, p^2 m, p^2 a, m^2, ma, a^2)
\end{align}

Now we can move toward a spurion analysis of the LO and NLO chiral Lagrangians. First, we must ask: when are two 
spurions to be considered equal? Each symmetry breaking term will not, in fact, contribute its own spurion. Two 
terms which transform under chiral symmetry in the same way and have the same power counting in $m$ and $a$ will 
contribute the same spurion to the chiral Lagrangian, as we only care about the symmetry patterns of the spurion field and 
can always make a field redefinition to absorb another spurion which transforms identically. 

We can now use this analysis to determine the general structure of the LO and NLO chiral Lagrangians. At dimension 
5, the Pauli term contributes a spurion $A$ which is first order in the lattice spacing $a$ and transforms as:
\begin{equation}
	A\mapsto L A R^\dagger
\end{equation}
This transformation keeps the term  $A_{ij}\overline(\psi_L)_i i\sigma_{\mu\nu} F^{\mu\nu}(\psi_R)_j + h.c.$ invariant under 
chiral transformations, and thus it can be added to the chiral Lagrangian. The $A$ field will be set to the constant value of 
$A_0 = a I$, where $I$ is the $N_f\times N_f$ identity matrix in flavor space. 

The fermion bilinears of dimension 6 will receive two types of spurions, which we will denote by $B$ and $C$. They 
transform as:
\begin{align}
    B := B_1\otimes B_2&\mapsto (L B_1 R^\dagger) \otimes (L B_2 R^\dagger) \nonumber \\
    C := C_1\otimes C_2 &\mapsto (R C_1 L^\dagger) \otimes (L C_2 R^\dagger) 
\end{align}
This will keep each bilinear in Equation~\ref{eq:dimension_six} invariant under chiral rotations. For example, each term 
comprising the operator $\mathcal O_1^{(6)} = (\overline\psi_L\psi_R + \overline\psi_R\psi_L)^2$ can be kept invariant 
by coupling to $B, C, B^\dagger$, or $C^\dagger$, as each term transforms like some variant of:
\begin{equation}
        B_{ijkl}(\overline\psi_L)_i(\psi_R)_j (\overline\psi_L)_k (\psi_R)_l = (\overline\psi_L B_1 \psi_R) (\overline\psi_L B_2 
        \psi_R)
\end{equation}

Using the transformation $\Sigma\mapsto L\Sigma R^\dagger$, we can thus write out the new terms which will appear in 
$\mathcal L_\chi$ at LO and NLO once the spurion fields are set to their constant values. These are:
\begin{align}
    \mathcal O(a) &:  tr\{A\Sigma^\dagger + A^\dagger\Sigma\}\rightarrow a\,tr\{\Sigma + \Sigma^\dagger\} \nonumber\\
    \mathcal O(a^2) &: tr\{B_1\Sigma^\dagger\}tr\{B_2\Sigma^\dagger\}\rightarrow a^2 tr\{\Sigma^\dagger\}^2 \nonumber\\
    &: tr\{B_1\Sigma^\dagger B_2 \Sigma^\dagger\}\rightarrow a^2 tr\{\Sigma^\dagger\Sigma^\dagger\} \nonumber\\
    &: tr\{C_1\Sigma\}\,tr\{C_2\Sigma^\dagger\}\rightarrow a^2 tr\{\Sigma\}\,tr\{\Sigma^\dagger\}
\end{align}
Note that $\Sigma = \exp(2i\Pi / f)$ is unitary, hence $tr\{C_1\Sigma C_2\Sigma^\dagger\}\rightarrow a^2\, 
tr\{\Sigma\Sigma^\dagger\} = a^2 N_f$ and we need not include this combination of spurions. Furthermore, we do not 
consider terms like $tr\{B_1 C_1\}\,tr\{B_2 C_2^\dagger\}$, as these come in at order $a^4$ and will not appear in the 
LO or NLO Lagrangian. 

Using these terms and the mass spurion as $\hat m = diag(m_1, ..., m_{N_f}$, we can expand the chiral Lagrangian to NLO. 
At first order, the chiral Lagrangian looks much like it previously did by adding the mass term, with an additional coupling 
proportional to the lattice spacing:
\begin{equation}
    \mathcal L^{(0)}_\chi = \frac{f^2}{4}\left[tr\left\{\partial_\mu\Sigma\partial^\mu\Sigma^\dagger\right\} - B_0\,tr\left\{\hat m 
    (\Sigma + \Sigma^\dagger)\right\} - a\,W_0\,tr\left\{\Sigma + \Sigma^\dagger\right\}\right]
\end{equation}
At NLO, we get a variety of new terms that depend on the lattice spacing. We will not expand the entire NLO Lagrangian for 
the sake of brevity, but we will note a few pertinent factors. Here, $\mathcal L_\chi^{cont\,(1)}$ is the NLO chiral Lagrangian 
in the continuum. The entire NLO Lagrangian can be found in Equation 22 of~\cite{bar}. 
    \begin{align}
        \mathcal L_\chi^{(1)} = \mathcal L_\chi^{cont\,(1)} &+ a\, W_4\,tr\left\{\partial_\mu\Sigma\partial^\mu\Sigma^\dagger\right\} \,tr\left\{\Sigma + \Sigma^\dagger\right\} + ... \\
        ...&+ a\, W_8\, tr\left\{\hat m(\Sigma^\dagger \Sigma^\dagger + \Sigma\Sigma)\right\} +... \\
        ...&+ a^2 W_8'\, tr\left\{\Sigma^\dagger\Sigma^\dagger + \Sigma\Sigma\right\}
    \end{align}
This completes our study of the chiral Lagrangian for Wilson fermions. We will next examine a different lattice action which 
is more suited to chiral symmetry, a mixed action comprised of Wilson fermions and Ginsparg-Wilson fermions.

\subsection{Mixed action}

As mentioned before, Wilson fermions wreak havoc on chiral symmetry. In many cases, they perform so poorly with 
chiral symmetry that chiral extrapolation is not even possible, because they quark masses are unable to enter the 
regime where $\mathcal L_\chi$ describes the dominant dynamics. Because of this, it is of interest to consider other 
types of actions which use fermions that are more chirally symmetric, for example Ginsparg-Wilson fermions. 
Ginsparg-Wilson fermions are too expensive to model a full lattice theory with, so we will pick and choose our battles: 
we will consider a \textbf{mixed action} with \textit{Wilson sea quarks} and \textit{Ginsparg-Wilson valence and ghost 
quarks}. This action has the advantage that it less computationally intensive than a full Ginsparg-Wilson theory, but more 
chiral symmetry than a theory with only Wilson fermions. 

We will consider a theory with $N_f$ Wilson sea quarks and $N_V$ Ginsparg-Wilson valence quarks. The sea degrees of 
freedom will be contained in $\psi_S$, and the valence and ghost degrees of freedom will be contained in $\psi_V$. In 
flavor space, $\psi_S$ is a $N_f$ dimensional spinor, and $\psi_V$ is a $2N_V$ dimensional spinor, as it contains valence 
quarks and ghosts. The action is:
\begin{equation}
	S = a^4\sum_x\left(\overline\psi_S D_W\psi_S + \overline\psi_V D_G\psi_V\right)
\end{equation}
Note that since the ghost fields in $\psi_V$ have bosonic statistics, the group of chiral symmetry transformations on 
$\psi_V$ is \textit{not} $SU(2N_V)$, but rather $G_{val} := SU(N_V | N_V)$, where $SU(N_V | N_V)$ is the graded group of 
superunitary transformations $U$ which leave the $2N_V\times 2N_V$ dimensional matrix $diag(1, ..., 1, -1, ..., -1)$ invariant. We 
denote by $G_M$ the full group of chiral symmetries of the mixed action, and this is:
\begin{equation}
	G_M := SU(N_f + N_V | N_V)_L\times SU(N_f + N_V | N_V)_R
\end{equation}

\newpage
\thispagestyle{empty}

\begin{thebibliography}{99}

\bibitem{oa_improvement}
L\"{u}scher, M., Sint, S., Sommer, R., Weisz, P. (1996). Chiral symmetry and O(a) improvement in lattice QCD arXiv  478(1-2), 365-397. https://dx.doi.org/10.1016/0550-3213(96)00378-1

\bibitem{bar}
Bar, O.Chiral perturbation theory at O($a^2$) for lattice QCD. https://arxiv.org/abs/hep-lat/0306021v2

\bibitem{nogo}
Nielsen, H. B., \& Ninomiya, M. (1981). A no-go theorem for regularizing chiral fermions. 
In Physics Letters B (Vol. 105). https://doi.org/10.1016/0370-2693(81)91026-1
	
	\bibitem{ginsparg}
	Ginsparg, P. H., \& Wilson, K. G. (1982). A remnant of chiral symmetry on the lattice (Vol. 25). 
	https://journals.aps.org/prd/pdf/10.1103/PhysRevD.25.2649
	
	\bibitem{neuberger}
	Neuberger, H. (1999). The Overlap Dirac Operator. hep-lat/9910040v1
	
	\bibitem{Luscher}
	L\"uscher, M. (1998). Exact chiral symmetry on the lattice and the Ginsparg-Wilson relation. hep-lat/9802011
	
	\bibitem{niedermayer}
	Niedermayer, F. (1998). Exact chiral symmetry, topological charge and related topics. hep-lat/9810026

\end{thebibliography}

\end{document}